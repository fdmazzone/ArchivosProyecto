\documentclass[twoside]{article}


%\usepackage{hyperref}
\usepackage{amssymb,amsthm}
\usepackage{amsmath}
\usepackage{color}
\usepackage{ esint }
\usepackage{mathabx}
\usepackage{MnSymbol}
\usepackage{fancyhdr}
%\usepackage{times}

%\usepackage[latin1]{inputenc}

\usepackage{comment}
\usepackage{url}
\usepackage{xcolor}
\usepackage{adjustbox}
\usepackage{hyperref}

\newtheorem{thm}{Theorem}[section]
\newtheorem{cor}[thm]{Corollary}
\newtheorem{lem}[thm]{Lemma}

\newtheorem{defi}[thm]{Definition}
\newtheorem{prop}[thm]{Proposition}
\theoremstyle{remark}
\newtheorem{comentario}{Remark}


\makeatletter
\newcommand{\labitem}[2]{%
\def\@itemlabel{\textbf{#1}}
\item
\def\@currentlabel{#1}\label{#2}}
\makeatother




\title{Periodic solutions for a Sitnikov restricted four-body problem with primaries in a colinear configuration}
\author{Gast\'on Beltritti \thanks{SECyT-UNRC and CONICET}\\
Dpto. de Matem\'atica, Facultad de Ciencias Exactas Físico-Químicas y Naturales\\
Universidad Nacional de R\'{i}o Cuarto\
(5800) R\'{\i}o Cuarto, C\'ordoba, Argentina,\\
\url{gbeltritti@exa.unrc.edu.ar}\\[3mm]
Fernando D. Mazzone \thanks{SECyT-UNRC, FCEyN-UNLPam}\\
Dpto. de Matem\'atica, Facultad de Ciencias Exactas, F\'{\i}sico-Qu\'{\i}micas y Naturales\\
Universidad Nacional de R\'{i}o Cuarto\\
(5800) R\'{\i}o Cuarto, C\'ordoba, Argentina,\\
\url{fmazzone@exa.unrc.edu.ar}\\
Martina G. Oviedo \thanks{SECyT-UNRC, CIN}\\
Dpto. de Matem\'atica, Facultad de Ciencias Exactas, F\'{\i}sico-Qu\'{\i}micas y Naturales\\
Universidad Nacional de R\'{i}o Cuarto\\
(5800) R\'{\i}o Cuarto, C\'ordoba, Argentina,\\
\url{martinagoviedo@gmail.com}
}

\date{}


\newcommand{\rr}{\mathbb{R}}
\newcommand{\nn}{\mathbb{N}}


\newcounter{example}

\setcounter{example}{1}


\newenvironment{example}{\noindent\textit{Example} \arabic{example}.}{\addtocounter{example}{1}}




\begin{document}


\maketitle
%
% \begingroup%Locallizing the change to `thefootnote'.
%     \renewcommand{\thefootnote}{}%Removing the footnote symbol.
%     %
%     \footnotetext{%
%     %   2010 Mathematics Subject Classification
%     %   http://www.ams.org/msc/
%     \textbf{2010  AMS Subject Classification.} Primary: .
%     Secondary: .
%     }%
%         \footnotetext{%
%     \textbf{Keywords and phrases.}  .
%     }%
%     \endgroup
%
%
%
%

\begin{abstract}


\end{abstract}




\pagestyle{fancy} \headheight 35pt \fancyhead{} \fancyfoot{}

\fancyfoot[C]{\thepage} \fancyhead[CE]{\nouppercase{G. Beltritti, F. Mazzone, M. Oviedo}} \fancyhead[CO]{\nouppercase{\section}}

\fancyhead[CO]{\nouppercase{\leftmark}}


%\tableofcontents




\section{Introduction}
In this paper we obtain existence of periodic solutions for the following restricted nonplanar Newtonian four-body problem (see figure \ref{fig:conf_esp}):
\begin{itemize}
 \item We have three primary bodies of masses $m_1,m_2,m_3$. The fourth body is masless.
 \item The primary bodies are in a central colinear rigid motion (see \cite[Section 2.9]{JaumeLlibre276}). This motion is carried out in a plane $\Pi$.
 \item The massless particle is moving on tha perpendicular line to $\Pi$ passing through the center of masses.
\end{itemize}


\begin{figure}[h]
 \begin{center}


\setlength{\unitlength}{4cm}
\begin{picture}(2.5, 1)(-.5, -.5)
  \qbezier(0, 0)(0,.25)(.5, .25)
  \qbezier(1, 0)(1,.25)(.5, .25)
    \qbezier(0, 0)(0,-.25)(.5, -.25)
  \qbezier(1, 0)(1,-.25)(.5, -.25)
  \setlength{\unitlength}{2cm}
    \put(.5,0){
    \qbezier(0, 0)(0,.25)(.5, .25)
  \qbezier(1, 0)(1,.25)(.5, .25)
    \qbezier(0, 0)(0,-.25)(.5, -.25)
  \qbezier(1, 0)(1,-.25)(.5, -.25)
}
  \setlength{\unitlength}{5cm}
    \put(-.1,0){
    \qbezier(0, 0)(0,.25)(.5, .25)
  \qbezier(1.01, 0)(1.01,.25)(.5, .25)
    \qbezier(0, 0)(0,-.25)(.5, -.25)
  \qbezier(1.01, 0)(1.01,-.25)(.5, -.25)
}
\put(-.2,0){\line(1,0){1.2}}
\put(-0.1,0){\circle*{.04}} \put(-0.19,-0.05){$m_1$}
\put(0.6,0){\circle*{.04}} \put(0.61,-0.05){$m_2$}
\put(0.8,0){\circle*{.04}}\put(0.805,-0.05){$m_3$}
\put(0.5,0){\circle*{.03}}\put(0.44,-0.05){$c$}
\put(0.5,-0.08){\line(0,1){.6}}
\put(0.5,0.4){\circle*{.04}}\put(0.53,0.4){$m_4\approx 0$}
\put(-.3,-0.3){\line(1,0){1.5}}
\put(-.3,-0.3){\line(1,5){.12}}
\put(1.2,-0.3){\line(-1,5){.12}}
\put(-0.18,0.3){\line(1,0){1.26}}
\put(1.1,-0.25){$\Pi$}
\end{picture}\caption{Four-body problem}\label{fig:conf_esp}
 \end{center}

\end{figure}

Problems like the one presented have been extensively discussed in the literature. In \cite{sitnikov1960existence} K. Sitnikov considered the problem of two body in a Keplerian motion and a massless particle moving in the perpendicular line to the orbital plane passing
through the center of masses. Sitnikov obtained deep results about existence of solutions, some of them periodic (see \cite[III(5)]{moser2016stable}. Since then many  other authors have studied Sitnikov problem, for instance Llibre and Ortega \cite{llibre2008families}, Chesley \cite{chesley1999global}, Hagel and Trnkler \cite{hagel1993computer}, Dvorak \cite{dvorak1993numerical}, {P{\'e}rez, Jim{\'e}nez and Lacomba \cite{perez2009periodic}, Liu, Zhou, and Sun \cite{liu1991numerical},
Llibre, Meyer and Soler \cite{llibre1999bridges},  Jim{\'e}nez-Lara a and Escalona-Buend{\'\i}a \cite{jimenez2001symmetries}, Dankowicz and Holmes \cite{dankowicz1995existence},



\section*{Acknowledgments}




 \bibliographystyle{apalike}
 \bibliography{mecanica_celeste}


\end{document}


