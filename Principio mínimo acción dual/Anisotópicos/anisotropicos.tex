\documentclass[twoside]{article}


\NeedsTeXFormat{LaTeX2e}
\ProvidesPackage{mathscinet}[2002/04/17 v1.05]
\RequirePackage{textcmds}\relax
\ProvideTextCommandDefault{\cprime}{\tprime}



%\usepackage{hyperref}
\usepackage{amssymb,amsthm}
\usepackage{amsmath}
\usepackage{color}
\usepackage{ esint }
\usepackage{mathabx}
\usepackage{MnSymbol}
\usepackage{fancyhdr}
\usepackage{times}

\usepackage[latin1]{inputenc}

\usepackage{comment}
\usepackage{url}
\usepackage{xcolor}
\usepackage{adjustbox}
\usepackage{hyperref}

\newtheorem{thm}{Theorem}[section]
\newtheorem{cor}[thm]{Corollary}
\newtheorem{lem}[thm]{Lemma}

\newtheorem{defi}[thm]{Definition}
\newtheorem{prop}[thm]{Proposition}
\theoremstyle{remark}
\newtheorem{comentario}{Remark}


\makeatletter
\newcommand{\labitem}[2]{%
\def\@itemlabel{\textbf{#1}}
\item
\def\@currentlabel{#1}\label{#2}}
\makeatother






% \newcommand{\orlnor}{\|_{L^{\Phi}}}
% \newcommand{\lurnor}{\|^{*}_{L^{\Phi}}}
\newcommand{\linf}{L^{\infty}}
\newcommand{\lphi}[2]{L^{\Phi}(#1,\rr^{#2})}
% \newcommand{\lphiuno}{L^{\Phi_1}}
% \newcommand{\lphidos}{L^{\Phi_2}}
% \newcommand{\lphii}{L^{\Phi_i}}
% \newcommand{\lpsi}{L^{\Psi}}
% \newcommand{\lpsiuno}{L^{\Psi_1}}
% \newcommand{\lpsidos}{L^{\Psi_2}}
% \newcommand{\lpsii}{L^{\Psi_i}}
% \newcommand{\lmuno}{L^{M_1}}
% \newcommand{\lmdos}{L^{M_2}}
% \newcommand{\lmj}{L^{M}}
% \newcommand{\lmn}{L^{M_n}}
\newcommand{\ephi}[2]{E^{\Phi}(#1,\rr^{#2})}
% \newcommand{\ephiuno}{E^{\Phi_1}}
% \newcommand{\ephidos}{E^{\Phi_2}}
% \newcommand{\ephin}{E^{\Phi_n}}
% \newcommand{\ephii}{E^{\Phi_i}}
\newcommand{\claseor}[2]{\tilde{L}^{\Phi}(#1,\rr^{#2})}
% \newcommand{\wphi}{W^{1}\lphi}
% \newcommand{\wphiuno}{W^{1}\lphiuno}
% \newcommand{\wphidos}{W^{1}\lphidos}
% \newcommand{\wphii}{W^{1}\lphii}
% \newcommand{\wphiet}{W^{1}\ephi_T}
% \newcommand{\wphie}{W^{1}\ephi}
% \newcommand{\sobnor}{\|_{W^{1}\lphi}}
% \newcommand{\domi}{\mathcal{E}^{\Phi}_d(\lambda)}
% \newcommand{\domiuno}{\mathcal{E}^{\Phi_1}_d(\lambda)}
% \newcommand{\domidos}{\mathcal{E}^{\Phi_2}_d(\lambda)}
% \newcommand{\domii}{\mathcal{E}^{\Phi_i}_d(\lambda)}
% \newcommand{\domin}{\mathcal{E}^{\Phi_n}_d(\lambda)}
\renewcommand{\b}[1]{\boldsymbol{#1}}
\newcommand{\rr}{\mathbb{R}}
% \newcommand{\nn}{\mathbb{N}}
% \newcommand{\ccdot}{\b{\cdot}}
% \renewcommand{\leq}{\leqslant} 
% \renewcommand{\geq}{\geqslant} 
% \newcommand{\epsi}{E^{\Psi}}
\newcommand{\franja}[2]{\Pi(\ephi_{#2},#1)} 



\newcounter{example}

\setcounter{example}{1}


\newenvironment{example}{\noindent\textit{Example} \arabic{example}.}{\addtocounter{example}{1}}




\begin{document}

$d_1,d_2,d_3$ positive integer. 
$\Omega\subset\rr^{d_1}$, $m(\Omega)<\infty$ y $f:\Omega\times \rr^{d_2}\to\rr^{d_3}$.

For Orlicz Spaces we will use the Orlicz norm.


Nemitski operator
\[\b{f}u(x)=f(x,u(x))\]
maps  $\rr^{d_2}$-valued functions defined on $\Omega\subset\rr^{d_1}$ into   
$\rr^{d_3}$-valued funcitions defined on $\Omega\subset\rr^{d_1}$.  Measurability 
seems follows the same lines that \cite[p. 349]{krasnosel2011integral}.???

Let $\Phi_1:\rr^{d_2}\to\rr$ and $\Phi_2:\rr^{d_3}\to\rr$  be anisotropic $N$-functions. 

\begin{thm}
 Deber \#1 Theorem similar to \cite[Lemma 17.2]{KR}
\end{thm}


\begin{proof}
Let $u\in\franja{r}{d_2}$ be. Adapting  the \cite[Prop. 3, p. 92-93]{rao1991theory} 
to $\rr^d$-valued functions (see also \cite{} By \cite[Th. 5.5]{schappacher2005notion} 
for Luxemburg norm and $N$-functions defined on infinite dimensional Banach space) 
(???) we obtain $\|u_0\|_{\lphi{\Omega}{d_2}}<r$ where $u_0=u-\chi_A u$ for certain 
set $A\subset\Omega$ where $\|u\|_{\linf(A,\rr^{d_2})}<\infty$, in particular 
$ \chi_A u \in \ephi{\Omega}{d_2}$. From  
\cite[Th. 5.4]{schappacher2005notion} we obtain that $\chi_A u$ has an absolutely 
continuous norm. Therefore we can finf $u_1,\ldots,u_n$ with $u_iu_j\equiv 0$, for 
$i\neq j$, $\|u_i\|_{\lphi{\Omega}{d_2}}<r$, $i=1,\ldots,n$ and $\chi_Au=u_1+\cdots+u_n$. The
result follows  
from the identity (see \cite[p. 353]{krasnosel2011integral})
\[\b{f}(u_0+\cdots +u_n)=\b{f}u_0+\cdots \b{f}u_n-(n-1)\b{f}0.\]

Deber \#2 completar otros enunciados
\end{proof}


Deber \#3  Acotacion y continuidad


Deber \#4   Nemitsky asociado a $f=\Phi'$. 



\begin{lem}[Lemma 9.1 KR] Suppose that $d\geq 1$ and 
$\|u\|_{L^{\Phi}(\Omega,\rr^d)}\leq 1$. Then $v(x)=\nabla \Phi(u(x))$ belong to 
$\tilde{L}^{\Psi}(\Omega,\rr^d)$ and   $\rho_{\Psi}(v)\leq 1$.

\end{lem}
\begin{proof} Sale igual, sólo hay que evitar escribir módulos
 
\end{proof}

El Lemma 9.2 KR sale sin cambios. 


 \bibliographystyle{plain} 
 \bibliography{biblio}


\end{document}
