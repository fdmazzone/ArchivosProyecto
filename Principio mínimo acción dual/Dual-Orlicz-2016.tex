\documentclass[twoside]{article}


%\usepackage{hyperref}
\usepackage{amssymb,amsthm}
\usepackage{amsmath}
\usepackage{color}
\usepackage{ esint }
\usepackage{mathabx}
\usepackage{MnSymbol}
\usepackage{fancyhdr}
\usepackage{times}
\usepackage{enumitem}
\usepackage[latin1]{inputenc}

\usepackage{comment}
\usepackage{url}
\usepackage{xcolor}
\usepackage{adjustbox}
\usepackage{hyperref}


\newtheorem{thm}{Theorem}[section]
\newtheorem{cor}[thm]{Corollary}
\newtheorem{lem}[thm]{Lemma}

\newtheorem{defi}[thm]{Definition}
\newtheorem{prop}[thm]{Proposition}
\theoremstyle{remark}
\newtheorem{comentario}{Remark}


\makeatletter
\newcommand{\labitem}[2]{%
\def\@itemlabel{\textbf{#1}}
\item
\def\@currentlabel{#1}\label{#2}}
\makeatother
\makeatletter
\def\namedlabel#1#2{\begingroup
    #2%
    \def\@currentlabel{#2}%
    \phantomsection\label{#1}\endgroup
}
\makeatother



\title{Periodic solutions of
Euler-Lagrange equations in an Orlicz-Sobolev space setting by the dual least action principle }
\author{Sonia Acinas \thanks{SECyT-UNRC and  FCEyN-UNLPam}\\
Dpto. de Matem\'atica, Facultad de Ciencias Exactas y Naturales\\
Universidad Nacional de La Pampa\\
(L6300CLB) Santa Rosa, La Pampa, Argentina\\
\url{sonia.acinas@gmail.com}\\[3mm]
Fernando D. Mazzone \thanks{SECyT-UNRC, FCEyN-UNLPam and CONICET}\\
Dpto. de Matem\'atica, Facultad de Ciencias Exactas, F\'{\i}sico-Qu\'{\i}micas y Naturales\\
Universidad Nacional de R\'{i}o Cuarto\\
(5800) R\'{\i}o Cuarto, C\'ordoba, Argentina,\\
\url{fmazzone@exa.unrc.edu.ar}
}

\date{}

\newcommand{\orlnor}{\|_{L^{\Phi}}}
\newcommand{\lurnor}{\|^{*}_{L^{\Phi}}}
\newcommand{\linf}{\|_{L^{\infty}}}
\newcommand{\lphi}{L^{\Phi}}
\newcommand{\lphiuno}{L^{\Phi_1}}
\newcommand{\lphidos}{L^{\Phi_2}}
\newcommand{\lphii}{L^{\Phi_i}}
\newcommand{\lpsi}{L^{\Phi^*}}
\newcommand{\lpsiuno}{L^{\Phie_1}}
\newcommand{\lpsidos}{L^{\Phie_2}}
\newcommand{\lpsii}{L^{\Phie_i}}
\newcommand{\lmuno}{L^{M_1}}
\newcommand{\lmdos}{L^{M_2}}
\newcommand{\lmj}{L^{M}}
\newcommand{\lmn}{L^{M_n}}
\newcommand{\ephi}{E^{\Phi}}
\newcommand{\ephiuno}{E^{\Phi_1}}
\newcommand{\ephidos}{E^{\Phi_2}}
\newcommand{\ephin}{E^{\Phi_n}}
\newcommand{\ephii}{E^{\Phi_i}}
\newcommand{\claseor}{C^{\Phi}}
\newcommand{\wphi}{W^{1}\lphi}
\newcommand{\wphiuno}{W^{1}\lphiuno}
\newcommand{\wphidos}{W^{1}\lphidos}
\newcommand{\wphii}{W^{1}\lphii}
\newcommand{\wphiet}{W^{1}\ephi_T}
\newcommand{\wphie}{W^{1}\ephi}
\newcommand{\sobnor}{\|_{W^{1}\lphi}}
\newcommand{\domi}{\mathcal{E}^{\Phi}_d(\lambda)}
\newcommand{\domiuno}{\mathcal{E}^{\Phi_1}_d(\lambda)}
\newcommand{\domidos}{\mathcal{E}^{\Phi_2}_d(\lambda)}
\newcommand{\domii}{\mathcal{E}^{\Phi_i}_d(\lambda)}
\newcommand{\domin}{\mathcal{E}^{\Phi_n}_d(\lambda)}
\renewcommand{\b}[1]{\boldsymbol{#1}}
\newcommand{\rr}{\mathbb{R}}
\newcommand{\nn}{\mathbb{N}}
\newcommand{\ccdot}{\b{\cdot}}
\renewcommand{\leq}{\leqslant} 
\renewcommand{\geq}{\geqslant} 
\newcommand{\epsi}{E^{\Phie}}
\newcommand{\Phie}{\Phi^{*}}

\newcounter{example}

\setcounter{example}{1}


\newenvironment{example}{\noindent\textbf{Example \arabic{example}}.}{\addtocounter{example}{1}}




\begin{document}


\maketitle
%
\begingroup%Locallizing the change to `thefootnote'.
    \renewcommand{\thefootnote}{}%Removing the footnote symbol.
    %
    \footnotetext{%
    %   2010 Mathematics Subject Classification
    %   http://www.ams.org/msc/
    \textbf{2010  AMS Subject Classification.} Primary: .
    Secondary: .
    }%
        \footnotetext{%
    \textbf{Keywords and phrases.}  .
    }%
    \endgroup
%
%
%
%

\begin{abstract}


\end{abstract}






\pagestyle{fancy} \headheight 35pt \fancyhead{} \fancyfoot{}

\fancyfoot[C]{\thepage} \fancyhead[CE]{\nouppercase{S. Acinas and F.D. Mazzone }} \fancyhead[CO]{\nouppercase{\section}}

\fancyhead[CO]{\nouppercase{\leftmark}}


%\tableofcontents




\section{Introduction}


This paper deals with system  of equations of the type:

\begin{equation}\label{ProbPrin-gral}
    \left\{%
\begin{array}{ll}
  \frac{d}{dt} D_{y}\mathcal{L}(t,u(t),u'(t))= D_{x}\mathcal{L}(t,u(t),u'(t)) \quad \hbox{a.e.}\ t \in (0,T)\\
    u(0)-u(T)=u'(0)-u'(T)=0,
\end{array}%
\right. ,\tag{$P_1$}
\end{equation}
where $\mathcal{L}:[0,T]\times\rr^d\times\rr^d\to\rr$, $d\geq 1$, is called the \emph{Lagrange function} or \emph{lagrangian} and the unknown function  $u:[0,T]\to\rr^d$ is absolutely continuous. In other words, we are interested in  finding \emph{periodic weak solutions} of \emph{Euler-Lagrange system}. This topic was deeply addressed for the \emph{Lagrange function}
\begin{equation}\label{eq:lagrange_cuad}
\mathcal{L}_{p,F}(t,x,y):=\frac{|y|^p}{p}+F(t,x),
\end{equation}
for $1<p<\infty$. For example, the classic book  \cite{mawhin2010critical} deals mainly with problem \eqref{ProbPrin-gral}, for the lagrangian $\mathcal{L}_{2,F}$, through various methods: direct, dual action, minimax, etc. The results in \cite{mawhin2010critical} were extended and improved in several articles, see  \cite{tang1995periodic,tang1998periodic,wu1999periodic,tang2001periodic,zhao2004periodic}  to cite some examples. Lagrange functions \eqref{eq:lagrange_cuad} for arbitrary $1<p<\infty$ were considered in  \cite{Tian2007192,tang2010periodic} and in this case \eqref{ProbPrin-gral}  is reduced to the $p$-laplacian system
\begin{equation}\label{ProbP-lapla}
    \left\{%
\begin{array}{ll}
   \frac{d}{dt}\left(u'(t)|u'|^{p-2}\right) = \nabla F(t,u(t)) \quad \hbox{a.e.}\ t \in (0,T)\\
    u(0)-u(T)=u'(0)-u'(T)=0.
\end{array}%
\right.\tag{$P_2$}
\end{equation}


In this context, it  is customary to call $F$ a  \emph{potential function}, and it is assumed that $F(t,x)$ is differentiable with respect to $x$ for a.e. $t\in [0,T]$ and the following conditions are verified:
\begin{enumerate}
\labitem{(C)}{item:condicion_c} $F$ and its gradient $\nabla F$, with respect to $x\in\rr^d$,  are  Carath\'eodory functions, i.e. they are measurable functions with respect to $t\in [0,T]$, for every  $x\in\rr^d$, and they are continuous functions with  respect to  $x\in\rr^d$ for a.e. $t \in [0,T]$.
 \labitem{(A)}{item:condicion_a}  For   a.e. $t\in [0,T]$, it holds that
\begin{equation}
|F(t,x)| + |\nabla F(t,x)|  \leq a(|x|)b(t).
\end{equation}
In this inequality we assume that the function  $a:[0,+\infty)\to [0,+\infty)$ is continuous and non decreasing and $0\leq b\in L^1([0,T],\rr)$.
\end{enumerate}


In \cite{ABGMS2015} it was treated  the case of a lagrangian $\mathcal{L}$ which is lower bounded by a Lagrange function
\begin{equation}\label{eq:lagrange_phi}
\mathcal{L}_{\Phi,F}(t,x,y)=\Phi(|y|)+F(t,x),
\end{equation}
where  $\Phi$ is an $N$-function (see section \ref{preliminares} for the definition of this concept).  
In the paper \cite{ABGMS2015} it was assumed  a condition of \emph{bounded oscillation} on $F$  (see xxxxx below). 
In this paper  we apply the dual method (\cite[Ch. 3]{mawhin2010critical}) to obtain solutions of \eqref{ProbPrin-gral}.



\section{Anisotropic Orlicz and Orlicz-Sobolev spaces}\label{preliminares}

In this section, we give a short introduction to  Orlicz and Orlicz-Sobolev spaces of vector valued functions associated to anisotropic Young functions $\Phi:\rr^n\to\rr_+$, i.e. functions such that $\Phi(x)$ depends on the direction of $x$, unlike the radial case where $\Phi(x)=\Phi(|x|)$.  References for  these topics are \cite{Orliczvectorial2005,Skaff1969, Desch2001}.

In the framework of anisotropic Sobolev-Orlicz spaces, we can study system of $p$-laplacian equations as the following example shows.


\begin{example} Let $1<p_1,p_2<\infty$. We define $\Phi_{p_1,p_2}:\rr^2\to\rr_+$  by
\[\Phi_{p_1,p_2}(y_1,y_2):=\frac{|y_1|}{p_1}+\frac{|y_2|}{p_2}.\]
Suppose the following Lagrange function
\[\mathcal{L}(t,x,y)=\Phi_{p_1,p_2}(y)+F(t,x).\]
Then the equations \eqref{ProbPrin-gral} becomes
\begin{equation}\label{eq:sist-p_lapa}
    \left\{%
\begin{array}{ll}
  \frac{d}{dt}\left(|u_1'|^{p_1-2}u_1'\right)=F_{x_1}(t,u) \quad \hbox{a.e.}\ t \in (0,T)\\
  \frac{d}{dt}\left(|u_2'|^{p_2-2}u_2'\right)=F_{x_2}(t,u) \quad \hbox{a.e.}\ t \in (0,T)\\
   u(0)-u(T)=u'(0)-u'(T)=0,
\end{array}%
\right., \tag{$P_3$}
\end{equation}


\end{example}

These equations are known in the literature as $(p_1,p_2)$-Laplacian system, see
\cite{yang2013existence,pasca2016periodic,yang2012periodic,pasca2010periodic,pacsca2010some,pasca2011some}.


On the other hand, anisotropic Orlicz-Sobolev spaces allow to simplify the writing, and they provide the natural frame of statements of the type \cite[Lemma 3.1]{Tian2007192}. This type of question was what motivated us to use these spaces.

Hereafter we denote  by $\mathbb{R}^+$  the set of all non negative real numbers. A function $\Phi:\mathbb{R}^d\to \mathbb{R}_+ $ is called an \emph{Young's function} if $\Phi$ is convex, $\Phi(0)=0$, $\Phi(-x)=\Phi(x)$ and $\Phi(x)\to +\infty$, when $|x|\to+\infty$. Additionally, we assume that the Young's functions which we deal with, satisfy that $\Phi(x)>0$ when $x\neq 0$. Following \cite{Orliczvectorial2005} we say that $\Phi$ is \emph{coercive} if
\[\lim_{|x|\to\infty}\frac{\Phi(x)}{|x|}=+\infty.\]

Given a Young's function $\Phi$, we define function $A_{\Phi}:\rr^+\to\rr^+$ by
\begin{equation}\label{eq:inversa-gral}
A_{\Phi}(s)=\min\left\{\Phi(x)\,\big|\,\|x\|=s\right\},
\end{equation}

Let us establish some elementary properties of $A_{\Phi}$ that we will use in this article.
\begin{prop}\label{prop:AsubPhi} The function $A_{\Phi}$ has the following properties:
\begin{enumerate}
 \item\label{it:prop1} $A_{\Phi}$ is continuous,
 \item\label{it:prop2} $A_{\Phi}(s)/s$ is increasing,
 \item\label{it:prop3} $A_{\Phi}(|x|)$ is the \emph{greatest radial minorant} of 
 $\Phi(x)$,
 \item\label{it:prop4} $\Phi$ is coercive if and only if $A_{\Phi}$ is.
\end{enumerate}
\end{prop}

\begin{proof} It is well known that finite and convex functions defined in finite dimensional 
vectorial spaces are locally Lipschitz functions (see \cite{clarke2013functional}). This fact 
imply item \ref{it:prop1} immediately. 

In order to prove item \ref{it:prop2}, suppose $0<r<s$ and $x\in\rr^d$ with $A_{\Phi}(s)
=\Phi(x)$. Then, from the definition of $A_{\Phi}$ and the convexity of $\Phi$,
\[\frac{A_{\Phi}(r)}{r}\leq \frac{\Phi\left(\frac{r}{s}x\right)}{r}\leq \frac{\Phi\left(x\right)}{s}=
 \frac{A_{\Phi}(s)}{s}.
\]
Property in items \ref{it:prop3} and \ref{it:prop4} are obtained easily.

 
\end{proof}






We also say that $\Phi:\mathbb{R}^d\rightarrow \mathbb{R}^+$ satisfies the  \emph{$\Delta_2^{\infty}$-condition}, denoted by $\Phi \in \Delta_2^{\infty}$,
if there exist  constants $K>0$ and  $M\geq 0$ such that
\begin{equation}\label{delta2defi}\Phi(2x)\leq KH(x),
\end{equation}
for every $|x|\geq M$.

If $\Phi$ is a Young's function we define its \emph{Fenchel conjugate}   $\Phi^*:\mathbb{R}^d\to \mathbb{R}^+ $ by:
\begin{equation}\label{eq:conjugada}
 \Phi^*(y)=\sup\limits_{x\in\mathbb{R}^d} x\cdot y-\Phi(x)
\end{equation}


 We denote by $\mathcal{M}:=\mathcal{M}([0,T],\rr^d)$, with $d\geq 1$,  the set of all measurable functions (i.e. functions which are limits of simple functions)  defined on $[0,T]$ with values on $\mathbb{R}^d$ and  we write $u=(u_1,\dots,u_d)$ for  $u\in \mathcal{M}$. For the set of functions $\mathcal{M}$, as for other similar sets, we will omit the reference to codomain $\mathbb{R}^d$ when $d=1$.


Given  an $N$-function $\Phi$ we define the \emph{modular function} 
$\rho_{\Phi}:\mathcal{M}\to \mathbb{R}^+\cup\{+\infty\}$ by
\[\rho_{\Phi}(u):= \int_0^T \Phi(u)\ dt.\]
Here $|\cdot|$ is the euclidean norm of $\mathbb{R}^d$.
Now, we introduce the \emph{Orlicz class} $C^{\Phi}=C^{\Phi}([0,T],\rr^d)$   by setting
\begin{equation}\label{claseOrlicz}
  C^{\Phi}:=\left\{u\in \mathcal{M} | \rho_{\Phi}(u)< \infty \right\}.
\end{equation}
The \emph{Orlicz space} $\lphi=L^{\Phi}([0,T],\rr^d)$ is the linear hull of $\claseor$;
equivalently,
\begin{equation}\label{espacioOrlicz}
\lphi:=\left\{ u\in \mathcal{M}| \exists \lambda>0: \rho_{\Phi}(\lambda u) < \infty   \right\}.
\end{equation}
  The Orlicz space $\lphi$ equipped with the \emph{Luxemburg norm}
\[
\|  u  \orlnor:=\inf \left\{ \lambda\bigg| \rho_{\Phi}\left(\frac{v}{\lambda}\right) dt\leq 1\right\},
\]
is a Banach space. By $u\b{\cdot} v$ we denote the usual dot product in $\mathbb{R}^{d}$ between $u$ and $v$.


The subspace $\ephi=\ephi([0,T],\rr^d)$ is defined as the closure in $\lphi$ of the subspace $L^{\infty}([0,T],\rr^d)$ of all $\mathbb{R}^d$-valued essentially bounded functions. It is shown that  (see \cite[Thm. 5.1]{Orliczvectorial2005}) $u\in\ephi$  if and only if $\rho_{\Phi}(\lambda u)<\infty$ for any $\lambda>0$. The equality $\lphi=\ephi$ is true if and only if $\Phi\in\Delta_2^{\infty}$ (see \cite[Thm. 5.2]{Orliczvectorial2005}). Another alternative characterization of $\ephi$, which is particularly useful for us, is that $u\in\ephi$ if and only if $u$ has  \emph{absolutely continuous norm}, i.e.   if $E_n\subset [0,T]$, $n=1,2,\ldots$ then $\|\chi_{E_n}u\|\to 0$ when $|E_n|\to 0$.

A generalized version of \emph{H\"older's inequality} holds in Orlicz spaces (see \cite[Thm. 4.1]{Skaff1969}). Namely, if $u\in\lphi$ and $v\in\lpsi$ then $u\cdot v\in L^1$ and
\begin{equation}\label{holder}
\int_0^Tv\cdot u\ dt\leq 2 \|u\orlnor\|v\|_{L^{\Phie}}.
\end{equation}


Like in \cite{KR} we will consider the subset $\Pi(\ephi,r)$ of $\lphi$ given by
\[\Pi(\ephi,r):=\{u\in\lphi| d(u,\ephi)<r\}.\]
This set is related to the Orlicz class $\claseor$ by means of inclusions, namely,
\begin{equation}\label{inclusiones}\Pi(\ephi, r )\subset r \claseor\subset\overline{\Pi(\ephi,r)}
\end{equation}
for any positive $r$ (see \cite[Thm. 5.6]{Orliczvectorial2005}).
If $\Phi \in \Delta_2^{\infty}$,  then the sets $\lphi$, $\ephi$, $\Pi(\ephi,r)$ and $\claseor$ are equal.

Following to \cite{Desch2001} we introduce the next definition.

\begin{defi} Let $u_n,u\in\lphi([0,T],\rr^d)$. We say that $u_n$ converges monotonically to $u$ if there exists $\alpha_n\in L^{\infty}([0,T],\rr)$, $n=1,2,\ldots$, such that $0\leq \alpha_n(t)\leq \alpha_{n+1}(t)$, $\alpha_n(t)\to 1$ a.e., when $n\to\infty$ and $u_n(t)=\alpha_n(t)u(t)$.

\end{defi}

 
As usual, if $(X,\|\cdot\|_X)$ is a normed space and $(Y,\|\cdot \|_Y)$ is a linear subspace of $X$,  we write $Y\hookrightarrow X$ and we say that $Y$ is \emph{embedded} in $X$  when there exists $C>0$ such that
$\|y\|_X\leq C\|y\|_Y$ for any $y\in Y$.  With this notation, H\"older's inequality states that  $\lpsi\hookrightarrow  \left[\lphi\right]^*$, where a function $v\in\lpsi$ is associated  to $\xi_v\in \left[\lphi\right]^*$ being
\begin{equation}\label{pairing}
  \xi_v(u)=\langle \xi_v,u\rangle=\int_0^Tv\cdot u\ dt,
\end{equation}
 In  \cite[Thm 2.9]{Desch2001}  it was characterized a subspace of   $\left[\lphi\right]^*$ which can be identified with $\lpsi$.

 \begin{prop} Let $F\in\left[\lphi([0,T],\rr^d)\right]^*$. Then the following statements are equivalent
 \begin{enumerate}
  \item $\xi\in \lpsi([0,T],\rr^d)$
  \item $\xi$ satisfies the \emph{monotone convergence property}, which is if $u_n$ converges monotonically to $u$ then $\langle \xi,u_n\rangle\to \langle \xi,u\rangle$.
 \end{enumerate}
 \end{prop}

 If $\Phi \in \Delta_2^{\infty}$ and $\Phi$ is coercive then $\lpsi([0,T],\rr^d)= \left[\lphi([0,T],\rr^d)\right]^*$  (see \cite[Thm. 2.9 , Thm. 2.10]{Desch2001}).

% It is easy to see that, for every coercive $\Phi$ we have that $L^{\infty}\hookrightarrow\lphi \hookrightarrow L^1$.




We define the \emph{Sobolev-Orlicz space} $\wphi$ by
\[\wphi([0,T],\rr^d):=\{u| u \hbox{ is absolutely continuous on $[0,T]$ and } u'\in \lphi([0,T],\rr^d)\}.\]
$\wphi([0,T],\rr^d)$ is a Banach space when equipped with the norm
\begin{equation}\label{def-norma-orlicz-sob}
\|  u  \|_{\wphi}= \|  u  \|_{\lphi} + \|u'\orlnor.
\end{equation}
And, we introduce the following subspaces of $\wphi$
%%
\begin{equation}\label{def-esp-orlicz-sob-per}
\begin{split}
\wphie&=\{u\in\wphi|u'\in\ephi\},\\
\wphie_T&=\{u\in\wphie|u(0)=u(T)\}.
\end{split}
\end{equation}

%
%
 We will use repeatedly the decomposition $u=\overline{u}+\widetilde{u}$ for a function $u\in L^1([0,T])$  where $\overline{u} =\frac1T\int_0^T u(t)\ dt$ and $\widetilde{u}=u-\overline{u}$.

 The following lemma is an elementary generalization to anisotropic Sobolev-Orlicz spaces of known results of Sobolev spaces.


%
%
%  Recall that a function   $w:\mathbb{R}^+\to \mathbb{R}^+$ is called  a \emph{modulus of continuity} if $w$ is a continuous increasing function which satisfies $w(0)=0$. For example, it can be easily shown that $w(s)=s\Phi^{-1}(1/s)$ is a modulus of  continuity for every $N$-function $\Phi$.  It is said that $u:[0,T]\to\rr^d$  has modulus of continuity $w$  when there exists a constant $C>0$ such that
% \begin{equation}\label{w-holder}|u(t)-u(s)|\leq Cw(|t-s|).
% \end{equation}
%
%
% We denote by $C^w([0,T],\rr^d)$  the space of  $w$-H\"older continuous functions that satisfy  \eqref{w-holder} for some $C>0$.
% This is a Banach space with norm
% \[\|u\|_{  C^w([0,T],\rr^d) }  :=\|u\|_{L^{\infty}}+\sup\limits_{t\neq s}\frac{|u(t)-u(s)|}{w(|t-s|)}.\]
%
%
%
%
% The following simple  embedding lemma, whose proof can be found in \cite{ABGMS2015}, will be used systematically.
%
%

\begin{lem}\label{inclusion orlicz} Let $\Phi:\rr^d\to [0,+\infty)$ be a Young's 
function and let $u\in\wphi([0,T],\rr^d)$. Let 
$A_{\Phi}: \rr^+ \to \rr^+$ be the function defined by \eqref{eq:inversa-gral}. Then
\begin{enumerate}
\item\label{inclusion orlicz_item1} For every $s,t\in [0,T]$, $s\neq t$,
\begin{align}
 &|u(t)-u(s)| \leq
 \|u'\orlnor |s-t|A_{\Phi}^{-1}\left(\frac{1}{|s-t|}\right)\tag{Morrey's inequality}\label{in-sob-cont}
\\
& \| u\linf \leq A_\Phi^{-1}\left(\frac{1}{T}\right)\max\{1,T\}\|u\sobnor\tag{Sobolev's inequality}\label{sobolev}
\end{align}
\item We have $\widetilde{u}\in L^{\infty}([0,T],\rr^d)$ and
\[
\|\widetilde u \linf \leq T A_{\Phi}^{-1}\left(\frac{1}{T}\right)\|u'\orlnor
\tag{Sobolev-Wirtinger's inequality}\label{wirtinger}
\]
\item\label{it:embeding} If $\Phi$ is coercive then the space $\wphi([0,T],\rr^d)$ is compactly embedded in the space of continuous functions $C([0,T],\rr^d)$.
\end{enumerate}
\end{lem}

\begin{proof} By the absolutely continuity of $u$, Jensen's inequality and the definition of 
the Luxemburg norm, we have

\[
 \begin{split}
    \Phi\left( \frac{u(t)-u(s)}{\|u'\orlnor |s-t|}\right) &\leq  \Phi\left( \frac{1}{ |s-t|}\int_s^t  \frac{u'(r)}{\|u'\orlnor }dr\right)\\
    &\leq   \frac{1}{ |s-t|}\int_s^t  \Phi\left(\frac{u'(r)}{\|u'\orlnor }\right)dr
    \leq \frac{1}{ |s-t|}.
 \end{split}
\]
By Proposition \ref{prop:AsubPhi}\eqref{it:prop3}  we have $A^{-1}_{\Phi}\Phi(x)\geq |x|$, therefore we get
\[
    \frac{|u(t)-u(s)|}{\|u'\orlnor |s-t|} 
    \leq  A_{\Phi}^{-1}\left(\frac{1}{ |s-t|}\right),
\]
then  \ref{inclusion orlicz_item1} holds.

Now, we use \ref{in-sob-cont} and  Proposition \ref{prop:AsubPhi} \eqref{it:prop2} and we have 
\[\begin{split}
\left|u(t)-\overline {u}\right|&=
\left|\frac{1}{T}\int_0^T u(t)-u(s)\,ds\right|
\\
&\leq \frac{1}{T} \int_0^T |u(t)-u(s)|\,ds
\\
&\leq \|u'\orlnor T A_{\Phi}^{-1}\left(\frac{1}{T}\right)
\end{split}
\] 
	CONTROLAR!!!!
In order to prove the Sobolev's inequality, we note that, using Jensen's inequality and 
the definition of $\|u\orlnor$, we obtain
\[ \Phi\left( \frac{ \overline{u}}{\|u\orlnor} \right) \leq
\frac{1}{T}\int_0^T\Phi\left(\frac{u(s)}{\|u\orlnor}\right)ds\leq\frac{1}{T}
\]
Then  by By Proposition \ref{prop:AsubPhi}\eqref{it:prop3} 
\[|\overline{u}|\leq A_{\Phi}^{-1}\left(\frac{1}{T}\right) \|u\orlnor.\]
Therefore, from this and \eqref{wirtinger} we get

\[\begin{split}
 \|u\linf &\leq |\overline{u}|+\|\tilde{u}\linf\\
 &\leq  
 A_{\Phi}^{-1}\left(\frac{1}{T}\right) \|u\orlnor+T ^{-1}\left(\frac{1}{T}\right)\|u'\orlnor\\
 &\leq A_{\Phi}^{-1}\left(\frac{1}{T}\right)\max\{1,T\}\|u\sobnor
 \end{split}
 \]
 



In order to prove item 3, we take a bounded sequence
$u_n$ in $\wphi([0,T],\rr^d)$. From \eqref{in-sob-cont} and Proposition \ref{prop:AsubPhi}\eqref{it:prop4}  we infer that $u_n$ are equicontinuous. Furthermore \eqref{sobolev} implies that $u_n$ is bounded in $C([0,T],\rr^d)$. Therefore by the Arzela-Ascoli Theorem we  obtain a subsequence $n_k$ and  $u\in C([0,T],\rr^d)$ with $u_{n_k}\to u$ in $C([0,T],\rr^d)$.

\end{proof}




\section{Superposition operators in anisotropic Orlicz spaces}
In this section we give a brief introduction to superposition operators between anistropic Orlicz Spaces.  We apply these results to obtain Gate\^aux differentiability of                                                                                action integrals associated to lagrangian functions defined in Sobolev-Orlicz spaces.

Henceforth we assume that $f$ is a \emph{Carath\'eodory function},

\begin{enumerate}
 \item[\namedlabel{eq:carathe}{(C)}] $f$ is measurable with respect to $t\in [0,T]$ for every  $x\in\rr^d$, and $f$ is a continuous function with  respect to  $x\in\rr^d$ for a.e. $t \in [0,T]$.
\end{enumerate}




\begin{defi}
 For $f:[0,T]\times \rr^d\to\rr$  we denote by $\b{f}$ the Nemytskii (o superposition) operator defined for functions $u:[0,T]\to\rr^d$ by
\[\b{f}u(t)=f(t,u(t))\]
\end{defi}

In the following Theorem we enumerate  some known properties for superposition operators definied in anisotropic Orlicz spaces of vectorial functions.   For the proofs of these results and additional  discussions see
\cite{zbMATH04038592,zbMATH03983966,zbMATH03942215}.

\begin{thm} We assume that $f$ satisfies condition \eqref{eq:carathe}. Then
\begin{enumerate}
 \item\label{it:measure}\emph{Measurability.}  The operator $\b{f}$ maps  masurable function into measurable functions
 \item\label{it:exten}\emph{Extensibility.?} If
 \item\label{it:exten}\emph{Continuity.?} If

\end{enumerate}




\end{thm}




\section*{Acknowledgments}
The authors are partially supported by a UNRC grant number 18/C417. The first author is  partially supported by a  UNSL grant number 22/F223. 




 \bibliographystyle{apalike}
 \bibliography{biblio}


\end{document}


