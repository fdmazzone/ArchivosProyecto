\documentclass[twoside]{article}


%\usepackage{hyperref}
\usepackage{amssymb,amsthm}
\usepackage{amsmath}
\usepackage{color}
\usepackage{ esint }
\usepackage{mathabx}
\usepackage{MnSymbol}
\usepackage{fancyhdr}
\usepackage{times}
\usepackage{enumitem}
\usepackage[latin1]{inputenc}
\usepackage{yfonts}

\usepackage{comment}
\usepackage{url}
\usepackage{xcolor}
\usepackage{adjustbox}
\usepackage{hyperref}



\newtheorem{thm}{Theorem}[section]
\newtheorem{cor}[thm]{Corollary}
\newtheorem{lem}[thm]{Lemma}

\newtheorem{defi}[thm]{Definition}
\newtheorem{prop}[thm]{Proposition}
\theoremstyle{remark}
\newtheorem{comentario}{Remark}


\makeatletter
\newcommand{\labitem}[2]{%
\def\@itemlabel{\textbf{#1}}
\item
\def\@currentlabel{#1}\label{#2}}
\makeatother
\makeatletter
\def\namedlabel#1#2{\begingroup
    #2%
    \def\@currentlabel{#2}%
    \phantomsection\label{#1}\endgroup
}
\makeatother



\title{Periodic solutions of
Euler-Lagrange equations in an anisotropic Orlicz-Sobolev space setting  }
\author{Sonia Acinas \thanks{SECyT-UNRC and  FCEyN-UNLPam}\\
Dpto. de Matem\'atica, Facultad de Ciencias Exactas y Naturales\\
Universidad Nacional de La Pampa\\
(L6300CLB) Santa Rosa, La Pampa, Argentina\\
\url{sonia.acinas@gmail.com}\\[3mm]
Fernando D. Mazzone \thanks{SECyT-UNRC, FCEyN-UNLPam and CONICET}\\
Dpto. de Matem\'atica, Facultad de Ciencias Exactas, F\'{\i}sico-Qu\'{\i}micas y Naturales\\
Universidad Nacional de R\'{i}o Cuarto\\
(5800) R\'{\i}o Cuarto, C\'ordoba, Argentina,\\
\url{fmazzone@exa.unrc.edu.ar}
}

\date{}

\newcommand{\orlnor}{\|_{L^{\Phi}}}
\newcommand{\lurnor}{\|^{*}_{L^{\Phi}}}
\newcommand{\linf}{\|_{L^{\infty}}}
\newcommand{\lphi}{L^{\Phi}}
\newcommand{\lphiuno}{L^{\Phi_1}}
\newcommand{\lphidos}{L^{\Phi_2}}
\newcommand{\lphii}{L^{\Phi_i}}
\newcommand{\lpsi}{L^{\Psi}}
\newcommand{\lpsiuno}{L^{\Phie_1}}
\newcommand{\lpsidos}{L^{\Phie_2}}
\newcommand{\lpsii}{L^{\Phie_i}}
\newcommand{\lmuno}{L^{M_1}}
\newcommand{\lmdos}{L^{M_2}}
\newcommand{\lmj}{L^{M}}
\newcommand{\lmn}{L^{M_n}}
\newcommand{\ephi}{E^{\Phi}}
\newcommand{\ephiuno}{E^{\Phi_1}}
\newcommand{\ephidos}{E^{\Phi_2}}
\newcommand{\ephin}{E^{\Phi_n}}
\newcommand{\ephii}{E^{\Phi_i}}
\newcommand{\claseor}{C^{\Phi}}
\newcommand{\wphi}{W^{1}\lphi}
\newcommand{\wphiuno}{W^{1}\lphiuno}
\newcommand{\wphidos}{W^{1}\lphidos}
\newcommand{\wphii}{W^{1}\lphii}
\newcommand{\wphiet}{W^{1}\ephi_T}
\newcommand{\wphie}{W^{1}\ephi}
\newcommand{\sobnor}{\|_{W^{1}\lphi}}
\newcommand{\domi}{\mathcal{E}^{\Phi}}
\newcommand{\domiuno}{\mathcal{E}^{\Phi_1}_d(\lambda)}
\newcommand{\domidos}{\mathcal{E}^{\Phi_2}_d(\lambda)}
\newcommand{\domii}{\mathcal{E}^{\Phi_i}_d(\lambda)}
\newcommand{\domin}{\mathcal{E}^{\Phi_n}_d(\lambda)}
\renewcommand{\b}[1]{\boldsymbol{#1}}
\newcommand{\rr}{\mathbb{R}}
\newcommand{\nn}{\mathbb{N}}
\newcommand{\ccdot}{\b{\cdot}}
\renewcommand{\leq}{\leqslant} 
\renewcommand{\geq}{\geqslant} 
\newcommand{\epsi}{E^{\Psi}}
\newcommand{\Phie}{\Phi^{*}}


\DeclareSymbolFont{symbolsC}{U}{txsyc}{m}{n}
\DeclareMathSymbol{\strictif}{\mathrel}{symbolsC}{74}


\newcounter{example}[section]

\setcounter{example}{0}

\makeatletter
\renewcommand{\p@example}{\thesection.} % "prefix" for cross-referencing
\makeatother
%\newenvironment{example}{\noindent\textit{Example \arabic{example}}.}{\addtocounter{example}{1}}
\newenvironment{example}{\refstepcounter{example}\noindent\textit{Example \arabic{section}.\arabic{example}}.}{ }



\begin{document}


\maketitle
%
\begingroup%Locallizing the change to `thefootnote'.
    \renewcommand{\thefootnote}{}%Removing the footnote symbol.
    %
    \footnotetext{%
    %   2010 Mathematics Subject Classification
    %   http://www.ams.org/msc/
    \textbf{2010  AMS Subject Classification.} Primary: .
    Secondary: .
    }%
        \footnotetext{%
    \textbf{Keywords and phrases.}  .
    }%
    \endgroup
%
%
%
%

\begin{abstract}


\end{abstract}






\pagestyle{fancy} \headheight 35pt \fancyhead{} \fancyfoot{}

\fancyfoot[C]{\thepage} \fancyhead[CE]{\nouppercase{S. Acinas and F.D. Mazzone }} \fancyhead[CO]{\nouppercase{\section}}

\fancyhead[CO]{\nouppercase{\leftmark}}


%\tableofcontents




\section{Introduction}


In this paper we obtain existence of solutions for systems  of equations of the type:

\begin{equation}\label{ProbPrin-gral}
    \left\{%
\begin{array}{ll}
  \frac{d}{dt} D_{y}\mathcal{L}(t,u(t),u'(t))= D_{x}\mathcal{L}(t,u(t),u'(t)) \quad \hbox{a.e.}\ t \in (0,T),\\
    u(0)-u(T)=u'(0)-u'(T)=0,
\end{array}%
\right. \tag{$\b{P}$}
\end{equation}
where the function $\mathcal{L}:[0,T]\times\rr^d\times\rr^d\to\rr$, $d\geq 1$ (called the \emph{Lagrange function} or \emph{lagrangian}) satisfying  that  it is measurable in $t$ for each $(x,y)\in \rr^d\times\rr^d$ and  continuously differentiable in $(x,y)$ for almost every $t \in [0,T]$. The unknown function  $u:[0,T]\to\rr^d$ is assumed absolutely continuous. 

Our approach involves the direct method of the calculus of variations in the framework of \emph{anisotropic Orlicz-Sobolev spaces}.  We suggest the articles  \cite{Orliczvectorial2005} for definitions and main results on anisotropic Orlicz spaces, see also \cite{chamra2017anisotropic}. These spaces allow us to unify and extend previous results on existences of solutions for systems like \eqref{ProbPrin-gral}. 

Through this article we say that  a function
$\Phi:\mathbb{R}^d\to [0,+\infty)$ is of $N_{\infty}$ class if $\Phi$ is convex, $\Phi(0)=0$, $\Phi(y)>0$ if $y\neq 0$ and $\Phi(-y)=\Phi(y)$,
 and 
\begin{equation}\label{eq:N-sub-inf}
\lim_{|y|\to\infty}\frac{\Phi(y)}{|y|}=+\infty.
\end{equation}
where $|\cdot|$ denotes the euclidean norm on $\rr^d$. From \cite[Cor. 2.35]{clarke2013functional} a $N_{\infty}$ function is  continuous.  


Associated to $\Phi$ we have the \emph{complementary function} $\Psi$ which is defined in $\xi\in\rr^d$ as
\begin{equation}\label{eq:conjugada}
 \Psi(\xi)=\sup\limits_{y\in\mathbb{R}^d} y\cdot \xi-\Phi(y)
\end{equation}
then, from the continuity of $\Phi$ and  \eqref{eq:N-sub-inf}, we have that $\Psi:\rr^d \to [0,\infty)$. 
Moreover, it is easy to see that $\Psi$ is a convex function such that $\Psi(0)=0$, $\Psi(-\xi)=\Psi(\xi)$    \cite[Chapter 2]{mawhin2010critical}. Moreover $\Psi$ satisfies \eqref{eq:N-sub-inf} (see \cite[Th. 2.2]{Orliczvectorial2005}). i.e. $\Psi$ is $N_{\infty}$ function.

Some examples of  $N_{\infty}$ functions are the following.

  \begin{example}\label{ex:p-laplac}
   $\Phi_p(y):=|y|^p/p$, for $1<p<\infty$.  In this case $\Psi(\xi)=|\xi|^q/q$, $q=p/(p-1)$.
  \end{example}

   \begin{example}
     If $\Phi:\rr\to [0,+\infty)$ is a $N_{\infty}$  function on $\rr$ then $\overline{\Phi}(y)=\Phi(|y|)$ is a $N_{\infty}$  function on $\rr^d$. In this example, as in the previous one, the function $\Phi$ is \emph{radial}, i.e. the value of $\Phi(y)$ depends on the norm of $y$  and not on its direction. These cases are not authentically anisotropic.
   \end{example}


   \begin{example}\label{ex:pq-laplac}
    An anisotropic function $\Phi(y)$ depends on the direction of $y$. For example, if  $1<p_1,p_2<\infty$, we define $\Phi_{p_1,p_2}:\rr^d\times \rr^d\to [0,+\infty)$   by
\[\Phi_{p_1,p_2}(y_1,y_2):=\frac{|y_1|^{p_1}}{p_1}+\frac{|y_2|^{p_2}}{p_2}.\]
Then $\Phi_{p_1,p_2}$ is a $N_{\infty}$ function.  In this case the complementary function is the function $\Phi_{q_1,q_2}$ with $q_i=p_i/(p_i-1)$. 

More generally, if $\Phi_k:\rr^d\to [0,+\infty)$, $k=1,\ldots,n$, are $N_{\infty}$ functions, then $\Phi:\rr^d\times\cdots\times\rr^d\to [0,+\infty)$ defined by $\Phi(y_1,\ldots,y_n)=\Phi_1(y_1)+\cdots+\Phi_n(y_n)$  is a $N_{\infty}$ function. These functions are truly anisotropic, i.e. $|x|=|y|$ does not imply that $\Phi(x)=\Phi(y)$.
   \end{example}


 
    \begin{example}
          If $\Phi:\rr\to [0,+\infty)$ is a $N_{\infty}$ function and  $O\in GL(d,\rr)$, then $\Phi(y)=\Phi(Oy)$  is a $N_{\infty}$ function.
    \end{example}

 

    \begin{example}
      An anisotropic $N_{\infty}$  function is not necessarily controlled by powers if it does not satisfy the $\Delta_2$ condition (see xxxxx). For example $\Phi:\rr^d:\to[0,+\infty)$ defined by $\Phi(y)=\exp(|y|)-1$ is $N_{\infty}$ function.
    \end{example}




 


 
 
The occurrence of Orlicz Spaces in this paper obeys to we will consider the 
following structure condition on the lagrangian:
\begin{equation}\label{eq:condicion-estructura}
  |\mathcal{L}|+ |\nabla_{x}\mathcal{L}|+\Psi\left(\frac{\nabla_{y}\mathcal{L}}{\lambda}\right)
\leq
a(x)\left\{b(t)+ \Phi\left(\frac{y}{\Lambda}\right)\right\}, \tag{$\b{S}$}
\end{equation}
for a.e. $t\in [0,T]$,
where  $a\in C\left(\mathbb{R}^d,[0,+\infty)\right)$, $b\in L^1\left([0,T],[0,+\infty)\right) $ and $\Lambda,\lambda>0$.

Our condition \eqref{eq:condicion-estructura} includes structure conditions that have previously been considered in the literature. For example, it is easy to see that, when $\Phi(x)$ is as in Example \ref{ex:p-laplac}, then  the condition \eqref{eq:condicion-estructura}  is equivalent to the structure condition in  \cite[Th. 1.4]{mawhin2010critical}.  If $\Phi$ is a radial $N_{\infty}$ function such that $\Psi$ satisfies that $\Delta_2$ function  then \eqref{eq:condicion-estructura} is essentially equivalent????? to conditions  \cite[Eq. (2)-(4)]{ABGMS2015} (see xxxx mas abajo).   If $\Phi$ is as in Example \ref{ex:pq-laplac} and $\mathcal{L}=\mathcal{L}(t,x_1,x_2,y_1,y_2)$ is a lagrangian with $\mathcal{L}:[0,T]\times\rr^d\times\rr^d\times\rr^d\times\rr^d\to\rr$ then inequality \eqref{eq:condicion-estructura} is related to estructure conditions like
\cite[Lemma 3.1, Eq. (3.1)]{Tian2007192}. As can be seen, condition \eqref{eq:condicion-estructura} is a more compact expression than \cite[Lemma 3.1, Eq. (3.1)]{Tian2007192} and moreover   weaker, because  \eqref{eq:condicion-estructura} does not imply a control of
$|D_{y_1}L|$ independent of $y_2$.  We will return to this point later.


An important example of lagrangian  is giving by:

\begin{equation}\label{eq:lagrange_phi}
\mathcal{L}_{\Phi,F}(t,x,y):=\Phi(y)+F(t,x).
\end{equation}
Here the function $F(t,x)$, which is often referred to potential,  be differentiable with respect to $x$ for a.e. $t\in [0,T]$. Moreover $F$ satisfies the following conditions:
\begin{enumerate}
\labitem{(C)}{item:condicion_c} $F$ and its gradient $\nabla_x F$, with respect to $x\in\rr^d$,  are  Carath\'eodory functions, i.e. they are measurable functions with respect to $t\in [0,T]$, for every  $x\in\rr^d$, and they are continuous functions with  respect to  $x\in\rr^d$ for a.e. $t \in [0,T]$.
 \labitem{(A)}{item:condicion_a}  For   a.e. $t\in [0,T]$, it holds that
\begin{equation}
|F(t,x)| + |\nabla_x F(t,x)|  \leq a(x)b(t).
\end{equation}
where  $a\in C\left(\rr^d,[0,+\infty)\right)$ and $0\leq b\in L^1([0,T],\rr)$.
\end{enumerate}

The lagrangian $\mathcal{L}_{\Phi,F}$ satisfies condition  \eqref{eq:condicion-estructura}. In order to prove this, the only non trivial fact that we should to establishis is that $ \Psi(\nabla_{y}\mathcal{L})
\leq
a(x)\left\{b(t)+ \Phi\left({y}/{\lambda}\right)\right\}$. But, from inequality xxxx below, 
$\Psi(\nabla_{y}\mathcal{L})=\Psi\left(\nabla\Phi(y)\right)\leq \Phi(2y)$.


The laplacian $\mathcal{L}_{\Phi,F}$ leads to the system

\begin{equation}\label{eq:ProbPhiLapla}
    \left\{%
\begin{array}{ll}
  \frac{d}{dt} \nabla \Phi(u'(t))= \nabla_{x}F(t,u(t)) \quad \hbox{a.e.}\ t \in (0,T),\\
    u(0)-u(T)=u'(0)-u'(T)=0,
\end{array}%
\right. \tag{$\b{P_{\Phi}}$}
\end{equation}

Problem \eqref{eq:ProbPhiLapla} contains, as a particular case, many problems that are usually considered in the literature.  For example, the classic book  \cite{mawhin2010critical} deals mainly with problem \eqref{ProbPrin-gral}, for the lagrangian $\mathcal{L}_{\Phi,F}$, with $\Phi(x)=|x|^2/2$, through various methods: direct, dual action, minimax, etc. The results in \cite{mawhin2010critical} were extended and improved in several articles,  see  \cite{tang1995periodic,tang1998periodic,wu1999periodic,tang2001periodic,zhao2004periodic}  to cite some examples. The case $\Phi(y)=|y|^p/p$, for arbitrary $1<p<\infty$ were considered in  \cite{Tian2007192,tang2010periodic}, among other papers, and in this case \eqref{eq:ProbPhiLapla} is reduced to the $p$-laplacian system
\begin{equation}\label{ProbP-lapla}
    \left\{%
\begin{array}{ll}
   \frac{d}{dt}\left(u'(t)|u'|^{p-2}\right) = \nabla F(t,u(t)) \quad \hbox{a.e.}\ t \in (0,T)\\
    u(0)-u(T)=u'(0)-u'(T)=0.
\end{array}%
\right.\tag{$\b{P_p}$}
\end{equation}


If $\Phi$ is as in Example \ref{ex:pq-laplac} and  $F:[0,T]\times\rr^d\times\rr^d\to\rr$ is a Carath\'eodory function, then the equations \eqref{eq:ProbPhiLapla} become
\begin{equation}\label{eq:sist-p_lapa}
    \left\{%
\begin{array}{ll}
  \frac{d}{dt}\left(|u_1'|^{p_1-2}u_1'\right)=F_{x_1}(t,u) \quad \hbox{a.e.}\ t \in (0,T)\\
  \frac{d}{dt}\left(|u_2'|^{p_2-2}u_2'\right)=F_{x_2}(t,u) \quad \hbox{a.e.}\ t \in (0,T)\\
   u(0)-u(T)=u'(0)-u'(T)=0,
\end{array}%
\right., \tag{$\b{P_{p_1,p_2}}$}
\end{equation}
where $x=(x_1,x_2)\in\rr^d\times\rr^d$ and $u(t)=(u_1(t),u_2(t))\in\rr^d\times\rr^d$. In the literature, these equations are known as $(p_1,p_2)$-Laplacian system, see
\cite{yang2013existence,pasca2016periodic,yang2012periodic,pasca2010periodic,pacsca2010some,pasca2011some,li2014periodic}.

In conclusion, the problem \eqref{ProbPrin-gral} with conditions \eqref{eq:condicion-estructura}  contains several problems that have been considered by many authors in the past. 
Moreover, our results still improve some results on $(p_1,p_2)$-lamplacian since our structure conditions are less restrictive even in that particular case. 

\section{Anisotropic Orlicz and Orlicz-Sobolev spaces}\label{preliminares}

In this section, we give a short introduction to  Orlicz and Orlicz-Sobolev spaces of vector valued functions associated to anisotropic $N_{\infty}$ functions $\Phi:\rr^n\to[0,+\infty)$.  References for  these topics are \cite{Desch2001,Orliczvectorial2005,Skaff1969,cianchi2000fully,cianchi2004optimal,chamra2017anisotropic,trudinger1974imbedding,gwiazda2013anisotropic}.
Note that, unlike in \cite{gwiazda2013anisotropic}, we do not require that $N_{\infty}$ functions be superlinear near from 0, i.e. $\Phi(x)/|x|\to 0$ when $|x|\to 0$. However, most of the results proved in \cite{gwiazda2013anisotropic} do not depend on this property.

If $\Phi$  is a $N_{\infty}$ function then from convexity and $\Phi(0)=0$ we obtain that
\begin{equation}\label{eq:escalar_ine}
 \Phi(\lambda x)\leq \lambda\Phi(x),\quad \lambda\in[0,1],x\in\rr^d.
\end{equation}

One of the greatest difficulties when dealing with anisotropic Orlicz spaces is the lack of  monotony  with respect to the Euclidean norm, i.e. $|x|\leq |y|$ does not imply $\Phi(x)\leq\Phi(y)$. This problem is avoided if we consider functions whose values on a sphere are comparable (see\cite{Skaff1969}). However, from \eqref{eq:escalar_ine}, we see that  $N_{\infty}$ functions have the following form of radial monotony: $|x|\leq |y|$ and $y=\lambda x$ imply $\Phi(x)\leq\Phi(y)$. 

We say that  $\Phi:\mathbb{R}^d\rightarrow [0,+\infty)$ satisfies the  \emph{$\Delta_2^{\infty}$-condition}, denoted by $\Phi \in \Delta_2^{\infty}$,
if there exist  constants $K>0$ and  $M\geq 0$ such that
\begin{equation}\label{delta2defi}\Phi(2x)\leq K \Phi(x),
\end{equation}
for every $|x|\geq M$. If $\Phi$ es a $\Delta_2$ function then $\Phi$ is bounded by powers functions (see \cite[Proof Lemma 2.4]{Desch2001} and \cite[Prop. 1]{cianchi2000local}), i.e. there exists $1<p<\infty$, $C>0$ and $r\geq 0$ such that
\[\Phi(x)\leq C|x|^p,\quad |x|\geq r_0.\] 

We consider that one of the most important aspects in considering $N_{\infty}$ functions is that it accounts for the Lagrange functions that present faster growth than powers, for example an exponential growth. Hence we consider it important to avoid imposing hypothesis that $\Phi$ to be $\Delta_2$. For some results we will need that $\Psi$ to be $\Delta_2$. 


Let $\Phi_1$ and $\Phi_2$ be   $N_{\infty}$ functions. Following to \cite{trudinger1974imbedding} we write $\Phi_1\strictif\Phi_2$ if there exists non negative numbers $k$ and $C$ such that
\begin{equation}\label{eq:orden} \Phi_1(x)\leq C+\Phi_2(kx),\quad x\in\rr^d.\end{equation}
For example if $\Phi$ is $\Delta_2$ then there exist $p\in (1,+\infty)$ such that $\Phi\strictif |x|^p$.  If for every $k>0$ there exists $C=C(k)>0$ such that \eqref{eq:orden} holds we write  $\Phi_1\llcurly\Phi_2$. 

If $\Phi_1\strictif \Phi_2$ then $\Psi_2\strictif\Psi_1$ as the following simple computation proves 
\[
\begin{split}
  \Psi_1(k\xi)&\geq \sup \left\{k\xi\cdot x-\Phi_2(kx)-C\right\}\\
&=\sup \left\{\xi\cdot x-\Phi_2(x)\right\}-C\\
&=\Psi_2\left(\xi\right)-C.
\end{split}
\]


As a consequence of the previous result, we obtain that if a Lagrange function $\mathcal{L}$ satisfies structure condition \eqref{eq:condicion-estructura} and $\Phi\strictif \Phi_0$ then $\mathcal{L}$ satisfies \eqref{eq:condicion-estructura} with $\Phi_0$ instead to $\Phi$ with other functions $b$, $a$ and constant $\Lambda$ and $\lambda$.  





 We denote by $\mathcal{M}:=\mathcal{M}\left([0,T],\rr^d\right)$, with $d\geq 1$,  the set of all measurable functions (i.e. functions which are limits of simple functions)  defined on $[0,T]$ with values on $\mathbb{R}^d$ and  we write $u=(u_1,\dots,u_d)$ for  $u\in \mathcal{M}$.

 Given  an $N_{\infty}$ function $\Phi$ we define the \emph{modular function} 
$\rho_{\Phi}:\mathcal{M}\to \mathbb{R}^+\cup\{+\infty\}$ by
\[\rho_{\Phi}(u):= \int_0^T \Phi(u)\ dt.\]

Now, we introduce the \emph{Orlicz class} $C^{\Phi}=C^{\Phi}\left([0,T],\rr^d\right)$   by setting
\begin{equation}\label{claseOrlicz}
  C^{\Phi}:=\left\{u\in \mathcal{M} | \rho_{\Phi}(u)< \infty \right\}.
\end{equation}
The \emph{Orlicz space} $\lphi=L^{\Phi}\left([0,T],\rr^d\right)$ is the linear hull of $\claseor$;
equivalently,
\begin{equation}\label{espacioOrlicz}
\lphi:=\left\{ u\in \mathcal{M}| \exists \lambda>0: \rho_{\Phi}(\lambda u) < \infty   \right\}.
\end{equation}
  The Orlicz space $\lphi$ equipped with the \emph{Luxemburg norm}
\[
\|  u  \orlnor:=\inf \left\{ \lambda\bigg| \rho_{\Phi}\left(\frac{v}{\lambda}\right) dt\leq 1\right\},
\]
is a Banach space. 


The subspace $\ephi=\ephi\left([0,T],\rr^d\right)$ is defined as the closure in $\lphi$ of the subspace $L^{\infty}\left([0,T],\rr^d\right)$ of all $\mathbb{R}^d$-valued essentially bounded functions. The equality $\lphi=\ephi$ is true if and only if $\Phi\in\Delta_2^{\infty}$ (see \cite[Cor. 5.1]{Orliczvectorial2005}). 

A generalized version of \emph{H\"older's inequality} holds in Orlicz spaces (see \cite[Thm. 7.2]{Orliczvectorial2005}). Namely, if $u\in\lphi$ and $v\in\lpsi$ then $u\cdot v\in L^1$ and
\begin{equation}\label{holder}
\int_0^Tv\cdot u\ dt\leq 2 \|u\orlnor\|v\|_{L^{\Psi}}.
\end{equation}
By $u\cdot v$ we denote the usual dot product in $\mathbb{R}^{d}$ between $u$ and $v$.

We consider the subset $\Pi(\ephi,r)$ of $\lphi$ given by
\[\Pi(\ephi,r):=\{u\in\lphi| d(u,\ephi)<r\}.\]
This set is related to the Orlicz class $\claseor$ by the following inclusions
\begin{equation}\label{eq:inclusiones}\Pi(\ephi, r )\subset r \claseor\subset\overline{\Pi(\ephi,r)}
\end{equation}
for any positive $r$. This relation is a trivial generalization of  \cite[Thm. 5.6]{Orliczvectorial2005}.
If $\Phi \in \Delta_2^{\infty}$,  then the sets $\lphi$, $\ephi$, $\Pi(\ephi,r)$ and $\claseor$ are equal.
 
As usual, if $(X,\|\cdot\|_X)$ is a normed space and $(Y,\|\cdot \|_Y)$ is a linear subspace of $X$,  we write $Y\hookrightarrow X$ and we say that $Y$ is \emph{embedded} in $X$  when there exists $C>0$ such that
$\|y\|_X\leq C\|y\|_Y$ for any $y\in Y$.  With this notation, H\"older's inequality states that  $\lphi\hookrightarrow  \left[\lpsi\right]^*$, where a function $v\in\lphi$ is associated  to $\xi_v\in \left[\lpsi\right]^*$ being
\begin{equation}\label{pairing}
  \langle \xi_v,u\rangle=\int_0^Tv\cdot u\ dt,
\end{equation}

We highlight the following result (see \cite[Th. 3.3]{gwiazda2013anisotropic}).

\begin{prop} $\lphi\left([0,T],\rr^d\right)=\left[\epsi\left([0,T],\rr^d\right)\right]^*$.
 
\end{prop}

Consequently  $\lphi\left([0,T],\rr^d\right)$ can be equipped with the weak* topology induced by $\epsi\left([0,T],\rr^d\right)$.






We define the \emph{Sobolev-Orlicz space} $\wphi\left([0,T],\rr^d\right)$ by
\[\wphi\left([0,T],\rr^d\right):=\left\{u| u\in AC\left([0,T],\rr^d\right) \hbox{ and } u'\in \lphi\left([0,T],\rr^d\right)\right\},\]
where $AC\left([0,T],\rr^d\right)$ denotes the space of all $\rr^d$ valued absolutely continuous functions defined on $[0,T]$. The space $\wphi\left([0,T],\rr^d\right)$ is a Banach space when equipped with the norm
\begin{equation}\label{def-norma-orlicz-sob}
\|  u  \|_{\wphi}= \|  u  \|_{\lphi} + \|u'\orlnor.
\end{equation}

Anisotropic Sobolev-Orlicz spaces were treated in \cite{cianchi2000fully,cianchi2004optimal,chamra2017anisotropic,trudinger1974imbedding}. Usually functions in Sobolev spaces are required to be weakly differentiable. In the particular and simplest case of functions of one variable, the weak differentiability implies absolute continuity. Hence we can assume $u\in AC\left([0,T],\rr^d\right)$ for functions $u\in\wphi\left([0,T],\rr^d\right)$.
% 
% 
% We introduce the following subspaces of $\wphi$
% %%
% \begin{equation}\label{def-esp-orlicz-sob-per}
%   \wphie =\{u\in\wphi|u'\in\ephi\},\quad \wphie_T=\{u\in\wphie|u(0)=u(T)\}.
% \end{equation}


As is well known, an active research topic in mathematical analysis are the Sobolev and Poincare inequalities. This topic have also been treated in the framework of Anisotropic Orlicz-Sobolev mainly in \cite{cianchi2000fully,cianchi2004optimal,trudinger1974imbedding} for several variables functions and in \cite{chamra2017anisotropic} for functions of one single variable, $\Phi$ and $\Psi$ functions of $\Delta_2^{\infty}$ class.   We do not know a reference for the embedding of Sobolev-Orlicz anisotropic spaces in the space of continuous functions when $\Phi$ or $\Psi$ are not $\Delta_2^{\infty}$. Below we present the results that we will require in this article and we show in detail the case of the incrustation in the space of continuous functions in the simple case of function of one variable.

In order to find a modulus of continuity for functios in $\wphi$, and from there, to obtain compact embedding of $\wphi$, we define the function $A_{\Phi}:\rr^+\to\rr^+$ by
\begin{equation}\label{eq:inversa-gral}
A_{\Phi}(s)=\min\left\{\Phi(x)\,\big|\,|x|=s\right\},
\end{equation}

Let us establish some elementary properties of $A_{\Phi}$.
\begin{prop}\label{prop:AsubPhi} The function $A_{\Phi}$ has the following properties:
\begin{enumerate}
 \item\label{it:prop1} $A_{\Phi}$ is continuous,
 \item\label{it:prop2} $A_{\Phi}(s)/s$ is increasing,
 \item\label{it:prop3} $A_{\Phi}(|x|)$ is the \emph{greatest radial minorant} of 
 $\Phi(x)$,
 \item\label{it:prop4} $\Phi$ is $N_{\infty}$ if and only if $\lim_{s\to+\infty} A_{\Phi}(s)/s=+\infty$.
\end{enumerate}
\end{prop}

\begin{proof} It is well known that finite and convex functions defined on finite dimensional 
vector spaces are locally Lipschitz functions (see \cite{clarke2013functional}). This fact 
implies item \ref{it:prop1} immediately. 

In order to prove item \ref{it:prop2}, suppose $0<r<s$ and $x\in\rr^d$ with $A_{\Phi}(s)
=\Phi(x)$. Then, from the definition of $A_{\Phi}$ and the convexity of $\Phi$,
\[\frac{A_{\Phi}(r)}{r}\leq \frac{\Phi\left(\frac{r}{s}x\right)}{r}\leq \frac{\Phi\left(x\right)}{s}=
 \frac{A_{\Phi}(s)}{s}.
\]
Property in items \ref{it:prop3} and \ref{it:prop4} are obtained easily.

 
\end{proof}

\begin{example} Let $\Phi=\Phi_{p_1,p_2}$ be the function given in Example \eqref{ex:pq-laplac}.  We show that 
 \[
 K\min\left\{\frac{r^{p_1}}{p_1}, \frac{r^{p_2}}{p_2}\right\}\leq A_{\Phi}(r)\leq \min\left\{\frac{r^{p_1}}{p_1}, \frac{r^{p_2}}{p_2}\right\}
\]
for some $K>0$, for every $1<p_1,p_2<\infty$. The second inequality follows directly from definition of $A_{\Phi}$. For the first inequality, we note that $|(y_1,y_2)|=r$ implies that $|y_1|\geq r/2$ or $|y_2|\geq r/2$. Then 
\begin{equation}\label{eq:aphi}\Phi_{p_1,p_2}(y_1,y_2)\geq \min\{2^{-p_1},2^{-p_2}\}\min\left\{\frac{r^{p_1}}{p_1}, \frac{r^{p_2}}{p_2}\right\}.\end{equation}
 
 
  

  Let us in a little digression to show that
 \[A_{\Phi}(r)= \min\left\{\frac{r^{p_1}}{p_1}, \frac{r^{p_2}}{p_2}\right\},\]
 when $1<p_1,p_2\leq 2$.   We apply the method of Lagrange multipliers (see \cite[Ch. 11]{luenberger2015linear}) to solve the next minimization problem subject to constraints
\[
\left\{ \begin{array}{ll}
  \hbox{minimize } \Phi_{p_1,p_2}(y_1,y_2)\\
	 \hbox{subject to }|y_1|^2+|y_2|^2=r^2\\
        \end{array}
\right.
.\]

The first order conditions  are
\begin{equation}\label{eq:mult_lagr}
\left
\{
\begin{array}{ccc}
|y_1|^{p_1-2}y_1+ \lambda y_1&=&0
\\
|y_2|^{p_2-2}y_2+\lambda y_2&=&0
\\
|y_1|^2+|y_2|^2&=&r^2
\end{array}
\right.
\end{equation}
These equations are solved, among others, by the following two sets  of citical points: 
a) $|x|=r$, $y=0$ and $\lambda=-r^{p_1-2}$ and b) $x=0$, $|y|=r$ and $\lambda=-r^{p_2-2}$. These sets are infinite when $d>1$. 
Associated with these critical points we have the following critical values: a) $r^{p_1}/p_1$ and b) $r^{p_2}/p_2$.



If $(y_1,y_2)$ solve \eqref{eq:mult_lagr} with $y_1\neq 0$ and $y_2 \neq 0$ then  $|y_2|=|y_1|^{\frac{p_1-2}{p_2-2}}$ and $\lambda=-|y_1|^{p_1-2}$.  We use second order conditions for constrained problems.
We have to consider  the tangent plane at the point $(y_1,y_2)\in \rr^{2n}$, i.e. 
$M=\{(\xi,\eta) \in \rr^{2n}: \xi y_1^t+\eta y_2^T=0\}$. Let $L$ be the Lagrangian associated to the constrained problem: $L(y_1,y_2,\lambda)=\Phi(y_1,y_2)+\lambda H(y_1,y_2)$ being $H=0$ the constraint. We must analize the positivity of the quadratic form associated to the matrix  of second partial derivatives $\mathcal{H}=D^2 \Phi+ \lambda D^2 H$ 
on the subspace $M$. By elementary computations we have for $(\xi,\eta)\in M$
\[(\xi,\eta)^t\mathcal{H}(\xi,\eta)=
|\lambda| (\xi^tx)^2 [|y_1|^{-2}(p_1-2)+(p_2-2)|y_2|^{-2}],\]
on the subspace $M$. We can assume that $p_1<2$ or $p_2<2$, otherwise the statement we intend to prove would be trivial. Under this assumption, we note that $(-y_2,y_1)\in M$ and $(-y_2,y_1)^t\mathcal{H}(-y_2,y_1)<0$.
Then, by second order necessary conditions \cite[p.333]{luenberger2015linear}, 
 there cannot be a minimum at $(y_1,y_2)$. Therefore follows \eqref{eq:aphi}.

  
\end{example}




 Recall that a function   $w:[0,+\infty)\to [0,+\infty)$ is called  a \emph{modulus of continuity} if $w$ is a continuous increasing function which satisfies $w(0)=0$. For example $w(s)=sA_{\Phi}^{-1}(1/s)$ is a modulus of  continuity for every $N$-function $\Phi$.  We say that $u:[0,T]\to\rr^d$  has modulus of continuity $w$  when there exists a constant $C>0$ such that
\begin{equation}\label{w-holder}|u(t)-u(s)|\leq Cw(|t-s|).
\end{equation}


We denote by $C^w([0,T],\rr^d)$  the space of  $w$-H\"older continuous functions. This is the space of all functions satisfying \eqref{w-holder} for some $C>0$ and it is a Banach space with norm
\[\|u\|_{  C^w([0,T],\rr^d) }  :=\|u\|_{L^{\infty}}+\sup\limits_{t\neq s}\frac{|u(t)-u(s)|}{w(|t-s|)}.\]





 As is customary, we will use the decomposition $u=\overline{u}+\widetilde{u}$ for a function $u\in L^1([0,T])$  where $\overline{u} =\frac1T\int_0^T u(t)\ dt$ and $\widetilde{u}=u-\overline{u}$.



\begin{lem}\label{lem:inclusion orlicz} Let $\Phi:\rr^d\to [0,+\infty)$ be a Young's 
function and let $u\in\wphi\left([0,T],\rr^d\right)$. Let 
$A_{\Phi}: \rr^+ \to \rr^+$ be the function defined by \eqref{eq:inversa-gral}. Then
\begin{enumerate}
\item\label{inclusion orlicz_item1} For every $s,t\in [0,T]$, $s\neq t$,
\begin{align}
 &|u(t)-u(s)| \leq
 \|u'\orlnor |s-t|A_{\Phi}^{-1}\left(\frac{1}{|s-t|}\right)\tag{Morrey's inequality}\label{in-sob-cont}
\\
& \| u\linf \leq A_\Phi^{-1}\left(\frac{1}{T}\right)\max\{1,T\}\|u\sobnor\tag{Sobolev's inequality}\label{eq:sobolev}
\end{align}
\item We have $\widetilde{u}\in L^{\infty}\left([0,T],\rr^d\right)$ and 

\begin{equation}\label{eq:wirtinger}
  \Phi\left(\tilde{u}(t)\right)\leq\frac{1}{T} \int_0^T \Phi\left(Tu'(r)\right)dr.\tag{Poincar\'e-Wirtinger's inequality}
\end{equation}

\item\label{it:embeding} If $\Phi$ is $N_{\infty}$ then the space $\wphi\left([0,T],\rr^d\right)$ is compactly embedded in the space of  continuous functions $C([0,T],\rr^d)$.
\end{enumerate}
\end{lem}

\begin{proof}  By the absolutely continuity of $u$, Jensen's inequality,  the definition of 
the Luxemburg norm and following a similar argument that in the deduction of \cite[Th. 4.5]{chamra2017anisotropic}, we have

\[
 \begin{split}
    \Phi\left( \frac{u(t)-u(s)}{\|u'\orlnor |s-t|}\right) &\leq  \Phi\left( \frac{1}{ |s-t|}\int_s^t  \frac{u'(r)}{\|u'\orlnor }dr\right)\\
    &\leq   \frac{1}{ |s-t|}\int_s^t  \Phi\left(\frac{u'(r)}{\|u'\orlnor }\right)dr
    \leq \frac{1}{ |s-t|}.
 \end{split}
\]
By Proposition \ref{prop:AsubPhi}\eqref{it:prop3}  we have $A^{-1}_{\Phi}\Phi(x)\geq |x|$, therefore we get
\[
    \frac{|u(t)-u(s)|}{\|u'\orlnor |s-t|} 
    \leq  A_{\Phi}^{-1}\left(\frac{1}{ |s-t|}\right),
\]
then  item \ref{inclusion orlicz_item1} holds.

Applying Jensen's inequality two times, we get
\begin{equation*}
\begin{split}
\Phi(\tilde{u}(t))&=\Phi\left(\frac{1}{T}\int_0^T \left(u(t)-u(s)\right) ds\right)\\
&\leq\frac{1}{T}\int_0^T \Phi(u(t)-u(s))ds\\
&\leq \frac{1}{T}\int_0^T \Phi\left(\int_s^t |t-s| u'(r)\frac{dr}{|t-s|}\right)ds\\
&\leq 
\frac{1}{T}\int_0^T \frac{1}{|t-s|} \int_s^t\Phi\left(|t-s| u'(r)\right)drds
\end{split}
\end{equation*}
From \eqref{eq:escalar_ine} we have that $\Phi(rx)/r$ is increasing with respecto to $r>0$ for $x\in\rr^d$ fix. Therefore, previous inequality  implies \eqref{eq:wirtinger}. If we apply this inequality to the function $\left(T\|u'\orlnor\right)^{-1}u$ we obtain
\[\Phi\left(\frac{\tilde{u}(t)}{T\|u'\orlnor}\right)\leq\frac{1}{T}\int_0^T \Phi\left(\frac{u'(r)}{\|u'\orlnor}\right)dr\leq\frac{1}{T}.\]
Using Proposition \ref{prop:AsubPhi}\eqref{it:prop3} we obtain $\tilde{u}\in L^{\infty}$ and
\begin{equation}\label{eq:cotau'}
 |\tilde{u}(t)|\leq TA_{\Phi}^{-1}\left(\frac{1}{T}\right)\|u'\orlnor.
\end{equation}




In order to prove the Sobolev's inequality, we note that, using Jensen's inequality and 
the definition of $\|u\orlnor$, we obtain
\[ \Phi\left( \frac{ \overline{u}}{\|u\orlnor} \right) \leq
\frac{1}{T}\int_0^T\Phi\left(\frac{u(s)}{\|u\orlnor}\right)ds\leq\frac{1}{T}
\]
Then  by By Proposition \ref{prop:AsubPhi}\eqref{it:prop3} 
\[|\overline{u}|\leq A_{\Phi}^{-1}\left(\frac{1}{T}\right) \|u\orlnor.\]
Therefore, from this and \eqref{eq:cotau'} we get

\[\begin{split}
 \|u\linf &\leq |\overline{u}|+\|\tilde{u}\linf\leq A_{\Phi}^{-1}\left(\frac{1}{T}\right)\max\{1,T\}\|u\sobnor
 \end{split}
 \]
 
\ref{in-sob-cont} and  \ref{eq:sobolev} imply that there exist $C_T>0$ with
\[\|u\|_{  C^w([0,T],\rr^d) }\leq   C_T\|u\sobnor,\]
i.e. $\wphi\left([0,T],\rr^d\right)\hookrightarrow C^w([0,T],\rr^d)$.
As a consequence of Arzela-Ascoli Theorem
 the embedding $C\left([0,T],\rr^d\right)\hookrightarrow C^w([0,T],\rr^d)$ is  compact. It was proved in  \cite[Prop. 5.13]{driver}   for the case $w(s)=|s|^{\alpha}$ with $0< \alpha\leq 1$. For  $w$  arbitrary, the proof follows with some obvious modifications. Consequently the embedding  $\wphi\left([0,T],\rr^d\right)\hookrightarrow C([0,T],\rr^d)$ is compact.
\end{proof}


Given a function $a:\mathbb{R}^d\to \mathbb{R}$, we define the composition operator $\b{a}:\mathcal{M}\to \mathcal{M}$ by $\b{a}(u)(x)= a(u(x))$.
We will often use the following result whose proof can be performed as that of  Corollary 2.3 in \cite{ABGMS2015}.
\begin{lem}\label{lem:cota-a}
\label{a_bound} If $a\in C(\mathbb{R}^d,\mathbb{R}^+)$ then $\b{a}:\wphi\to L^{\infty}([0,T])$ is bounded.
More concretely,  there exists a non decreasing function $A:[0,+\infty)\to[0,+\infty)$ such that
 $\|\b{a}(u)\|_{L^{\infty}([0,T])}\leq A(\|u\|_{\wphi})$.
\end{lem}




\begin{lem}\label{lem:segundo lema}
Let  $\{{u}_n\}_{n\in \mathbb{N}}$ be a sequence of  functions in $\Pi(\ephi,\lambda)$ converging to  ${u}\in \Pi(\ephi,\lambda)$  in the $\lphi$-norm. Then, there exist a subsequence
$\{u_{n_k}\}$ and a function $h\in L^1([0,T],\rr)$ such that
 $\Phi(\frac{{u}_{n_k}}{\lambda})\rightarrow \Phi(\frac{u}{\lambda}) \quad\text{a.e.}$ and
$\Phi(\frac{{u}_{n_k}}{\lambda})\leq h\quad\text{a.e.}$
\end{lem}


\begin{proof}
As $u \in \Pi(\ephi,\lambda)$, we consider  $\Lambda \in (0,\lambda)$.
In this way, $d(u,\ephi)<\Lambda<\lambda$ and,
taking into account \eqref{eq:inclusiones}, $\Phi(\frac{u}{\lambda})\in L^1([0,T],\rr)$.

Applying \cite[Lemma 3.1]{chamra2017anisotropic} with $x+y=\frac{u_n}{\lambda}$, $x=\frac{u}{\lambda}$,
$k=\frac{\lambda}{\Lambda}$, $0<\epsilon<\frac{\Lambda}{\lambda}$ and
$C_{\epsilon}=\frac{\Lambda}{\epsilon(\lambda-\Lambda)}$, we have
\begin{equation}\label{eq:conv-acot-l1}
\begin{split}
&\int_0^T \left|\Phi\left(\frac{u_n}{\lambda}\right)-\Phi\left(\frac{u}{\lambda}\right)\right|\,dt
\\
&\leq \epsilon \int_0^T \left|\Phi\left(\frac{u}{\lambda}\right)-\frac{\lambda}{\Lambda}\Phi\left(\frac{u}{\lambda}\right)\right|\,dt+
2\int_0^T \Phi\left( \frac{\Lambda}{\epsilon(\lambda-\Lambda)}\frac{u_n-u}{\lambda}\right)\,dt.
\end{split}
\end{equation}
Let $\eta>0$. Since $\Phi(\frac{u}{\lambda})\in L^1([0,T,\rr])$,  we can choose $\epsilon$ such that
\begin{equation}\label{eq:cota-l1}
\epsilon \int_0^T \left|\Phi\left(\frac{u}{\lambda}\right)-\frac{\lambda}{\Lambda}\Phi\left(\frac{u}{\lambda}\right)\right|\,dt
<\eta.
\end{equation}
From the fact that $u_n \to u$ in the $L^{\Phi}$-norm, there exists $n_0$ such that
$\|u_n-u\orlnor <\epsilon^2(\lambda-\Lambda)$
for every $n\geq n_0$.
Then, by the convexity of $\Phi$ and the definition of Orlicz norm, we get
\begin{equation}\label{eq:cota-conv-orl}
\int_0^T \Phi\left( \frac{\Lambda}{\epsilon(\lambda-\Lambda)}\frac{u_n-u}{\lambda}\right)\leq
\epsilon\frac{\Lambda}{\lambda} \int_0^T\Phi\left( \frac{u_n-u}{\epsilon^2(\lambda-\Lambda)}\right)<\epsilon
\end{equation}
Thus, from \eqref{eq:conv-acot-l1}, \eqref{eq:cota-l1} and \eqref{eq:cota-conv-orl},
we obtain  that $\Phi(\frac{u_n}{\lambda})$ converges to $\Phi(\frac{u}{\lambda})$ in the $L^1$-norm.
Now,  \cite[Thm. 4.9]{brezis2010functional} implies that there exist a subsequence $\Phi(\frac{u_{n_k}}{\lambda})$
and a function $h \in L^1([0,T],\rr)$
such that $\Phi(\frac{u_{n_k}}{\lambda})\to \Phi(\frac{u}{\lambda})$ a.e. and  $\Phi(\frac{u_{n_k}}{\lambda})\leq h$ a.e.
\end{proof}

\begin{proof}
Since $d({u},\ephi)<1$ and ${u}_n$ converges to ${u}$, there exists $u_0\in\ephi$, a subsequence of $u_n$ (again denoted $u_n$) and $0<r<1$  such that $d(u_n,u_0)<r$. Let $\lambda_0\in (r,1)$.  By extracting more subsequences, if necessary, we can assume that $u_n\to u$ a.e. and
\[\lambda_n:=\|{u}_{n+1}-{u}_{n}\orlnor<\frac{1-\lambda_0}{2^n},\quad\hbox{for } n\geq 1.\]
We can assume $\lambda_n>0$ for every $n=0,\ldots$.

Let $\lambda:=1-\sum_{n=0}^{\infty}\lambda_n$ and define $h:[0,T]\rightarrow\mathbb{R}$  by
\begin{equation}\label{eq:serie} h(x)= \lambda\Phi\left(\frac{u_0}{\lambda}\right)+\sum_{n=0}^{\infty}\lambda_n\Phi\left(\frac{u_{n+1}-u_n}{\lambda_n}\right).
\end{equation}
Note that $\sum_{n=0}^{\infty}\lambda_n+\lambda=1$, therefore for any $n=1,\ldots$


\[
 \begin{split}
   \Phi(u_n) &=\Phi\left(  \lambda\frac{u_0}{\lambda}+   \sum_{j=0}^{n-1}\lambda_j\frac{u_{j+1}-u_j}{\lambda_j} \right)\\
   &\leq
   \lambda\Phi\left(\frac{u_0}{\lambda}\right)+\sum_{j=0}^{n-1}\lambda_j\Phi\left(\frac{u_{j+1}-u_j}{\lambda_j}\right) \leq h
 \end{split}
\]

Since $u_0\in\ephi\subset \claseor$ and $\ephi$ is a subspace we have that $\Phi(u_0/\lambda)\in L^1([0,T],\rr)$. 
On the other hand $\|u_{n+1}-u_n\orlnor \leq \lambda_n$, therefore
\[
 \int_0^T\Phi\left(\frac{u_{j+1}-u_j}{\lambda_j}\right)dt\leq 1.
\]
Then $h\in L^1([0,T],\rr)$.


\end{proof}




\section{Differentiability Gate\^aux of action integrals in anisotropic Orlicz spaces}

Next, we deal with the differentiability of the action integral 
\begin{equation}\label{eq:integral_accion}
I(u)=\int_{0}^T \mathcal{L}(t,u(t),\dot{u}(t))\ dt.
\end{equation}

\begin{thm}\label{teo:diferenciabilidad}
Let $\mathcal{L}$ be a differentiable Carath\'eodory function satisfying \eqref{eq:condicion-estructura}.
Then the following statements hold:
\begin{enumerate}
\item \label{it:T1item1} \label{A1} The action integral given by \eqref{eq:integral_accion}
is finitely defined on $\domi:=W^{1}\lphi\cap\{u|\dot{u}\in\Pi(\ephi,1)\}$.

\item\label{it:T1item3} The function  $I$ is G\^ateaux differentiable on $\domi$ and  its derivative $I'$ is demicontinuous from 
$\domi$  into $\left[\wphi \right]^*$. Moreover, $I'$ is given by the following expression
\begin{equation}\label{eq:DerAccion}
\langle  I'(u),v\rangle= \int_0^T \left\{D_{x}\mathcal{L}\big(t,u,\dot{u}\big)\ccdot v
+ D_{y}\mathcal{L}\big(t,u,\dot{u}\big)\ccdot\dot{v}\right\} \ dt.
\end{equation}

\item\label{it:T1item4}  If  $\Psi \in \Delta_2$ then 
  $I'$ is continuous from $\domi$ into $\left[\wphi\right]^*$ when both spaces are equipped with the strong topology.
\end{enumerate}
\end{thm}


\begin{proof}
Let $u\in \domi$.
As 
\begin{equation}\label{eq:inclusion3}
\dot{u}\in\Pi(\ephi,1)\subset \claseor_1
\end{equation}
and \eqref{eq:inclusiones}, then $\Phi( \dot{u}(t)) \in L^1$.
Now,
 \begin{equation}\label{eq:cota-condicion-estructura}
|\mathcal{L}(\cdot,u,\dot{u})|+ |D_{x}\mathcal{L}(\cdot,u,\dot{u})|
+\Psi(D_{y}\mathcal{L}(\cdot,u,\dot{u}))
\leq A(\|u\sobnor ) (b+ \Phi (\dot{u})) \in
 L^1,
\end{equation}
by  \eqref{eq:condicion-estructura} and Lemma \ref{lem:cota-a}.
Thus item \eqref{it:T1item1} is proved.

We split up the proof of item \ref{it:T1item3} into four steps.

\noindent\emph{Step 1. The non linear operator  $u \mapsto D_{x}\mathcal{L}(t,u,\dot{u})$ is continuous from $\domi$ into $L^{1}([0,T])$ with the strong topology on both sets.} 


%If $u\in \domi$, from \eqref{eq:condicion-estructura} and \eqref{eq:inclusion3}, we obtain 
%\begin{equation}\label{eq:DxL1}
%|D_{x}\mathcal{L}(\cdot,u,\dot{u})|\leq A(\|u\sobnor) \left(b+\Phi\left(\dot{u})\right) \in L^1.
%\end{equation}


Let   $\{u_n\}_{n\in \mathbb{N}}$ be a sequence of  functions in $\domi$  
and let $u\in \domi$  such that $u_n\rightarrow u$ in $\wphi$.
By \eqref{eq:sobolev}, we have 
\[
|u_n(t)-u(t)|\leq T A_{\Phi}^{-1}\left(\frac{1}{T}\right) \|u_n-u\orlnor
\]
then $u_n \to u$ uniformly.
As $\dot{u}_n\rightarrow \dot{u}\in\domi$, by 
  Lemma \ref{segundo lema}, there exist a subsequence of  $\dot{u}_{n_k}$ (again denoted $\dot{u}_{n_k}$) and a function  
	$h\in L^1([0,T],\rr)$
	such that  $\dot{u}_{n_k}\rightarrow \dot{u} \quad\text{a.e.}$ and $\Phi(\dot{u}_{n_k})\leq h\quad\text{a.e}$.  

Since $u_{n_k}$, $k=1,2,\ldots,$ is a strong convergent sequence in $\wphi$, it is a bounded sequence in $\wphi$. 
According to item \eqref{it:embeding} of   Lemma \ref{lem:inclusion orlicz}, 
% and Lemma \eqref{lem:cota-a}, 
there exists $M>0$ such that $\|\b{a}(u_{n_k})\|_{L^{\infty}} \leq M$, $k=1,2,\ldots$.  
From the previous facts and \eqref{eq:cota-condicion-estructura}, we get
\begin{equation*}\label{eq:DxL1-bis}
|D_{x}\mathcal{L}(\cdot,u_{n_k},\dot{u}_{n_k})|\leq a(|u_{n_k}|)(b+\Phi(\dot{u}_{n_k}))\leq
M (b+h) \in L^1.
\end{equation*}
On the other hand, by the continuous differentiability of $\mathcal{L}$, we have
\[D_{x}\mathcal{L}(t,u_{n_k}(t),\dot{u}_{n_k}(t))\to D_{x}\mathcal{L}(t,u(t),\dot{u}(t))\quad\hbox{ for a.e. } t\in[0,T].\]
Applying the Dominated Convergence Theorem we conclude the proof of step 1.


\noindent\emph{Step 2. The non linear operator   $u
 \mapsto  D_{y}\mathcal{L}(t,u,\dot{u})$ is continuous from $\domi$ with the strong topology  
into $\left[\lphi\right]^*$  with the weak$^*$ topology.}

 Let $u\in \domi$.  From  \eqref{eq:cota-condicion-estructura} it follows that 
%\begin{equation}\label{eq:DyLpsi}
%\Psi(D_y\mathcal{L}(\cdot,u,\dot{u}))\leq a(|u|)(b+\Phi(\dot{u}))\in L^1
%\end{equation}
%then
\begin{equation}\label{eq:DyLpsi-clase}
D_{y}\mathcal{L}(\cdot,u,\dot{u})\in C^{\Psi}.
\end{equation}

As\'i? o conviene poner la cota de $\Psi(D_y)$ expl\'icitamente???

 Note that \eqref{eq:cota-condicion-estructura},  \eqref{eq:DyLpsi-clase} and the imbeddings $\wphi \hookrightarrow L^{\infty}$ and  
$\lpsi\hookrightarrow  \left[\lphi\right]^*$ imply that the second member of
\eqref{eq:DerAccion} defines an element of $\left[\wphi\right]^*$.



Let $u_n,u\in \domi$ such that $u_n\to u$ in the norm of $\wphi$. 
We must prove that  $D_{y}\mathcal{L}(\cdot,u_n,\dot{u}_n)\overset{w^*}{\rightharpoonup} 
D_{y}\mathcal{L}(\cdot,u,\dot{u})$. 
On the contrary, there exist $v\in\lphi$, $\epsilon>0$ and a subsequence of $\{u_n\}$ (denoted  $\{u_n\}$ for simplicity)  such that
\begin{equation}\label{cota_eps}
 \left| \langle D_{y}\mathcal{L}(\cdot,u_n,\dot{u}_n),v \rangle - 
\langle  D_{y}\mathcal{L}(\cdot,u,\dot{u}),v \rangle\right|\geq \epsilon.
\end{equation}
We have $u_n\rightarrow u$ in $\lphi$ and
$\dot{u}_n\rightarrow \dot{u}$ in $\lphi$.
 By Lemma \ref{segundo lema}, there exist a subsequence of $\{u_n\}$ (again denoted  $\{u_n\}$ for simplicity) 
and a function $h\in L^1([0,T],\rr)$ such that 
$u_n\rightarrow u$ uniformly, $\dot{u}_n\rightarrow \dot{u} \quad\text{a.e.}$ and $\Phi(\dot{u}_n)\leq h\quad\text{a.e.}$ 
As in the previous step, since $u_n$ is a convergent sequence, 
Lemma \ref{lem:cota-a} implies that $a(|u_n(t)|)$ is uniformly bounded by a certain constant $M>0$. 
Therefore,   from inequality  \eqref{eq:cota-condicion-estructura} with $u_n$ instead of $u$, we have 
\begin{equation}\label{eq:Dy-suc}
  \Psi(D_{y}\mathcal{L}(\cdot,u_n,\dot{u}_n))   
	\leq M (b+h)\in L^1.
\end{equation}
As $v \in \lphi$ there exists $\lambda>0$ such that $\Phi(\frac{v}{\lambda})\in L^1$. 
Now, by Young inequality and \eqref{eq:Dy-suc}, we have
\begin{equation}\label{eq:Dy_lambda-Psi}
\begin{split}
&\lambda D_{y}\mathcal{L}(\cdot,u_{n_k},\dot{u}_{n_k})\ccdot \frac{v(t)}{\lambda} 
\\
&
\leq 
\lambda\left[\Psi(D_{y}\mathcal{L}(\cdot,u_{n_k},\dot{u}_{n_k}))+\Phi\left(\frac{v}{\lambda}\right)\right]
\\
&\leq \lambda M (b+h)+\lambda \Phi\left(\frac{v}{\lambda}\right)\in L^1
\end{split}
\end{equation}
  Finally, from the Lebesgue Dominated Convergence Theorem, we deduce
\begin{equation}\label{conv_debil}
\int_0^T  D_{y}\mathcal{L}(t,u_{n_k},\dot{u}_{n_k})
\ccdot  v \,dt 
\to 
\int_0^T D_{y}\mathcal{L}(t,u,\dot{u})\ccdot v\, dt \end{equation}
which contradicts the inequality \eqref{cota_eps}. This completes the proof of step 2.

\emph{Step 3.} We will prove \eqref{eq:DerAccion}. 
%The proof follows similar lines as \cite[Thm. 1.4]{mawhin2010critical}. 
For $u\in \domi$ and $0\neq v\in\wphi$, we define the function
\[H(s,t):=\mathcal{L}(t,u(t)+s v(t),\dot{u}(t)+s\dot{v}(t)).\]

%From \cite[Lemma 10.1]{KR} (or \cite[Thm. 5.5]{Orliczvectorial2005} ) 
%we obtain that if $|u|\leq |v|$ then    $d(u,\ephi)\leq d(v,\ephi)$. Esto va????

For  $|s|\leq s_0:=\min\{\left(1-d(\dot{u},\ephi)\right)/\|v\sobnor, 1-d(\dot{u},\ephi)\}$, 
using triangle inequality we get 
$
d \left(\dot{u}+s \dot{v}, \ephi \right)<1$ and thus $\dot{u}+s\dot{v} \in \Pi(\ephi,1)$. 
These facts imply, in virtue of Theorem \ref{teo:diferenciabilidad} item \ref{it:T1item1}, 
that $I(u+s v)$ is well defined and finite for $|s|\leq s_0$. 

We also have 
$
\|u+sv\sobnor\leq \|u\sobnor+s_0\|v\sobnor;
$
then, by Lemma \ref{lem:cota-a}, there exists $M>0$ such that 
$\|a(u+sv)\linf\leq M$.

Let $\lambda>0$ such that $\Phi(\frac{\dot{v}}{\lambda})\in L^1$.
On the other hand, if $\dot{v}\in\lphi$ and $|s|\leq s_0 \lambda^{-1}$,
from the convexity and the parity of $\Phi$, we get
\[
\begin{split}
&\Phi(\dot{u}+s\dot{v})=
\Phi\left((1-s_0)\frac{\dot{u}}{1-s_0}+s_0 \frac{s}{s_0}\dot{v}\right)
\leq
(1-s_0)\Phi\left(\frac{\dot{u}}{1-s_0}\right)+s_0 \Phi\left(\frac{s}{s_0}\dot{v}\right)
\\
&\leq
(1-s_0)\Phi\left(\frac{\dot{u}}{1-s_0}\right)+s_0 \Phi\left(\frac{\dot{v}}{\lambda}\right)
\in L^1
\end{split}
\]
As $\dot{u}\in\Pi(\ephi,1)$ then
\[
d\left(\frac{\dot{u}}{1-s_0},E^{\Phi}\right)=\frac{1}{1-s_0}d(\dot{u}, E^{\Phi})<1
\]
and therefore $\frac{\dot{u}}{1-s_0}\in C^\Phi$.

Now, applying \eqref{eq:cota-condicion-estructura}, \eqref{eq:Dy_lambda-Psi},  
%the monotonicity of $\varphi$ and $\Phi$, 
the fact that $v \in L^{\infty}$ and $\dot{v}\in\lphi$, 
%and H\"older's inequality, 
we get
\begin{equation}\label{ctg}
\begin{split}
|D_s H(s,t)|&=\left| D_{x}\mathcal{L}(t,u+sv,\dot{u}+s\dot{v})\ccdot v +  
\lambda D_{y}\mathcal{L}(t, u+s v, \dot{u}+s\dot{v})\ccdot\frac{\dot{v}}{\lambda}\right| \\
  & \leq M \left[ b(t)+ \Phi(\dot{u}+s\dot{v})\right]|v|\\
 &\quad+ \lambda\left[\Psi(D_{y}\mathcal{L}(t,u+sv,\dot{u}+s\dot{v}))+\Phi\left(\frac{\dot{v}}{\lambda}\right) \right]
\\
 &\leq M \left\{\left[ b(t)+ \Phi(\dot{u}+s\dot{v})\right]|v|\right\}+
 \lambda M[ b(t)+ \Phi(\dot{u}+s\dot{v})]+\lambda \Phi\left(\frac{\dot{v}}{\lambda}\right)
 \\
 &=
 M [ b(t)+ \Phi(\dot{u}+s\dot{v})] (|v|+\lambda) +\lambda \Phi\left(\frac{\dot{v}}{\lambda}\right)
 \in L^1.
\end{split}
\end{equation}

Consequently, $I$ has a directional derivative and
\[
\langle I'(u),v \rangle=\frac{d}{ds}I(u+s v)\big|_{s=0}=\int_0^T  
\left\{D_{x}\mathcal{L}(t,u,\dot{u})\ccdot v+ D_{y}\mathcal{L}(t,u,\dot{u})\ccdot \dot{v}\right\} \ dt.
\]
Moreover, from the previous formula, \eqref{eq:cota-condicion-estructura},  \eqref{eq:DyLpsi-clase}, and
Lemma \ref{lem:inclusion orlicz}, we obtain
\[
|\langle I'(u),v \rangle| \leq \|D_{x}\mathcal{L}\|_{L^1} \| v\linf + 
\|D_{y}\mathcal{L}\|_{\lpsi} \|\dot{v}\orlnor \leq C \|v\sobnor
\]
with a appropriate constant $C$.

This completes the proof of the G\^ateaux differentiability of $I$. 



\emph{Step 4. The operator $I':\domi  \to \left[\wphi_d
\right]^* $ is demicontinuous.}
This is a consequence  of the continuity of the mappings $u \mapsto D_{x}\mathcal{L}(t,u,\dot{u})$ and $u \mapsto
D_{y}\mathcal{L}(t,u,\dot{u})$. Indeed, if $u_n,u\in \domi$ with $u_n\to u$ in the norm of $\wphi$ and $v \in
\wphi$, then
\[
\begin{split}
\left\langle  I'(u_{n}),v \right\rangle &= \int_0^T \left\{  D_{x}\mathcal{L}\left(t,u_n,\dot{u}_n\right)\ccdot
v +
 D_{y}\mathcal{L}\left(t,u_n,\dot{u}_n\right)\ccdot \dot{v}\right\} \ dt\\
&\rightarrow \int_0^T \left\{ D_{x}\mathcal{L}\left(t,u,\dot{u}\right)\ccdot v+ 
D_{y}\mathcal{L}\left(t,u,\dot{u}\right)\ccdot \dot{v}\right\} \ dt\\
&=\left\langle  I'(u),v \right\rangle.
\end{split}
\]


In order to prove item  \ref{it:T1item4}, it is necessary to see that the maps $u\mapsto D_{x}\mathcal{L}(t,u,\dot{u})$  
and $u\mapsto D_{y}\mathcal{L}(t,u,\dot{u})$  are norm continuous
from $\domi $ into $L^1$ and
 $\lpsi$, respectively.  

The continuity of the first map has already been proved in step 1. 

Si eliminamos la demicontinuidad del segundo item, hay que copiar 
la continuidad de $D_x$ aqu\'i!!!

Let $u_n, u \in \domi$ with $\|u_n- u\sobnor\to 0$.  

Applying Lemma \ref{segundo lema} to $\dot{u}_n$, 
there exists a subsequence (denoted $\dot{u}_n$ for simplicity)
such that
$\dot{u}_n \in L^\Phi$ and a function  $h \in L^1$  such that  $\Psi(\dot{u}_n)\leq h$ and $\dot{u}_n \to \dot{u}$ a.e.

Then, by \eqref{eq:Dy_lambda-Psi} we have
 $ \Psi(v_n) 
	\leq m(t) \in L^1$
 being  $v_n:=D_{y}\mathcal{L}(\cdot,u_n,\dot{u}_n)$ and $m(t):= M (b+h)$. 
In addition, from the continuous differentiability of $\mathcal{L}$, we have that
$v_n \to v$ a.e. where $D_y\mathcal{L}(\cdot,u,\dot{u})$.

As  $\Psi\in\Delta_2$, there exists $c:\rr^+\to ???$ such that 
$\Psi(\lambda x)\leq c(|\lambda|)\Psi(x)$. 
Then, $\Psi(\frac{v_n-v}{\lambda})\leq c(|\lambda|^{-1})\Psi(v_n-v)$ for every $\lambda \in \rr$.

Therefore, $\Psi(\frac{v_n-v}{\lambda})\to 0$ a.e. as $n \to \infty$ and 
$\Psi(\frac{v_n-v}{\lambda})\leq c(|\lambda|^{-1})K\Psi(v_n)+\Psi(v))
\leq c(|\lambda|^{-1}) K [m(t)+\Psi(v)])\in L^1$.

Now, by Dominated Convergence Theorem, we get
$
\int \Psi(\frac{v_n-v}{\lambda})\,dt \to 0
$
for every $\lambda>0$. Thus, $v_n \to v$ in $L^{\Psi}$.

The continuity of $I'$  follows  from the continuity 
of $D_{x}\mathcal{L}$ and $D_{y}\mathcal{L}$ using the formula \eqref{eq:DerAccion}.
\end{proof}







\section*{Acknowledgments}
The authors are partially supported by a UNRC grant number 18/C417. The first author is  partially supported by a  UNSL grant number 22/F223. 




% \bibliographystyle{apalike}
 \bibliographystyle{plain}
 
\bibliography{biblio}


\end{document}


