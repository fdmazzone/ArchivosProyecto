\documentclass[twoside]{article}


\NeedsTeXFormat{LaTeX2e}
\ProvidesPackage{mathscinet}[2002/04/17 v1.05]
\RequirePackage{textcmds}\relax
\ProvideTextCommandDefault{\cprime}{\tprime}



%\usepackage{hyperref}
\usepackage{amssymb,amsthm}
\usepackage{amsmath}
\usepackage{color}
\usepackage{ esint }
\usepackage{mathabx}
\usepackage{MnSymbol}
\usepackage{fancyhdr}
\usepackage{times}

\usepackage[latin1]{inputenc}

\usepackage{comment}
\usepackage{url}
\usepackage{xcolor}
\usepackage{adjustbox}
\usepackage{hyperref}

\newtheorem{thm}{Theorem}[section]
\newtheorem{cor}[thm]{Corollary}
\newtheorem{lem}[thm]{Lemma}

\newtheorem{defi}[thm]{Definition}
\newtheorem{prop}[thm]{Proposition}
\theoremstyle{remark}
\newtheorem{comentario}{Remark}


\makeatletter
\newcommand{\labitem}[2]{%
\def\@itemlabel{\textbf{#1}}
\item
\def\@currentlabel{#1}\label{#2}}
\makeatother




\title{Periodic solutions of 
Euler-Lagrange equations in an Orlicz-Sobolev space setting by the dual least action principle }
\author{Sonia Acinas \thanks{SECyT-UNRC, UNSL and CONICET}\\
Instituto de Matem\'atica Aplicada San Luis (IMASL)\\ 
Universidad Nacional de San Luis and CONICET\\
Ej\'ercito de los Andes 950,
(D5700HDW) San Luis, Argentina\\
Universidad Nacional de La Pampa\\
(L6300CLB) Santa Rosa, La Pampa, Argentina\\
\url{sonia.acinas@gmail.com}\\[3mm]
Fernando D. Mazzone \thanks{SECyT-UNRC and CONICET}\\
Dpto. de Matem\'atica, Facultad de Ciencias Exactas, F\'{\i}sico-Qu\'{\i}micas y Naturales\\
Universidad Nacional de R\'{i}o Cuarto\\
(5800) R\'{\i}o Cuarto, C\'ordoba, Argentina,\\
\url{fmazzone@exa.unrc.edu.ar}
}

\date{}

\newcommand{\orlnor}{\|_{L^{\Phi}}}
\newcommand{\lurnor}{\|^{*}_{L^{\Phi}}}
\newcommand{\linf}{\|_{L^{\infty}}}
\newcommand{\lphi}{L^{\Phi}}
\newcommand{\lphiuno}{L^{\Phi_1}}
\newcommand{\lphidos}{L^{\Phi_2}}
\newcommand{\lphii}{L^{\Phi_i}}
\newcommand{\lpsi}{L^{\Psi}}
\newcommand{\lpsiuno}{L^{\Psi_1}}
\newcommand{\lpsidos}{L^{\Psi_2}}
\newcommand{\lpsii}{L^{\Psi_i}}
\newcommand{\lmuno}{L^{M_1}}
\newcommand{\lmdos}{L^{M_2}}
\newcommand{\lmj}{L^{M}}
\newcommand{\lmn}{L^{M_n}}
\newcommand{\ephi}{E^{\Phi}}
\newcommand{\ephiuno}{E^{\Phi_1}}
\newcommand{\ephidos}{E^{\Phi_2}}
\newcommand{\ephin}{E^{\Phi_n}}
\newcommand{\ephii}{E^{\Phi_i}}
\newcommand{\claseor}{C^{\Phi}}
\newcommand{\wphi}{W^{1}\lphi}
\newcommand{\wphiuno}{W^{1}\lphiuno}
\newcommand{\wphidos}{W^{1}\lphidos}
\newcommand{\wphii}{W^{1}\lphii}
\newcommand{\wphiet}{W^{1}\ephi_T}
\newcommand{\wphie}{W^{1}\ephi}
\newcommand{\sobnor}{\|_{W^{1}\lphi}}
\newcommand{\domi}{\mathcal{E}^{\Phi}_d(\lambda)}
\newcommand{\domiuno}{\mathcal{E}^{\Phi_1}_d(\lambda)}
\newcommand{\domidos}{\mathcal{E}^{\Phi_2}_d(\lambda)}
\newcommand{\domii}{\mathcal{E}^{\Phi_i}_d(\lambda)}
\newcommand{\domin}{\mathcal{E}^{\Phi_n}_d(\lambda)}
\renewcommand{\b}[1]{\boldsymbol{#1}}
\newcommand{\rr}{\mathbb{R}}
\newcommand{\nn}{\mathbb{N}}
\newcommand{\ccdot}{\b{\cdot}}
\renewcommand{\leq}{\leqslant} 
\renewcommand{\geq}{\geqslant} 
\newcommand{\epsi}{E^{\Psi}}


\newcounter{example}

\setcounter{example}{1}


\newenvironment{example}{\noindent\textit{Example} \arabic{example}.}{\addtocounter{example}{1}}




\begin{document}


\maketitle
%
\begingroup%Locallizing the change to `thefootnote'.
    \renewcommand{\thefootnote}{}%Removing the footnote symbol.
    %
    \footnotetext{%
    %   2010 Mathematics Subject Classification
    %   http://www.ams.org/msc/
    \textbf{2010  AMS Subject Classification.} Primary: .
    Secondary: .
    }%
        \footnotetext{%
    \textbf{Keywords and phrases.}  .
    }%
    \endgroup
%
%
%
%

\begin{abstract}

In this paper we obtain existence of periodic solutions, in the Orlicz-Sobolev space $\wphi([0,T])$, of hamiltonian systems with a potential  function $F$ satisfying the inequality  $|\nabla F(t,x)|\leq b_1(t) \Phi_0'(|x|)+b_2(t)$, with    $b_1, b_2\in L^1$ and for certain $N$-functions $\Phi_0$ employing the dual least action principle.

\end{abstract}




\pagestyle{fancy} \headheight 35pt \fancyhead{} \fancyfoot{}

\fancyfoot[C]{\thepage} \fancyhead[CE]{\nouppercase{S. Acinas and F.D. Mazzone }} \fancyhead[CO]{\nouppercase{\section}}

\fancyhead[CO]{\nouppercase{\leftmark}}


%\tableofcontents




\section{Introduction}
This paper deals with system  of equations of the type:

\begin{equation}\label{ProbPrin-gral}
    \left\{%
\begin{array}{ll}
  \frac{d}{dt} D_{y}\mathcal{L}(t,u(t),u'(t))= D_{x}\mathcal{L}(t,u(t),u'(t)) \quad \hbox{a.e.}\ t \in (0,T)\\
    u(0)-u(T)=u'(0)-u'(T)=0,
\end{array}%
\right.,
\end{equation}
where $\mathcal{L}:[0,T]\times\rr^d\times\rr^d\to\rr$, $d\geq 1$, is called the \emph{Lagrange function} or \emph{lagrangian} and the unknown function  $u:[0,T]\to\rr^d$ is absolutely continuous. In other words, we are interested in  finding \emph{periodic weak solutions} of \emph{Euler-Lagrange system of ordinary equations}. This topic was deeply addressed???(studied, treated????) for the \emph{Lagrange function}
\begin{equation}\label{eq:lagrange_cuad}
\mathcal{L}_{p,F}(t,x,y)=\frac{|y|^p}{p}+F(t,x),
\end{equation}
for $1<p<\infty$. For example, the classic book  \cite{mawhin2010critical} deals mainly with problem \eqref{ProbPrin-gral}, for the lagrangian $\mathcal{L}_{2,F}$, through various methods: direct, dual action, minimax, etc. The results in \cite{mawhin2010critical} were extended and improved in several articles, see  \cite{tang1995periodic,tang1998periodic,wu1999periodic,tang2001periodic,zhao2004periodic}  to cite some examples. Lagrange functions \eqref{eq:lagrange_cuad} for arbitrary $1<p<\infty$ were considered in  \cite{Tian2007192,tang2010periodic} and in this case \eqref{ProbPrin-gral}  is reduced to the $p$-laplacian system
\begin{equation}\label{ProbP-lapla}
    \left\{%
\begin{array}{ll}
   \frac{d}{dt}\left(u'(t)|u'|^{p-2}\right) = \nabla F(t,u(t)) \quad \hbox{a.e.}\ t \in (0,T)\\
    u(0)-u(T)=u'(0)-u'(T)=0.
\end{array}%
\right.
\end{equation}


In this context, it  is customary to call $F$ a  \emph{potential function}, and it is assumed that $F(t,x)$ is differentiable with respect to $x$ for a.e. $t\in [0,T]$ and the following conditions are verified:
\begin{enumerate}
\labitem{(C)}{item:condicion_c} $F$ and its gradient $\nabla F$, with respect to $x\in\rr^d$,  are  Carath\'eodory functions, i.e. they are measurable functions with respect to $t\in [0,T]$, for every  $x\in\rr^d$, and they are continuous functions with  respect to  $x\in\rr^d$ for a.e. $t \in [0,T]$.
 \labitem{(A)}{item:condicion_a}  For   a.e. $t\in [0,T]$, it holds that
\begin{equation}
|F(t,x)| + |\nabla F(t,x)|  \leq a(|x|)b(t).
\end{equation}
In this inequality we assume that the function  $a:[0,+\infty)\to [0,+\infty)$ is continuous and nondecreasing and $0\leq b\in L^1([0,T],\rr)$.
\end{enumerate}


In \cite{ABGMS2015} it was treated  the case of a lagrangian $\mathcal{L}$ which is lower bounded by a Lagrange function
\begin{equation}\label{eq:lagrange_phi}
\mathcal{L}_{\Phi,F}(t,x,y)=\Phi(|y|)+F(t,x),
\end{equation}

where  $\Phi$ is an $N$-function (see section \ref{preliminares} for the definition of this concept).  
In the paper \cite{ABGMS2015} it was assumed  a condition of \emph{bounded oscillation} on $F$  (see xxxxx below). 
In this paper we shall study the condition of \emph{sublinearity} (see  \cite{tang1998periodic,wu1999periodic,zhao2004periodic,tang2010periodic,zhao2005existence}) on $\nabla F$ for the lagrangian  $\mathcal{L}_{\Phi,F}$, or more generally for lagrangians which are lower bounded by $\mathcal{L}_{\Phi,F}$.



The problem \eqref{ProbPrin-gral} comes from a variational one, that is,  the equation in  \eqref{ProbPrin} ESTA EQ DESAPARECIO!!!!  
is the Euler-Lagrange equation associated to the \emph{action integral}
\begin{equation}\label{integral_accion}
I(u)=\int_{0}^T \mathcal{L}(t,u(t),u'(t))\ dt.
\end{equation}

The paper is organized as follows. .....

\section{Preliminaries}\label{preliminares}


For reader convenience, we give a short introduction to Orlicz and Orlicz-Sobolev spaces of vector valued functions. Classic references for these topics are \cite{adams_sobolev,KR,rao1991theory}. 

Hereafter we denote  by $\mathbb{R}^+$  the set of all non negative real numbers. A function $\Phi:\mathbb{R}^+\to \mathbb{R}^+ $ is called an \emph{$N$-function} if $\Phi$ is convex and it also satisfies that
\[
\lim_{t\to+\infty}\frac{\Phi(t)}{t}=+\infty\quad\text{and}\quad \lim_{t\to 0}\frac{\Phi(t)}{t}=0.
\]
In addition,  in this paper for the sake of simplicity  we assume that $\Phi$ is differentiable and we call $\varphi$  the derivative of $\Phi$. 
On these assumptions, $\varphi:\mathbb{R}^+\rightarrow \mathbb{R}^+$ is a homeomorphism whose inverse will be denoted by $\psi$. 
We denote by $\Psi$ the primitive of $\psi$ that satisfies $\Psi(0)=0$. Then, $\Psi$ is an $N$-function which  is called the \emph{complementary function} of $\Phi$.

 We recall that an $N$-function $\Phi(u)$ has \emph{principal part} $f(u)$ if $\Phi(u)=f(u)$ for large values of the argument (see \cite[p. 16]{KR} and \cite[Sec. 7]{KR} for  properties of principal part).

There exist several orders and equivalence relations between $N$-functions (see \cite[Sec. 2.2]{rao1991theory}).
Following \cite[Def. 1, pp. 15-16]{rao1991theory} we say that the   $N$-function $\Phi_2$ is \emph{stronger} than the $N$-function  $\Phi_1$, in symbols  $\Phi_1\prec\Phi_2$, if  there exist $a>0$ and $x_0\geq 0$ such that
\begin{equation}\label{eq:prec}\Phi_1(x)\leq \Phi_2(ax), \quad x\geq x_0.\end{equation}
 The $N$-functions  $\Phi_1$ and   $\Phi_2$ are \emph{equivalent} ($\Phi_1\sim\Phi_2$)  when  $\Phi_1\prec\Phi_2$ and $\Phi_2\prec\Phi_1$.
We say that  $\Phi_2$ is \emph{essentially stronger} than  $\Phi_1$  ($\Phi_1\llcurly\Phi_2$) if and only if for every $a>0$ there exists $x_0=x_0(a)\geq 0$ such that \eqref{eq:prec} holds. Finally, we say that  $\Phi_2$ is \emph{completely stronger} than  $\Phi_1$  ($\Phi_1\closedprec\Phi_2$) if and only if for every $a>0$ there exist $K=K(a)>0$ and  $x_0=x_0(a)\geq 0$ such that


\begin{equation}\label{eq:prec2}\Phi_1(x)\leq K\Phi_2(ax), \quad x\geq x_0.\end{equation}


We also say that a non decreasing function $\eta:\mathbb{R}^+\rightarrow \mathbb{R}^+$ satisfies the  \emph{$\Delta_2^{\infty}$-condition}, denoted by $\eta \in \Delta_2^{\infty}$,
if there exist  constants $K>0$ and  $x_0\geq 0$ such that
\begin{equation}\label{delta2defi}\eta(2x)\leq K\eta(x),
\end{equation}
for every $x\geq x_0$. We note that $\eta \in \Delta_2^{\infty}$ if and only if $\eta\closedprec\eta$.
If $x_0=0$,  the function   $\eta:\mathbb{R}^+\rightarrow \mathbb{R}^+$ is said to satisfy the 
\emph{$\Delta_2$-condition} ($\eta \in \Delta_2$). 
If there exists $x_0>0$ such that  inequality \eqref{delta2defi} holds for $x\leq x_0$, 
we will say that $\Phi$ satisfies the 
\emph{$\Delta_2^0$-condition} ($\Phi\in\Delta_2^0$).

We denote by $\alpha_{\eta}$ and $\beta_{\eta}$ the so called  \emph{Matuszewska-Orlicz indices} of the function $\eta$, which are defined next. Given
an increasing, unbounded, continuous function   $\eta:[ 0,+\infty)\to [0,+\infty)$ such that $\eta(0)=0$, we define
\begin{equation}\label{MO_indices}
    \alpha_{\eta}:=\lim\limits_{t\to 0^{+}}\frac{\log \left (\sup\limits_{u>0}\frac{\eta(t u)}{\eta(u)} \right ) }{\log(t)},\quad
    \beta_{\eta}:=\lim\limits_{t\to +\infty}\frac{\log \left  (\sup\limits_{u>0}\frac{\eta(t u)}{\eta(u)}\right )}{\log(t)}.
\end{equation}
It is known that the previous limits exist and  $0\leq \alpha_{\eta}\leq \beta_{\eta}\leq +\infty$ 
(see \cite[p. 84]{M}). The relation $\beta_{\eta}<+\infty$ holds true if and only if $\eta \in \Delta_2$
(\cite[Thm. 11.7]{M}). If $(\Phi,\Psi)$ is a complementary pair of  $N$-functions then
\begin{equation}\label{compl_ind}
 \frac{1}{\alpha_{\Phi}}+\frac{1}{\beta_{\Psi}}=1,
\end{equation}
(see \cite[Cor. 11.6]{M}). Therefore $1\leq \alpha_{\Phi}\leq\beta_{\Phi}\leq \infty $.

 If $\eta$ is an increasing function that satisfies the $\Delta_2$-condition, then $\eta$ is controlled by above and below
 by power functions (\cite[Sec. 1]{Gustavsson1977}, \cite[Eq. (2.3)-(2.4)]{fiorenza1997indices} and \cite[Thm. 11.13]{M}).   More concretely, for every $\epsilon>0$ there exists a
constant $K=K(\eta,\epsilon)$ such that, for every $t,u\geq 0$,
\begin{equation}\label{delta2-potencias}
    K^{-1}\min\big\{t^{\beta_{\eta}+\epsilon},t^{\alpha_{\eta}-\epsilon} \big\}\eta(u)\leq \eta(t u)\leq
    K\max\big\{t^{\beta_{\eta}+\epsilon},t^{\alpha_{\eta}-\epsilon} \big\}\eta(u).
\end{equation}



Let $d$ be a positive integer. We denote by $\mathcal{M}:=\mathcal{M}([0,T],\rr^d)$  the set of all measurable functions defined on $[0,T]$ with values on $\mathbb{R}^d$ and  we write $u=(u_1,\dots,u_d)$ for  $u\in \mathcal{M}$. For the set of functions $\mathcal{M}$, as for other similar sets, we will omit the reference to codomain $\mathbb{R}^d$ when $d=1$.


Given  an $N$-function $\Phi$ we define the \emph{modular function} 
$\rho_{\Phi}:\mathcal{M}\to \mathbb{R}^+\cup\{+\infty\}$ by
\[\rho_{\Phi}(u):= \int_0^T \Phi(|u|)\ dt.\]
Here $|\cdot|$ is the euclidean norm of $\mathbb{R}^d$.
Now, we introduce the \emph{Orlicz class} $C^{\Phi}=C^{\Phi}([0,T],\rr^d)$   by setting
\begin{equation}\label{claseOrlicz}
  C^{\Phi}:=\left\{u\in \mathcal{M} | \rho_{\Phi}(u)< \infty \right\}.
\end{equation}
The \emph{Orlicz space} $\lphi=L^{\Phi}([0,T],\rr^d)$ is the linear hull of $\claseor$;
equivalently,
\begin{equation}\label{espacioOrlicz}
\lphi:=\left\{ u\in \mathcal{M}| \exists \lambda>0: \rho_{\Phi}(\lambda u) < \infty   \right\}.
\end{equation}
  The Orlicz space $\lphi$ equipped with the \emph{Orlicz norm}
\[
\|  u  \orlnor:=\sup \left\{  \int_0^T u\b{\cdot} v\ dt \big| \rho_{\Psi}(v)\leq 1\right\},
\]
is a Banach space. By $u\b{\cdot} v$ we denote the usual dot product in $\mathbb{R}^{d}$ between $u$ and $v$.

The following  inequality holds for any $u\in\lphi$
\begin{equation}\label{amemiya-ine}
\|u\orlnor\leq \frac{1}{k}\left\{1+\rho_{\Phi}(ku)\right\},\quad\text{for every } k>0.
\end{equation}
In fact, $\|u\orlnor$ is the infimum for $k>0$ of the right hand side in above expression  (see \cite[Thm. 10.5]{KR} and \cite{hudzik2000amemiya}). 


The subspace $\ephi=\ephi([0,T],\rr^d)$ is defined as the closure in $\lphi$ of the subspace $L^{\infty}([0,T],\rr^d)$ of all $\mathbb{R}^d$-valued essentially bounded functions. It is shown that  $\ephi$ is the only one maximal subspace contained in the Orlicz class $\claseor$, i.e.
$u\in\ephi$ if and only if $\rho_{\Phi}(\lambda u)<\infty$ for any $\lambda>0$. The equality $\lphi=\ephi$ is true if and only if $\Phi\in\Delta_2^{\infty}$.

A generalized version of \emph{H\"older's inequality} holds in Orlicz spaces (see \cite[Thm. 9.3]{KR}). Namely, if $u\in\lphi$ and $v\in\lpsi$ then $u\cdot v\in L^1$ and
\begin{equation}\label{holder}
\int_0^Tv\cdot u\ dt\leq \|u\orlnor\|v\|_{L^{\Psi}}.
\end{equation}


Like in \cite{KR}, we will consider the subset $\Pi(\ephi_d,r)$ of $\lphi_d$ given by
\[\Pi(\ephi_d,r):=\{\b{u}\in\lphi_d| d(\b{u},\ephi_d)<r\}.\]
This set is related to the Orlicz class $\claseor_d$ by means of inclusions, namely,
\begin{equation}\label{inclusiones}\Pi(\ephi_d, r )\subset r \claseor_d\subset\overline{\Pi(\ephi_d,r)}
\end{equation}
for any positive $r$.
If $\Phi \in \Delta_2$,  then the sets $\lphi_d$, $\ephi_d$, $\Pi(\ephi_d,r)$ and $\claseor_d$ are equal.

 
Let $\domii:=W^{1}\lphii_d\cap\{u|\dot{u}\in\Pi(\ephii_d,\lambda)\}$.

If $X$ and $Y$ are  Banach spaces such that  $Y\subset X^*$, we denote by $\langle\cdot,\cdot\rangle:Y\times X\to\mathbb{R}$ the bilinear pairing  map given by $\langle x^*,x\rangle=x^*(x)$. H\"older's inequality shows that $\lpsi\subset \left[\lphi\right]^*$, where the pairing
$\langle v, u\rangle$
is defined by 
\begin{equation}\label{pairing}
  \langle v,u\rangle=\int_0^Tv\cdot u\ dt,
\end{equation}
with  $u\in\lphi$ and $v\in\lpsi$.
 Unless $\Phi \in \Delta_2^{\infty}$, the relation $\lpsi= \left[\lphi\right]^*$ will not be satisfied. 
In general, it is true  that  $\left[\ephi\right]^*=\lpsi$.



We define the \emph{Sobolev-Orlicz space} $\wphi$ (see \cite{adams_sobolev}) by
\[\wphi:=\{u| u \hbox{ is absolutely continuous on $[0,T]$ and } u'\in \lphi\}.\]
$\wphi$ is a Banach space when equipped with the norm
\begin{equation}\label{def-norma-orlicz-sob}
\|  u  \|_{\wphi}= \|  u  \|_{\lphi} + \|u'\orlnor.
\end{equation}
And, we introduce the following subspaces of $\wphi$
%%
\begin{equation}\label{def-esp-orlicz-sob-per}
\begin{split}
\wphie&=\{u\in\wphi|u'\in\ephi\},\\
\wphie_T&=\{u\in\wphie|u(0)=u(T)\}.
\end{split}
\end{equation}



We will use repeatedly the decomposition $u=\overline{u}+\widetilde{u}$ for a function $u\in L^1([0,T])$  where $\overline{u} =\frac1T\int_0^T u(t)\ dt$ and $\widetilde{u}=u-\overline{u}$.

As usual, if $(X,\|\cdot\|_X)$ is a Banach space and $(Y,\|\cdot \|_Y)$ is a subspace of $X$,  we write $Y\hookrightarrow X$ and we say that $Y$ is \emph{embedded} in $X$  when the restricted identity map $i_Y:Y\to X$ is bounded. That is, there exists $C>0$ such that  for any $y\in Y$ we have $\|y\|_X\leq C\|y\|_Y$.  With this notation, H\"older's inequality states that  $\lpsi\hookrightarrow  \left[\lphi\right]^*$; and, it is easy to see that for every $N$-function $\Phi$ we have that $L^{\infty}\hookrightarrow\lphi \hookrightarrow L^1$.


 Recall that a function   $w:\mathbb{R}^+\to \mathbb{R}^+$ is called  a \emph{modulus of continuity} if $w$ is a continuous increasing function which satisfies $w(0)=0$. For example, it can be easily shown that $w(s)=s\Phi^{-1}(1/s)$ is a modulus of  continuity for every $N$-function $\Phi$.  It is said that $u:[0,T]\to\rr^d$  has modulus of continuity $w$  when there exists a constant $C>0$ such that
\begin{equation}\label{w-holder}|u(t)-u(s)|\leq Cw(|t-s|).
\end{equation}


We denote by $C^w([0,T],\rr^d)$  the space of  $w$-H\"older continuous functions that satisfy  \eqref{w-holder} for some $C>0$. 
This is a Banach space with norm
\[\|u\|_{  C^w([0,T],\rr^d) }  :=\|u\|_{L^{\infty}}+\sup\limits_{t\neq s}\frac{|u(t)-u(s)|}{w(|t-s|)}.\]




The following simple  embedding lemma, whose proof can be found in \cite{ABGMS2015}, will be used systematically.




\begin{lem}\label{inclusion orlicz} Let  $w(s):= s\Phi^{-1}(1/s)$. Then, the following statements hold:
\begin{enumerate}
\item\label{inclusion orlicz_item1} $\wphi\hookrightarrow C^w([0,T],\rr^d) $ and for every $u\in\wphi$
\begin{align}
 &\left|u(t)-u(s) \right| \leq  \|u'\orlnor w(| t-s|)&\text{  (Morrey's inequality),}\label{in-sob-cont}
\\
& \|u\|_{L^{\infty}} \leq\Phi^{-1}\left(\frac{1}{T}\right)\max\{1,T\}\|u\sobnor&\text{  (Sobolev's inequality).}\label{sobolev}
\end{align}
\item For every $u\in\wphi$ we have $\widetilde{u}\in L^{\infty}_d$ and
\begin{align}
& \|\widetilde{u}\|_{L^{\infty}} \leq T\Phi^{-1}\left(\frac{1}{T}\right)\|u'\orlnor&
\text{  (Sobolev-Wirtinger's inequality).}\label{wirtinger}
\end{align}




\end{enumerate}
\end{lem}


\section{Once upon a time...}

Vamos escribiendo lo que queremos...(de acuerdo a mis apuntes y sin ver las hojitas de la semana pasada)

For $f:[0,T]\times \rr^d\to\rr$  we denote by $\mathfrak{f}$ the Nemytskii (o superposition) operator defined for functions $u:[0,T]\to\rr^d$ by 
\[\mathfrak{f}u(t)=f(t,u(t))\]

Referencias y alguna propiedad interesante medibles en medibles? \cite{krasnosel2011integral,KR}


\begin{thm}
Let $\Phi_1,\Phi_2,\dots,\Phi_n$ be $N$-functions. 
Assume that $M$ is another $N$-functions that satisfy the $\Delta_2$-condition. 
We write $x=(x_1,\dots,x_n)$  and $y=(y_1,\dots,y_n)$ with $x_i\in \rr^d$, $y_i\in \rr^d$.
Let $f(t,x_1,\ldots,x_n,y_1,\ldots,y_n)$ be a function Chatratheodory? with $f:[0,T]\times {(\rr^d)}^n\times {(\rr^d)}^n \to \rr^{d'}$.

Suppose that $a:(\rr^d)^n\to [0,+\infty)$ is a bounded function on bounded sets and 
$b \in L^{M}([0,T])$, for a.e. $t \in [0, T]$ such that 
\begin{equation}\label{eq:condicion estru gral}
|f|\leq a(x)[ b(t)+\sum_{i=1}^{n} M^{-1}(\Phi_i(|y_i|))],
\end{equation}
then 
\[
\mathfrak{f}:\left(\prod\limits_{i=1}^n L^{\infty}([0,T],\rr^d)\right) \times \left(\prod\limits_{i=1}^n \Pi(E^{\Phi_i}([0,T],\rr^d),\lambda=1)\right)
\to \lmj.\]
\end{thm}

\begin{proof}
If $(u,v)\in  \left(\prod\limits_{i=1}^n \linf_d\right) \times \left(\prod\limits_{i=1}^n \Pi(E_d^{\Phi_i},\lambda=1)\right)$.
By \cite[ Thm. 17.6]{KR} (y otras cosas), we get 
 \[|\mathfrak{f}u(t)|=|f(t,u(t),v(t))| \leq 
M_a [b_j(t) +\sum_{i=1}^{n} M_j^{-1}(\Phi_i(|v_i(t)|))]
\in
 L_1^{M_j}.\]
\end{proof}

We define the space $X$ by
$X=\{v=(v_1,v_2):v_1 \in W^{1}L^{\Phi_1}_T,v_2\in W^{1}L^{\Phi_2}_T\}$
and 
$X^*=\{v=(v_1,v_2):v_1 \in (W^{1}L^{\Phi_1}_T)^*,v_2\in (W^{1}L^{\Phi_2}_T)^*\}$
where $(W^{1}L^{\Phi_i}_T)^*$ stands for the conjugate space of $W^{1}L^{\Phi_i}_T$ for $i=1,2$.

\begin{cor}
We will consider the Lagrange function
$\mathcal{L}:[0,T]\times\rr^d\times\rr^d\times\rr^d\times\rr^d\to\rr$, $(t,x_1,x_2,y_1,y_2)\to \mathcal{L}(t,x_1,x_2,y_1,y_2)$
which is measurable in $t$ for each $(x_1,x_2,y_1,y_2)\in \rr^d\times\rr^d\times\rr^d\times\rr^d$ and continuously differentiable 
in $(x_1,x_2,y_1,y_2)$ for almost every $t \in [0,T]$. 

Let $x=(x_1,x_2)$, $y=(y_1,y_2)$ with $x_i \in \rr^d$ and $y_i \in \rr^d$ and let 
\begin{equation}\label{integral_accion}
I(x)=\int_{0}^T \mathcal{L}(t,x,y)\ dt
\end{equation}


If there exist $a\in C(\rr^+,\rr^+)$, $i=1,2,$
$b \in L^1_1([0,T])$, $j=1,\dots,d'$ for a.e. $t \in [0, T]$ and 
every $(x_1,x_2,y_1,y_2)\in \rr^d\times\rr^d\times\rr^d\times\rr^d$ satisfying the structure conditions 
\begin{eqnarray}
|\mathcal{L}(t,x,y)|+ \sum_{i=1}^2
|D_{x_i}\mathcal{L}(t,x,y)|&\leq a(|x|)(b(t)+ \Phi_1(|y_1|)+\Phi_2(|y_2|)),\label{cotaL}\\
|D_{y_i}\mathcal{L}(t,x,y)| &\leq a(|x|)(c_i(t)+\sum_{j=1}^n \Psi^{-1}_i(\Phi_j(|y_j|))\;i=1,2.\label{cotaDyL}
\end{eqnarray}


The nonlinear operator $(x_1,x_2)\mapsto D_x\mathcal{L}(t,x_1,y_1, y_2)$ is continuous from $\domiuno\times\domidos\times\dots\times\domin$ with the strong topology  into $L^1([0,T])$  with the strong topology on both sets.

The nonlinear operator $(x_1,x_2)\mapsto D_y\mathcal{L}(t,x_1,y_1, y_2)$ is continuous from $\domiuno\times\domidos\times\dots\times\domin$ with the strong topology  into $X$  with the weak$^*$ topology.


The function  $I$ is G\^ateaux differentiable on $\domiuno\times \domidos$ and  its derivative $I'$ is demicontinuous from $\domiuno\times\domidos$  into $X^*$. Moreover, $I'$ is given by the following expression
\begin{equation}\label{DerAccion}
\begin{split}
\left\langle I'(x),w \right\rangle=\int_0^T 
[
(D_{x_1}\mathcal{L}(t,x_1(t),x_2(t),y_1(t),y_2(t)),w_1(t))+
\\
(D_{x_2}\mathcal{L}(t,x_1(t),x_2(t),y_1(t),y_2(t)),w_2(t))+
\\
(D_{y_1}\mathcal{L}(t,x_1(t),x_2(t),y_1(t),y_2(t)),w'_1(t))+
\\
(D_{y_2}\mathcal{L}(t,x_1(t),x_2(t),y_1(t),y_2(t)),w'_2(t))
]\,dt
\end{split}
\end{equation}

If  $\Psi \in \Delta_2$ then 
  $I'$ is continuous from $\domiuno\times\domidos$ into $X^*$ when both spaces are equipped with the strong topology.
\end{cor}


We denote by $\mathfrak{A}(a,b,c,\lambda,f,\Phi)$ the set of all Lagrange functions satisfying  \eqref{eq:estru1}, \eqref{eq:estru2} and \eqref{eq:estru3}.


\begin{proof} 

{\bf OJO!!!! Es algo que ten\'iamos del trabajo anterior!!! 
con algunas adaptaciones a 2 variables sin controlar y a lo bruto!!!!!}

Let $\b{u}\in \domiuno\times\domidos$.

\noindent\emph{Step 1. The non linear operator  $(x_1,x_2) \mapsto (D_{x_1}\mathcal{L}(t,x_1,x_2,y_1,y_2),
D_{x_1}\mathcal{L}(t,x_1,x_2,y_1,y_2))$ is continuous from $\domiuno\times\domidos$ into $L^{1}_d([0,T])\times L^1_d([0,T])$ with the strong topology on both sets.} 


If $u\in \domiuno\times\domidos$, from \eqref{cotaDxL} and \eqref{inclusion3}, we obtain 
Let   $\{x_n=({x_1}_n,{x_2}_n)\}_{n\in \mathbb{N}}$ be a sequence of  functions in $\domiuno\times\domidos$  and let 
$x=(x_1,x_2)\in \domiuno\times\domidos$  such that $x_n\rightarrow x$ in $X$.
From  ${x_i}_n\rightarrow x_i$ in $\lphii$, there exists a subsequence ${x_i}_{n_k}$ such that ${x_i}_{n_k}\rightarrow x_i \quad\text{a.e.}$; and, as ${x_i}_n\rightarrow x_i \in\domi$, by 
  Lemma \ref{segundo lema}, there exist a subsequence of  ${x_i}_{n_k}$ (again denoted ${x_i}_{n_k}$) and a function  $h_i\in \Pi(\ephi_1,\lambda))$
such that  ${x_i}_{n_k}\rightarrow u_i \quad\text{a.e.}$ and $|{x_i}_{n_k}|\leq h_i\quad\text{a.e}$.  
Since ${x_i}_{n_k}$, $k=1,2,\ldots,$ is a strong convergent sequence in $\wphii_d$, it is a bounded sequence in $\wphii_d$. According to Lemma \ref{inclusion orlicz} and Corollary \ref{a_bound}, there exist $M_i>0$ such that $\|\b{a}({x_i}_{n_k})\|_{L^{\infty}} \leq M_i$, $k=1,2,\ldots$.  From the previous facts and \eqref{DxL1}, we get
\begin{equation*}\label{DxL1-bis}
|D_{x_i}\mathcal{L}(\cdot,{x_1}_{n_k},{x_2}_{n_k},{y_1}_{n_k},{y_2}_{n_k})|\leq 
M_i (b+\Phi_i(|h_i|)) \in L^1_1\;\;i=1,2.
\end{equation*}
On the other hand, by the continuous differentiability of $\mathcal{L}$, we have
\[D_{x_i}\mathcal{L}(t,{x_i}_{n_k}(t),{y_i}_{n_k}(t))\to D_{x_i}\mathcal{L}(t,x_i(t), y_i(t))\quad\hbox{ for a.e. } t\in[0,T].\]
Applying the Dominated Convergence Theorem we conclude the proof of step 1.

\noindent\emph{Step 2. The non linear operator   
$(x_1,x_2)\mapsto (D_{y_1}\mathcal{L}(t,x_1,y_1, D_{y_2}\mathcal{L}(t,x_2,y_2)$ is continuous from $\domiuno\times\domidos$ with the strong topology  into $X$  with the weak$^*$ topology.}


 Note that \eqref{DxL1},  \eqref{DyLpsi} and the imbeddings $\wphi_d \hookrightarrow L_d^{\infty}$ and  $\lpsi_d\hookrightarrow  \left[\lphi_d\right]^*$ imply that the second member of
\eqref{DerAccion} defines an element in $\left[\wphi_d\right]^*$.

Let $({x_1}_n,{x_2}_n)\in \domi$ such that $({x_1}_n,{x_2}_n)\to (x_1,x_2)$ in the norm of $X$. 
We must prove that  $D_{y_i}\mathcal{L}(\cdot,{x_1}_n,{x_2}_n)\overset{w^*}{\rightharpoonup} 
D_{y_i}\mathcal{L}(\cdot,x_1,x_2,y_1,y_2)$ para $i=1,2$.
On the contrary, there exist $v=(v_1,v_2)\in\lphiuno\times\lphidos$, $\epsilon>0$ and a subsequence of $\{x_n\}$ (denoted  $\{x_n\}$ for simplicity)  such that
\begin{equation}\label{cota_eps}
 \left| \langle D_{y_i}\mathcal{L}(\cdot, {x_1}_n,{x_2}_n,{y_1}_n,{y_2}_n),\
v \rangle - \langle  D_{y_i}\mathcal{L}(\cdot,x_1,x_2, y_1,y_2,v \rangle\right|\geq \epsilon.
\end{equation}
We have $x_n \rightarrow x$ in $X$ and
$y_n\rightarrow y$ in $X$. By Lemma \ref{segundo lema}, 
there exist a subsequence $x_{n_k}$ and a function $h\in \Pi(\ephiuno_1,\lambda)\times\Pi(\ephidos_1,\lambda) $ such that $x_{n_k}\rightarrow x \quad\text{a.e.}$, $y_{n_k}\rightarrow y \quad\text{a.e.}$ and $|y_{n_k}|\leq h\quad\text{a.e.}$ 
As in the previous step, since $x_n$ is a convergent sequence, the Corollary \ref{a_bound} implies that $a(|y_n(t)|)$ is uniformly bounded by a certain constant $M>0$. 
Therefore,  with $x_{n_k}$ instead of $x$, inequality  \eqref{DyLpsi} becomes 
\begin{equation}\label{Dy-suc}
  \left | D_{y_i}\mathcal{L}(\cdot,x_{n_k},y_{n_k})  \right| 
	\leq M_i(c_i+\varphi_i(h_i)+\Psi_i^{-1}(\Phi_j(|y_j|)))\in \lpsii_1.
\end{equation}
Consequently, as $v \in \lphi_d$ and employing H\"older's inequality, we obtain that
\[\sup_k|D_{\b{y}}\mathcal{L}(\cdot,\b{u}_{n_k},\b{\dot{u}}_{n_k})\ccdot v| \in L^1_1.\]
  Finally, from the Lebesgue Dominated Convergence Theorem, we deduce
\begin{equation}\label{conv_debil}\int_0^T  D_{\b{y}}\mathcal{L}(t,\b{u}_{n_k},\b{\dot{u}}_{n_k})\ccdot\b{ v} \ dt \to \int_0^T D_{\b{y}}\mathcal{L}(t,\b{u},\b{\dot{u}})\ccdot\b{ v}\ dt \end{equation}
which contradicts the inequality \eqref{cota_eps}. This completes the proof of step 2.

\emph{Step 3.} We will prove \eqref{DerAccion}. The proof follows similar lines as \cite[Thm. 1.4]{mawhin2010critical}. For $\b{u}\in \domi$ and $\b{0}\neq\b{v}\in\wphi_d$, we define the function
\[H(s,t):=\mathcal{L}(t,\b{u}(t)+s\b{v}(t),\b{\dot{u}}(t)+s\b{\dot{v}}(t)).\]

From \cite[Lemma 10.1]{KR} (or \cite[Thm. 5.5]{Orliczvectorial2005} ) we obtain that if $|\b{u}|\leq |\b{v}|$ then    $d(\b{u},\ephi_d)\leq d(\b{v},\ephi_d)$. 
Therefore, for  $|s|\leq s_0:=\left(\lambda-d(\b{\dot{u}},\ephi_d)\right)/\|\b{v}\sobnor$ we have
\[
d \left(\b{\dot{u}}+s\b{\dot{v}}, \ephi_d \right)
\leq
d \left(|\b{\dot{u}}|+s|\b{\dot{v}}|, \ephi_1 \right)
\leq d \left(|\b{\dot{u}}|,\ephi_1 \right)+ s \|\b{\dot{v}}\orlnor < \lambda.
\]
Thus $\b{\dot{u}}+s\b{\dot{v}} \in \Pi(\ephi_d,\lambda)$ and  $|\b{\dot{u}}|+s|\b{\dot{v}}| \in \Pi(\ephi_1,\lambda)$. These facts imply, in virtue of Theorem \ref{teorema_acotacion} item \ref{T1item1}, that $I(\b{u}+s\b{v})$ is well defined and finite for $|s|\leq s_0$. 
And, using  Corollary \ref{a_bound}, we also see that
\[ \|a(|\b{u}+s\b{v}|)\|_{L^{\infty}}\leq  A(\|\b{u}+s\b{v}\sobnor)\leq
 A(\|\b{u}\sobnor+s_0\|\b{v}\sobnor)=:M
\]
Now, applying Chain Rule, \eqref{DxL1}, \eqref{DyLpsi} the monotonicity of $\varphi$ and $\Phi$, 
the fact that $\b{v}\in L^{\infty}_d$ and $\b{\dot{v}}\in\lphi_d$ and H\"older's inequality, we get
\begin{equation}\label{ctg}
\begin{split}
|D_s H(s,t)|&=\left| D_{\b{x}}\mathcal{L}(t,\b{u}+s\b{v},\b{\dot{u}}+s\b{\dot{v}})\ccdot \b{v} +  D_{\b{y}}\mathcal{L}(t,\b{u}+s\b{v},\b{\dot{u}}+s\b{\dot{v}})\ccdot\b{\dot{v}}\right| \\
 & \leq M \left[\left( b(t)+ \Phi\left(\frac{|\b{\dot{u}}|+s_0|\b{\dot{v}}|}{\lambda}+f(t)\right)\right)|\b{v}|\right.\\
&\left. \quad+ \left(c(t)+ \varphi\left (\frac{|\b{\dot{u}}|+s_0|\b{\dot{v}}|}{\lambda}+f(t)\right)\right)|\b{\dot{v}}| \right]\in L^1_1.
\end{split}
\end{equation}
Consequently, $I$ has a directional derivative and
\[
\langle I'(\b{u}),\b{v} \rangle=\frac{d}{ds}I(\b{u}+s\b{v})\big|_{s=0}=\int_0^T  
\left\{D_{\b{x}}\mathcal{L}(t,\b{u},\b{\dot{u}})\ccdot \b{v}+ D_{\b{y}}\mathcal{L}(t,\b{u},\b{\dot{u}})\ccdot\b{\dot{v}}\right\} \ dt.
\]
Moreover, from \eqref{DxL1}, \eqref{DyLpsi}, Lemma \ref{inclusion orlicz} and the previous formula, we obtain
\[
|\langle I'(\b{u}),\b{v} \rangle| \leq \|D_{\b{x}}\mathcal{L}\|_{L^1} \| \b{v}\linf + 
\|D_{\b{y}}\mathcal{L}\|_{\lpsi} \|\b{\dot{v}}\orlnor \leq C \|\b{v}\sobnor
\]
with a appropriate constant $C$.
This completes the proof of the G\^ateaux differentiability of $I$. 

\emph{Step 4. The operator $I':\domi  \to \left[\wphi_d
\right]^* $ is demicontinuous.}
This is a consequence  of the continuity of the mappings $\b{u} \mapsto D_{\b{x}}\mathcal{L}(t,\b{u},\b{\dot{u}})$ and $\b{u} \mapsto
D_{\b{y}}\mathcal{L}(t,\b{u},\b{\dot{u}})$. Indeed, if $\b{u}_n,\b{u}\in \domi$ with $\b{u}_n\to \b{u}$ in the norm of $\wphi_d$ and $\b{v} \in
\wphi_d$, then
\[
\begin{split}
\left\langle  I'(\b{u}_{n}),\b{v} \right\rangle &= \int_0^T \left\{  D_{\b{x}}\mathcal{L}\left(t,\b{u}_n,\b{\dot{u}}_n\right)\ccdot
\b{v} +
 D_{\b{y}}\mathcal{L}\left(t,\b{u}_n,\b{\dot{u}}_n\right)\ccdot\b{\dot{v}}\right\} \ dt\\
&\rightarrow \int_0^T \left\{ D_{\b{x}}\mathcal{L}\left(t,\b{u},\b{\dot{u}}\right)\ccdot \b{v}+ 
D_{\b{y}}\mathcal{L}\left(t,\b{u},\b{\dot{u}}\right)\ccdot\b{\dot{v}}\right\} \ dt\\
&=\left\langle  I'(\b{u}),\b{v} \right\rangle.
\end{split}
\]


In order to prove item  \ref{T1item4}, it is necessary to see that the maps $\b{u}\mapsto D_{\b{x}}\mathcal{L}(t,\b{u},\b{\dot{u}})$  and $\b{u}\mapsto D_{\b{y}}\mathcal{L}(t,\b{u},\b{\dot{u}})$  are norm continuous
from $\domi $ into $L^1_d$ and
 $\lpsi_d$ respectively.  The continuity of the first map has already been proved in step 1. 
Let $\b{u}_n, \b{u} \in \domi$ with $\|\b{u}_n- \b{u}\sobnor\to 0$. Therefore,   there exist a subsequence $\b{u}_{n_k}\in \domi$ and a function $h\in\Pi(\ephi_1,\lambda)$  such that   \eqref{Dy-suc} holds true. 
And, as  $\Psi\in\Delta_2$ then   the right hand side of  \eqref{Dy-suc} belongs to $\epsi_1$. 
Now, invoking  Lemma \ref{lema_conv_may}, we  prove that
  from any sequence $\b{u}_n$ which converges to $\b{u}$ in $\wphi_d$ we can
extract a subsequence such that   $D_{\b{y}}\mathcal{L}(t,\b{u}_{n_k},\b{\dot{u}}_{n_k})\to D_{\b{y}}\mathcal{L}(t,\b{u},\b{\dot{u}})$ in the strong topology. The desired result is obtained by a standard argument.

The continuity of $I'$  follows  from the continuity 
of $D_{\b{x}}\mathcal{L}$ and $D_{\b{y}}\mathcal{L}$ using the formula \eqref{DerAccion}.
\end{proof}





\section*{Acknowledgments}
The authors are partially supported by a UNRC grant number 18/C417. The first author is  partially supported by a  UNSL grant number 22/F223. 




 \bibliographystyle{elsarticle-num} 
 \bibliography{biblio}


\end{document}
