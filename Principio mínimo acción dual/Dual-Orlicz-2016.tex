\documentclass[twoside]{article}


%\usepackage{hyperref}
\usepackage{amssymb,amsthm}
\usepackage{amsmath}
\usepackage{color}
\usepackage{ esint }
\usepackage{mathabx}
\usepackage{MnSymbol}
\usepackage{fancyhdr}
\usepackage{times}
\usepackage{enumitem}
\usepackage[latin1]{inputenc}

\usepackage{comment}
\usepackage{url}
\usepackage{xcolor}
\usepackage{adjustbox}
\usepackage{hyperref}


\newtheorem{thm}{Theorem}[section]
\newtheorem{cor}[thm]{Corollary}
\newtheorem{lem}[thm]{Lemma}

\newtheorem{defi}[thm]{Definition}
\newtheorem{prop}[thm]{Proposition}
\theoremstyle{remark}
\newtheorem{comentario}{Remark}


\makeatletter
\newcommand{\labitem}[2]{%
\def\@itemlabel{\textbf{#1}}
\item
\def\@currentlabel{#1}\label{#2}}
\makeatother
\makeatletter
\def\namedlabel#1#2{\begingroup
    #2%
    \def\@currentlabel{#2}%
    \phantomsection\label{#1}\endgroup
}
\makeatother



\title{Periodic solutions of
Euler-Lagrange equations in an Orlicz-Sobolev space setting by the dual least action principle }
\author{Sonia Acinas \thanks{SECyT-UNRC and  FCEyN-UNLPam}\\
Dpto. de Matem\'atica, Facultad de Ciencias Exactas y Naturales\\
Universidad Nacional de La Pampa\\
(L6300CLB) Santa Rosa, La Pampa, Argentina\\
\url{sonia.acinas@gmail.com}\\[3mm]
Fernando D. Mazzone \thanks{SECyT-UNRC, FCEyN-UNLPam and CONICET}\\
Dpto. de Matem\'atica, Facultad de Ciencias Exactas, F\'{\i}sico-Qu\'{\i}micas y Naturales\\
Universidad Nacional de R\'{i}o Cuarto\\
(5800) R\'{\i}o Cuarto, C\'ordoba, Argentina,\\
\url{fmazzone@exa.unrc.edu.ar}
}

\date{}

\newcommand{\orlnor}{\|_{L^{\Phi}}}
\newcommand{\lurnor}{\|^{*}_{L^{\Phi}}}
\newcommand{\linf}{\|_{L^{\infty}}}
\newcommand{\lphi}{L^{\Phi}}
\newcommand{\lphiuno}{L^{\Phi_1}}
\newcommand{\lphidos}{L^{\Phi_2}}
\newcommand{\lphii}{L^{\Phi_i}}
\newcommand{\lpsi}{L^{\Psi}}
\newcommand{\lpsiuno}{L^{\Phie_1}}
\newcommand{\lpsidos}{L^{\Phie_2}}
\newcommand{\lpsii}{L^{\Phie_i}}
\newcommand{\lmuno}{L^{M_1}}
\newcommand{\lmdos}{L^{M_2}}
\newcommand{\lmj}{L^{M}}
\newcommand{\lmn}{L^{M_n}}
\newcommand{\ephi}{E^{\Phi}}
\newcommand{\ephiuno}{E^{\Phi_1}}
\newcommand{\ephidos}{E^{\Phi_2}}
\newcommand{\ephin}{E^{\Phi_n}}
\newcommand{\ephii}{E^{\Phi_i}}
\newcommand{\claseor}{C^{\Phi}}
\newcommand{\wphi}{W^{1}\lphi}
\newcommand{\wphiuno}{W^{1}\lphiuno}
\newcommand{\wphidos}{W^{1}\lphidos}
\newcommand{\wphii}{W^{1}\lphii}
\newcommand{\wphiet}{W^{1}\ephi_T}
\newcommand{\wphie}{W^{1}\ephi}
\newcommand{\sobnor}{\|_{W^{1}\lphi}}
\newcommand{\domi}{\mathcal{E}^{\Phi}}
\newcommand{\domiuno}{\mathcal{E}^{\Phi_1}_d(\lambda)}
\newcommand{\domidos}{\mathcal{E}^{\Phi_2}_d(\lambda)}
\newcommand{\domii}{\mathcal{E}^{\Phi_i}_d(\lambda)}
\newcommand{\domin}{\mathcal{E}^{\Phi_n}_d(\lambda)}
\renewcommand{\b}[1]{\boldsymbol{#1}}
\newcommand{\rr}{\mathbb{R}}
\newcommand{\nn}{\mathbb{N}}
\newcommand{\ccdot}{\b{\cdot}}
\renewcommand{\leq}{\leqslant} 
\renewcommand{\geq}{\geqslant} 
\newcommand{\epsi}{E^{\Phie}}
\newcommand{\Phie}{\Phi^{*}}

\newcounter{example}

\setcounter{example}{1}


\newenvironment{example}{\noindent\textbf{Example \arabic{example}}.}{\addtocounter{example}{1}}




\begin{document}


\maketitle
%
\begingroup%Locallizing the change to `thefootnote'.
    \renewcommand{\thefootnote}{}%Removing the footnote symbol.
    %
    \footnotetext{%
    %   2010 Mathematics Subject Classification
    %   http://www.ams.org/msc/
    \textbf{2010  AMS Subject Classification.} Primary: .
    Secondary: .
    }%
        \footnotetext{%
    \textbf{Keywords and phrases.}  .
    }%
    \endgroup
%
%
%
%

\begin{abstract}


\end{abstract}






\pagestyle{fancy} \headheight 35pt \fancyhead{} \fancyfoot{}

\fancyfoot[C]{\thepage} \fancyhead[CE]{\nouppercase{S. Acinas and F.D. Mazzone }} \fancyhead[CO]{\nouppercase{\section}}

\fancyhead[CO]{\nouppercase{\leftmark}}


%\tableofcontents




\section{Introduction}


This paper deals with system  of equations of the type:

\begin{equation}\label{ProbPrin-gral}
    \left\{%
\begin{array}{ll}
  \frac{d}{dt} D_{y}\mathcal{L}(t,u(t),u'(t))= D_{x}\mathcal{L}(t,u(t),u'(t)) \quad \hbox{a.e.}\ t \in (0,T)\\
    u(0)-u(T)=u'(0)-u'(T)=0,
\end{array}%
\right. ,\tag{$P_1$}
\end{equation}
where $\mathcal{L}:[0,T]\times\rr^d\times\rr^d\to\rr$, $d\geq 1$, is called the \emph{Lagrange function} or \emph{lagrangian} and the unknown function  $u:[0,T]\to\rr^d$ is absolutely continuous. In other words, we are interested in  finding \emph{periodic weak solutions} of \emph{Euler-Lagrange systems of ordinary equations}.

This topic was deeply addressed for the several types of \emph{Lagrange functions}.  For example,
\begin{equation}\label{eq:lagrange_cuad}
\mathcal{L}_{p,F}(t,x,y):=\frac{|y|^p}{p}+F(t,x),
\end{equation}
for $1<p<\infty$. For example, the classic book  \cite{mawhin2010critical} deals mainly with problem \eqref{ProbPrin-gral}, for the lagrangian $\mathcal{L}_{2,F}$, through various methods: direct, dual action, minimax, etc. The results in \cite{mawhin2010critical} were extended and improved in several articles,  see  \cite{tang1995periodic,tang1998periodic,wu1999periodic,tang2001periodic,zhao2004periodic}  to cite some examples. Lagrange functions \eqref{eq:lagrange_cuad} for arbitrary $1<p<\infty$ were considered in  \cite{Tian2007192,tang2010periodic} and in this case \eqref{ProbPrin-gral}  is reduced to the $p$-laplacian system
\begin{equation}\label{ProbP-lapla}
    \left\{%
\begin{array}{ll}
   \frac{d}{dt}\left(u'(t)|u'|^{p-2}\right) = \nabla F(t,u(t)) \quad \hbox{a.e.}\ t \in (0,T)\\
    u(0)-u(T)=u'(0)-u'(T)=0.
\end{array}%
\right.\tag{$P_2$}
\end{equation}


In this context, it  is customary to call $F$ a  \emph{potential function}, and it is assumed that $F(t,x)$ is differentiable with respect to $x$ for a.e. $t\in [0,T]$ and the following conditions are verified:
\begin{enumerate}
\labitem{(C)}{item:condicion_c} $F$ and its gradient $\nabla F$, with respect to $x\in\rr^d$,  are  Carath\'eodory functions, i.e. they are measurable functions with respect to $t\in [0,T]$, for every  $x\in\rr^d$, and they are continuous functions with  respect to  $x\in\rr^d$ for a.e. $t \in [0,T]$.
 \labitem{(A)}{item:condicion_a}  For   a.e. $t\in [0,T]$, it holds that
\begin{equation}
|F(t,x)| + |\nabla F(t,x)|  \leq a(|x|)b(t).
\end{equation}
In this inequality we assume that the function  $a:[0,+\infty)\to [0,+\infty)$ is continuous and non decreasing and $0\leq b\in L^1([0,T],\rr)$.
\end{enumerate}


In the framework of anisotropic Sobolev-Orlicz spaces, we can study system of $p$-laplacian equations as the following example shows.


\begin{example}\label{ex:phip1p2} Let $1<p_1,p_2<\infty$. We define $\Phi_{p_1,p_2}:\rr^n\times \rr^n\to\rr_+$   by
\[\Phi_{p_1,p_2}(y_1,y_2):=\frac{|y_1|^{p_1}}{p_1}+\frac{|y_2|^{p_2}}{p_2}.\]
where $|\cdot|$ is the Euclidean norm on $\rr^n$.
And, we consider the following Lagrange function
\[\mathcal{L}(t,x,y)=\Phi_{p_1,p_2}(y)+F(t,x).\]
where $y=(y_1,y_2), x=(x_1,x_2)\in \rr^{2n}$...o algo as\'i?????

Then the equations \eqref{ProbPrin-gral} become
\begin{equation}\label{eq:sist-p_lapa}
    \left\{%
\begin{array}{ll}
  \frac{d}{dt}\left(|u_1'|^{p_1-2}u_1'\right)=F_{x_1}(t,u) \quad \hbox{a.e.}\ t \in (0,T)\\
  \frac{d}{dt}\left(|u_2'|^{p_2-2}u_2'\right)=F_{x_2}(t,u) \quad \hbox{a.e.}\ t \in (0,T)\\
   u(0)-u(T)=u'(0)-u'(T)=0,
\end{array}%
\right., \tag{$P_3$}
\end{equation}



\end{example}


In the literature, these equations are known as $(p_1,p_2)$-Laplacian system, see
\cite{yang2013existence,pasca2016periodic,yang2012periodic,pasca2010periodic,pacsca2010some,pasca2011some}.



In \cite{ABGMS2015} it is treated  the case of a lagrangian $\mathcal{L}$ which is lower bounded by a Lagrange function like
\begin{equation}\label{eq:lagrange_phi}
\mathcal{L}_{\Phi,F}(t,x,y):=\Phi(|y|)+F(t,x),
\end{equation}
where  $\Phi$ is an $N$-function (see section \ref{preliminares} for the definition of this concept).


\section{Anisotropic Orlicz and Orlicz-Sobolev spaces}\label{preliminares}

In this section, we give a short introduction to  Orlicz and Orlicz-Sobolev spaces of vector valued functions associated to anisotropic Young functions $\Phi:\rr^n\to\rr_+$, i.e. functions such that $\Phi(x)$ depends on the direction of $x$, unlike the radial case where $\Phi(x)=\Phi(|x|)$.  References for  these topics are \cite{Orliczvectorial2005,Skaff1969, Desch2001}.



On the other hand, anisotropic Orlicz-Sobolev spaces allow us to simplify the writing, and they provide the natural frame for statements of the type \cite[Lemma 3.1]{Tian2007192}. This type of question motivated us to use these spaces.

Hereafter we denote  by $\mathbb{R}^+$  the set of all non negative real numbers. A function $\Phi:\mathbb{R}^d\to \mathbb{R}_+ $ is called an \emph{Young's function} if $\Phi$ is convex, $\Phi(0)=0$, $\Phi(-x)=\Phi(x)$ and $\Phi(x)\to +\infty$, when $|x|\to+\infty$. Additionally, we assume that  Young's functions which we deal with, satisfy that $\Phi(x)>0$ when $x\neq 0$. Following \cite{Orliczvectorial2005} we say that $\Phi$ is an $N_{\infty}$-function  if
\[\lim_{|x|\to\infty}\frac{\Phi(x)}{|x|}=+\infty.\]

Given a Young's function $\Phi$, we define function $A_{\Phi}:\rr^+\to\rr^+$ by
\begin{equation}\label{eq:inversa-gral}
A_{\Phi}(s)=\min\left\{\Phi(x)\,\big|\,|x|=s\right\},
\end{equation}

Let us establish some elementary properties of $A_{\Phi}$ that we will use in this article.
\begin{prop}\label{prop:AsubPhi} The function $A_{\Phi}$ has the following properties:
\begin{enumerate}
 \item\label{it:prop1} $A_{\Phi}$ is continuous,
 \item\label{it:prop2} $A_{\Phi}(s)/s$ is increasing,
 \item\label{it:prop3} $A_{\Phi}(|x|)$ is the \emph{greatest radial minorant} of 
 $\Phi(x)$,
 \item\label{it:prop4} $\Phi$ is $N_{\infty}$ if and only if $A_{\Phi}$ is.
\end{enumerate}
\end{prop}

\begin{proof} It is well known that finite and convex functions defined on finite dimensional 
vector spaces are locally Lipschitz functions (see \cite{clarke2013functional}). This fact 
implies item \ref{it:prop1} immediately. 

In order to prove item \ref{it:prop2}, suppose $0<r<s$ and $x\in\rr^d$ with $A_{\Phi}(s)
=\Phi(x)$. Then, from the definition of $A_{\Phi}$ and the convexity of $\Phi$,
\[\frac{A_{\Phi}(r)}{r}\leq \frac{\Phi\left(\frac{r}{s}x\right)}{r}\leq \frac{\Phi\left(x\right)}{s}=
 \frac{A_{\Phi}(s)}{s}.
\]
Property in items \ref{it:prop3} and \ref{it:prop4} are obtained easily.

 
\end{proof}

\begin{example} We compute $A_{\Phi}$ for the function $\Phi=\Phi_{p_1,p_2}$ given in Example \eqref{ex:phip1p2}.

We apply the method of Lagrange multipliers to solve the  problem
\[
G(r)=\min\left\{\Phi(x,y):|(x,y)|_2^2=r^2\right\}\]

The first order equations are
\[
\left
\{
\begin{array}{ccc}
|x|^{p_1-2}x+ \lambda x&=&0
\\
|y|^{p_2-2}y+\lambda y&=&0
\\
|x|^2+|y|^2&=&r^2
\end{array}
\right.
\]
These equations are solved, among others, by the following sets (if $n>1$ infinite) of citical values: 
a) $|x|=r$, $y=0$ and $\lambda=-r^{p_1-2}$ and b) $x=0$, $|y|=r$ and $\lambda=-r^{p_2-2}$. 
Associated with these critical points we have the following critical values: a) $r^{p_1}/p_1$ and b) $r^{p_2}/p_2$.

Now, suppose that  $x\neq 0$ and $y \neq 0$ then $|x|^2+|y|^2=r^2$ and $|y|=|x|^{\frac{p_1-2}{p_2-2}}$and $\lambda=-|x|^{p_1-2}$.

We have to split the analysis in several cases. 

Now, we consider $p_1\leq 2$ and $p_2\leq 2$ with of them different to 2.

There exists $(z,w)$ such that $zx^t+wy^t=0$ (z=-y, w=x) where 
$H=|\lambda||y|^2|x|^2[(p_1-2)|x|^{-2}+(p_2-2)|y|^{-2}]<0$
 
{\bf (aclarar algo de H, poner un nombre adecuado y cambiar el formato de letra)}

Then, by the second order criteria \cite[Thm....]{Ye}, 
at $(x,y)$ there cannot be a minimum. Therefore, the minima occur at $x=0$ or $y=0$.

The remaining cases can be treated with similar techniques. 

Finally, we conclude that 
\[
K_1\min\{r^{p_1}, r^{p_2}\}\leq A_{\Phi}\leq K_2\max\{r^{p_1}, r^{p_2}\}
\]
with $K_1,K_2>0$.
\end{example}





We also say that $\Phi:\mathbb{R}^d\rightarrow \mathbb{R}^+$ satisfies the  \emph{$\Delta_2^{\infty}$-condition}, denoted by $\Phi \in \Delta_2^{\infty}$,
if there exist  constants $K>0$ and  $M\geq 0$ such that
\begin{equation}\label{delta2defi}\Phi(2x)\leq KH(x),
\end{equation}
for every $|x|\geq M$.

If $\Phi$ is a Young's function we define its \emph{Fenchel conjugate}   $\Phi^*:\mathbb{R}^d\to \mathbb{R}^+ $ by:
\begin{equation}\label{eq:conjugada}
 \Phi^*(y)=\sup\limits_{x\in\mathbb{R}^d} x\cdot y-\Phi(x)
\end{equation}


 We denote by $\mathcal{M}:=\mathcal{M}([0,T],\rr^d)$, with $d\geq 1$,  the set of all measurable functions (i.e. functions which are limits of simple functions)  defined on $[0,T]$ with values on $\mathbb{R}^d$ and  we write $u=(u_1,\dots,u_d)$ for  $u\in \mathcal{M}$. For the set of functions $\mathcal{M}$, as for other similar sets, we will omit the reference to codomain $\mathbb{R}^d$ when $d=1$.


Given  an $N$-function $\Phi$ we define the \emph{modular function} 
$\rho_{\Phi}:\mathcal{M}\to \mathbb{R}^+\cup\{+\infty\}$ by
\[\rho_{\Phi}(u):= \int_0^T \Phi(u)\ dt.\]
Here $|\cdot|$ is the euclidean norm of $\mathbb{R}^d$.
Now, we introduce the \emph{Orlicz class} $C^{\Phi}=C^{\Phi}([0,T],\rr^d)$   by setting
\begin{equation}\label{claseOrlicz}
  C^{\Phi}:=\left\{u\in \mathcal{M} | \rho_{\Phi}(u)< \infty \right\}.
\end{equation}
The \emph{Orlicz space} $\lphi=L^{\Phi}([0,T],\rr^d)$ is the linear hull of $\claseor$;
equivalently,
\begin{equation}\label{espacioOrlicz}
\lphi:=\left\{ u\in \mathcal{M}| \exists \lambda>0: \rho_{\Phi}(\lambda u) < \infty   \right\}.
\end{equation}
  The Orlicz space $\lphi$ equipped with the \emph{Luxemburg norm}
\[
\|  u  \orlnor:=\inf \left\{ \lambda\bigg| \rho_{\Phi}\left(\frac{v}{\lambda}\right) dt\leq 1\right\},
\]
is a Banach space. By $u\b{\cdot} v$ we denote the usual dot product in $\mathbb{R}^{d}$ between $u$ and $v$.


The subspace $\ephi=\ephi([0,T],\rr^d)$ is defined as the closure in $\lphi$ of the subspace $L^{\infty}([0,T],\rr^d)$ of all $\mathbb{R}^d$-valued essentially bounded functions. It is shown that  (see \cite[Thm. 5.1]{Orliczvectorial2005}) $u\in\ephi$  if and only if $\rho_{\Phi}(\lambda u)<\infty$ for any $\lambda>0$. The equality $\lphi=\ephi$ is true if and only if $\Phi\in\Delta_2^{\infty}$ (see \cite[Thm. 5.2]{Orliczvectorial2005}). Another alternative characterization of $\ephi$, which is particularly useful for us, is that $u\in\ephi$ if and only if $u$ has  \emph{absolutely continuous norm}, i.e.   if $E_n\subset [0,T]$, $n=1,2,\ldots$ then $\|\chi_{E_n}u\|\to 0$ when $|E_n|\to 0$.

A generalized version of \emph{H\"older's inequality} holds in Orlicz spaces (see \cite[Thm. 4.1]{Skaff1969}). Namely, if $u\in\lphi$ and $v\in\lpsi$ then $u\cdot v\in L^1$ and
\begin{equation}\label{holder}
\int_0^Tv\cdot u\ dt\leq 2 \|u\orlnor\|v\|_{L^{\Phie}}.
\end{equation}


Like in \cite{KR} we will consider the subset $\Pi(\ephi,r)$ of $\lphi$ given by
\[\Pi(\ephi,r):=\{u\in\lphi| d(u,\ephi)<r\}.\]
This set is related to the Orlicz class $\claseor$ by means of inclusions, namely,
\begin{equation}\label{eq:inclusiones}\Pi(\ephi, r )\subset r \claseor\subset\overline{\Pi(\ephi,r)}
\end{equation}
for any positive $r$ (see \cite[Thm. 5.6]{Orliczvectorial2005}).
If $\Phi \in \Delta_2^{\infty}$,  then the sets $\lphi$, $\ephi$, $\Pi(\ephi,r)$ and $\claseor$ are equal.

Following to \cite{Desch2001} we introduce the next definition.

\begin{defi} Let $u_n,u\in\lphi([0,T],\rr^d)$. We say that $u_n$ converges monotonically to $u$ if there exists $\alpha_n\in L^{\infty}([0,T],\rr)$, $n=1,2,\ldots$, such that $0\leq \alpha_n(t)\leq \alpha_{n+1}(t)$, $\alpha_n(t)\to 1$ a.e., when $n\to\infty$ and $u_n(t)=\alpha_n(t)u(t)$.

\end{defi}

 
As usual, if $(X,\|\cdot\|_X)$ is a normed space and $(Y,\|\cdot \|_Y)$ is a linear subspace of $X$,  we write $Y\hookrightarrow X$ and we say that $Y$ is \emph{embedded} in $X$  when there exists $C>0$ such that
$\|y\|_X\leq C\|y\|_Y$ for any $y\in Y$.  With this notation, H\"older's inequality states that  $\lpsi\hookrightarrow  \left[\lphi\right]^*$, where a function $v\in\lpsi$ is associated  to $\xi_v\in \left[\lphi\right]^*$ being
\begin{equation}\label{pairing}
  \xi_v(u)=\langle \xi_v,u\rangle=\int_0^Tv\cdot u\ dt,
\end{equation}
 In  \cite[Thm 2.9]{Desch2001}  it was characterized a subspace of   $\left[\lphi\right]^*$ which can be identified with $\lpsi$.

 \begin{prop} Let $F\in\left[\lphi([0,T],\rr^d)\right]^*$. Then the following statements are equivalent
 \begin{enumerate}
  \item $\xi\in \lpsi([0,T],\rr^d)$
  \item $\xi$ satisfies the \emph{monotone convergence property}, which is if $u_n$ converges monotonically to $u$ then $\langle \xi,u_n\rangle\to \langle \xi,u\rangle$.
 \end{enumerate}
 \end{prop}

 If $\Phi \in \Delta_2^{\infty}$ and $\Phi$ is $N_{\infty}$ then $\lpsi([0,T],\rr^d)= \left[\lphi([0,T],\rr^d)\right]^*$  (see \cite[Thm. 2.9 , Thm. 2.10]{Desch2001}).

% It is easy to see that, for every $N_{\infty}$ $\Phi$ we have that $L^{\infty}\hookrightarrow\lphi \hookrightarrow L^1$.




We define the \emph{Sobolev-Orlicz space} $\wphi$ by
\[\wphi([0,T],\rr^d):=\{u| u \hbox{ is absolutely continuous on $[0,T]$ and } u'\in \lphi([0,T],\rr^d)\}.\]
$\wphi([0,T],\rr^d)$ is a Banach space when equipped with the norm
\begin{equation}\label{def-norma-orlicz-sob}
\|  u  \|_{\wphi}= \|  u  \|_{\lphi} + \|u'\orlnor.
\end{equation}
And, we introduce the following subspaces of $\wphi$
%%
\begin{equation}\label{def-esp-orlicz-sob-per}
\begin{split}
\wphie&=\{u\in\wphi|u'\in\ephi\},\\
\wphie_T&=\{u\in\wphie|u(0)=u(T)\}.
\end{split}
\end{equation}

%
%
 We will use repeatedly the decomposition $u=\overline{u}+\widetilde{u}$ for a function $u\in L^1([0,T])$  where $\overline{u} =\frac1T\int_0^T u(t)\ dt$ and $\widetilde{u}=u-\overline{u}$.

 The following lemma is an elementary generalization to anisotropic Sobolev-Orlicz spaces of known results of Sobolev spaces.



\begin{lem}\label{lem:inclusion orlicz} Let $\Phi:\rr^d\to [0,+\infty)$ be a Young's 
function and let $u\in\wphi([0,T],\rr^d)$. Let 
$A_{\Phi}: \rr^+ \to \rr^+$ be the function defined by \eqref{eq:inversa-gral}. Then
\begin{enumerate}
\item\label{inclusion orlicz_item1} For every $s,t\in [0,T]$, $s\neq t$,
\begin{align}
 &|u(t)-u(s)| \leq
 \|u'\orlnor |s-t|A_{\Phi}^{-1}\left(\frac{1}{|s-t|}\right)\tag{Morrey's inequality}\label{in-sob-cont}
\\
& \| u\linf \leq A_\Phi^{-1}\left(\frac{1}{T}\right)\max\{1,T\}\|u\sobnor\tag{Sobolev's inequality}\label{eq:sobolev}
\end{align}
\item We have $\widetilde{u}\in L^{\infty}([0,T],\rr^d)$ and
\[
\|\widetilde u \linf \leq T A_{\Phi}^{-1}\left(\frac{1}{T}\right)\|u'\orlnor
\tag{Sobolev-Wirtinger's inequality}\label{wirtinger}
\]
\item\label{it:embeding} If $\Phi$ is $N_{\infty}$ then the space $\wphi([0,T],\rr^d)$ is compactly embedded in the space of continuous functions $C([0,T],\rr^d)$.
\end{enumerate}
\end{lem}

\begin{proof} By the absolutely continuity of $u$, Jensen's inequality and the definition of 
the Luxemburg norm, we have

\[
 \begin{split}
    \Phi\left( \frac{u(t)-u(s)}{\|u'\orlnor |s-t|}\right) &\leq  \Phi\left( \frac{1}{ |s-t|}\int_s^t  \frac{u'(r)}{\|u'\orlnor }dr\right)\\
    &\leq   \frac{1}{ |s-t|}\int_s^t  \Phi\left(\frac{u'(r)}{\|u'\orlnor }\right)dr
    \leq \frac{1}{ |s-t|}.
 \end{split}
\]
By Proposition \ref{prop:AsubPhi}\eqref{it:prop3}  we have $A^{-1}_{\Phi}\Phi(x)\geq |x|$, therefore we get
\[
    \frac{|u(t)-u(s)|}{\|u'\orlnor |s-t|} 
    \leq  A_{\Phi}^{-1}\left(\frac{1}{ |s-t|}\right),
\]
then  \ref{inclusion orlicz_item1} holds.

Now, we use \ref{in-sob-cont} and  Proposition \ref{prop:AsubPhi} \eqref{it:prop2} and we have 
\[\begin{split}
\left|u(t)-\overline {u}\right|&=
\left|\frac{1}{T}\int_0^T u(t)-u(s)\,ds\right|
\\
&\leq \frac{1}{T} \int_0^T |u(t)-u(s)|\,ds
\\
&\leq \|u'\orlnor T A_{\Phi}^{-1}\left(\frac{1}{T}\right)
\end{split}
\] 

In order to prove the Sobolev's inequality, we note that, using Jensen's inequality and 
the definition of $\|u\orlnor$, we obtain
\[ \Phi\left( \frac{ \overline{u}}{\|u\orlnor} \right) \leq
\frac{1}{T}\int_0^T\Phi\left(\frac{u(s)}{\|u\orlnor}\right)ds\leq\frac{1}{T}
\]
Then  by By Proposition \ref{prop:AsubPhi}\eqref{it:prop3} 
\[|\overline{u}|\leq A_{\Phi}^{-1}\left(\frac{1}{T}\right) \|u\orlnor.\]
Therefore, from this and \eqref{wirtinger} we get

\[\begin{split}
 \|u\linf &\leq |\overline{u}|+\|\tilde{u}\linf\\
 &\leq  
 A_{\Phi}^{-1}\left(\frac{1}{T}\right) \|u\orlnor+T A_{\Phi}^{-1}\left(\frac{1}{T}\right)\|u'\orlnor\\
 &\leq A_{\Phi}^{-1}\left(\frac{1}{T}\right)\max\{1,T\}\|u\sobnor
 \end{split}
 \]
 



In order to prove item 3, we take a bounded sequence
$u_n$ in $\wphi([0,T],\rr^d)$. Since $\Phi$ is $N_{\infty}$, from Proposition \ref{prop:AsubPhi}\eqref{it:prop4}  we obtain $sA_{\Phi}^{-1}(1/s)\to 0$ when $s\to 0$. Therefore \eqref{in-sob-cont} implies that $u_n$ are equicontinuous. Furthermore \eqref{sobolev} implies that $u_n$ is bounded in $C([0,T],\rr^d)$. Therefore by the Arzela-Ascoli Theorem we  obtain a subsequence $n_k$ and  $u\in C([0,T],\rr^d)$ with $u_{n_k}\to u$ in $C([0,T],\rr^d)$.

\end{proof}


\begin{lem}\label{segundo lema}
Let  $\{{u}_n\}_{n\in \mathbb{N}}$ be a sequence of  functions in $\Pi(\ephi,1)$ converging to  ${u}\in \Pi(\ephi,1)$  in the $\lphi$-norm. Then, there exist a subsequence
${u}_{n_k}$ and a real valued function $h\in L^1([0,T],\rr)$ such that ${u}_{n_k}\rightarrow {u} \quad\text{a.e.}$ and $\Phi({u}_{n_k})\leq h\quad\text{a.e.}$
\end{lem}



\begin{proof}
Since $d({u},\ephi)<1$ and ${u}_n$ converges to ${u}$, there exists $u_0\in\ephi$, a subsequence of $u_n$ (again denoted $u_n$) and $0<r<1$  such that $d(u_n,u_0)<r$. Let $\lambda_0\in (r,1)$.  By extracting more subsequences, if necessary, we can assume that $u_n\to u$ a.e. and
\[\lambda_n:=\|{u}_{n+1}-{u}_{n}\orlnor<\frac{1-\lambda_0}{2^n},\quad\hbox{for } n\geq 1.\]
We can assume $\lambda_n>0$ for every $n=0,\ldots$.

Let $\lambda:=1-\sum_{n=0}^{\infty}\lambda_n$ and define $h:[0,T]\rightarrow\mathbb{R}$  by
\begin{equation}\label{eq:serie} h(x)= \lambda\Phi\left(\frac{u_0}{\lambda}\right)+\sum_{n=0}^{\infty}\lambda_n\Phi\left(\frac{u_{n+1}-u_n}{\lambda_n}\right).
\end{equation}
Note that $\sum_{n=0}^{\infty}\lambda_n+\lambda=1$, therefore for any $n=1,\ldots$


\[
 \begin{split}
   \Phi(u_n) &=\Phi\left(  \lambda\frac{u_0}{\lambda}+   \sum_{j=0}^{n-1}\lambda_j\frac{u_{j+1}-u_j}{\lambda_j} \right)\\
   &\leq
   \lambda\Phi\left(\frac{u_0}{\lambda}\right)+\sum_{j=0}^{n-1}\lambda_j\Phi\left(\frac{u_{j+1}-u_j}{\lambda_j}\right) \leq h
 \end{split}
\]

Since $u_0\in\ephi\subset \claseor$ and $\ephi$ is a subspace we have that $\Phi(u_0/\lambda)\in L^1([0,T],\rr)$. 
On the other hand $\|u_{n+1}-u_n\orlnor \leq \lambda_n$, therefore
\[
 \int_0^T\Phi\left(\frac{u_{j+1}-u_j}{\lambda_j}\right)dt\leq 1.
\]
Then $h\in L^1([0,T],\rr)$.


\end{proof}




\section{Differentiability Gate\^aux of action integrals in anisotropic Orlicz spaces}
In this section we give a brief introduction to superposition operators between anistropic Orlicz Spaces.  
We apply these results to obtain Gate\^aux differentiability of                                                                                action integrals associated to lagrangian functions defined on Sobolev-Orlicz spaces.

Henceforth we assume that  $f:[0,T]\times \rr^d\to\rr^d$is a \emph{Carath\'eodory function}, i.e.

\begin{enumerate}
 \item[\namedlabel{eq:carathe}{(C)}] $f$ is measurable with respect to $t\in [0,T]$ for every  $x\in\rr^d$, and $f$ is a continuous function with  respect to  $x\in\rr^d$ for a.e. $t \in [0,T]$.
\end{enumerate}




\begin{defi}
 For $f:[0,T]\times \rr^d\to\rr^d$  we denote by $\b{f}$ the Nemytskii (o superposition) operator defined for functions $u:[0,T]\to\rr^d$ by
\[\b{f}u(t)=f(t,u(t))\]
\end{defi}

In the following Theorem we enumerate  some known properties for superposition operators defined on anisotropic Orlicz spaces of vector functions.   For the proofs see \cite{krasnosel2011integral} for scalar functions  and
\cite{zbMATH04038592,zbMATH03983966,zbMATH03942215} for the generalization to  $\mathbb{R}^d$-valued  (moreover Banach spaces valued)  functions in a anisotropic Orlicz Spaces (moreover modular anisotropic spaces).

\begin{thm} We assume that $f$ satisfies condition \eqref{eq:carathe} and that $\Phi_1,\Phi_2:\rr^d\to [0,+\infty)$ are anisotropic Young functions. Then
\begin{enumerate}
 \item\label{it:measure}\emph{Measurability.}  The operator $\b{f}$ maps  measurable function into measurable functions
 \item\label{it:exten}\emph{Extensibility.}  If the operator $\b{f}$ acts from the ball $B_{L^{\Phi_1}}(r):=\{ u\in L^{\Phi_1}| \|u\|_{L^{\Phi_1}}<r\}$ into the space $L^{\Phi_2}$ or the space $E^{\Phi_2}$ then $\b{f}$ can be extended  from $\Pi(E^{\Phi_1},r)$ into space $L^{\Phi_2}$ or  $E^{\Phi_2}$, respectively.
 \item\label{it:exten}\emph{Continuity.} If the operator $\b{f}$ acts from $\Pi(E^{\Phi_1},r)$ into space $E^{\Phi_2}$, then $\b{f}$ is continuous.
\end{enumerate}
\end{thm}



Given a continuous function $a\in C(\mathbb{R}^n,\mathbb{R}^+)$, we define the composition operator $a:\mathcal{M}_d\to \mathcal{M}_d$ by $\b{a}(u)(x)= a(u(x))$.

We will often use the following result whose proof can be performed as that of  Corollary 2.3 in \cite{ABGMS2015}. 
\begin{lem}\label{lem:cota-a}
\label{a_bound} If $a\in C(\mathbb{R}^d,\mathbb{R}^+)$ then $\b{a}:\wphi\to L^{\infty}([0,T])$ is bounded. 
More concretely,  there exists a non decreasing function $A:\mathbb{R}^+\to\mathbb{R}^+$ such that
 $\|\b{a}(u)\|_{L^{\infty}([0,T])}\leq A(\|u\|_{\wphi})$.
\end{lem}

Quiz\'as no sea necesaria la prueba, si dejamos  el comentario de arriba???

\begin{proof}  Let $A \in C(\mathbb{R}^+,\mathbb{R}^+)$ be a  non decreasing, continuous function defined by  
$\alpha(s):=\sup_{\|x\|\leq s, x \in \rr^d}|a(x)|$.  If $u \in \wphi_d$ then, by  \ref{eq:sobolev}, 
\[a(u(x))\leq \alpha (\|u\|_{L^{\infty}})\leq 
\alpha \left(
A_\Phi^{-1}\left(\frac{1}{T}\right)\max\{1,T\}\|u\sobnor\right)=: 
A(\|u\sobnor).\]
\end{proof}



HABR\'IA QUE VER D\'ONDE SE UBICA LA CONDICI\'ON DE ESTRUCTURA...QUIZ\'AS EN LA INTRODUCCI\'ON?....

We assume that the \emph{Lagrangian} $\mathcal{L}:[0,T]\times\rr^d\times\rr^d\to\rr$ is 
Carath\'eodory and differentiable function satisfying 
\begin{equation}\label{eq:condicion-estructura}
|\mathcal{L}(t,x,y)|+ |D_{x}\mathcal{L}(t,x,y)|+\Psi(D_{y}\mathcal{L}(t,x,y))
\leq
a(|x|)\left(b(t)+ \Phi(y)),
\end{equation}
where  $a\in C(\mathbb{R}^+,\mathbb{R}^+)$, $b\in L^1_1([0,T]) $, $\Phi$ and $\Psi$ are $N_{\infty}$-functions (complementary???? o en el teorema o nunca?)

Next, we deal with the differentiability of the action integral 
\begin{equation}\label{eq:integral_accion}
I(u)=\int_{0}^T \mathcal{L}(t,u(t),\dot{u}(t))\ dt.
\end{equation}


\begin{thm}\label{teo:diferenciabilidad}
Let $\mathcal{L}$ be a differentiable Carath\'eodory function satisfying \eqref{eq:condicion-estructura}.
Then the following statements hold:
\begin{enumerate}
\item \label{it:T1item1} \label{A1} The action integral given by \eqref{eq:integral_accion}
is finitely defined on $\domi:=W^{1}\lphi\cap\{u|\dot{u}\in\Pi(\ephi,1)\}$.

\item\label{it:T1item3} The function  $I$ is G\^ateaux differentiable on $\domi$ and  its derivative $I'$ is demicontinuous from 
$\domi$  into $\left[\wphi \right]^*$. Moreover, $I'$ is given by the following expression
\begin{equation}\label{eq:DerAccion}
\langle  I'(u),v\rangle= \int_0^T \left\{D_{x}\mathcal{L}\big(t,u,\dot{u}\big)\ccdot v
+ D_{y}\mathcal{L}\big(t,u,\dot{u}\big)\ccdot\dot{v}\right\} \ dt.
\end{equation}

\item\label{it:T1item4}  If  $\Psi \in \Delta_2$ then 
  $I'$ is continuous from $\domi$ into $\left[\wphi\right]^*$ when both spaces are equipped with the strong topology.
\end{enumerate}
\end{thm}


\begin{proof}
Let $u\in \domi$.
As 
\begin{equation}\label{eq:inclusion3}
\dot{u}\in\Pi(\ephi,1)\subset \claseor_1
\end{equation}
and \eqref{eq:inclusiones}, then $\Phi( \dot{u}(t)) \in L^1$.
Now,
 \begin{equation}\label{eq:cota-condicion-estructura}
|\mathcal{L}(\cdot,u,\dot{u})|+ |D_{x}\mathcal{L}(\cdot,u,\dot{u})|
+\Psi(D_{y}\mathcal{L}(\cdot,u,\dot{u}))
\leq A(\|u\sobnor ) (b+ \Phi (\dot{u})) \in
 L^1,
\end{equation}
by  \eqref{eq:condicion-estructura} and Lemma \ref{lem:cota-a}.
Thus item \eqref{it:T1item1} is proved.

We split up the proof of item \ref{it:T1item3} into four steps.

\noindent\emph{Step 1. The non linear operator  $u \mapsto D_{x}\mathcal{L}(t,u,\dot{u})$ is continuous from $\domi$ into $L^{1}([0,T])$ with the strong topology on both sets.} 


%If $u\in \domi$, from \eqref{eq:condicion-estructura} and \eqref{eq:inclusion3}, we obtain 
%\begin{equation}\label{eq:DxL1}
%|D_{x}\mathcal{L}(\cdot,u,\dot{u})|\leq A(\|u\sobnor) \left(b+\Phi\left(\dot{u})\right) \in L^1.
%\end{equation}


Let   $\{u_n\}_{n\in \mathbb{N}}$ be a sequence of  functions in $\domi$  
and let $u\in \domi$  such that $u_n\rightarrow u$ in $\wphi$.
By \eqref{eq:sobolev}, we have 
\[
|u_n(t)-u(t)|\leq T A_{\Phi}^{-1}\left(\frac{1}{T}\right) \|u_n-u\orlnor
\]
then $u_n \to u$ uniformly.
As $\dot{u}_n\rightarrow \dot{u}\in\domi$, by 
  Lemma \ref{segundo lema}, there exist a subsequence of  $\dot{u}_{n_k}$ (again denoted $\dot{u}_{n_k}$) and a function  
	$h\in L^1([0,T],\rr)$
	such that  $\dot{u}_{n_k}\rightarrow \dot{u} \quad\text{a.e.}$ and $\Phi(\dot{u}_{n_k})\leq h\quad\text{a.e}$.  

Since $u_{n_k}$, $k=1,2,\ldots,$ is a strong convergent sequence in $\wphi$, it is a bounded sequence in $\wphi$. 
According to item \eqref{it:embeding} of   Lemma \ref{lem:inclusion orlicz}, 
% and Lemma \eqref{lem:cota-a}, 
there exists $M>0$ such that $\|\b{a}(u_{n_k})\|_{L^{\infty}} \leq M$, $k=1,2,\ldots$.  
From the previous facts and \eqref{eq:cota-condicion-estructura}, we get
\begin{equation*}\label{eq:DxL1-bis}
|D_{x}\mathcal{L}(\cdot,u_{n_k},\dot{u}_{n_k})|\leq a(|u_{n_k}|)(b+\Phi(\dot{u}_{n_k}))\leq
M (b+h) \in L^1.
\end{equation*}
On the other hand, by the continuous differentiability of $\mathcal{L}$, we have
\[D_{x}\mathcal{L}(t,u_{n_k}(t),\dot{u}_{n_k}(t))\to D_{x}\mathcal{L}(t,u(t),\dot{u}(t))\quad\hbox{ for a.e. } t\in[0,T].\]
Applying the Dominated Convergence Theorem we conclude the proof of step 1.


\noindent\emph{Step 2. The non linear operator   $u
 \mapsto  D_{y}\mathcal{L}(t,u,\dot{u})$ is continuous from $\domi$ with the strong topology  
into $\left[\lphi\right]^*$  with the weak$^*$ topology.}

 Let $u\in \domi$.  From  \eqref{eq:cota-condicion-estructura} it follows that 
%\begin{equation}\label{eq:DyLpsi}
%\Psi(D_y\mathcal{L}(\cdot,u,\dot{u}))\leq a(|u|)(b+\Phi(\dot{u}))\in L^1
%\end{equation}
%then
\begin{equation}\label{eq:DyLpsi-clase}
D_{y}\mathcal{L}(\cdot,u,\dot{u})\in C^{\Psi}.
\end{equation}

As\'i? o conviene poner la cota de $\Psi(D_y)$ expl\'icitamente???

 Note that \eqref{eq:cota-condicion-estructura},  \eqref{eq:DyLpsi-clase} and the imbeddings $\wphi \hookrightarrow L^{\infty}$ and  
$\lpsi\hookrightarrow  \left[\lphi\right]^*$ imply that the second member of
\eqref{eq:DerAccion} defines an element of $\left[\wphi\right]^*$.



Let $u_n,u\in \domi$ such that $u_n\to u$ in the norm of $\wphi$. 
We must prove that  $D_{y}\mathcal{L}(\cdot,u_n,\dot{u}_n)\overset{w^*}{\rightharpoonup} 
D_{y}\mathcal{L}(\cdot,u,\dot{u})$. 
On the contrary, there exist $v\in\lphi$, $\epsilon>0$ and a subsequence of $\{u_n\}$ (denoted  $\{u_n\}$ for simplicity)  such that
\begin{equation}\label{cota_eps}
 \left| \langle D_{y}\mathcal{L}(\cdot,u_n,\dot{u}_n),v \rangle - 
\langle  D_{y}\mathcal{L}(\cdot,u,\dot{u}),v \rangle\right|\geq \epsilon.
\end{equation}
We have $u_n\rightarrow u$ in $\lphi$ and
$\dot{u}_n\rightarrow \dot{u}$ in $\lphi$.
 By Lemma \ref{segundo lema}, there exist a subsequence of $\{u_n\}$ (again denoted  $\{u_n\}$ for simplicity) 
and a function $h\in L^1([0,T],\rr)$ such that 
$u_n\rightarrow u$ uniformly, $\dot{u}_n\rightarrow \dot{u} \quad\text{a.e.}$ and $\Phi(\dot{u}_n)\leq h\quad\text{a.e.}$ 
As in the previous step, since $u_n$ is a convergent sequence, 
Lemma \ref{lem:cota-a} implies that $a(|u_n(t)|)$ is uniformly bounded by a certain constant $M>0$. 
Therefore,   from inequality  \eqref{eq:cota-condicion-estructura} with $u_n}$ instead of $u$, we have 
\begin{equation}\label{eq:Dy-suc}
  \Psi(D_{y}\mathcal{L}(\cdot,u_n,\dot{u}_n))   
	\leq M (b+h)\in L^1.
\end{equation}
As $v \in \lphi$ there exists $\lambda>0$ such that $\Phi(\frac{v}{\lambda})\in L^1$. 
Now, by Young inequality and \eqref{eq:Dy-suc}, we have
\begin{equation}\label{eq:Dy_lambda-Psi}
\begin{split}
&\lambda D_{y}\mathcal{L}(\cdot,u_{n_k},\dot{u}_{n_k})\ccdot \frac{v(t)}{\lambda} 
\\
&
\leq 
\lambda\left[\Psi(D_{y}\mathcal{L}(\cdot,u_{n_k},\dot{u}_{n_k}))+\Phi\left(\frac{v}{\lambda}\right)\right]
\\
&\leq \lambda M (b+h)+\lambda \Phi\left(\frac{v}{\lambda}\right)\in L^1
\end{split}
\end{equation}
  Finally, from the Lebesgue Dominated Convergence Theorem, we deduce
\begin{equation}\label{conv_debil}
\int_0^T  D_{y}\mathcal{L}(t,u_{n_k},\dot{u}_{n_k})
\ccdot  v \,dt 
\to 
\int_0^T D_{y}\mathcal{L}(t,u,\dot{u})\ccdot v\, dt \end{equation}
which contradicts the inequality \eqref{cota_eps}. This completes the proof of step 2.

\emph{Step 3.} We will prove \eqref{eq:DerAccion}. 
%The proof follows similar lines as \cite[Thm. 1.4]{mawhin2010critical}. 
For $u\in \domi$ and $0\neq v\in\wphi$, we define the function
\[H(s,t):=\mathcal{L}(t,u(t)+s v(t),\dot{u}(t)+s\dot{v}(t)).\]

%From \cite[Lemma 10.1]{KR} (or \cite[Thm. 5.5]{Orliczvectorial2005} ) 
%we obtain that if $|u|\leq |v|$ then    $d(u,\ephi)\leq d(v,\ephi)$. Esto va????

For  $|s|\leq s_0:=\min\{\left(1-d(\dot{u},\ephi)\right)/\|v\sobnor, 1-d(\dot{u},\ephi)\}$, 
using triangle inequality we get 
$
d \left(\dot{u}+s \dot{v}, \ephi \right)<1$ and thus $\dot{u}+s\dot{v} \in \Pi(\ephi,1)$. 
These facts imply, in virtue of Theorem \ref{teo:diferenciabilidad} item \ref{it:T1item1}, 
that $I(u+s v)$ is well defined and finite for $|s|\leq s_0$. 

We also have 
$
\|u+sv\sobnor\leq \|u\sobnor+s_0\|v\sobnor;
$
then, by Lemma \ref{lem:cota-a}, there exists $M>0$ such that 
$\|a(u+sv)\linf\leq M$.

Let $\lambda>0$ such that $\Phi(\frac{\dot{v}}{\lambda})\in L^1$.
On the other hand, if $\dot{v}\in\lphi$ and $|s|\leq s_0 \lambda^{-1}$,
from the convexity and the parity of $\Phi$, we get
\[
\begin{split}
&\Phi(\dot{u}+s\dot{v})=
\Phi\left((1-s_0)\frac{\dot{u}}{1-s_0}+s_0 \frac{s}{s_0}\dot{v}\right)
\leq
(1-s_0)\Phi\left(\frac{\dot{u}}{1-s_0}\right)+s_0 \Phi\left(\frac{s}{s_0}\dot{v}\right)
\\
&\leq
(1-s_0)\Phi\left(\frac{\dot{u}}{1-s_0}\right)+s_0 \Phi\left(\frac{\dot{v}}{\lambda}\right)
\in L^1
\end{split}
\]
As $\dot{u}\in\Pi(\ephi,1)$ then
\[
d\left(\frac{\dot{u}}{1-s_0},E^{\Phi}\right)=\frac{1}{1-s_0}d(\dot{u}, E^{\Phi})<1
\]
and therefore $\frac{\dot{u}}{1-s_0}\in C^\Phi$.

Now, applying \eqref{eq:cota-condicion-estructura}, \eqref{eq:Dy_lambda-Psi},  
%the monotonicity of $\varphi$ and $\Phi$, 
the fact that $v \in L^{\infty}$ and $\dot{v}\in\lphi$, 
%and H\"older's inequality, 
we get
\begin{equation}\label{ctg}
\begin{split}
|D_s H(s,t)|&=\left| D_{x}\mathcal{L}(t,u+sv,\dot{u}+s\dot{v})\ccdot v +  
\lambda D_{y}\mathcal{L}(t, u+s v, \dot{u}+s\dot{v})\ccdot\frac{\dot{v}}{\lambda}\right| \\
 & \leq M \left\{\left[ b(t)+ \Phi(\dot{u}+s\dot{v})\right]|v|\}\\
&+ \lambda\left[\Psi(D_{y}\mathcal{L}(t,u+sv,\dot{u}+s\dot{v}))+\Phi\left(\frac{\dot{v}}{\lambda}\right) \right]
\\
&\leq M \left\{\left[ b(t)+ \Phi(\dot{u}+s\dot{v})\right]|v|\}+
\lambda M[ b(t)+ \Phi(\dot{u}+s\dot{v})]+\lambda \Phi\left(\frac{\dot{v}}{\lambda}\right)
\\&=
M [ b(t)+ \Phi(\dot{u}+s\dot{v})] (|v|+\lambda) +\lambda \Phi\left(\frac{\dot{v}}{\lambda}\right)
\in L^1.
\end{split}
\end{equation}

Consequently, $I$ has a directional derivative and
\[
\langle I'(u),v \rangle=\frac{d}{ds}I(u+s v)\big|_{s=0}=\int_0^T  
\left\{D_{x}\mathcal{L}(t,u,\dot{u})\ccdot v+ D_{y}\mathcal{L}(t,u,\dot{u})\ccdot \dot{v}\right\} \ dt.
\]
Moreover, from the previous formula, \eqref{eq:cota-condicion-estructura},  \eqref{eq:DyLpsi-clase}, and
Lemma \ref{lem:inclusion orlicz}, we obtain
\[
|\langle I'(u),v \rangle| \leq \|D_{x}\mathcal{L}\|_{L^1} \| v\linf + 
\|D_{y}\mathcal{L}\|_{\lpsi} \|\dot{v}\orlnor \leq C \|v\sobnor
\]
with a appropriate constant $C$.

This completes the proof of the G\^ateaux differentiability of $I$. 




LO QUE SIGUE NO IR\'IA ???? PORQUE TOMAR\'IAMOS $\Psi \in \Delta_2$

\emph{Step 4. The operator $I':\domi  \to \left[\wphi_d
\right]^* $ is demicontinuous.}
This is a consequence  of the continuity of the mappings $u \mapsto D_{x}\mathcal{L}(t,u,\dot{u})$ and $u \mapsto
D_{y}\mathcal{L}(t,u,\dot{u})$. Indeed, if $u_n,u\in \domi$ with $u_n\to u$ in the norm of $\wphi$ and $v \in
\wphi$, then
\[
\begin{split}
\left\langle  I'(u_{n}),v \right\rangle &= \int_0^T \left\{  D_{x}\mathcal{L}\left(t,u_n,\dot{u}_n\right)\ccdot
v +
 D_{y}\mathcal{L}\left(t,u_n,\dot{u}_n\right)\ccdot \dot{v}\right\} \ dt\\
&\rightarrow \int_0^T \left\{ D_{x}\mathcal{L}\left(t,u,\dot{u}\right)\ccdot v+ 
D_{y}\mathcal{L}\left(t,u,\dot{u}\right)\ccdot \dot{v}\right\} \ dt\\
&=\left\langle  I'(u),v \right\rangle.
\end{split}
\]


In order to prove item  \ref{it:T1item4}, it is necessary to see that the maps $u\mapsto D_{x}\mathcal{L}(t,u,\dot{u})$  
and $u\mapsto D_{y}\mathcal{L}(t,u,\dot{u})$  are norm continuous
from $\domi $ into $L^1$ and
 $\lpsi$, respectively.  

The continuity of the first map has already been proved in step 1. 

Let $u_n, u \in \domi$ with $\|u_n- u\sobnor\to 0$ and suppose that 
$D_y\mathcal{L}(t,u_n,\dot{u}_n)$ does not converge to $D_y\mathcal{L}(t,u,\dot{u})$ in $L^{\Psi}$.
Applying Lemma \ref{segundo lema}  there exist a subsequence of $u_n$ 
(denoted $u_n$ for simplicity)
$u_n \in \domi$ and a function  $h \in L^1$  such that  $\Psi(u_n)\leq h$ and $u_n \to u$ a.e.
Then, by \eqref{eq:Dy_lambda-Psi} we have
 $ \Psi(v_n) 
	\leq m(t) \in L^1$
 being  $v_n:=D_{y}\mathcal{L}(\cdot,u_n,\dot{u}_n)$ and $m(t):= M (b+h)$, 
and $v_n \to v$ a.e. where $D_y\mathcal{L}(\cdot,u,\dot{u})$.

As  $\Psi\in\Delta_2$, there exists $c:\rr^+\to C^{}$ such that 
$\Psi(\lambda x)\leq c(|\lambda|)\Psi(x)$. 

FALTA LA \'ULTIMA PARTE DE HOJA 9!!
%then   the right hand side of  \eqref{Dy-suc} belongs to $\epsi$. 
%Now, invoking  Lemma \ref{lema_conv_may}, we  prove that
  %from any sequence $u_n$ which converges to $u$ in $\wphi$ we can
%extract a subsequence such that   $D_{y}\mathcal{L}(t,u_{n_k},\dot{u}_{n_k})\to 
%D_{y}\mathcal{L}(t,u,\dot{u})$ in the strong topology. The desired result is obtained by a standard argument.

The continuity of $I'$  follows  from the continuity 
of $D_{x}\mathcal{L}$ and $D_{y}\mathcal{L}$ using the formula \eqref{eq:DerAccion}.

\end{proof}


\section*{Acknowledgments}
The authors are partially supported by a UNRC grant number 18/C417. The first author is  partially supported by a  UNSL grant number 22/F223. 




 \bibliographystyle{apalike}
 \bibliography{biblio}


\end{document}


