\documentclass[twoside]{article}


%\usepackage{hyperref}
\usepackage{amssymb,amsthm}
\usepackage{amsmath}
\usepackage{color}
\usepackage{ esint }
\usepackage{mathabx}
\usepackage{MnSymbol}
\usepackage{fancyhdr}
\usepackage{times}

\usepackage[latin1]{inputenc}

\usepackage{comment}
\usepackage{url}
\usepackage{xcolor}
\usepackage{adjustbox}
\usepackage{hyperref}

\newtheorem{thm}{Theorem}[section]
\newtheorem{cor}[thm]{Corollary}
\newtheorem{lem}[thm]{Lemma}

\newtheorem{defi}[thm]{Definition}
\newtheorem{prop}[thm]{Proposition}
\theoremstyle{remark}
\newtheorem{comentario}{Remark}


\makeatletter
\newcommand{\labitem}[2]{%
\def\@itemlabel{\textbf{#1}}
\item
\def\@currentlabel{#1}\label{#2}}
\makeatother




\title{Periodic solutions of 
Euler-Lagrange equations in an Orlicz-Sobolev space setting by the dual least action principle }
\author{Sonia Acinas \thanks{SECyT-UNRC and  FCEyN-UNLPam}\\
Dpto. de Matem\'atica, Facultad de Ciencias Exactas y Naturales\\
Universidad Nacional de La Pampa\\
(L6300CLB) Santa Rosa, La Pampa, Argentina\\
\url{sonia.acinas@gmail.com}\\[3mm]
Fernando D. Mazzone \thanks{SECyT-UNRC, FCEyN-UNLPam and CONICET}\\
Dpto. de Matem\'atica, Facultad de Ciencias Exactas, F\'{\i}sico-Qu\'{\i}micas y Naturales\\
Universidad Nacional de R\'{i}o Cuarto\\
(5800) R\'{\i}o Cuarto, C\'ordoba, Argentina,\\
\url{fmazzone@exa.unrc.edu.ar}
}

\date{}

\newcommand{\orlnor}{\|_{L^{\Phi}}}
\newcommand{\lurnor}{\|^{*}_{L^{\Phi}}}
\newcommand{\linf}{\|_{L^{\infty}}}
\newcommand{\lphi}{L^{\Phi}}
\newcommand{\lphiuno}{L^{\Phi_1}}
\newcommand{\lphidos}{L^{\Phi_2}}
\newcommand{\lphii}{L^{\Phi_i}}
\newcommand{\lpsi}{L^{\Phi^*}}
\newcommand{\lpsiuno}{L^{\Phie_1}}
\newcommand{\lpsidos}{L^{\Phie_2}}
\newcommand{\lpsii}{L^{\Phie_i}}
\newcommand{\lmuno}{L^{M_1}}
\newcommand{\lmdos}{L^{M_2}}
\newcommand{\lmj}{L^{M}}
\newcommand{\lmn}{L^{M_n}}
\newcommand{\ephi}{E^{\Phi}}
\newcommand{\ephiuno}{E^{\Phi_1}}
\newcommand{\ephidos}{E^{\Phi_2}}
\newcommand{\ephin}{E^{\Phi_n}}
\newcommand{\ephii}{E^{\Phi_i}}
\newcommand{\claseor}{C^{\Phi}}
\newcommand{\wphi}{W^{1}\lphi}
\newcommand{\wphiuno}{W^{1}\lphiuno}
\newcommand{\wphidos}{W^{1}\lphidos}
\newcommand{\wphii}{W^{1}\lphii}
\newcommand{\wphiet}{W^{1}\ephi_T}
\newcommand{\wphie}{W^{1}\ephi}
\newcommand{\sobnor}{\|_{W^{1}\lphi}}
\newcommand{\domi}{\mathcal{E}^{\Phi}_d(\lambda)}
\newcommand{\domiuno}{\mathcal{E}^{\Phi_1}_d(\lambda)}
\newcommand{\domidos}{\mathcal{E}^{\Phi_2}_d(\lambda)}
\newcommand{\domii}{\mathcal{E}^{\Phi_i}_d(\lambda)}
\newcommand{\domin}{\mathcal{E}^{\Phi_n}_d(\lambda)}
\renewcommand{\b}[1]{\boldsymbol{#1}}
\newcommand{\rr}{\mathbb{R}}
\newcommand{\nn}{\mathbb{N}}
\newcommand{\ccdot}{\b{\cdot}}
\renewcommand{\leq}{\leqslant} 
\renewcommand{\geq}{\geqslant} 
\newcommand{\epsi}{E^{\Phie}}
\newcommand{\Phie}{\Phi^{*}}

\newcounter{example}

\setcounter{example}{1}


\newenvironment{example}{\noindent\textit{Example} \arabic{example}.}{\addtocounter{example}{1}}




\begin{document}


\maketitle
%
\begingroup%Locallizing the change to `thefootnote'.
    \renewcommand{\thefootnote}{}%Removing the footnote symbol.
    %
    \footnotetext{%
    %   2010 Mathematics Subject Classification
    %   http://www.ams.org/msc/
    \textbf{2010  AMS Subject Classification.} Primary: .
    Secondary: .
    }%
        \footnotetext{%
    \textbf{Keywords and phrases.}  .
    }%
    \endgroup
%
%
%
%

\begin{abstract}


\end{abstract}




\pagestyle{fancy} \headheight 35pt \fancyhead{} \fancyfoot{}

\fancyfoot[C]{\thepage} \fancyhead[CE]{\nouppercase{S. Acinas and F.D. Mazzone }} \fancyhead[CO]{\nouppercase{\section}}

\fancyhead[CO]{\nouppercase{\leftmark}}


%\tableofcontents




\section{Introduction}
This paper deals with system  of equations of the type:

\begin{equation}\label{ProbPrin-gral}
    \left\{%
\begin{array}{ll}
  \frac{d}{dt} D_{y}\mathcal{L}(t,u(t),u'(t))= D_{x}\mathcal{L}(t,u(t),u'(t)) \quad \hbox{a.e.}\ t \in (0,T)\\
    u(0)-u(T)=u'(0)-u'(T)=0,
\end{array}%
\right.,
\end{equation}
where $\mathcal{L}:[0,T]\times\rr^d\times\rr^d\to\rr$, $d\geq 1$, is called the \emph{Lagrange function} or \emph{lagrangian} and the unknown function  $u:[0,T]\to\rr^d$ is absolutely continuous. In other words, we are interested in  finding \emph{periodic weak solutions} of \emph{Euler-Lagrange system of ordinary equations}. This topic was deeply addressed for the \emph{Lagrange function}
\begin{equation}\label{eq:lagrange_cuad}
\mathcal{L}_{p,F}(t,x,y)=\frac{|y|^p}{p}+F(t,x),
\end{equation}
for $1<p<\infty$. For example, the classic book  \cite{mawhin2010critical} deals mainly with problem \eqref{ProbPrin-gral}, for the lagrangian $\mathcal{L}_{2,F}$, through various methods: direct, dual action, minimax, etc. The results in \cite{mawhin2010critical} were extended and improved in several articles, see  \cite{tang1995periodic,tang1998periodic,wu1999periodic,tang2001periodic,zhao2004periodic}  to cite some examples. Lagrange functions \eqref{eq:lagrange_cuad} for arbitrary $1<p<\infty$ were considered in  \cite{Tian2007192,tang2010periodic} and in this case \eqref{ProbPrin-gral}  is reduced to the $p$-laplacian system
\begin{equation}\label{ProbP-lapla}
    \left\{%
\begin{array}{ll}
   \frac{d}{dt}\left(u'(t)|u'|^{p-2}\right) = \nabla F(t,u(t)) \quad \hbox{a.e.}\ t \in (0,T)\\
    u(0)-u(T)=u'(0)-u'(T)=0.
\end{array}%
\right.
\end{equation}


In this context, it  is customary to call $F$ a  \emph{potential function}, and it is assumed that $F(t,x)$ is differentiable with respect to $x$ for a.e. $t\in [0,T]$ and the following conditions are verified:
\begin{enumerate}
\labitem{(C)}{item:condicion_c} $F$ and its gradient $\nabla F$, with respect to $x\in\rr^d$,  are  Carath\'eodory functions, i.e. they are measurable functions with respect to $t\in [0,T]$, for every  $x\in\rr^d$, and they are continuous functions with  respect to  $x\in\rr^d$ for a.e. $t \in [0,T]$.
 \labitem{(A)}{item:condicion_a}  For   a.e. $t\in [0,T]$, it holds that
\begin{equation}
|F(t,x)| + |\nabla F(t,x)|  \leq a(|x|)b(t).
\end{equation}
In this inequality we assume that the function  $a:[0,+\infty)\to [0,+\infty)$ is continuous and non decreasing and $0\leq b\in L^1([0,T],\rr)$.
\end{enumerate}


In \cite{ABGMS2015} it was treated  the case of a lagrangian $\mathcal{L}$ which is lower bounded by a Lagrange function
\begin{equation}\label{eq:lagrange_phi}
\mathcal{L}_{\Phi,F}(t,x,y)=\Phi(|y|)+F(t,x),
\end{equation}
where  $\Phi$ is an $N$-function (see section \ref{preliminares} for the definition of this concept).  
In the paper \cite{ABGMS2015} it was assumed  a condition of \emph{bounded oscillation} on $F$  (see xxxxx below). 
In this paper  we apply the dual method (\cite[Ch. 3]{mawhin2010critical}) to obtain solutions of \eqref{ProbPrin-gral}.



\section{Preliminaries}\label{preliminares}

In this section, we give a short introduction to known results on Orlicz and Orlicz-Sobolev spaces of vector valued functions (anisotropic Orlicz Spaces) and other brief introduction to superposition operators between these spaces. References for  these topics are \cite{Orliczvectorial2005,Skaff1969, Desch2001} and
\cite{zbMATH04038592,zbMATH04009182,zbMATH03983966,zbMATH03942215}. 

Hereafter we denote  by $\mathbb{R}^+$  the set of all non negative real numbers. A function $\Phi:\mathbb{R}^d\to \mathbb{R}_+ $ is called an \emph{Young's function} if $\Phi$ is convex, $\Phi(0)=0$, $\Phi(-x)=\Phi(x)$ and $\Phi(x)\to +\infty$, when $|x|\to+\infty$.

Following \cite{Orliczvectorial2005} we say that $\Phi$ is \emph{coercive} if
\[\lim_{|x|\to\infty}\frac{\Phi(x)}{|x|}=+\infty.\]

We define the function the $F$ by
\begin{equation}\label{eq:inversa-gral}
F(s)=\sup\{|x|:\Phi(x)\leq s\},
\end{equation}
where $\Phi$ is a Young's function. 


As $\alpha \Phi(\frac{x}{\alpha})$ is decreasing with respect to $\alpha$, 
we get that the function $\alpha F(\frac{x}{\alpha})$ is increasing 
with respect to $\alpha$. That is,
if $0<\alpha \leq \beta$, we have
\[
\begin{split}
\alpha F\left(\frac{s}{\alpha}\right)=
\alpha\sup\left\{|x|:\Phi(x)\leq \frac{s}{\alpha}\right\}=\sup\left\{\alpha|x|:\alpha\Phi(x)\leq s\right\}=
\\
\sup\left\{|y|:\alpha\Phi\left(\frac{y}{\alpha}\right)\leq s\right\}
\leq \sup\left\{|y|:\beta\Phi\left(\frac{y}{\beta}\right)\leq s\right\}=
\\
\sup\left\{\beta|x|:\beta \Phi(x)\leq s\}=\sup\beta \{|x|:\Phi(x)\leq\frac{x}{\beta} \right\}=
\\
\beta F\left(\frac{x}{\beta}\right).
\end{split}
\]

We note that  for every $K>0$, if $|x|>F(K)$ 
then $\Phi(x)> K$ and therefore we see that  $|x|\leq F(\Phi(x))$. If $d=1$ 
then $F=\Phi^{-1}$.





We also say that a non decreasing function $\eta:\mathbb{R}^+\rightarrow \mathbb{R}^+$ satisfies the  \emph{$\Delta_2^{\infty}$-condition}, denoted by $\eta \in \Delta_2^{\infty}$,
if there exist  constants $K>0$ and  $M\geq 0$ such that
\begin{equation}\label{delta2defi}\eta(2x)\leq K\eta(x),
\end{equation}
for every $|x|\geq M$.


If $\Phi$ is a Young function we define its \emph{Fenchel conjugate}   $\Phi^*:\mathbb{R}^d\to \mathbb{R}_+ $by:
\begin{equation}\label{eq:conjugada}
 \Phi^*(y)=\sup\limits_{x\in\mathbb{R}^d} x\cdot y-\Phi(x)
\end{equation}


Let $d$ be a positive integer. We denote by $\mathcal{M}:=\mathcal{M}([0,T],\rr^d)$  the set of all measurable functions (i.e. functions which are limits of simple functions)  defined on $[0,T]$ with values on $\mathbb{R}^d$ and  we write $u=(u_1,\dots,u_d)$ for  $u\in \mathcal{M}$. For the set of functions $\mathcal{M}$, as for other similar sets, we will omit the reference to codomain $\mathbb{R}^d$ when $d=1$.


Given  an $N$-function $\Phi$ we define the \emph{modular function} 
$\rho_{\Phi}:\mathcal{M}\to \mathbb{R}^+\cup\{+\infty\}$ by
\[\rho_{\Phi}(u):= \int_0^T \Phi(u)\ dt.\]
Here $|\cdot|$ is the euclidean norm of $\mathbb{R}^d$.
Now, we introduce the \emph{Orlicz class} $C^{\Phi}=C^{\Phi}([0,T],\rr^d)$   by setting
\begin{equation}\label{claseOrlicz}
  C^{\Phi}:=\left\{u\in \mathcal{M} | \rho_{\Phi}(u)< \infty \right\}.
\end{equation}
The \emph{Orlicz space} $\lphi=L^{\Phi}([0,T],\rr^d)$ is the linear hull of $\claseor$;
equivalently,
\begin{equation}\label{espacioOrlicz}
\lphi:=\left\{ u\in \mathcal{M}| \exists \lambda>0: \rho_{\Phi}(\lambda u) < \infty   \right\}.
\end{equation}
  The Orlicz space $\lphi$ equipped with the \emph{Luxemburg norm}
\[
\|  u  \orlnor:=\inf \left\{ \lambda\bigg| \rho_{\Phi}\left(\frac{v}{\lambda}\right) dt\leq 1\right\},
\]
is a Banach space. By $u\b{\cdot} v$ we denote the usual dot product in $\mathbb{R}^{d}$ between $u$ and $v$.


The subspace $\ephi=\ephi([0,T],\rr^d)$ is defined as the closure in $\lphi$ of the subspace $L^{\infty}([0,T],\rr^d)$ of all $\mathbb{R}^d$-valued essentially bounded functions. It is shown that  (see \cite[Thm. 5.1]{Orliczvectorial2005}) $u\in\ephi$  if and only if $\rho_{\Phi}(\lambda u)<\infty$ for any $\lambda>0$. The equality $\lphi=\ephi$ is true if and only if $\Phi\in\Delta_2^{\infty}$ (see \cite[Thm. 5.2]{Orliczvectorial2005}). Another alternative characterization of $\ephi$, which is particularly useful for us, is that $u\in\ephi$ if and only if $u$ has  \emph{absolutely continuous norm}, i.e.   if $E_n\subset [0,T]$, $n=1,2,\ldots$ then $\|\chi_{E_n}u\|\to 0$ when $|E_n|\to 0$.

A generalized version of \emph{H\"older's inequality} holds in Orlicz spaces (see \cite[Thm. 4.1]{Skaff1969}). Namely, if $u\in\lphi$ and $v\in\lpsi$ then $u\cdot v\in L^1$ and
\begin{equation}\label{holder}
\int_0^Tv\cdot u\ dt\leq 2 \|u\orlnor\|v\|_{L^{\Phie}}.
\end{equation}


Like in \cite{KR} we will consider the subset $\Pi(\ephi,r)$ of $\lphi$ given by
\[\Pi(\ephi,r):=\{u\in\lphi| d(u,\ephi)<r\}.\]
This set is related to the Orlicz class $\claseor$ by means of inclusions, namely,
\begin{equation}\label{inclusiones}\Pi(\ephi, r )\subset r \claseor\subset\overline{\Pi(\ephi,r)}
\end{equation}
for any positive $r$ (see \cite[Thm. 5.6]{Orliczvectorial2005}).
If $\Phi \in \Delta_2^{\infty}$,  then the sets $\lphi$, $\ephi$, $\Pi(\ephi,r)$ and $\claseor$ are equal.

Following to \cite{Desch2001} we introduce the next definition.

\begin{defi} Let $u_n,u\lphi([0,T],\rr^d)$. We say that $u_n$ converges monotonically to $u$ if there exists $\alpha_n\in\linf([0,T],\rr^d)$, $n=1,2,\ldots$, such that $0\leq \alpha_n(t)\leq \alpha_{n+1}(t)$, $\alpha_n(t)\to 1$ a.e., when $n\to\infty$ and $u_n(t)=\alpha_n(t)u(t)$.

\end{defi}

 
As usual, if $(X,\|\cdot\|_X)$ is a normed space and $(Y,\|\cdot \|_Y)$ is a linear subspace of $X$,  we write $Y\hookrightarrow X$ and we say that $Y$ is \emph{embedded} in $X$  when there exists $C>0$ such that
$\|y\|_X\leq C\|y\|_Y$ for any $y\in Y$.  With this notation, H\"older's inequality states that  $\lpsi\hookrightarrow  \left[\lphi\right]^*$, where a function $v\in\lpsi$ is associated  to $F_v\in \left[\lphi\right]^*$ where
\begin{equation}\label{pairing}
  F_v(u):=\langle v,u\rangle=\int_0^Tv\cdot u\ dt,
\end{equation}
 In  \cite[Thm 2.9]{Desch2001}  it was characterized a subspace of   $\left[\lphi\right]^*$ which is identified with $\lpsi$. Namely $\lpsi=P^{\Phie}([0,T],\rr^d)$ where $F\in P^{\Phie}([0,T],\rr^d)$ if and only if $F\in\left[\lphi\right]^*$ and satisfying the \emph{monotone convergence property}, which is if $u_n$ converges monotonically to $u$ then $F(u_n)\to F(u)$.

 If $\Phi \in \Delta_2^{\infty}$ and $\Phi$ is coercive then $\lpsi([0,T],\rr^d)= \left[\lphi([0,T],\rr^d)\right]^*$ is satisfied (see \cite[Thm. 2.9 , Thm. 2.10]{Desch2001}).

% It is easy to see that, for every coercive $\Phi$ we have that $L^{\infty}\hookrightarrow\lphi \hookrightarrow L^1$.




We define the \emph{Sobolev-Orlicz space} $\wphi$ by
\[\wphi([0,T],\rr^d):=\{u| u \hbox{ is absolutely continuous on $[0,T]$ and } u'\in \lphi([0,T],\rr^d)\}.\]
$\wphi([0,T],\rr^d)$ is a Banach space when equipped with the norm
\begin{equation}\label{def-norma-orlicz-sob}
\|  u  \|_{\wphi}= \|  u  \|_{\lphi} + \|u'\orlnor.
\end{equation}
And, we introduce the following subspaces of $\wphi$
%%
\begin{equation}\label{def-esp-orlicz-sob-per}
\begin{split}
\wphie&=\{u\in\wphi|u'\in\ephi\},\\
\wphie_T&=\{u\in\wphie|u(0)=u(T)\}.
\end{split}
\end{equation}

%
%
 We will use repeatedly the decomposition $u=\overline{u}+\widetilde{u}$ for a function $u\in L^1([0,T])$  where $\overline{u} =\frac1T\int_0^T u(t)\ dt$ and $\widetilde{u}=u-\overline{u}$.

 The following lemma is an elementary generalization to anisotropic Sobolev-Orlicz spaces of known results of Sobolev spaces.


%
%
%  Recall that a function   $w:\mathbb{R}^+\to \mathbb{R}^+$ is called  a \emph{modulus of continuity} if $w$ is a continuous increasing function which satisfies $w(0)=0$. For example, it can be easily shown that $w(s)=s\Phi^{-1}(1/s)$ is a modulus of  continuity for every $N$-function $\Phi$.  It is said that $u:[0,T]\to\rr^d$  has modulus of continuity $w$  when there exists a constant $C>0$ such that
% \begin{equation}\label{w-holder}|u(t)-u(s)|\leq Cw(|t-s|).
% \end{equation}
%
%
% We denote by $C^w([0,T],\rr^d)$  the space of  $w$-H\"older continuous functions that satisfy  \eqref{w-holder} for some $C>0$.
% This is a Banach space with norm
% \[\|u\|_{  C^w([0,T],\rr^d) }  :=\|u\|_{L^{\infty}}+\sup\limits_{t\neq s}\frac{|u(t)-u(s)|}{w(|t-s|)}.\]
%
%
%
%
% The following simple  embedding lemma, whose proof can be found in \cite{ABGMS2015}, will be used systematically.
%
%


\begin{lem}\label{inclusion orlicz} Let $\Phi:\rr^d\to [0,+\infty)$ be a Young's 
function and let $u\in\wphi([0,T],\rr^d)$. Let 
$F: \rr^+ \to \rr^+$ be the function defined by \eqref{eq:inversa-gral}. Then 
\begin{enumerate}
\item\label{inclusion orlicz_item1} For every $s,t\in [0,T]$, $s\neq t$,
\begin{align}
 &|u(t)-u(s)| \leq
 \|u'\orlnor |s-t|F\left(\frac{1}{|s-t|}\right)\tag{Morrey's inequality}\label{in-sob-cont}
\\
& \| u\linf \leq F\left(\frac{1}{T}\right)\max\{1,T\}\|u\sobnor\tag{Sobolev's inequality}\label{sobolev}
\end{align}
\item We have $\widetilde{u}\in L^{\infty}([0,T],\rr^d)$ and
\[
\|\widetilde u \linf \leq T F\left(\frac{1}{T}\right)\|u'\orlnor
\tag{Sobolev-Wirtinger's inequality}\label{wirtinger}
\]
\item\label{it:embeding} The space $\wphi([0,T],\rr^d)$ is compactly embedded in the space of continuous functions $C([0,T],\rr^d)$.
\end{enumerate}
\end{lem}

\begin{proof} By the absolutely continuity of $u$, Jensen's inequality and the definition of 
the Luxemburg norm, we have

\[
 \begin{split}
    \Phi\left( \frac{u(t)-u(s)}{\|u'\orlnor |s-t|}\right) &\leq  \Phi\left( \frac{1}{ |s-t|}\int_s^t  \frac{u'(r)}{\|u'\orlnor }dr\right)\\
    &\leq   \frac{1}{ |s-t|}\int_s^t  \Phi\left(\frac{u'(r)}{\|u'\orlnor }\right)dr
    \leq \frac{1}{ |s-t|}.
 \end{split}
\]
By \eqref{eq:inversa-gral} we get
\[
    \frac{|u(t)-u(s)|}{\|u'\orlnor |s-t|} 
    \leq  F\left(\frac{1}{ |s-t|}\right),
\]
then  \ref{inclusion orlicz_item1} holds.

\ref{in-sob-cont}  implies \ref{wirtinger} according to the following argument.  
Taking into account that $\alpha F(1/\alpha)$ is an increasing function 
with respect to $\alpha\in [0,\infty)$ we have
\[|u(t)-\overline{u}|\leq  \|u'\orlnor T F\left(\frac{1}{T}\right),\]
and Sobolev-Wirtinger's inequality follows easily.

In order to prove the Sobolev's inequality, we note that, using Jensen's inequality and 
the definition of $\|u\orlnor$, we obtain
\[ \Phi\left( \frac{ \overline{u}}{\|u\orlnor} \right) \leq
\frac{1}{T}\int_0^T\Phi\left(\frac{u(s)}{\|u\orlnor}\right)ds\leq\frac{1}{T}
\]
Then
\[|\overline{u}|\leq F\left(\frac{1}{T}\right) \|u\orlnor.\]
Therefore, from this and \eqref{wirtinger} we get

\[\begin{split}
 \|u\linf &\leq |\overline{u}|+\|\tilde{u}\linf\\
 &\leq  
 F\left(\frac{1}{T}\right) \|u\orlnor+T F\left(\frac{1}{T}\right)\|u'\orlnor\\
 &\leq F\left(\frac{1}{T}\right)\max\{1,T\}\|u\sobnor
 \end{split}
 \]
 



We take a bounded sequence
$u_n$ in $\wphi([0,T],\rr^d)$ and suppose that $u_n$ has not convergent subsequence.


\end{proof}




\section{Superposition operators in anisotropic Orlicz spaces}

Vamos escribiendo lo que queremos...(de acuerdo a mis apuntes y sin ver las hojitas de la semana pasada)

For $f:[0,T]\times \rr^d\to\rr$  we denote by $\mathfrak{f}$ the Nemytskii (o superposition) operator defined for functions $u:[0,T]\to\rr^d$ by 
\[\mathfrak{f}u(t)=f(t,u(t))\]

Referencias y alguna propiedad interesante medibles en medibles? \cite{krasnosel2011integral,KR}


\begin{thm}
Let $\Phi_1,\Phi_2,\dots,\Phi_n$ be $N$-functions. 
Assume that $M$ is another $N$-functions that satisfy the $\Delta_2$-condition. 
We write $x=(x_1,\dots,x_n)$  and $y=(y_1,\dots,y_n)$ with $x_i\in \rr^d$, $y_i\in \rr^d$.
Let $f(t,x_1,\ldots,x_n,y_1,\ldots,y_n)$ be a function Charath\'eodory? with $f:[0,T]\times {(\rr^d)}^n\times {(\rr^d)}^n \to \rr^{d'}$.

Suppose that $a:(\rr^d)^n\to [0,+\infty)$ is a bounded function on bounded sets and 
$b \in L^{M}([0,T])$, for a.e. $t \in [0, T]$ such that 
\begin{equation}\label{eq:condicion estru gral}
|f|\leq a(x)[ b(t)+\sum_{i=1}^{n} M^{-1}(\Phi_i(|y_i|))],
\end{equation}
then 
\[
\mathfrak{f}:\left(\prod\limits_{i=1}^n L^{\infty}([0,T],\rr^d)\right) \times \left(\prod\limits_{i=1}^n \Pi(E^{\Phi_i}([0,T],\rr^d),\lambda=1)\right)
\to \lmj.\]
\end{thm}

\begin{proof}
If $(u,v)\in \left(\prod\limits_{i=1}^n L^{\infty}([0,T],\rr^d)\right) \times \left(\prod\limits_{i=1}^n \Pi(E_d^{\Phi_i},\lambda=1)\right)$.
By \cite[ Thm. 17.6]{KR} (y otras cosas), we get 
 \[|\mathfrak{f}u(t)|=|f(t,u(t),v(t))| \leq 
M_a [b_j(t) +\sum_{i=1}^{n} M_j^{-1}(\Phi_i(|v_i(t)|))]
\in
 L_1^{M_j}.\]
\end{proof}

We define the space $X$ by
$X=\{v=(v_1,v_2):v_1 \in W^{1}L^{\Phi_1}_T,v_2\in W^{1}L^{\Phi_2}_T\}$
and 
$X^*=\{v=(v_1,v_2):v_1 \in (W^{1}L^{\Phi_1}_T)^*,v_2\in (W^{1}L^{\Phi_2}_T)^*\}$
where $(W^{1}L^{\Phi_i}_T)^*$ stands for the conjugate space of $W^{1}L^{\Phi_i}_T$ for $i=1,2$.

\begin{cor}
We will consider the Lagrange function
$\mathcal{L}:[0,T]\times\rr^d\times\rr^d\times\rr^d\times\rr^d\to\rr$, $(t,x_1,x_2,y_1,y_2)\to \mathcal{L}(t,x_1,x_2,y_1,y_2)$
which is measurable in $t$ for each $(x_1,x_2,y_1,y_2)\in \rr^d\times\rr^d\times\rr^d\times\rr^d$ and continuously differentiable 
in $(x_1,x_2,y_1,y_2)$ for almost every $t \in [0,T]$. 

Let $x=(x_1,x_2)$, $y=(y_1,y_2)$ with $x_i \in \rr^d$ and $y_i \in \rr^d$ and let 
\begin{equation}\label{integral_accion}
I(x)=\int_{0}^T \mathcal{L}(t,x,y)\ dt
\end{equation}


If there exist $a\in C(\rr^+,\rr^+)$, $i=1,2,$
$b \in L^1_1([0,T])$, $j=1,\dots,d'$ for a.e. $t \in [0, T]$ and 
every $(x_1,x_2,y_1,y_2)\in \rr^d\times\rr^d\times\rr^d\times\rr^d$ satisfying the structure conditions 
% \begin{eqnarray}
% |\mathcal{L}(t,x,y)|+ \sum_{i=1}^2
% |D_{x_i}\mathcal{L}(t,x,y)|&\leq a(|x|)(b(t)+ \Phi_1(|y_1|)+\Phi_2(|y_2|)),\label{cotaL}\\
% |D_{y_i}\mathcal{L}(t,x,y)| &\leq a(|x|)(c_i(t)+\sum_{j=1}^n \Phie^{-1}_i(\Phi_j(|y_j|))\;i=1,2.\label{cotaDyL}
% \end{eqnarray}


The nonlinear operator $(x_1,x_2)\mapsto D_x\mathcal{L}(t,x_1,y_1, y_2)$ is continuous from $\domiuno\times\domidos\times\dots\times\domin$ with the strong topology  into $L^1([0,T])$  with the strong topology on both sets.

The nonlinear operator $(x_1,x_2)\mapsto D_y\mathcal{L}(t,x_1,y_1, y_2)$ is continuous from $\domiuno\times\domidos\times\dots\times\domin$ with the strong topology  into $X$  with the weak$^*$ topology.


The function  $I$ is G\^ateaux differentiable on $\domiuno\times \domidos$ and  its derivative $I'$ is demicontinuous from $\domiuno\times\domidos$  into $X^*$. Moreover, $I'$ is given by the following expression
\begin{equation}\label{DerAccion}
\begin{split}
\left\langle I'(x),w \right\rangle=\int_0^T 
[
(D_{x_1}\mathcal{L}(t,x_1(t),x_2(t),y_1(t),y_2(t)),w_1(t))+
\\
(D_{x_2}\mathcal{L}(t,x_1(t),x_2(t),y_1(t),y_2(t)),w_2(t))+
\\
(D_{y_1}\mathcal{L}(t,x_1(t),x_2(t),y_1(t),y_2(t)),w'_1(t))+
\\
(D_{y_2}\mathcal{L}(t,x_1(t),x_2(t),y_1(t),y_2(t)),w'_2(t))
]\,dt
\end{split}
\end{equation}

If  $\Phie \in \Delta_2$ then
  $I'$ is continuous from $\domiuno\times\domidos$ into $X^*$ when both spaces are equipped with the strong topology.
\end{cor}


We denote by $\mathfrak{A}(a,b,c,\lambda,f,\Phi)$ the set of all Lagrange functions satisfying  \eqref{eq:estru1}, \eqref{eq:estru2} and \eqref{eq:estru3}.


\begin{proof} 

{\bf OJO!!!! Es algo que ten\'iamos del trabajo anterior!!! 
con algunas adaptaciones a 2 variables sin controlar y a lo bruto!!!!!}

Let $u\in \domiuno\times\domidos$.

\noindent\emph{Step 1. The non linear operator  $(x_1,x_2) \mapsto (D_{x_1}\mathcal{L}(t,x_1,x_2,y_1,y_2),
D_{x_1}\mathcal{L}(t,x_1,x_2,y_1,y_2))$ is continuous from $\domiuno\times\domidos$ into $L^{1}_d([0,T])\times L^1_d([0,T])$ with the strong topology on both sets.} 


If $u\in \domiuno\times\domidos$, from \eqref{cotaDxL} and \eqref{inclusion3}, we obtain 
Let   $\{x_n=({x_1}_n,{x_2}_n)\}_{n\in \mathbb{N}}$ be a sequence of  functions in $\domiuno\times\domidos$  and let 
$x=(x_1,x_2)\in \domiuno\times\domidos$  such that $x_n\rightarrow x$ in $X$.
From  ${x_i}_n\rightarrow x_i$ in $\lphii$, there exists a subsequence ${x_i}_{n_k}$ such that ${x_i}_{n_k}\rightarrow x_i \quad\text{a.e.}$; and, as ${x_i}_n\rightarrow x_i \in\domi$, by 
  Lemma \ref{segundo lema}, there exist a subsequence of  ${x_i}_{n_k}$ (again denoted ${x_i}_{n_k}$) and a function  $h_i\in \Pi(\ephi_1,\lambda))$
such that  ${x_i}_{n_k}\rightarrow u_i \quad\text{a.e.}$ and $|{x_i}_{n_k}|\leq h_i\quad\text{a.e}$.  
Since ${x_i}_{n_k}$, $k=1,2,\ldots,$ is a strong convergent sequence in $\wphii_d$, it is a bounded sequence in $\wphii_d$. According to Lemma \ref{inclusion orlicz} and Corollary \ref{a_bound}, there exist $M_i>0$ such that $\|\b{a}({x_i}_{n_k})\|_{L^{\infty}} \leq M_i$, $k=1,2,\ldots$.  From the previous facts and \eqref{DxL1}, we get
\begin{equation*}\label{DxL1-bis}
|D_{x_i}\mathcal{L}(\cdot,{x_1}_{n_k},{x_2}_{n_k},{y_1}_{n_k},{y_2}_{n_k})|\leq 
M_i (b+\Phi_i(|h_i|)) \in L^1_1\;\;i=1,2.
\end{equation*}
On the other hand, by the continuous differentiability of $\mathcal{L}$, we have
\[D_{x_i}\mathcal{L}(t,{x_i}_{n_k}(t),{y_i}_{n_k}(t))\to D_{x_i}\mathcal{L}(t,x_i(t), y_i(t))\quad\hbox{ for a.e. } t\in[0,T].\]
Applying the Dominated Convergence Theorem we conclude the proof of step 1.

\noindent\emph{Step 2. The non linear operator   
$(x_1,x_2)\mapsto (D_{y_1}\mathcal{L}(t,x_1,y_1, D_{y_2}\mathcal{L}(t,x_2,y_2)$ is continuous from $\domiuno\times\domidos$ with the strong topology  into $X$  with the weak$^*$ topology.}


 Note that \eqref{DxL1},  \eqref{DyLpsi} and the imbeddings $\wphi_d \hookrightarrow L_d^{\infty}$ and  $\lpsi_d\hookrightarrow  \left[\lphi\right]^*$ imply that the second member of
\eqref{DerAccion} defines an element in $\left[\wphi_d\right]^*$.

Let $({x_1}_n,{x_2}_n)\in \domi$ such that $({x_1}_n,{x_2}_n)\to (x_1,x_2)$ in the norm of $X$. 
We must prove that  $D_{y_i}\mathcal{L}(\cdot,{x_1}_n,{x_2}_n)\overset{w^*}{\rightharpoonup} 
D_{y_i}\mathcal{L}(\cdot,x_1,x_2,y_1,y_2)$ para $i=1,2$.
On the contrary, there exist $v=(v_1,v_2)\in\lphiuno\times\lphidos$, $\epsilon>0$ and a subsequence of $\{x_n\}$ (denoted  $\{x_n\}$ for simplicity)  such that
\begin{equation}\label{cota_eps}
 \left| \langle D_{y_i}\mathcal{L}(\cdot, {x_1}_n,{x_2}_n,{y_1}_n,{y_2}_n),\
v \rangle - \langle  D_{y_i}\mathcal{L}(\cdot,x_1,x_2, y_1,y_2,v \rangle\right|\geq \epsilon.
\end{equation}
We have $x_n \rightarrow x$ in $X$ and
$y_n\rightarrow y$ in $X$. By Lemma \ref{segundo lema}, 
there exist a subsequence $x_{n_k}$ and a function $h\in \Pi(\ephiuno_1,\lambda)\times\Pi(\ephidos_1,\lambda) $ such that $x_{n_k}\rightarrow x \quad\text{a.e.}$, $y_{n_k}\rightarrow y \quad\text{a.e.}$ and $|y_{n_k}|\leq h\quad\text{a.e.}$ 
As in the previous step, since $x_n$ is a convergent sequence, the Corollary \ref{a_bound} implies that $a(|y_n(t)|)$ is uniformly bounded by a certain constant $M>0$. 
Therefore,  with $x_{n_k}$ instead of $x$, inequality  \eqref{DyLpsi} becomes 
% \begin{equation}\label{Dy-suc}
%   \left | D_{y_i}\mathcal{L}(\cdot,x_{n_k},y_{n_k})  \right|
% 	\leq M_i(c_i+\varphi_i(h_i)+\Phie_i^{-1}(\Phi_j(|y_j|)))\in \lpsii_1.
% \end{equation}
Consequently, as $v \in \lphi$ and employing H\"older's inequality, we obtain that
\[\sup_k|D_{\b{y}}\mathcal{L}(\cdot,u_{n_k},\b{\dot{u}}_{n_k})\ccdot v| \in L^1_1.\]
  Finally, from the Lebesgue Dominated Convergence Theorem, we deduce
\begin{equation}\label{conv_debil}\int_0^T  D_{\b{y}}\mathcal{L}(t,u_{n_k},\b{\dot{u}}_{n_k})\ccdot\b{ v} \ dt \to \int_0^T D_{\b{y}}\mathcal{L}(t,u,\b{\dot{u}})\ccdot\b{ v}\ dt \end{equation}
which contradicts the inequality \eqref{cota_eps}. This completes the proof of step 2.

\emph{Step 3.} We will prove \eqref{DerAccion}. The proof follows similar lines as \cite[Thm. 1.4]{mawhin2010critical}. For $u\in \domi$ and $\b{0}\neq v\in\wphi_d$, we define the function
\[H(s,t):=\mathcal{L}(t,u(t)+sv(t),\b{\dot{u}}(t)+s\b{\dot{v}}(t)).\]

From \cite[Lemma 10.1]{KR} (or \cite[Thm. 5.5]{Orliczvectorial2005} ) we obtain that if $|u|\leq |v|$ then    $d(u,\ephi)\leq d(v,\ephi)$.
Therefore, for  $|s|\leq s_0:=\left(\lambda-d(\b{\dot{u}},\ephi)\right)/\|v\sobnor$ we have
\[
d \left(\b{\dot{u}}+s\b{\dot{v}}, \ephi \right)
\leq
d \left(|\b{\dot{u}}|+s|\b{\dot{v}}|, \ephi_1 \right)
\leq d \left(|\b{\dot{u}}|,\ephi_1 \right)+ s \|\b{\dot{v}}\orlnor < \lambda.
\]
Thus $\b{\dot{u}}+s\b{\dot{v}} \in \Pi(\ephi,\lambda)$ and  $|\b{\dot{u}}|+s|\b{\dot{v}}| \in \Pi(\ephi_1,\lambda)$. These facts imply, in virtue of Theorem \ref{teorema_acotacion} item \ref{T1item1}, that $I(u+sv)$ is well defined and finite for $|s|\leq s_0$.
And, using  Corollary \ref{a_bound}, we also see that
\[ \|a(|u+sv|)\|_{L^{\infty}}\leq  A(\|u+sv\sobnor)\leq
 A(\|u\sobnor+s_0\|v\sobnor)=:M
\]
Now, applying Chain Rule, \eqref{DxL1}, \eqref{DyLpsi} the monotonicity of $\varphi$ and $\Phi$, 
the fact that $v\in L^{\infty}_d$ and $\b{\dot{v}}\in\lphi$ and H\"older's inequality, we get
\begin{equation}\label{ctg}
\begin{split}
|D_s H(s,t)|&=\left| D_{\b{x}}\mathcal{L}(t,u+sv,\b{\dot{u}}+s\b{\dot{v}})\ccdot v +  D_{\b{y}}\mathcal{L}(t,u+sv,\b{\dot{u}}+s\b{\dot{v}})\ccdot\b{\dot{v}}\right| \\
 & \leq M \left[\left( b(t)+ \Phi\left(\frac{|\b{\dot{u}}|+s_0|\b{\dot{v}}|}{\lambda}+f(t)\right)\right)|v|\right.\\
&\left. \quad+ \left(c(t)+ \varphi\left (\frac{|\b{\dot{u}}|+s_0|\b{\dot{v}}|}{\lambda}+f(t)\right)\right)|\b{\dot{v}}| \right]\in L^1_1.
\end{split}
\end{equation}
Consequently, $I$ has a directional derivative and
\[
\langle I'(u),v \rangle=\frac{d}{ds}I(u+sv)\big|_{s=0}=\int_0^T
\left\{D_{\b{x}}\mathcal{L}(t,u,\b{\dot{u}})\ccdot v+ D_{\b{y}}\mathcal{L}(t,u,\b{\dot{u}})\ccdot\b{\dot{v}}\right\} \ dt.
\]
Moreover, from \eqref{DxL1}, \eqref{DyLpsi}, Lemma \ref{inclusion orlicz} and the previous formula, we obtain
\[
|\langle I'(u),v \rangle| \leq \|D_{\b{x}}\mathcal{L}\|_{L^1} \| v\linf +
\|D_{\b{y}}\mathcal{L}\|_{\lpsi} \|\b{\dot{v}}\orlnor \leq C \|v\sobnor
\]
with a appropriate constant $C$.
This completes the proof of the G\^ateaux differentiability of $I$. 

\emph{Step 4. The operator $I':\domi  \to \left[\wphi_d
\right]^* $ is demicontinuous.}
This is a consequence  of the continuity of the mappings $u \mapsto D_{\b{x}}\mathcal{L}(t,u,\b{\dot{u}})$ and $u \mapsto
D_{\b{y}}\mathcal{L}(t,u,\b{\dot{u}})$. Indeed, if $u_n,u\in \domi$ with $u_n\to u$ in the norm of $\wphi_d$ and $v \in
\wphi_d$, then
\[
\begin{split}
\left\langle  I'(u_{n}),v \right\rangle &= \int_0^T \left\{  D_{\b{x}}\mathcal{L}\left(t,u_n,\b{\dot{u}}_n\right)\ccdot
v +
 D_{\b{y}}\mathcal{L}\left(t,u_n,\b{\dot{u}}_n\right)\ccdot\b{\dot{v}}\right\} \ dt\\
&\rightarrow \int_0^T \left\{ D_{\b{x}}\mathcal{L}\left(t,u,\b{\dot{u}}\right)\ccdot v+
D_{\b{y}}\mathcal{L}\left(t,u,\b{\dot{u}}\right)\ccdot\b{\dot{v}}\right\} \ dt\\
&=\left\langle  I'(u),v \right\rangle.
\end{split}
\]


In order to prove item  \ref{T1item4}, it is necessary to see that the maps $u\mapsto D_{\b{x}}\mathcal{L}(t,u,\b{\dot{u}})$  and $u\mapsto D_{\b{y}}\mathcal{L}(t,u,\b{\dot{u}})$  are norm continuous
from $\domi $ into $L^1_d$ and
 $\lpsi_d$ respectively.  The continuity of the first map has already been proved in step 1. 
Let $u_n, u \in \domi$ with $\|u_n- u\sobnor\to 0$. Therefore,   there exist a subsequence $u_{n_k}\in \domi$ and a function $h\in\Pi(\ephi_1,\lambda)$  such that   \eqref{Dy-suc} holds true.
And, as  $\Phie\in\Delta_2$ then   the right hand side of  \eqref{Dy-suc} belongs to $\epsi_1$.
Now, invoking  Lemma \ref{lema_conv_may}, we  prove that
  from any sequence $u_n$ which converges to $u$ in $\wphi_d$ we can
extract a subsequence such that   $D_{\b{y}}\mathcal{L}(t,u_{n_k},\b{\dot{u}}_{n_k})\to D_{\b{y}}\mathcal{L}(t,u,\b{\dot{u}})$ in the strong topology. The desired result is obtained by a standard argument.

The continuity of $I'$  follows  from the continuity 
of $D_{\b{x}}\mathcal{L}$ and $D_{\b{y}}\mathcal{L}$ using the formula \eqref{DerAccion}.
\end{proof}





\section*{Acknowledgments}
The authors are partially supported by a UNRC grant number 18/C417. The first author is  partially supported by a  UNSL grant number 22/F223. 




 \bibliographystyle{apalike}
 \bibliography{biblio}


\end{document}


