\documentclass[final,hyperref={pdfpagelabels=false}]{beamer}
\usepackage{grffile}
\mode<presentation>{\usetheme{I6pd2}}
\usepackage[english]{babel}
\usepackage[latin1]{inputenc}
\usepackage{amsmath,amsthm, amssymb, latexsym}
\usepackage{mathabx}
%\usepackage{times}\usefonttheme{professionalfonts}  % obsolete
%\usefonttheme[onlymath]{serif}
%\boldmath
\usepackage[orientation=portrait,width=900mm,height=1200mm,debug]{beamerposter}
% change list indention level
% \setdefaultleftmargin{3em}{}{}{}{}{}


%\usepackage{snapshot} % will write a .dep file with all dependencies, allows for easy bundling

\usepackage{array,booktabs,tabularx}
\newcolumntype{Z}{>{\centering\arraybackslash}X} % centered tabularx columns
\newcommand{\pphantom}{\textcolor{ta3aluminium}} % phantom introduces a vertical space in p formatted table columns??!!
\newtheorem{thm}{Theorem}
\newtheorem{cor}[thm]{Corollary}
\newtheorem{lem}[thm]{Lemma}
\newtheorem{rem}[thm]{Remark}
\newtheorem{defn}[thm]{Definition}
\newtheorem{prop}[thm]{Proposition}
\newtheorem{exmp}[thm]{Example}
\listfiles

%%%%%%%%%%%%%%%%%%%%%%%%%%%%%%%%%%%%%%%%%%%%%%%%%%%%%%%%%%%%%%%%%%%%%%%%%%%%%%%%%%%%%%
\graphicspath{{figures/}}
 
\title{\huge Métodos Variacionales y Sistemas Hamiltonianos
\vspace{1cm}\\
{Jornadas de Ciencia y Técnica 2016 -  UNLPam\\
20 años de las Jornadas de Ciencia y Técnica}\\
\normalsize{Resolución CD FCEyN N$^\circ$ 38/16 - Período: 01/01/2016-31/12/2018.}
}
\author{{\bf Director}: Fernando Mazzone${}^{(1,2,3)}$.
\\
{\bf Integrantes}: Sonia Acinas${}^{(1)}$, Lorenzo Sierra${}^{(4)}$, Stefanía Demaría${}^{(2,3)}$ y Leopoldo Buri${}^{(2)}$. }
\institute[UNRC]{ 
 ${}^{(1)}$ Dpto. de Matemática, Facultad de Ciencias Exactas y Naturales, Universidad Nacional de La Pampa.\\
 ${}^{(2)}$ Dpto. de Matemática, Universidad Nacional de R\'{\i}o Cuarto.\\
${}^{(3)}$ CONICET.
\\
${}^{(4)}$ Facultad de Ingeniería, Universidad Nacional de La Pampa.
}
\date[14 de Octubre de 2016]{ Jornadas de Ciencia y Técnica 2016}

%%%%%%%%%%%%%%%%%%%%%%%%%%%%%%%%%%%%%%%%%%%%%%%%%%%%%%%%%%%%%%%%%%%%%%%%%%%%%%%%%%%%%%
\newlength{\columnheight}
\setlength{\columnheight}{105cm}


%%%%%%%%%%%%%%%%%%%%%%%%%%%%%%%%%%%%%%%%%%%%%%%%%%%%%%%%%%%%%%%%%%%%%%%%%%%%%%%%%%%%%%
\begin{document}

\newcommand{\orlnor}{\|_{L^{\Phi}}}
\newcommand{\lurnor}{\|^{*}_{L^{\Phi}}}
\newcommand{\linf}{\|_{L^{\infty}}}
\newcommand{\lphi}{L^{\Phi}}
\newcommand{\lpsi}{L^{\Psi}}
\newcommand{\ephi}{E^{\Phi}}
\newcommand{\claseor}{C^{\Phi}}
\newcommand{\wphi}{W^{1}\lphi}
\newcommand{\sobnor}{\|_{W^{1}\lphi}}
\newcommand{\domi}{\mathcal{E}^{\Phi}_d(\lambda)}
\renewcommand{\b}[1]{#1}
\newcommand{\rr}{\mathbb{R}}
\newcommand{\nn}{\mathbb{N}}
\newcommand{\ccdot}{\b{\cdot}}
\renewcommand{\leq}{\leqslant} 
\newcommand{\epsi}{E^{\Psi}}



\newcommand{\pro}{\operatornamewithlimits{\int\hspace{-4.6mm}{\diagup}}}
\newcommand{\es}{\operatornamewithlimits{\hbox{ess sup }}}
\newcommand{\dive}{\hbox{div }}



\begin{frame}
  \begin{columns}
    % ---------------------------------------------------------%
    % Set up a column
    \begin{column}{.49\textwidth}
      \begin{beamercolorbox}[center,wd=\textwidth]{postercolumn}
        \begin{minipage}[T]{.95\textwidth}  % tweaks the width, makes a new \textwidth
          \parbox[t][\columnheight]{\textwidth}{ % must be some better way to set the the height, width and textwidth simultaneously
            % Since all columns are the same length, it is all nice and tidy.  You have to get the height empirically
            % ---------------------------------------------------------%
            % fill each column with content
            \begin{block}{Ecuaciones de Euler-Lagrange}

	      Muchos sistemas físicos se modelan con un sistema de ecuaciones del tipo
	       \begin{equation}\label{ProbPrin}
                    \frac{d}{dt} D_{y}\mathcal{L}(t,u(t),\dot{u}(t))= D_{x}\mathcal{L}(t,u(t),\dot{u}(t)) \quad \hbox{en c.t.p.}\ t \in (0,T)
              \end{equation}
              donde $T>0$, $u:[0,T]\to\rr^d$ y la función   $\mathcal{L}:[0,T]\times\rr^d\times\rr^d\to\rr$ se denomina \emph{Lagrangiano}.




            \end{block}
\begin{block}{Ejemplos}
 \begin{itemize}
  \item Las ecuaciones que gobiernan el movimiento de un cuerpo de masa $m$ sujeta a un  resorte son de la forma \eqref{ProbPrin} con $\mathcal{L}=m\frac{y^2}{2}+k\frac{x^2}{2}$. Aquí $x$ representa el desplazamiento de la masa  desde su posición de equilibrio.
 \item Las ecuaciones que gobiernan el movimiento de una masa sujeta a un péndulo son de la forma \eqref{ProbPrin} con $\mathcal{L}=m\frac{y^2}{2}+\sqrt{\frac{g}{l}}\sin(x)$. En este caso, $x$ representa el desplazamiento angular desde la vertical del péndulo.
 \item Las ecuaciones que gobiernan el movimiento de $n$-cuerpos graves interactuando entre sí a través de sus campos gravitatorios son de la forma \eqref{ProbPrin} con
 \[\mathcal{L}(t,x,y)=\frac12\sum_{i}^nm_i\|y_i\|^2+G\sum_{i<j}\frac{m_im_j}{\|x_i-x_j\|},\]
 con $x=(x_1,\ldots,x_n)$ siendo $x_i(t)\in\mathbb{R}^3$ las posiciones de los cuerpos.
 \item En general, las ecuaciones \eqref{ProbPrin} son características de sistemas mecánicos conservativos. O sea, sistemas donde la energía total $E=T+U$ se conserva, con $T$ la energía cinética, $U$ la energía potencial y $\mathcal{L}=T-U$.

 \end{itemize}

\end{block}


  \begin{block}{Soluciones Periódicas}

  Es importante saber si los sistemas mecánicos poseen soluciones periódicas. 
	Por ejemplo, si se trata del problema de los $n$-cuerpos esta información resulta útil para la planificación de misiones espaciales.

  Soluciones periódicas se hallan resolviendo las ecuaciones del sistema \eqref{ProbPrin} sujeto a las condiciones de contorno
  \begin{equation}\label{eq:cond_contorno} u(0)-u(T)=u'(0)-u'(T)=0.\end{equation}

    \end{block}











  \begin{block}{Objetivo y Metodología}
  En este proyecto pretendemos encontrar soluciones de \eqref{ProbPrin} sujeto a  \eqref{eq:cond_contorno} a través de métodos variacionales.

  Estos métodos se basan en la observación de que las soluciones del problema de contorno
  son puntos críticos de la \emph{integral de acción}
              \begin{equation}\label{integral_accion}
                I(u)=\int_{0}^T \mathcal{L}(t,u(t),\dot{u}(t))\ dt.
              \end{equation}

  Un punto crítico es una función $u$ tal que cuando la integral de acción es perturbada infinitesimalmente $I(u+\epsilon v)$, la integral de acción cambia en un infinitésimo menor al de la perturbación para toda función $v$ que satisface las condiciones de contorno. O sea, 

  \begin{equation}
   \lim_{\epsilon\to 0}\frac{1}{\epsilon}\int_0^T(\mathcal{L}(t,u+\epsilon v,u'+\epsilon v')-\mathcal{L}(t,u,u'))dt=0.
  \end{equation}

Cuando el límite anterior existe (independientemente que de 0) y el resultado del límite es una función lineal y acotada $\Lambda$ de $v$, se dice que $\Lambda$ es la  \textbf{derivada G\^ateaux} de $I$. Decir que $\Lambda$ es función lineal y acotada  es afirmar que $\Lambda$ pertenece al espacio dual del espacio en el que consideramos a $u$.

Un problema importante es hallar condiciones sobre $  \mathcal{L}$ para que $I$ tenga una derivada G\^ateaux.

Hay una variedad muy grande de métodos variacionales: directo,  dual, teorema de punto silla de Rabinowitz, teoría de Morse, teoría de Leray-Schauder, teoremas minimax, etc.

En una primera etapa estamos investigando la aplicabilidad del método directo y del método dual a cierto tipo especial de Lagrangianos.




 \end{block}
%==============================================================================
 \begin{block}{Método directo}
  Se basa en el hecho de que los puntos extremos de $I$ son puntos críticos. 
	Es así que, pretendemos determinar condiciones que garanticen la existencia de puntos extremos, normalmente mínimos.

 El éxito del método radica en establecer: 
\begin{itemize}
\item \textbf{semicontinuidad inferior} de $I$, o sea, si $u_n\to u$ entonces
% \[
$I(u)\leq \liminf_{n\to\infty}I(u_n);$
%\]
\item \textbf{coercitividad} de $I$, esto es,
% \[
$\lim_{\|u\|\to\infty} I(u)=\infty.$
%\]
\end{itemize}



\end{block}


 %         \vfill

%===========================================================================
   \begin{block}{Lagrangianos, integrales de acción y sus dominios}
Nuestro interés es estudiar 
Lagrangianos que satisfacen las siguientes condiciones estructurales
 \begin{equation}\label{cotaL}
                |\mathcal{L}(t,x,y)| \leq a(|x|)\left(b(t)+ \Phi\left(\frac{|y|}{\lambda}+f(t) \right)\right)
              \end{equation}
     \begin{equation}\label{cotaDxL}
|D_{x}\mathcal{L}(t,x,y)| \leq a(|x|)\left(b(t)+ \Phi\left(\frac{|y|}{\lambda}+f(t) \right)\right),
\end{equation}
  \begin{equation}\label{cotaDyL}
|D_{y}\mathcal{L}(t,x,y)| \leq a(|x|)\left(c(t)+ \varphi\left(\frac{|y|}{\lambda}+f(t)\right)  \right).
\end{equation}

Aquí hemos introducido varios entes nuevos, a saber

              \begin{itemize}
              \item $\lambda$ un número positivo.

              \item  $a:\mathbb{R}^+\to \mathbb{R}^+$ continua.

              \item  $\Phi$ es una $N$-función (este es un concepto  técnico y remitimos a \cite{KR} para su tratamiento) y $\varphi=\Phi'$.

              \item $b\in L^1([0,T])$, $c\in \lpsi$ y $f\in\ephi$, donde $\Psi$ es la $N$-función complementaria de $\Phi$ y $L^1$ es el espacio de Banach clásico de funciones integrables y $\lphi$, $\lpsi$ y $\ephi$ son espacios de Orlicz y ciertos subespacios de ellos (ver \cite{KR} ).

              \item $f\in\ephi$.

              \end{itemize}

Consideramos este tipo de estructura para los Lagrangianos porque generaliza  la tratada previamente en la literatura y por nuestra experiencia en el trabajo con $N$-funciones.

Las condiciones de estructura \eqref{cotaL},\eqref{cotaDxL} y \eqref{cotaDyL} garantizan que nuestras integrales de acción tienen como dominios espacios de Sobolev-Orlicz \cite{adams_sobolev, ABGMS2015}.
\end{block}

\begin{block}{Algunos resultados}
Presentaremos resultados de derivabilidad G\^ateaux y resultados de coercitividad. 

La semicontinuidad inferior de las integrales de acción se obtiene a partir de resultados generales que esencialmente ya existen en la literatura. 


\end{block}



            \vfill
}
\end{minipage}
\end{beamercolorbox}
\end{column}







% ---------------------------------------------------------%
    % end the column

    % ---------------------------------------------------------%
    % Set up a column
    \begin{column}{.49\textwidth}
      \begin{beamercolorbox}[center,wd=\textwidth]{postercolumn}
        \begin{minipage}[T]{.95\textwidth} % tweaks the width, makes a new \textwidth
          \parbox[t][\columnheight]{\textwidth}{ % must be some better way to set the the height, width and textwidth simultaneously
            % Since all columns are the same length, it is all nice and tidy.  You have to get the height empirically
            % ---------------------------------------------------------%
            % fill each column with content


      %      \vfill
%===========================PUNTOS CRITICOS================================



\begin{block}{Derivabilidad G\^ateaux}
\begin{minipage}[T]{.9\textwidth}
\begin{block}{Teorema \cite{ABGMS2015}}
Sea  $\mathcal{L}$ Lagrangiano que satisface \eqref{cotaL}, \eqref{cotaDxL} y \eqref{cotaDyL}.
Entonces $I$ es G\^ateaux diferenciable sobre $W^1E^{\Phi}$ y  
\[\left\langle I'(u),v\right\rangle=
\int_0^T
\{
D_x\mathcal{L}(t,u(t),u'(t))\cdot v(t) +
D_y\mathcal{L}(t,u(t),u'(t))\cdot v'(t) 
\}\,dt.
\]
\end{block}

De este resultado se infiere que los puntos críticos de \eqref{integral_accion}
son solución de \eqref{ProbPrin}.

\end{minipage}
\end{block}



%=============================

            \begin{block}{Coercitividad}
Sea  $\mathcal{L}$ Lagrangiano que satisface \eqref{cotaL}, \eqref{cotaDxL} y \eqref{cotaDyL}.

Además consideramos 
\begin{equation}\label{cota_inf}
\mathcal{L}(t,x,y)\geq \Phi\left(|y|\right)+ F(t,x),
\end{equation}
donde $F:\rr\times\rr^d\to\rr$ tal que$F(t,x)$ es  medible con respecto a $t$ para cada  $x\in\rr^d$ fijo y $F$ es continua en  $x$ para c.t.p. $t\in [0,T]$. 

Y supongamos  que 
\begin{equation}\label{propiedad1coercividad}
\int_{0}^{T}F(t,x)\ dt \rightarrow \infty \quad \hbox{as} \quad |x|\rightarrow \infty.
\end{equation}

Entonces, la integral de acción $I$ es coercitiva si

\begin{minipage}[T]{.9\textwidth}
\begin{block}{Teorema \cite{ABGMS2015}: Coercitividad I}\label{coercitividad1}
\begin{itemize} 
\item existen $b_0\in L^1([0,T])$ y $a:\mathbb{R}^+\to \mathbb{R}^+$ continua tales que 
\begin{equation}\label{condA1} |F(t,x)|\leq a(|x|)b_0(t),\quad\text{en c.t.p. }t\in [0,T] \quad\text{y para cada } x\in\rr^d;
\end{equation}
\item existen una función no negativa  $b_1 \in L^1([0,T])$ y constantes $0<\mu<\alpha_{\Phi}$ 
tales que para cualquier  $x_1,x_2\in\rr^d$ y en c.t.p. $t\in [0,T]$
\begin{equation}\label{holder_cont}
  \left| F(t,x_2)- F(t,x_1) \right|\leq b_1(t)(1+|x_2-x_1|^{\mu}),
\end{equation}
% con $\mu< \alpha_{\Phi}$,  
%en el caso que $\Psi\in\Delta_2$; y,  $\mu=1$ si $\Psi$ es una $N$-función arbitraria;
\item\label{hipot_coer}  $\Psi\in\Delta_2$.
%or, alternatively, $\alpha_0^{-1}T\Phi^{-1}\left(1/T\right)\|b_1\|_{L^1}\Lambda<1$.
\end{itemize}
%Entonces, la integral de acción $I$ es coercitiva.
\end{block}
\end{minipage}

\begin{minipage}[T]{.9\textwidth}
\begin{block}{Teorema \cite{acmaz2015}: Coercitividad II}
\begin{itemize}
\item existen $a:\mathbb{R}^+\to \mathbb{R}^+$ continua, no decreciente y $0\leq b \in L^{1}([0,T],\mathbb{R})$ tal que 
\begin{equation}\label{item:condicion_cya}
|F(t,x)|+|\nabla F(t,x)|\leq a(|x|)b(t),\tag{$C\; y\; A$}
\end{equation}
\item existen $b_1,b_2 \in L^1([0,T])$ y ciertas N-funciones $\Phi_0$ relacionadas con $\Phi$ tales que 
\begin{equation}\label{holder_cont-mu}
  \left| \nabla F(t,x) \right|\leq b_1(t)\Phi_0'(|x|)+b_2(t).
  \tag{$A_5$}
\end{equation}
\end{itemize}
\end{block}
\end{minipage}
\end{block}




\begin{block}{Herramientas para coercitividad}
	\begin{minipage}[T]{.9\textwidth}
	\begin{block}{Lema CM$\rho$: Coercitividad del modular $\rho_{\Phi}(u)$}
	\label{lem_coer}
	Sean  $\Phi,\Psi$  $N$-funciones complementarias con $\Psi \in \Delta_2^{\infty}$. 

Entonces existe una $N$-funci\'on $\Phi^*$ con $\Phi^*\prec\Phi$,
tal que para cada $N$-funci\'on $\Phi_0$ que satisface 
$\Phi_0\llcurly\Phi^*$ 
y para cada 
$k>0$, tenemos
\begin{equation}\label{eq:coer_mod}
\lim\limits_{\|u\orlnor\to \infty}
\frac{\int_0^T \Phi(|u|)\,dt}{\Phi_0(k\|u\orlnor)}=\infty.
\end{equation}
Rec\'iprocamente, si  \eqref{eq:coer_mod} vale para alguna $N$-funci\'on $\Phi_0$,  entonces $\Psi\in\Delta_2^{\infty}$.
	\end{block}
	\end{minipage}
	\end{block}



\begin{block}{GPS: Dónde estamos y hacia dónde vamos}
\begin{minipage}[T]{.9\textwidth}

\begin{itemize}
\item Espacios de Orlicz anisotrópicos. Nos interesa estudiar funciones con valores en $\rr^d$  en espacios normados cuya norma es disímil respecto a la dirección  (anisotropía). Esto hemos visto que tiene aplicaciones al siguiente punto.
\item Método dual. Este método consiste en pasar las ecuaciones de Euler-Lagrange a ecuaciones Hamiltonianas, y considerar, para ellas, la llamada acción dual . Este método provee soluciones del las ecuaciones originales (dualización de Clarke). El método fue empleado con éxito  en espacios de Sobolev $W^{1,p}$ (\cite{Tian2007192}). Queremos generalizar estos resultados.

\end{itemize}
\end{minipage}
\end{block}





  %\end{block}
         %   \vfill
				%=================================================MAIN RESULT=================================================
           
\begin{block}{Bibliografía}
  \bibliographystyle{apalike}
 \bibliography{biblio}
\end{block}




                      \vfill

          }
          % ---------------------------------------------------------%
          % end the column
        \end{minipage}
      \end{beamercolorbox}
    \end{column}
    % ---------------------------------------------------------%
    % end the column
  \end{columns}

\end{frame}

\end{document}


%%%%%%%%%%%%%%%%%%%%%%%%%%%%%%%%%%%%%%%%%%%%%%%%%%%%%%%%%%%%%%%%%%%%%%%%%%%%%%%%%%%%%%%%%%%%%%%%%%%%
%%% Local Variables: 
%%% mode: latex
%%% TeX-PDF-mode: t
%%% End:
