
\documentclass[twoside]{article}


\NeedsTeXFormat{LaTeX2e}
\ProvidesPackage{mathscinet}[2002/04/17 v1.05]
\RequirePackage{textcmds}\relax
\ProvideTextCommandDefault{\cprime}{\tprime}



%\usepackage{hyperref}
\usepackage{amssymb,amsthm}
\usepackage{amsmath}
\usepackage{color}
\usepackage{ esint }

\usepackage{fancyhdr}
\usepackage{times}

\usepackage[latin1]{inputenc}

\usepackage{comment}
\usepackage{url}
\usepackage{xcolor}
\usepackage{adjustbox}
\newtheorem{thm}{Theorem}[section]
\newtheorem{cor}[thm]{Corollary}
\newtheorem{lem}[thm]{Lemma}
\newtheorem{rem}[thm]{Remark}
\newtheorem{defi}[thm]{Definition}
\newtheorem{prop}[thm]{Proposition}
\theoremstyle{remark}
\newtheorem{comentario}{Remark}


\title{Periodic solutions of 
Euler-Lagrange equations with ``sublinear nonlinearity'' in an Orlicz-Sobolev space setting}
\author{Sonia Acinas \thanks{SECyT-UNRC, UNSL and CONICET}\\
Instituto de Matem\'atica Aplicada San Luis (CONICET-UNSL)\\
(5700) San Luis, Argentina\\
Universidad Nacional de La Pampa\\
(6300) Santa Rosa, La Pampa, Argentina\\
\url{sonia.acinas@gmail.com}\\[3mm]
Fernando D. Mazzone \thanks{SECyT-UNRC and CONICET}\\
Dpto. de Matem\'atica, Facultad de Ciencias Exactas, F\'{\i}sico-Qu\'{\i}micas y Naturales\\
Universidad Nacional de R\'{i}o Cuarto\\
(5800) R\'{\i}o Cuarto, C\'ordoba, Argentina,\\
\url{fmazzone@exa.unrc.edu.ar}
}

\date{}

\newcommand{\orlnor}{\|_{L^{\Phi}}}
\newcommand{\lurnor}{\|^{*}_{L^{\Phi}}}
\newcommand{\linf}{\|_{L^{\infty}}}
\newcommand{\lphi}{L^{\Phi}}
\newcommand{\lpsi}{L^{\Psi}}
\newcommand{\ephi}{E^{\Phi}}
\newcommand{\claseor}{C^{\Phi}}
\newcommand{\wphi}{W^{1}\lphi}
\newcommand{\sobnor}{\|_{W^{1}\lphi}}
\newcommand{\domi}{\mathcal{E}^{\Phi}_d(\lambda)}
\renewcommand{\b}[1]{\boldsymbol{#1}}
\newcommand{\rr}{\mathbb{R}}
\newcommand{\nn}{\mathbb{N}}
\newcommand{\ccdot}{\b{\cdot}}
\renewcommand{\leq}{\leqslant} 
\newcommand{\epsi}{E^{\Psi}}

\begin{document}



\maketitle
%
\begingroup%Locallizing the change to `thefootnote'.
    \renewcommand{\thefootnote}{}%Removing the footnote symbol.
    %
    \footnotetext{%
    %   2010 Mathematics Subject Classification
    %   http://www.ams.org/msc/
    \textbf{2010  AMS Subject Classification.} Primary: .
    Secondary: .
    }%
        \footnotetext{%
    \textbf{Keywords and phrases.}  .
    }%
    \endgroup
%
%
%
%

\begin{abstract}
In this paper we....
\end{abstract}




\pagestyle{fancy} \headheight 35pt \fancyhead{} \fancyfoot{}

\fancyfoot[C]{\thepage} \fancyhead[CE]{\nouppercase{S. Acinas and F.D. Mazzone }} \fancyhead[CO]{\nouppercase{\section}}

\fancyhead[CO]{\nouppercase{\leftmark}}


%\tableofcontents




\section{Introduction}
This paper is concerned with the existence of periodic solutions of the problem
\begin{equation}\label{ProbPrin}
    \left\{%
\begin{array}{ll}
   \frac{d}{dt} D_{y}\mathcal{L}(t,\b{u}(t),\b{\dot{u}}(t))= D_{\b{x}}\mathcal{L}(t,\b{u}(t),\b{\dot{u}}(t)) \quad \hbox{a.e.}\ t \in (0,T)\\
    \b{u}(0)-\b{u}(T)=\b{\dot{u}}(0)-\b{\dot{u}}(T)=0
\end{array}%
\right.
\end{equation}
where $T>0$, $\b{u}:[0,T]\to\rr^d$ is absolutely continuous and the \emph{Lagrangian} $\mathcal{L}:[0,T]\times\rr^d\times\rr^d\to\rr$ is a Carath\'eodory function satisfying the conditions
\begin{eqnarray}
|\mathcal{L}(t,\b{x},\b{y})| &\leq a(|\b{x}|)\left(b(t)+ \Phi\left(\frac{|\b{y}|}{\lambda}+f(t) \right)\right),\label{cotaL}\\
|D_{\b{x}}\mathcal{L}(t,\b{x},\b{y})| &\leq a(|\b{x}|)\left(b(t)+ \Phi\left(\frac{|\b{y}|}{\lambda}+f(t) \right)\right),\label{cotaDxL}\\
|D_{\b{y}}\mathcal{L}(t,\b{x},\b{y})| &\leq a(|\b{x}|)\left(c(t)+ \varphi\left(\frac{|\b{y}|}{\lambda}+f(t)\right)  \right).\label{cotaDyL}
\end{eqnarray}
In these inequalities we assume that  $a\in C(\mathbb{R}^+,\mathbb{R}^+)$, $\lambda>0$, $\Phi$ is an $N$-function (see section  Preliminaries  for definitions), $\varphi$ is the right continuous derivative of $\Phi$. The non negative functions $b,c$ and $f$ satisfy that  $b\in L^1_1([0,T]) $,  $c\in\lpsi_1([0,T])$ and  $f\in \ephi_1([0,T])$, where  the Banach spaces $ L^1_1([0,T]), \lpsi_1([0,T])$ and  $\ephi_1([0,T])$  will be defined later.


It is well known that problem \eqref{ProbPrin} comes from a variational one, that is,  a solution of \eqref{ProbPrin}  
is a critical point of the \emph{action integral}
\begin{equation}\label{integral_accion}
I(\b{u})=\int_{0}^T \mathcal{L}(t,\b{u}(t),\b{\dot{u}}(t))\ dt.
\end{equation}




\section{Preliminaries}\label{preliminares}

For reader convenience, we give a short introduction to Orlicz and Orlicz-Sobolev spaces of vector valued functions and a  list  of results that we will use throughout the article. 
Classic references for Orlicz spaces of real valued functions are \cite{adams_sobolev,KR,rao1991theory}.
For  Orlicz spaces of vector valued functions, see \cite{Orliczvectorial2005} and the references therein.

Hereafter we denote  by $\mathbb{R}^+$  the set of all non negative real numbers. A function $\Phi:\mathbb{R}^+\to \mathbb{R}^+ $ is called an \emph{$N$-function} if $\Phi$ is given by 
\[
\Phi(t)=\int_{0}^t \varphi(\tau)\ d\tau,\quad\hbox{for } t\geq 0,
\]
where $\varphi:\mathbb{R}^+\rightarrow \mathbb{R}^+$ is a right continuous non decreasing function  satisfying   $\varphi(0)=0$, $\varphi(t)>0$ for $t>0$ and
$\lim_{t\rightarrow \infty}\varphi(t)=+\infty$.

Given a function $\varphi$ as above, we  consider the so-called right inverse function $\psi$ of $\varphi$ which is 
defined by $\psi(s)=\sup_{\varphi(t)\leq s}t$.
The function $\psi$ satisfies the same properties as the function $\varphi$, therefore we have an $N$-function $\Psi$ such that $\Psi'=\psi$ .
 The function $\Psi$ is called the \emph{complementary function} of $\Phi$.


We say that $\Phi$ satisfies the  \emph{$\Delta_2$-condition}, denoted by $\Phi \in \Delta_2$, 
if there exist  constants $K>0$ and  $t_0\geq 0$ such that 
\begin{equation}\label{delta2defi}\Phi(2t)\leq K\Phi(t)
\end{equation}
for every $t\geq t_0$. 
If $t_0=0$,  we say that $\Phi$ satisfies the \emph{$\Delta_2$-condition globally} ($\Phi \in \Delta_2$ globally).  

% and plain symbols indicate scalars.

Let $d$ be a positive integer. We denote by $\mathcal{M}_d:=\mathcal{M}_d([0,T])$ the set of all measurable functions defined on $[0,T]$ with values on $\mathbb{R}^d$ and  we write $\b{u}=(u_1,\dots,u_d)$ for  $\b{u}\in \mathcal{M}_d$.
In this paper we adopt the convention that bold symbols denote points in $\mathbb{R}^d$.


Given  an $N$-function $\Phi$ we define the \emph{modular function} 
$\rho_{\Phi}:\mathcal{M}_d\to \mathbb{R}^+\cup\{+\infty\}$ by
\[\rho_{\Phi}(\b{u}):= \int_0^T \Phi(|\b{u}|)\ dt.\]
Here $|\cdot|$ is the euclidean norm of $\mathbb{R}^d$.
The \emph{Orlicz class} $C_d^{\Phi}=C_d^{\Phi}([0,T])$  is given  by
\begin{equation}\label{claseOrlicz}
  C^{\Phi}_d:=\left\{\b{u}\in \mathcal{M}_d | \rho_{\Phi}(\b{u})< \infty \right\}.
\end{equation}
The \emph{Orlicz space} $\lphi_d=L^{\Phi}_d([0,T])$ is the linear hull of $\claseor_d$;
equivalently,
\begin{equation}\label{espacioOrlicz}
\lphi_d:=\left\{ \b{u}\in \mathcal{M}_d | \exists \lambda>0: \rho_{\Phi}(\lambda \b{u}) < \infty   \right\}.
\end{equation}
  The Orlicz space $\lphi_d$ equipped with the \emph{Orlicz norm}
\[
\|  \b{u}  \orlnor:=\sup \left\{  \int_0^T \b{u}\b{\cdot} \b{v}\ dt \big| \rho_{\Psi}(\b{v})\leq 1\right\},
\]
is a Banach space. By $\b{u}\b{\cdot} \b{v}$ we denote the usual dot product in $\mathbb{R}^{d}$ between $\b{u}$ and $\b{v}$.  
The following alternative expression for the norm, known as \emph{Amemiya norm},     will  be useful (see \cite[Thm. 10.5]{KR} and \cite{hudzik2000amemiya}). For every $\b{u}\in\lphi$,

\begin{equation}\label{amemiya}
\|\b{u}\orlnor=\inf\limits_{k>0}\frac{1}{k}\left\{1+\rho_{\Phi}(k\b{u})\right\}.
\end{equation}



The subspace $\ephi_d=\ephi_d([0,T])$ is defined as the closure in $\lphi_d$ of the subspace $L^{\infty}_d$ of all $\mathbb{R}^d$-valued essentially bounded functions. It is shown that  $\ephi_d$ is the only one maximal subspace contained in the Orlicz class $\claseor_d$, i.e. 
$\b{u}\in\ephi_d$ if and only if $\rho_{\Phi}(\lambda \b{u})<\infty$ for any $\lambda>0$.  

A generalized version of \emph{H\"older's inequality} holds in Orlicz spaces (see \cite[Th. 9.3]{KR}). Namely, if $\b{u}\in\lphi_d$ and $\b{v}\in\lpsi_d$ then $\b{u}\ccdot\b{v}\in L_1^1$ and
\begin{equation}\label{holder}
\int_0^T\b{v}\ccdot\b{u}\ dt\leq \|\b{u}\orlnor\|\b{v}\|_{L^{\Psi}}.
\end{equation}




If $X$ and $Y$ are  Banach spaces such that  $Y\subset X^*$, we denote by $\langle\cdot,\cdot\rangle:Y\times X\to\mathbb{R}$ the bilinear pairing  map given by $\langle x^*,x\rangle=x^*(x)$. H\"older's inequality shows that $\lpsi_d\subset \left[\lphi_d\right]^*$, where the pairing  
$\langle \b{v}, \b{u}\rangle$
is defined by 
\begin{equation}\label{pairing}
  \langle \b{v},\b{u}\rangle=\int_0^T\b{v}\ccdot\b{u}\ dt
\end{equation}
with  $\b{u}\in\lphi_d$ and $\b{v}\in\lpsi_d$.
 Unless $\Phi \in \Delta_2$, the relation $\lpsi_d= \left[\lphi_d\right]^*$ will not hold. In general, it is true  that  $\left[\ephi_d\right]^*=\lpsi_d$.


Like in \cite{KR}, we will consider the subset $\Pi(\ephi_d,r)$ of $\lphi_d$ given by
\[\Pi(\ephi_d,r):=\{\b{u}\in\lphi_d| d(\b{u},\ephi_d)<r\}.\]
This set is related to the Orlicz class $\claseor_d$ by means of inclusions, namely,
\begin{equation}\label{inclusiones}\Pi(\ephi_d, r )\subset r \claseor_d\subset\overline{\Pi(\ephi_d,r)}
\end{equation}
for any positive $r$.
If $\Phi \in \Delta_2$,  then the sets $\lphi_d$, $\ephi_d$, $\Pi(\ephi_d,r)$ and $\claseor_d$ are equal.



We define the \emph{Sobolev-Orlicz space} $\wphi_d$ (see \cite{adams_sobolev}) by
\[\wphi_d:=\{\b{u}| \b{u} \hbox{ is absolutely continuous and } \b{\dot{u}}\in \lphi_d\}.\]
$\wphi_d$ is a Banach space when equipped with the norm
\[
\|  \b{u}  \|_{\wphi}= \|  \b{u}  \|_{\lphi} + \|\b{\dot{u}}\orlnor.
\]



For a  function $\b{u}\in L^1_d([0,T])$, we write $\b{u}=\overline{\b{u}}+\widetilde{\b{u}}$ where $\overline{\b{u}} =\frac1T\int_0^T \b{u}(t)\ dt$ and $\widetilde{\b{u}}=\b{u}-\overline{\b{u}}$.

As usual, if $(X,\|\cdot\|_X)$ is a Banach space and $(Y,\|\cdot \|_Y)$ is a subspace of $X$,  we write $Y\hookrightarrow X$ and we say that $Y$ is \emph{embedded} in $X$  when the restricted identity map $i_Y:Y\to X$ is bounded. That is, there exists $C>0$ such that  for any $y\in Y$ we have $\|y\|_X\leq C\|y\|_Y$.  With this notation, H\"older's inequality states that  $\lpsi_d\hookrightarrow  \left[\lphi_d\right]^*$; and, it is easy to see that for every $N$-function $\Phi$ we have that $L^{\infty}_d\hookrightarrow\lphi_d \hookrightarrow L^1_d$.


 Recall that a function   $w:\mathbb{R}^+\to \mathbb{R}^+$ is called  a \emph{modulus of continuity} if $w$ is a continuous increasing function which satisfies $w(0)=0$. For example, it can be easily shown that $w(s)=s\Phi^{-1}(1/s)$ is a modulus of  continuity for every $N$-function $\Phi$.  We say that $\b{u}:[0,T]\to\rr^d$  has modulus of continuity $w$  when there exists a constant $C>0$ such that 
\begin{equation}\label{w-holder}|\b{u}(t)-\b{u}(s)|\leq Cw(|t-s|).
\end{equation}


We denote by $C^w([0,T],\rr^d)$  the space of  $w$-H\"older continuous functions. This is the space of all functions satisfying \eqref{w-holder} for some $C>0$ and it is a Banach space with norm
\[\|\b{u}\|_{  C^w([0,T],\rr^d) }  :=\|\b{u}\|_{L^{\infty}}+\sup\limits_{t\neq s}\frac{|\b{u}(t)-\b{u}(s)|}{w(|t-s|)}.\]





 An important aspect of the theory of Sobolev spaces is related to embedding theorems. There is an extensive literature on this question in the  Orlicz-Sobolev space setting, see for example
 \cite{cianchi2000fully,cianchi1999some,claverooptimal,edmunds2000optimal,kerman2006optimal}.
The next simple lemma is essentially known and we will use it systematically. For the sake of completeness, we include a brief proof of it.



\begin{lem}\label{inclusion orlicz} Let  $w(s):= s\Phi^{-1}(1/s)$. Then, the following statements hold:
\begin{enumerate}
\item\label{inclusion orlicz_item1} $\wphi\hookrightarrow C^w([0,T],\rr^d) $ and for every $\b{u}\in\wphi$
\begin{align}
 &\left|\b{u}(t)-\b{u}(s) \right| \leq  \|\b{\dot{u}}\orlnor w(| t-s|),&\label{in-sob-cont}
\\
& \|\b{u}\|_{L^{\infty}} \leq\Phi^{-1}\left(\frac{1}{T}\right)\max\{1,T\}\|\b{u}\sobnor&\label{sobolev}
\end{align}
\item For every $\b{u}\in\wphi$ we have $\widetilde{\b{u}}\in L^{\infty}_d$ and 
\begin{align}
& \|\widetilde{\b{u}}\|_{L^{\infty}} \leq T\Phi^{-1}\left(\frac{1}{T}\right)\|\b{\dot u}\orlnor&\text{  (Sobolev's inequality).}\label{wirtinger}
\end{align}




\end{enumerate}
\end{lem}


The next result is analogous to some lemmata in $W^1L^p_d$, see \cite{xu2007some}.
\begin{lem}\label{infinito-a-prom-upunto}
If $\|\b{u}\sobnor\to \infty$, then $(|\b{\overline u}|+\|\b{\dot u}\orlnor)\to \infty$.
\end{lem}

\begin{proof}
We have
\[
\|\b{u}\orlnor=
\|\b{\overline u}+\b{\tilde{u}}\orlnor\leq 
\|\b{\overline u}\orlnor+\|\b{\tilde{u}}\orlnor=
|\b{\overline u}|\|1\orlnor+\|\b{\tilde{u}}\orlnor
\]
We know that Holder's inequality implies that $L^{\infty}_d\hookrightarrow\lphi_d$, that is,
there exists $C>0$ such that for any $\b{\tilde{u}}\in L^{\infty}_d$ we have 
\[
\|\b{\tilde{u}}\orlnor
\leq 
C
\|\b{\tilde{u}}\|_{L^{\infty}}
\]
and, applying  Sobolev's inequality to the previous formula,  we get  
\[
\|\b{\tilde{u}}\orlnor
\leq 
C\|\b{\dot{u}}\orlnor
\]
{\bf La desigualdad anterior ser\'ia del tipo Wirtinger's que no tenemos enunciada en ning\'un lado.}
\\
Therefore, 
\begin{equation}
\|\b{u}\orlnor\leq 
C(|\b{\overline u}|+\|\b{\dot{u}}\orlnor)
\end{equation}
As 
$\|\b{u}\sobnor=\|\b{u}\orlnor+\|\b{\dot{u}}\orlnor$, then 
\[
\|\b{u}\sobnor\leq
C(|\b{\overline u}|+\|\b{\dot{u}}\orlnor)
\]
 and by hypothesis $\|\b{u}\sobnor\to \infty$, then 
$|\b{\overline u}|+\|\b{\dot{u}}\orlnor\to \infty$.
\end{proof}

{\bf Esta definici\'on va as\'i o requiere modificaciones/adaptaciones???}

\begin{defi} We say that a function $\mathcal{L}:[0,T]\times \mathbb{R}^d \times \mathbb{R}^d \rightarrow \mathbb{R}$ is a Carath\'eodory function if for fixed $(\b{x},\b{y})$
the map $t \mapsto \mathcal{L}(t, \b{x},\b{y})$ is measurable  and for fixed $t$ the map  $(\b{x},\b{y}) \mapsto \mathcal{L}(t, \b{x}, \b{y})$ is continuously differentiable for almost everywhere $t\in [0,T]$.
\end{defi}

In \cite{ABGMS2015} we proved the next results.

\begin{thm}\label{teorema_acotacion}
Let $\mathcal{L}$ be a Carath\'eodory function satisfying \eqref{cotaL}, \eqref{cotaDxL} and \eqref{cotaDyL}. 
Then the following statements hold:
\begin{enumerate}
\item \label{T1item1} \label{A1} The action integral given by \eqref{integral_accion}
is finitely defined on $\domi:=W^{1}\lphi_d\cap\{\b{u}|\b{\dot{u}}\in\Pi(\ephi_d,\lambda)\}$.

\item\label{T1item3} The function  $I$ is G\^ateaux differentiable on $\domi$ and  its derivative $I'$ is demicontinuous from $\domi$  into $\left[\wphi_d \right]^*$. Moreover, $I'$ is given by the following expression
\begin{equation}\label{DerAccion}
\langle  I'(\b{u}),\b{v}\rangle= \int_0^T \left\{D_{\b{x}}\mathcal{L}\big(t,\b{u},\b{\dot{u}}\big)\ccdot \b{v}+ D_{\b{y}}\mathcal{L}\big(t,\b{u},\b{\dot{u}}\big)\ccdot\b{\dot{v}}\right\} \ dt.
\end{equation}

\item\label{T1item4}  If  $\Psi \in \Delta_2$ then 
  $I'$ is continuous from $\domi$ into $\left[\wphi_d\right]^*$ when both spaces are equipped with the strong topology.
\end{enumerate}
\end{thm}


%\begin{proof} Let $\b{u}\in \domi$.
 %Since  $\lambda\Pi(\ephi_d,r)=\Pi(\ephi_d,\lambda r)$, we have   $\b{\dot{u}}/\lambda\in\Pi(\ephi_d,1)$. 
%Thus, as $f\in\ephi_1$ and attending to \eqref{inclusiones}, we get 
%
%\begin{equation}\label{inclusion3}
%|\b{\dot{u}}|/\lambda+f\in\Pi(\ephi_1,1)\subset \claseor_1.
%\end{equation}
%By Corollary \ref{a_bound} and \eqref{cotaL}, we get 
 %\[|\mathcal{L}(\cdot,\b{u},\b{\dot{u}})| \leq A(\|\b{u}\sobnor ) \left(b+ \Phi\left (\frac{|\b{\dot{u}}|}{\lambda}+f\right)  \right)\in
 %L^1_1.\]
%This fact proves item \ref{T1item1}.
%
 %We split up the proof of item \ref{T1item3} into four steps.
%
%\noindent\emph{Step 1. The non linear operator  $\b{u} \mapsto D_{\b{x}}\mathcal{L}(t,\b{u},\b{\dot{u}})$ is continuous from $\domi$ into $L^{1}_d([0,T])$ with the strong topology on both sets.} 
%
%
%If $\b{u}\in \domi$, from \eqref{cotaDxL} and \eqref{inclusion3}, we obtain 
%\begin{equation}\label{DxL1}
%|D_{\b{x}}\mathcal{L}(\cdot,\b{u},\b{\dot{u}})|\leq A(\|u\sobnor) \left(b+\Phi\left(\frac{|\b{\dot{u}}|}{\lambda}+f\right)\right) \in L^1_1.
%\end{equation}
%
%
%Let   $\{\b{u}_n\}_{n\in \mathbb{N}}$ be a sequence of  functions in $\domi$  and let $\b{u}\in \domi$  such that $\b{u}_n\rightarrow \b{u}$ in $\wphi_d$.
%From  $\b{u}_n\rightarrow \b{u}$ in $\lphi_d$, there exists a subsequence $\b{u}_{n_k}$ such that $\b{u}_{n_k}\rightarrow \b{u} \quad\text{a.e.}$; and, as $\b{\dot{u}}_n\rightarrow \b{\dot{u}}\in\domi$, by 
  %Lemma \ref{segundo lema}, there exist a subsequence of  $\b{u}_{n_k}$ (again denoted $\b{u}_{n_k}$) and a function  $h\in \Pi(\ephi_1,\lambda))$
%such that  $\b{\dot{u}}_{n_k}\rightarrow \b{\dot{u}} \quad\text{a.e.}$ and $|\b{\dot{u}}_{n_k}|\leq h\quad\text{a.e}$.  Since $\b{u}_{n_k}$, $k=1,2,\ldots,$ is a strong convergent sequence in $\wphi_d$, it is a bounded sequence in $\wphi_d$. According to Lemma \ref{inclusion orlicz} and Corollary \ref{a_bound}, there exists $M>0$ such that $\|\b{a}(\b{u}_{n_k})\|_{L^{\infty}} \leq M$, $k=1,2,\ldots$.  From the previous facts and \eqref{DxL1}, we get
%\begin{equation*}\label{DxL1-bis}
%|D_{\b{x}}\mathcal{L}(\cdot,\b{u}_{n_k},\b{\dot{u}}_{n_k})|\leq M\left(b+\Phi\left(\frac{|h|}{\lambda}+f\right)\right) \in L^1_1.
%\end{equation*}
%On the other hand, by the Carath\'eodory condition, we have
%\[D_{\b{x}}\mathcal{L}(t,\b{u}_{n_k}(t),\b{\dot{u}}_{n_k}(t))\to D_{\b{x}}\mathcal{L}(t,\b{u}(t),\b{\dot{u}}(t))\quad\hbox{ for a.e. } t\in[0,T].\]
%Applying the Dominated Convergence Theorem we conclude the proof of step 1.
%
%\noindent\emph{Step 2. The non linear operator   $\b{u}
 %\mapsto  D_{y}\mathcal{L}(t,\b{u},\b{\dot{u}})$ is continuous from $\domi$ with the strong topology  into $\left[\lphi_d\right]^*$  with the weak$^*$ topology.}
%
 %Let $\b{u}\in \domi$.  From  \eqref{inclusion3} and Lemma \ref{phi_comp},  it follows that 
%\begin{equation}\label{AcotOperphi}
%\varphi\left(\frac{|\b{\dot{u}}|}{\lambda}+f\right)\in C^{\Psi}_1;
%\end{equation}
%and, Corollary \ref{a_bound} implies $\b{a}(\b{u})\in L^{\infty}_1$. 
%Therefore, in virtue of  \eqref{cotaDyL} we get
%\begin{equation}\label{DyLpsi}
   %\left|D_{\b{y}}\mathcal{L}(\cdot,\b{u},\b{\dot{u}})\right|\leq  A(\|\b{u}\|_{\wphi} )  \left(c+\varphi\left( \frac{|\b{\dot{u}}|}{\lambda}+f\right  ) \right)\in\lpsi_1.
%\end{equation}
 %Note that \eqref{DxL1},  \eqref{DyLpsi} and the imbeddings $\wphi_d \hookrightarrow L_d^{\infty}$ and  $\lpsi_d\hookrightarrow  \left[\lphi_d\right]^*$ imply that the second member of
%\eqref{DerAccion} defines an element in $\left[\wphi_d\right]^*$.
%
%Let $\b{u}_n,\b{u}\in \domi$ such that $\b{u}_n\to \b{u}$ in the norm of $\wphi_d$. 
%We must prove that  $D_{\b{y}}\mathcal{L}(\cdot,\b{u}_n,\dot{\b{u}}_n)\overset{w^*}{\rightharpoonup} D_{\b{y}}\mathcal{L}(\cdot,\b{u},\b{\dot{u}})$. On the contrary, there exist $\b{v}\in\lphi_d$, $\epsilon>0$ and a subsequence of $\{\b{u}_n\}$ (denoted  $\{\b{u}_n\}$ for simplicity)  such that
%\begin{equation}\label{cota_eps}
 %\left| \langle D_{\b{y}}\mathcal{L}(\cdot,\b{u}_n,\b{\dot{u}}_n),\b{v} \rangle - \langle  D_{\b{y}}\mathcal{L}(\cdot,\b{u},\b{\dot{u}}),\b{v} \rangle\right|\geq \epsilon.
%\end{equation}
%We have $\b{u}_n\rightarrow \b{u}$ in $\lphi_d$ and
%$\b{\dot{u}}_n\rightarrow \b{\dot{u}}$ in $\lphi_d$. By Lemma \ref{segundo lema}, there exist a subsequence $\b{u}_{n_k}$ and a function $h\in \Pi(\ephi_1,\lambda)$ such that $\b{u}_{n_k}\rightarrow \b{u} \quad\text{a.e.}$, $\b{\dot{u}}_{n_k}\rightarrow \b{\dot{u}} \quad\text{a.e.}$ and $|\b{\dot{u}}_{n_k}|\leq h\quad\text{a.e.}$ 
%As in the previous step, since $\b{u}_n$ is a convergent sequence, the Corollary \ref{a_bound} implies that $a(|\b{u}_n(t)|)$ is uniformly bounded by a certain constant $M>0$. 
%Therefore,  with $\b{u}_{n_k}$ instead of $\b{u}$, inequality  \eqref{DyLpsi} becomes 
%\begin{equation}\label{Dy-suc}
  %\left | D_{\b{y}}\mathcal{L}(\cdot,\b{u}_{n_k},\b{\dot{u}}_{n_k})  \right| 
	%\leq M\left(c+\varphi\left(\frac{h}{\lambda}+f\right)\right)\in \lpsi_1.
%\end{equation}
%Consequently, as $v \in \lphi_d$ and employing H\"older's inequality, we obtain that
%\[\sup_k|D_{\b{y}}\mathcal{L}(\cdot,\b{u}_{n_k},\b{\dot{u}}_{n_k})\ccdot v| \in L^1_1.\]
  %Finally, from the Lebesgue Dominated Convergence Theorem, we deduce
%\begin{equation}\label{conv_debil}\int_0^T  D_{\b{y}}\mathcal{L}(t,\b{u}_{n_k},\b{\dot{u}}_{n_k})\ccdot\b{ v} \ dt \to \int_0^T D_{\b{y}}\mathcal{L}(t,\b{u},\b{\dot{u}})\ccdot\b{ v}\ dt \end{equation}
%which contradicts the inequality \eqref{cota_eps}. This completes the proof of step 2.
%
%\emph{Step 3.} We will prove \eqref{DerAccion}. The proof follows similar lines as \cite[Thm. 1.4]{mawhin2010critical}. For $\b{u}\in \domi$ and $\b{0}\neq\b{v}\in\wphi_d$, we define the function
%\[H(s,t):=\mathcal{L}(t,\b{u}(t)+s\b{v}(t),\b{\dot{u}}(t)+s\b{\dot{v}}(t)).\]
%
%From \cite[Lemma 10.1]{KR} (or \cite[Thm. 5.5]{Orliczvectorial2005} ) we obtain that if $|\b{u}|\leq |\b{v}|$ then    $d(\b{u},\ephi_d)\leq d(\b{v},\ephi_d)$. 
%Therefore, for  $|s|\leq s_0:=\left(\lambda-d(\b{\dot{u}},\ephi_d)\right)/\|\b{v}\sobnor$ we have
%\[
%d \left(\b{\dot{u}}+s\b{\dot{v}}, \ephi_d \right)
%\leq
%d \left(|\b{\dot{u}}|+s|\b{\dot{v}}|, \ephi_1 \right)
%\leq d \left(|\b{\dot{u}}|,\ephi_1 \right)+ s \|\b{\dot{v}}\orlnor < \lambda.
%\]
%Thus $\b{\dot{u}}+s\b{\dot{v}} \in \Pi(\ephi_d,\lambda)$ and  $|\b{\dot{u}}|+s|\b{\dot{v}}| \in \Pi(\ephi_1,\lambda)$. These facts imply, in virtue of Theorem \ref{teorema_acotacion} item \ref{T1item1}, that $I(\b{u}+s\b{v})$ is well defined and finite for $|s|\leq s_0$. 
%And, using  Corollary \ref{a_bound}, we also see that
%\[ \|a(|\b{u}+s\b{v}|)\|_{L^{\infty}}\leq  A(\|\b{u}+s\b{v}\sobnor)\leq
 %A(\|\b{u}\sobnor+s_0\|\b{v}\sobnor)=:M
%\]
%Now, applying Chain Rule, \eqref{DxL1}, \eqref{DyLpsi} the monotonicity of $\varphi$ and $\Phi$, 
%the fact that $\b{v}\in L^{\infty}_d$ and $\b{\dot{v}}\in\lphi_d$ and H\"older's inequality, we get
%\begin{equation}\label{ctg}
%\begin{split}
%|D_s H(s,t)|&=\left| D_{\b{x}}\mathcal{L}(t,\b{u}+s\b{v},\b{\dot{u}}+s\b{\dot{v}})\ccdot \b{v} +  D_{\b{y}}\mathcal{L}(t,\b{u}+s\b{v},\b{\dot{u}}+s\b{\dot{v}})\ccdot\b{\dot{v}}\right| \\
 %& \leq M \left[\left( b(t)+ \Phi\left(\frac{|\b{\dot{u}}|+s_0|\b{\dot{v}}|}{\lambda}+f(t)\right)\right)|\b{v}|\right.\\
%&\left. \quad+ \left(c(t)+ \varphi\left (\frac{|\b{\dot{u}}|+s_0|\b{\dot{v}}|}{\lambda}+f(t)\right)\right)|\b{\dot{v}}| \right]\in L^1_1.
%\end{split}
%\end{equation}
%Consequently, $I$ has a directional derivative and
%\[
%\langle I'(\b{u}),\b{v} \rangle=\frac{d}{ds}I(\b{u}+s\b{v})\big|_{s=0}=\int_0^T  
%\left\{D_{\b{x}}\mathcal{L}(t,\b{u},\b{\dot{u}})\ccdot \b{v}+ D_{\b{y}}\mathcal{L}(t,\b{u},\b{\dot{u}})\ccdot\b{\dot{v}}\right\} \ dt.
%\]
%Moreover, from \eqref{DxL1}, \eqref{DyLpsi}, Lemma \ref{inclusion orlicz} and the previous formula, we obtain
%\[
%|\langle I'(\b{u}),\b{v} \rangle| \leq \|D_{\b{x}}\mathcal{L}\|_{L^1} \| \b{v}\linf + 
%\|D_{\b{y}}\mathcal{L}\|_{\lpsi} \|\b{\dot{v}}\orlnor \leq C \|\b{v}\sobnor
%\]
%with a appropriate constant $C$.
%This completes the proof of the G\^ateaux differentiability of $I$. 
%
%\emph{Step 4. The operator $I':\domi  \to \left[\wphi_d
%\right]^* $ is demicontinuous.}
%This is a consequence  of the continuity of the mappings $\b{u} \mapsto D_{\b{x}}\mathcal{L}(t,\b{u},\b{\dot{u}})$ and $\b{u} \mapsto
%D_{\b{y}}\mathcal{L}(t,\b{u},\b{\dot{u}})$. Indeed, if $\b{u}_n,\b{u}\in \domi$ with $\b{u}_n\to \b{u}$ in the norm of $\wphi_d$ and $\b{v} \in
%\wphi_d$, then
%\[
%\begin{split}
%\left\langle  I'(\b{u}_{n}),\b{v} \right\rangle &= \int_0^T \left\{  D_{\b{x}}\mathcal{L}\left(t,\b{u}_n,\b{\dot{u}}_n\right)\ccdot
%\b{v} +
 %D_{\b{y}}\mathcal{L}\left(t,\b{u}_n,\b{\dot{u}}_n\right)\ccdot\b{\dot{v}}\right\} \ dt\\
%&\rightarrow \int_0^T \left\{ D_{\b{x}}\mathcal{L}\left(t,\b{u},\b{\dot{u}}\right)\ccdot \b{v}+ 
%D_{\b{y}}\mathcal{L}\left(t,\b{u},\b{\dot{u}}\right)\ccdot\b{\dot{v}}\right\} \ dt\\
%&=\left\langle  I'(\b{u}),\b{v} \right\rangle.
%\end{split}
%\]
%
%
%In order to prove item  \ref{T1item4}, it is necessary to see that the maps $\b{u}\mapsto D_{\b{x}}\mathcal{L}(t,\b{u},\b{\dot{u}})$  and $\b{u}\mapsto D_{\b{y}}\mathcal{L}(t,\b{u},\b{\dot{u}})$  are norm continuous
%from $\domi $ into $L^1_d$ and
 %$\lpsi_d$ respectively.  The continuity of the first map has already been proved in step 1. 
%Let $\b{u}_n, \b{u} \in \domi$ with $\|\b{u}_n- \b{u}\sobnor\to 0$. Therefore,   there exist a subsequence $\b{u}_{n_k}\in \domi$ and a function $h\in\Pi(\ephi_1,\lambda)$  such that   \eqref{Dy-suc} holds true. 
%And, as  $\Psi\in\Delta_2$ then   the right hand side of  \eqref{Dy-suc} belongs to $\epsi_1$. 
%Now, invoking  Lemma \ref{lema_conv_may}, we  prove that
  %from any sequence $\b{u}_n$ which converges to $\b{u}$ in $\wphi_d$ we can
%extract a subsequence such that   $D_{\b{y}}\mathcal{L}(t,\b{u}_{n_k},\b{\dot{u}}_{n_k})\to D_{\b{y}}\mathcal{L}(t,\b{u},\b{\dot{u}})$ in the strong topology. The desired result is obtained by a standard argument.
%
%The continuity of $I'$  follows  from the continuity 
%of $D_{\b{x}}\mathcal{L}$ and $D_{\b{y}}\mathcal{L}$ using the formula \eqref{DerAccion}.
%\end{proof}



%\section{Critical points and Euler-Lagrange equations}\label{sec:equa-min}

In \cite{ABGMS2015} we derived the Euler-Lagrange equations associated to critical points of action integrals on the subspace of $T$-periodic functions.  
We denote by $\wphi_T$ the subspace of $\wphi_d$ containing all  $T$-periodic functions. As usual, when $Y$ is a subspace of
the Banach space $X$, we denote by $Y^{\perp}$ the \emph{annihilator subspace} of $X^*$, i.e. the subspace
that consists of all  bounded linear functions which are identically zero on $Y$.

We recall that  a function $f: \mathbb{R}^d \to \mathbb{R}$ is called \emph{strictly convex} if 
$f\left(\tfrac{\b{x}+\b{y}}{2}\right)< \tfrac{1}{2} \left(f\left(
\b{x}\right)+f\left( \b{y}\right)\right)$ for  $\b{x}\neq\b{y}$.  
It is  well known that if $f$ is a strictly convex and differentiable function, then
$D_{\b{x}}f:\mathbb{R}^d\to\mathbb{R}^d$ is a one-to-one map  (see, e.g. \cite[Thm. 12.17]{rockafellar2009variational}).


\begin{thm}\label{critpoint} Let $\b{u}\in\domi$ be  a $T$-periodic function. The following statements are equivalent:
\begin{enumerate}
 \item $I'(\b{u})\in\left( \wphi_T\right)^{\perp}$.
 \item  $D_{\b{y}}\mathcal{L}(t,\b{u}(t),\b{\dot{u}}(t))$ is an absolutely continuous function and $\b{u}$ solves the following boundary value problem
 \begin{equation}\label{ecualagran2}
    \left\{%
\begin{array}{ll}
   \frac{d}{dt} D_{y}\mathcal{L}(t,\b{u}(t),\b{\dot{u}}(t))= D_{\b{x}}\mathcal{L}(t,\b{u}(t),\b{\dot{u}}(t)) \quad \hbox{a.e.}\ t \in (0,T)\\
    \b{u}(0)-\b{u}(T)=D_{\b{y}}\mathcal{L}(0,\b{u}(0),\b{\dot{u}}(0))-D_{\b{y}}\mathcal{L}(T,\b{u}(T),\b{\dot{u}}(T))=0.
\end{array}%
\right.
\end{equation}
\end{enumerate}
Moreover if $D_{\b{y}}\mathcal{L}(t,x,y)$ is $T$-periodic with respect to the variable $t$ and strictly convex with respect to $\b{y}$, then
$D_{\b{y}}\mathcal{L}(0,\b{u}(0),\b{\b{\dot{\b{u}}}}(0))-D_{\b{y}}\mathcal{L}(T,\b{u}(T),\b{\dot{u}}(T))=0$ is equivalent to $\b{\dot{u}}(0)=\b{\dot{u}}(T)$.
\end{thm}





DECIR ALGO DE LOS \'INDICES AC\'A O EN LA INTRO...????
\\
Por ac\'a aparec\'ia la NOTACI\'ON DE $J_{C,\nu}$, que usamos en alguna demostraci\'on m\'as adelante.
Me parece que no es necesaria, s\'olo hay que corregir la prueba donde se la usaba!!!! 



\begin{lem}\label{lem_coer} Let $\Phi$ and $\Psi$ be complementary $N$-functions. Then:
\begin{enumerate}
  \item $\|\b{u}\orlnor=O\left( \rho_{\Phi}\left(\b{u} \right)  \right)$.
  
  \item If $\Psi \in \Delta_2$ globally, then there exists a constant $\alpha_{\Phi}>1$ such that, for any $0<\mu<\alpha_{\Phi}$,
\begin{equation}\label{coer_modular} \|\b{u}\orlnor^{\mu} =o\left(\rho_{\Phi}\left(\b{u}\right)\right).
\end{equation}
Reciprocally, if \eqref{coer_modular} holds for $\mu\geq 1$ then $\Psi \in \Delta_2$.  
\end{enumerate}
\end{lem}









Based on \cite{mawhin2010critical} we say that $F$ satisfies the condition (A) if  $F(t,\b{x})$ is a Carath\'eo\-dory function and  $F$ is continuously differentiable with respect to $\b{x}$. Moreover, the next inequality holds 
\begin{equation}\label{condA2}|F(t,\b{x})|+ |D_{\b{x}}F(t,\b{x})|\leq a(|\b{x}|)b_0(t),\quad\text{for a.e. }t\in [0,T], \forall\b{x}\in\rr^d.
\end{equation}

\section{Lagrangians with sublinear nonlinearity}
We define the following functionals $J_{C,\mu}:\lphi\to (-\infty,+\infty]$ and $  H_{C,\sigma}:\rr^n\to \rr$, with $C,\nu,\sigma>0$, by
\begin{equation}\label{func_phi}
  J_{C,\nu}(\b{u}):= \rho_{\Phi}\left(\b{u}\right)-C\|\b{u}\orlnor^{\nu},
\end{equation}
 and

\begin{equation}\label{eq:functional_H}
 H_{C,\sigma}(\b{x})=\int_0^TF(t,\b{x})dt-C|\b{x}|^{\sigma}
\end{equation}
respectively.

Like in \cite{ABGMS2015} we consider Lagrangians $\mathcal{L}$ which are lower bounded as follows 
\begin{equation}\label{cota_inf}
\mathcal{L}(t,\b{x},\b{y})\geq \alpha_0\Phi\left(\frac{|\b{y}|}{\Lambda}\right)+ F(t,\b{x}).
\end{equation}


Now, we have another result about coercivity of $I$ assuming some conditions on the  $\nabla F$. 

\begin{thm}\label{coercitividad-r}
Let  $\mathcal{L}$ be a lagrangian function satisfying \eqref{cotaL}, \eqref{cotaDxL}, \eqref{cotaDyL}, \eqref{cota_inf}  and $F$ satisfies condition (A). We assume the following conditions:
\begin{enumerate}
\item $\Psi\in\Delta_2$.
\item There exist  non negative functions  $b_1,b_2 \in L^1_1$ and a constant $1<\mu<\alpha_{\Phi}$  such that 
for any $\b{x}\in\rr^d$ and a.e. $t\in [0,T]$
\begin{equation}\label{holder_cont-mu}
  \left| \nabla F(t,\b{x}) \right|\leq b_1(t)|\b{x}|^{\mu-1}+b_2(t).
\end{equation}
\item There exists a real positive number $\sigma$ such that $\sigma>(\mu-1)\beta_{\Psi}$ and
\begin{equation}\label{propiedad-coercividad-mu}
|\b{x}|^{\sigma}=o\left(\int_{0}^{T}F(t,\b{x})\ dt\right)\;\;\mbox{as}\;\;|\b{x}|\to \infty.
\end{equation}
\end{enumerate}
Then  the action integral $I$ is coercive.
\end{thm}




\begin{proof} 
By the decomposition $u=\overline{u}+\b{\tilde{u}}$,  Mean Value Theorem, Cauchy-Schwarz inequality 
and \eqref{holder_cont-mu}, we have
\begin{equation}\label{cota-diferencia-F}
\begin{split}
&\left|\int_0^T F(t,\b{u})-F(t,\b{\overline{u}})\,dt\right|=
\left|\int_0^T \int_0^1 \nabla F(t,\b{\overline{u}}+s\b{\tilde{u}}(t))\ccdot \b{\tilde{u}}(t) \,ds \,dt\right|
\\
&\leq \int_0^T \int_0^1 b_1(t)|\b{\overline{u}}+s\b{\tilde{u}}(t)|^{\mu-1}|\b{\tilde{u}}(t)|\,ds\,dt+
\int_0^T \int_0^1 b_2(t)|\b{\tilde{u}}(t)|\,ds\,dt
\\
&=I_1+I_2.
\end{split}
\end{equation}
On the one hand, by H\"older's inequality and Sobolev's inequality, we estimate $I_2$ as follows
\begin{equation}\label{cota-i2}
I_2\leq \|b_2\|_{L^1} \|\b{\tilde{u}}\|_{L^{\infty}}\leq
C_1\|\b{\dot u}\orlnor.
\end{equation}
 where $C_1=C_1(\|b_2\|_{L^1}, T)$. On the other hand, as $s\in [0,1]$, we have
\begin{equation}\label{pot-suma}
|\b{\overline{u}}+s\b{\tilde{u}}(t)|^{\mu-1}\leq
C(\mu)(|\b{\overline{u}}|^{\mu-1}+\|\b{\tilde{u}}\|_{L^{\infty}}^{\mu-1}).
\end{equation}
where $C(\mu)=2^{\mu-2}$, for $\mu\geq 2$ and $C(\mu)=1$, for $1<\mu<2$. Now,  inequality \eqref{pot-suma}, H\"older's inequality and Sobolev's inequality imply that
\begin{equation}\label{cota-i1}
\begin{split}
I_1&\leq 
C(\mu)\left(|\b{\overline{u}}|^{\mu-1} \int_0^T b_1(t) |\b{\tilde{u}}(t)|\,dt+
\|\b{\tilde{u}}\|^{\mu-1}_{L^{\infty}} \int_0^T b_1(t)|\b{\tilde{u}}(t)| \,dt\right)
\\
&\leq C(\mu)\bigg\{ |\b{\overline{u}}|^{\mu-1} \|b_1\|_{L^1} \|\b{\tilde{u}}\|_{L^{\infty}}+
 \|b_1\|_{L^1}\|\b{\tilde{u}}\|^{\mu}_{L^\infty}\bigg\}
\\
&\leq C_2 \bigg\{ |\b{\overline{u}}|^{\mu-1} \|\b{\dot{u}}\orlnor+ \|\b{\dot u}\orlnor^{\mu}\bigg\},
\end{split}
\end{equation}
where $C_2=C_2(\mu,T, \|b_1\|_{L^1} )$. Let $\mu'$ be a positive constant such that $1<\mu\leq \mu'<\alpha_{\Phi}$. 
Next, using Young's inequality with conjugate exponents $\mu'$ and $\frac{\mu'}{\mu'-1}$ 
 we get
\begin{equation}\label{cota-i1-parcial}
|\b{\overline{u}}|^{\mu-1}   \|\b{\dot{u}}\orlnor
\leq \frac{(\mu'-1)}{\mu'}|\b{\overline{u}|^{\sigma}}
+\frac{1}{\mu'} \|\b{\dot{u}}\orlnor^{\mu'}
\end{equation}
where $\sigma=\frac{(\mu-1) \mu'}{\mu'-1}$. We note that $\sigma$ is an arbitrary positive constant bigger than $(\mu-1)b_{\Psi}$.

From \eqref{cota-i1},\eqref{cota-i1-parcial}, \eqref{cota-i2} and the inequality $x^{r_1}\leq x^{r_2}+1$, for any $x\geq 0$ and $r_1\leq r_2$ we have
\begin{equation}\label{cota-i1-i2}
\begin{split}
I_1+I_2
&\leq C_3\bigg\{ |\b{\overline{u}}|^{\sigma}
+ \|\b{\dot u}\orlnor^{\mu'}
+ \|\b{\dot u}\orlnor^{\mu}
+\|\b{\dot u}\orlnor\bigg\}\\
&\leq C_3\bigg\{ |\b{\overline{u}}|^{\sigma}
+ \|\b{\dot u}\orlnor^{\mu'}
+1\bigg\}
\end{split}
\end{equation}
with $C_3= C_3(\mu,T, \|b_1\|_{L^1},\mu' )$. In the subsequent estimates, we use the decomposition $u=\overline{u}+\b{\tilde{u}}$, \eqref{cota_inf}, \eqref{cota-diferencia-F},
\eqref{cota-i1-i2} and we get
\begin{equation}\label{cota_inf_I}
\begin{split}
I(\b{u})&\geq\alpha_0\rho_{\Phi}\left( \frac{\b{\dot{u}}}{\Lambda}\right)+\int_0^TF(t,\b{u})\ dt
\\ 
&=\alpha_0\rho_{\Phi}\left( \frac{\b{\dot{u}}}{\Lambda}\right)+ \int_0^T \left[F(t,\b{u})-F(t,\b{\overline{u}})\right]\ dt 
+  \int_0^TF(t,\b{\overline{u}})\ dt
\\
&\geq \alpha_0\rho_{\Phi}\left( \frac{\b{\dot{u}}}{\Lambda}\right)
-C_3 \|\b{\dot u}\orlnor^{\mu'}
+\int_0^TF(t,\b{\overline{u}})\ dt-
C_3 |\b{\overline{u}}|^{\sigma}-C_3\\
&=\alpha_0J_{C_4,\mu'}\left(\frac{\b{\dot u}}{\Lambda}\right)
+ H_{C_3,\sigma}(\b{\overline{u}})-C_3,
\end{split}
\end{equation}
where $C_4=\Lambda^{\mu'}C_3/\alpha_0$.


Let $\b{u}_n$ be  a sequence in $\domi$ with 
$\|\b{u}_n\sobnor\to\infty$ and we have to prove that $I(\b{u}_n)\to\infty$. 
\\
On the contrary, suppose  that for a subsequence, 
still denoted by $\b{u}_n$, $I(\b{u}_n)$ is upper bounded, that is, there exists $M>0$ such that $|I(\b{u}_{n})|\leq M$. 
As $\|\b{u}_n\sobnor\to\infty$, from Lemma \ref{infinito-a-prom-upunto},  we have $|\overline{\b{u}}_n|+\|\b{\dot{u}}_n\orlnor\to \infty$.
Then, there exists subsequence of the subsequence $\{\b{u}_n\}$, still denoted by $\b{u}_n$, which is not bounded.
Then, 
$\b{\overline u}_n\to \infty$ or $\|\b{\dot{u}}_n\orlnor\to \infty$.
Now, as the functionals $J_{C,\mu'(\b{\dot u})}$ and $\gamma(\b{\overline{u}})$ are coercive, then 
$J_{C,\mu'(\b{\dot u}_n)} \to \infty$ or $\gamma(\b{\overline{u}}_n)\to \infty$.
By \eqref{condA2}, the functional $\gamma(\b{\overline{u}}_n)$ is lower bounded and 
$J_{C,\mu'(\b{\dot u}_n)}$ is also lower bounded on a bounded set because the modular $\rho_{\Phi}\left(\frac{\b{u}}{\Lambda}\right)$ is always bigger than zero. 
Therefore,  $I(\b{u}_n)\to\infty$ as $\|\b{u}_n\sobnor\to\infty$ which contradits the initial assumption on the behavior of $I(\b{u}_n)$. 
\end{proof}
REVISAR LA PRUEBA ANTERIOR Y MEJORAR LA ESCRITURA, y adaptar quitando $J$ si fuera el caso!!!!

\section{Limit case $\mu=\alpha_{\Phi}$}

In \cite{} coercivity was obtained even in the limit case $\mu=1$ and $\mu=p$ assuming additional conditions on ... 
This was possible because in $L^p$ spaces, the norm and the modular coincides, that is, $\|\cdot\|_p^p=O(\int_0^T |\cdot|^p\,dt)$.
In Orlicz spaces, $\|\cdot\orlnor^\mu$ can be upper controled by a modular provided that $\mu<\alpha_{\Phi}$ for any
$N$-function $\Phi$. But,  the limit case does  not hold for any $\Phi$, i.e. in general $\|\cdot\orlnor^{\alpha_{\Phi}}=O(\int_0^T \Phi(|u|)\,dt)$ is false as can be seen as follows.


Let $\Phi, \Psi \in \Delta_2$, then the next inequality $\Phi(tu)\geq t^{\alpha_{\Phi}}\Phi(u)$ for any $u>0$ and for any $t\geq 1$ is false.

In fact, let 
$\Phi(u)=
\left\{
\begin{array}{ll}
\frac{p-1}{p}u^p&u\leq e
\\
\frac{u^p}{\log u}-\frac{e^p}{p}&u>e
\end{array}
\right.$

\begin{thm}
If $p\geq \frac{1+\sqrt 2}{2}$, then $\Phi$ is an $N$-function.
\end{thm}


\begin{proof}
We have 
\[\Phi'(u)=\left\{
\begin{array}{cl}
(p-1)u^{p-1}=:\varphi_1(u)&u<e
\\
\frac{u^{p-1}}{\log u}(p-\frac{1}{\log u}):=\varphi_2(u)&u>e
\end{array}
\right.
\]
and $\Phi$ is differentiable at $e$ because $\varphi_1(e)=\varphi_2(e)=(p-1)e^{p-1}$.

{\bf Tendr\'iamos que ver que $\varphi_1,\varphi_2$ son crecientes y que $\varphi_2\to \infty$ cuando
$u \to \infty????$ o basta con ver que $\varphi_2$ es creciente????}

$\varphi_1$ is an increasing function provided that $p>1$ and $\varphi_1(u)\to 0$ as $u \to 0$.

In addition, $\varphi_2(u)\to \infty$ as $u \to \infty$ provided that $p>1$.
And 
\[0<\varphi_2'(u)=
\frac{u^{p-2}}{\log u} \left(p^2-p-\frac{2p}{\log u}+\frac{1}{\log u}+\frac{2}{\log^2 u}\right)\] on $[e,\infty)$ if and only if 
\[
\left(p^2-p-\frac{2p}{\log u}+\frac{1}{\log u}+\frac{2}{\log^2 u}\right)>0.
\]
If we take $\alpha:=\frac{1}{\log u}$, then we need 
\[
2\alpha^2+(1-2p)\alpha+(p^2-p)\geq 0
\]
which is true if and only if $p \notin(\frac{1-\sqrt2}{2},\frac{1+\sqrt2}{2})$.
Therefore, $\varphi_2$ is an icreasing function when $p\geq \frac{1+\sqrt2}{2}$.
\end{proof}



\begin{thm}
There exists a constant $C>0$ such that 
\begin{equation}
\Phi(tu)\leq ct^p\Phi(u)\;\;t\geq 1, u>0.
\end{equation}
For every $\varepsilon>0$ there exists a constant $C=C(\varepsilon,p)$ such that
\begin{equation}
\Phi(tu)\geq Ct^{p-\varepsilon}\Phi(u)\;\;t\geq 1,u>0.
\end{equation}
\end{thm}

\begin{proof}
Resumir la prueba
\end{proof}
\begin{rem}
The inequality 
\[
\Phi(tu)\geq Ct^p\Phi(u)
\] 
is false for every $C$ because for every $u\geq e$ we have 
\[
\lim\limits_{t \to \infty}\frac{\Phi(tu)}{t^p\Phi(u)}=0
\]
\end{rem}

\begin{thm}
$\alpha_{\Phi}=\beta_{\Phi}=p$
\end{thm}

\begin{proof}
Resumir la prueba.
\end{proof}
Now, we are able to see that 
\[
\rho_{\Phi}(u)=\int_0^T \Phi(|u|)\,dx\geq C\|u\orlnor^{\alpha_{\Phi}}=C\|u\orlnor^p
\]
is false.

If we take $u\equiv t>0$, then $\|u\orlnor^p=C_1t^p$ where $C_1=\|1\orlnor$ and
$\int_0^T \Phi(|u|)\,dx=C_2\Phi(t)$ with $C_2=T$. 
Then, if $\rho_{\Phi}(u)\geq C\|u\orlnor^p$ were true, then $\Phi(t)\geq C t^p$ were also true but this
last inequality is false.
 

\section*{Acknowledgments}
The authors are partially supported by a UNRC grant number 18/C417. The first author is  partially supported by a  UNSL grant number 22/F223. 


No s\'e por qu\'e pero parece que funciona...., en realidad quit\'e el .bib...
Y reci\'en ahora me doy cuenta que las referencias se ordenan de acuerdo al orden de menci\'on/aparici\'on....
 \bibliographystyle{elsarticle-num} 
 \bibliography{biblio}


\end{document}
