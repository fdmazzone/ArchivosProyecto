\documentclass[twoside]{article}


\NeedsTeXFormat{LaTeX2e}
\ProvidesPackage{mathscinet}[2002/04/17 v1.05]
\RequirePackage{textcmds}\relax
\ProvideTextCommandDefault{\cprime}{\tprime}



%\usepackage{hyperref}
\usepackage{amssymb,amsthm}
\usepackage{amsmath}
\usepackage{color}
\usepackage{ esint }
\usepackage{mathabx}
\usepackage{fancyhdr}
\usepackage{times}

\usepackage[latin1]{inputenc}

\usepackage{comment}
\usepackage{url}
\usepackage{xcolor}
\usepackage{adjustbox}
\usepackage{hyperref}

\newtheorem{thm}{Theorem}[section]
\newtheorem{cor}[thm]{Corollary}
\newtheorem{lem}[thm]{Lemma}

\newtheorem{defi}[thm]{Definition}
\newtheorem{prop}[thm]{Proposition}
\theoremstyle{remark}
\newtheorem{comentario}{Remark}



\title{Periodic solutions of 
Euler-Lagrange equations with ``sublinear nonlinearity'' in an Orlicz-Sobolev space setting}
\author{Sonia Acinas \thanks{SECyT-UNRC, UNSL and CONICET}\\
Instituto de Matem\'atica Aplicada San Luis (IMASL)\\ 
Universidad Nacional de San Luis and CONICET\\
Ej\'ercito de los Andes 950,
(D5700HDW) San Luis, Argentina\\
Universidad Nacional de La Pampa\\
(L6300CLB) Santa Rosa, La Pampa, Argentina\\
\url{sonia.acinas@gmail.com}\\[3mm]
Fernando D. Mazzone \thanks{SECyT-UNRC and CONICET}\\
Dpto. de Matem\'atica, Facultad de Ciencias Exactas, F\'{\i}sico-Qu\'{\i}micas y Naturales\\
Universidad Nacional de R\'{i}o Cuarto\\
(5800) R\'{\i}o Cuarto, C\'ordoba, Argentina,\\
\url{fmazzone@exa.unrc.edu.ar}
}

\date{}

\newcommand{\orlnor}{\|_{L^{\Phi}}}
\newcommand{\lurnor}{\|^{*}_{L^{\Phi}}}
\newcommand{\linf}{\|_{L^{\infty}}}
\newcommand{\lphi}{L^{\Phi}}
\newcommand{\lpsi}{L^{\Psi}}
\newcommand{\ephi}{E^{\Phi}}
\newcommand{\claseor}{C^{\Phi}}
\newcommand{\wphi}{W^{1}\lphi}
\newcommand{\wphiet}{W^{1}\ephi_T}
\newcommand{\wphie}{W^{1}\ephi}
\newcommand{\sobnor}{\|_{W^{1}\lphi}}
\newcommand{\domi}{\mathcal{E}^{\Phi}_d(\lambda)}
\renewcommand{\b}[1]{\boldsymbol{#1}}
\newcommand{\rr}{\mathbb{R}}
\newcommand{\nn}{\mathbb{N}}
\newcommand{\ccdot}{\b{\cdot}}
\renewcommand{\leq}{\leqslant} 
\renewcommand{\geq}{\geqslant} 
\newcommand{\epsi}{E^{\Psi}}

\begin{document}



\maketitle
%
\begingroup%Locallizing the change to `thefootnote'.
    \renewcommand{\thefootnote}{}%Removing the footnote symbol.
    %
    \footnotetext{%
    %   2010 Mathematics Subject Classification
    %   http://www.ams.org/msc/
    \textbf{2010  AMS Subject Classification.} Primary: .
    Secondary: .
    }%
        \footnotetext{%
    \textbf{Keywords and phrases.}  .
    }%
    \endgroup
%
%
%
%

\begin{abstract}

In this paper we obtain existence of periodic solutions, in the Orlicz-Sobolev space $\wphi([0,T])$, of hamiltonian systems with a potential  function $F$ satisfying the inequality  $|\nabla F(t,x)|\leq b_1(t) \varphi_0(|x|)+b_2(t)$, with    $b_1(t), b_2(t)\in L^1$ and for certain functions $\varphi_0$.

\end{abstract}




\pagestyle{fancy} \headheight 35pt \fancyhead{} \fancyfoot{}

\fancyfoot[C]{\thepage} \fancyhead[CE]{\nouppercase{S. Acinas and F.D. Mazzone }} \fancyhead[CO]{\nouppercase{\section}}

\fancyhead[CO]{\nouppercase{\leftmark}}


%\tableofcontents




\section{Introduction}
The purpose of this paper is to study the existence  of periodic solution for the
following non-autonomous second-order systems:

\begin{equation}\label{ProbPrin}
    \left\{%
\begin{array}{ll}
   \frac{d}{dt}\left(u'(t)\frac{\varphi(|u'|)}{|u'|}\right) = \nabla F(t,u(t)) \quad \hbox{a.e.}\ t \in (0,T)\\
    u(0)-u(T)=u'(0)-u'(T)=0
\end{array}%
\right.
\end{equation}
where $T>0$, $u:[0,T]\to\rr^d$ is absolutely continuous and  $\varphi=\Phi'$ where $\Phi$ is an differentiable  $N$-function (see preliminaries section  for definitions). Furthermore, the \emph{potential} $F:[0,T]\times\rr^d\to\rr$  satisfy the following conditions
\begin{enumerate}

 \item[(C)] $F$ and its gradient $\nabla F$ are  Carath\'eodory functions, i.e. they are measurable functions with respecto to $t\in [0,T]$, for every  $x\in\rr^d$, and   continuous functions with  respect to  $x\in\rr^d$ for a.e. $t \in [0,T]$.

 \item[(A)]   For   a.e. $t\in [0,T]$ we have that
\begin{equation}
|F(t,x)| + |\nabla F(t,x)|  \leq a(|x|)b(t)
\end{equation}
In these inequalities we assume that the function  $a:[0,+\infty)\to [0,+\infty)$ is continuous and nondecreasing and $0\leq b\in L^1([0,T],\rr)$.


\end{enumerate}

We call the differential operator.
\[L_{\Phi}[u]=\frac{d}{dt}\left(u'(t)\frac{\varphi(|u'|)}{|u'|}\right) \]
the
\emph{$\Phi$-laplacian operator}. If $\Phi(x)=|x|^p$, $1<p<\infty$, $L_{\Phi}$ is the well known $p$-laplacian operator.



The problem \eqref{ProbPrin} comes from a variational one, that is,  the equation in  \eqref{ProbPrin}  is the Euler-Lagrange equation associeted to the \emph{action integral}
\begin{equation}\label{integral_accion}
I(u)=\int_{0}^T \Phi(|u'(t)|)+F(t,u(t))\ dt.
\end{equation}




\section{Preliminaries}\label{preliminares}

For reader convenience, we give a short introduction to Orlicz and Orlicz-Sobolev spaces of vector valued functions and a  list  of results that we will use throughout the article. 
Classic references for Orlicz spaces of real valued functions are \cite{adams_sobolev,KR,rao1991theory}.
For  Orlicz spaces of vector valued functions, see \cite{Orliczvectorial2005} and the references therein.

Hereafter we denote  by $\mathbb{R}^+$  the set of all non negative real numbers. A function $\Phi:\mathbb{R}^+\to \mathbb{R}^+ $ is called an \emph{$N$-function} if $\Phi$ is convex and satisfies that
\[
\lim_{t\to+\infty}\frac{\Phi(t)}{t}=+\infty\quad\text{and}\quad \lim_{t\to 0}\frac{\Phi(t)}{t}=0
\]
In addition,  in this paper,  we asssume that $\Phi$ is differentiable, and we call $\varphi$ to the derivative of $\Phi$. With these assumptions, $\varphi:\mathbb{R}^+\rightarrow \mathbb{R}^+$ is a homeomorphism, with inverse $\psi$. We denote by $\Psi$ the primitive of $\psi$ that satisfies $\Psi(0)=0$. Then $\Psi$ is a $N$-function which  is called the \emph{complementary function} of $\Phi$.


There exists several order relations between $N$-functions (see \cite[Section 2.2]{rao1991theory}). Following \cite[Def. 1, p.15]{rao1991theory} we said that the $N$-function $\Phi_2$ is \emph{essentially stronger} than the $N$-function  $\Phi_1$  ($\Phi_1\llcurly\Phi_2$) if and only if there exists $x_0\geq 0$ such that $\Phi_1(x)\leq \Phi_2(ax)$, for every $a>0$ and $x\geq x_0$.






We say that a function $\eta:\mathbb{R}^+\rightarrow \mathbb{R}^+$ satisfies the  \emph{$\Delta_2$-condition}, denoted by $\eta \in \Delta_2$,
if there exist  constants $K>0$ and  $t_0\geq 0$ such that 
\begin{equation}\label{delta2defi}\eta(2t)\leq K\eta(t)
\end{equation}
for every $t\geq t_0$. 
If $t_0=0$,  we say that a function   $\eta:\mathbb{R}^+\rightarrow \mathbb{R}^+$ satisfies the \emph{$\Delta_2$-condition globally} ($\eta \in \Delta_2$ globally).


Let $d$ be a positive integer. We denote by $\mathcal{M}_d:=\mathcal{M}_d([0,T],\rr^d)$ the set of all measurable functions defined on $[0,T]$ with values on $\mathbb{R}^d$ and  we write $u=(u_1,\dots,u_d)$ for  $u\in \mathcal{M}_d$.



Given  an $N$-function $\Phi$ we define the \emph{modular function} 
$\rho_{\Phi}:\mathcal{M}_d\to \mathbb{R}^+\cup\{+\infty\}$ by
\[\rho_{\Phi}(u):= \int_0^T \Phi(|u|)\ dt.\]
Here $|\cdot|$ is the euclidean norm of $\mathbb{R}^d$.
The \emph{Orlicz class} $C_d^{\Phi}=C_d^{\Phi}([0,T],\rr^d)$  is given  by
\begin{equation}\label{claseOrlicz}
  C^{\Phi}_d:=\left\{u\in \mathcal{M}_d | \rho_{\Phi}(u)< \infty \right\}.
\end{equation}
The \emph{Orlicz space} $\lphi=L^{\Phi}_d([0,T],\rr^d)$ is the linear hull of $\claseor$;
equivalently,
\begin{equation}\label{espacioOrlicz}
\lphi:=\left\{ u\in \mathcal{M}_d | \exists \lambda>0: \rho_{\Phi}(\lambda u) < \infty   \right\}.
\end{equation}
  The Orlicz space $\lphi$ equipped with the \emph{Orlicz norm}
\[
\|  u  \orlnor:=\sup \left\{  \int_0^T u\b{\cdot} v\ dt \big| \rho_{\Psi}(v)\leq 1\right\},
\]
is a Banach space. By $u\b{\cdot} v$ we denote the usual dot product in $\mathbb{R}^{d}$ between $u$ and $v$.
The following alternative expression for the norm, known as \emph{Amemiya norm},     will  be useful (see \cite[Thm. 10.5]{KR} and \cite{hudzik2000amemiya}). For every $u\in\lphi$,

\begin{equation}\label{amemiya}
\|u\orlnor=\inf\limits_{k>0}\frac{1}{k}\left\{1+\rho_{\Phi}(ku)\right\}.
\end{equation}
In particular
\begin{equation}\label{amemiya-ine}
\|u\orlnor\leq \frac{1}{k}\left\{1+\rho_{\Phi}(ku)\right\},\quad\text{for every } k>0.
\end{equation}


The subspace $\ephi=\ephi([0,T],\rr^d)$ is defined as the closure in $\lphi$ of the subspace $L^{\infty}_d([0,T],\rr^d)$ of all $\mathbb{R}^d$-valued essentially bounded functions. It is shown that  $\ephi$ is the only one maximal subspace contained in the Orlicz class $\claseor$, i.e.
$u\in\ephi$ if and only if $\rho_{\Phi}(\lambda u)<\infty$ for any $\lambda>0$.

A generalized version of \emph{H\"older's inequality} holds in Orlicz spaces (see \cite[Th. 9.3]{KR}). Namely, if $u\in\lphi$ and $v\in\lpsi$ then $u\cdot v\in L_1^1$ and
\begin{equation}\label{holder}
\int_0^Tv\cdot u\ dt\leq \|u\orlnor\|v\|_{L^{\Psi}}.
\end{equation}




If $X$ and $Y$ are  Banach spaces such that  $Y\subset X^*$, we denote by $\langle\cdot,\cdot\rangle:Y\times X\to\mathbb{R}$ the bilinear pairing  map given by $\langle x^*,x\rangle=x^*(x)$. H\"older's inequality shows that $\lpsi\subset \left[\lphi\right]^*$, where the pairing
$\langle v, u\rangle$
is defined by 
\begin{equation}\label{pairing}
  \langle v,u\rangle=\int_0^Tv\cdot u\ dt
\end{equation}
with  $u\in\lphi$ and $v\in\lpsi$.
 Unless $\Phi \in \Delta_2$, the relation $\lpsi= \left[\lphi\right]^*$ will not hold. In general, it is true  that  $\left[\ephi\right]^*=\lpsi$.


Like in \cite{KR}, we will consider the subset $\Pi(\ephi,r)$ of $\lphi$ given by
\[\Pi(\ephi,r):=\{u\in\lphi| d(u,\ephi)<r\}.\]
This set is related to the Orlicz class $\claseor$ by means of inclusions, namely,
\begin{equation}\label{inclusiones}\Pi(\ephi, r )\subset r \claseor\subset\overline{\Pi(\ephi,r)}
\end{equation}
for any positive $r$.
If $\Phi \in \Delta_2$,  then the sets $\lphi$, $\ephi$, $\Pi(\ephi,r)$ and $\claseor$ are equal.



We define the \emph{Sobolev-Orlicz space} $\wphi$ (see \cite{adams_sobolev}) by
\[\wphi:=\{u| u \hbox{ is absolutely continuous in $[0,T]$ and } u'\in \lphi\}.\]
$\wphi$ is a Banach space when equipped with the norm
\begin{equation}\label{def-norma-orlicz-sob}
\|  u  \|_{\wphi}= \|  u  \|_{\lphi} + \|u'\orlnor.
\end{equation}



For a  function $u\in L^1_d([0,T])$, we write $u=\overline{u}+\widetilde{u}$ where $\overline{u} =\frac1T\int_0^T u(t)\ dt$ and $\widetilde{u}=u-\overline{u}$.

As usual, if $(X,\|\cdot\|_X)$ is a Banach space and $(Y,\|\cdot \|_Y)$ is a subspace of $X$,  we write $Y\hookrightarrow X$ and we say that $Y$ is \emph{embedded} in $X$  when the restricted identity map $i_Y:Y\to X$ is bounded. That is, there exists $C>0$ such that  for any $y\in Y$ we have $\|y\|_X\leq C\|y\|_Y$.  With this notation, H\"older's inequality states that  $\lpsi\hookrightarrow  \left[\lphi\right]^*$; and, it is easy to see that for every $N$-function $\Phi$ we have that $L^{\infty}_d\hookrightarrow\lphi \hookrightarrow L^1_d$.


 Recall that a function   $w:\mathbb{R}^+\to \mathbb{R}^+$ is called  a \emph{modulus of continuity} if $w$ is a continuous increasing function which satisfies $w(0)=0$. For example, it can be easily shown that $w(s)=s\Phi^{-1}(1/s)$ is a modulus of  continuity for every $N$-function $\Phi$.  We say that $u:[0,T]\to\rr^d$  has modulus of continuity $w$  when there exists a constant $C>0$ such that
\begin{equation}\label{w-holder}|u(t)-u(s)|\leq Cw(|t-s|).
\end{equation}


We denote by $C^w([0,T],\rr^d)$  the space of  $w$-H\"older continuous functions. This is the space of all functions satisfying \eqref{w-holder} for some $C>0$ and it is a Banach space with norm
\[\|u\|_{  C^w([0,T],\rr^d) }  :=\|u\|_{L^{\infty}}+\sup\limits_{t\neq s}\frac{|u(t)-u(s)|}{w(|t-s|)}.\]





 An important aspect of the theory of Sobolev spaces is related to embedding theorems. There is an extensive literature on this question in the  Orlicz-Sobolev space setting, see for example
 \cite{cianchi2000fully,cianchi1999some,claverooptimal,edmunds2000optimal,kerman2006optimal}.
The next simple lemma, whose proof can be found in \cite{ABGMS2015}, will be used systematically.




\begin{lem}\label{inclusion orlicz} Let  $w(s):= s\Phi^{-1}(1/s)$. Then, the following statements hold:
\begin{enumerate}
\item\label{inclusion orlicz_item1} $\wphi\hookrightarrow C^w([0,T],\rr^d) $ and for every $u\in\wphi$
\begin{align}
 &\left|u(t)-u(s) \right| \leq  \|u'\orlnor w(| t-s|),&\text{  (Morrey inequality).}\label{in-sob-cont}
\\
& \|u\|_{L^{\infty}} \leq\Phi^{-1}\left(\frac{1}{T}\right)\max\{1,T\}\|u\sobnor&\text{  (Sobolev inequality).}\label{sobolev}
\end{align}
\item For every $u\in\wphi$ we have $\widetilde{u}\in L^{\infty}_d$ and
\begin{align}
& \|\widetilde{u}\|_{L^{\infty}} \leq T\Phi^{-1}\left(\frac{1}{T}\right)\|\dot{u}\orlnor&\text{  (Sobolev-Wirtinger inequality).}\label{wirtinger}
\end{align}




\end{enumerate}
\end{lem}


The following result is analogous to some lemmata in $W^1L^p_d$, see \cite{xu2007some}.
\begin{lem}\label{infinito-a-prom-upunto}
If $\|u\sobnor\to \infty$, then $(|\overline{u}|+\|\dot{u}\orlnor)\to \infty$.
\end{lem}

\begin{proof}
By the decomposition $u=\overline{u}+\tilde{u}$ and some elementary operations,
we get
\begin{equation}\label{cota-u-lphi}
\|u\orlnor=
\|\overline{u}+\tilde{u}\orlnor\leq
\|\overline{u}\orlnor+\|\tilde{u}\orlnor=
|\overline{u}|\|1\orlnor+\|\tilde{u}\orlnor.
\end{equation}
It is known that $L^{\infty}_d\hookrightarrow\lphi$, i.e.
there exists $C_1=C_1(T)>0$ such that for any $\tilde{u}\in L^{\infty}_d$ we have
\[
\|\tilde{u}\orlnor
\leq 
C_1 \|\tilde{u}\|_{L^{\infty}};
\]
and, applying  Sobolev's inequality,  we obtain Wirtinger's inequality, that is there exists $C_2=C_2(T)>0$ such that 
\begin{equation}\label{cota-u-tilde}
\|\tilde{u}\orlnor
\leq 
C_2\|u'\orlnor.
\end{equation}

Therefore, from \eqref{cota-u-lphi}, \eqref{cota-u-tilde} and \eqref{def-norma-orlicz-sob}, 
we get
\[
\|u\sobnor\leq
C_3(|\overline{u}|+\|u'\orlnor)
\]
where $C_3=C_3(T)$. Finally, as $\|u\sobnor\to \infty$ we conclude that
$(|\overline{u}|+\|u'\orlnor)\to \infty$.
\end{proof}



\section{Lagrangians satisfying  sublinear nonlinearity type conditions}
\begin{lem}
Let $\Phi,\Psi$ complementary functions.
The next statements are equivalent:
\begin{enumerate}
\item\label{item1} $\Psi \in \Delta_2$ globally.
\item\label{item2} There exists an $N$-function $\Phi_1$ such that
\begin{equation}\label{eq:caract_delta2}
\Phi(rs)\geq \Phi_1(r)\Phi(s)\;\;\mbox{for every}\;\;r\geq1,\;\;s\geq 0.
\end{equation}
\end{enumerate}
\end{lem}

\begin{proof}
\ref{item1})$\Rightarrow$\ref{item2})
In virtue of the $\Delta_2$-condition on $\Psi$, \cite[Thm. 11.7]{M} and \cite[Cor. 11.6]{M} (see also  \cite[Eq. (2.8)]{AGMS}), we get constants $K>0$ and $\alpha_{\Phi}>1$ such that
\begin{equation}\label{delta2-consecuencia}
\Phi(r s)\geq Kr^{\nu}\Phi(s)
\end{equation}
for any $1<\nu<\alpha_{\Phi}$,  $s\geq 0$ and $r>1$. This proves  \eqref{eq:caract_delta2} with $\Phi_1(r)=kr^\nu$, which is is an $N$-function.

\ref{item2})$\Rightarrow$\ref{item1})
Next, we follow  \cite[p. 32, Prop. 13]{rao1991theory} and \cite[p. 29, Prop. 9]{rao1991theory}.
Assume that 
\[
\Phi_1(r)\Phi(s)\leq \Phi(rs)\;\;r>1,\;s\geq 0.
\]
Let $u=\Phi_1(r)\geq \Phi_1(1)$ and $v=\Phi(s)\geq 0$. By a well known inequality \cite[p. 13, Prop. 1]{rao1991theory} and \eqref{eq:caract_delta2},  we have  for $u\geq \Phi_1(1)$ and $v> 0$
\[
\frac{uv}{\Psi^{-1}(uv)}\leq \Phi^{-1}(uv)\leq\Phi_1^{-1}(u)\Phi^{-1}(v)\leq
\frac{4uv}{\Psi_1^{-1}(u)\Psi^{-1}(v)},
\]
then 
\[
\Psi^{-1}_1(u)\Psi^{-1}(v)\leq 4 \Psi^{-1}(uv).
\]
If we take $x=\Psi^{-1}_1(u)\geq \Psi^{-1}_1(\Phi_1(1))$ and $y=\Psi^{-1}(v)\geq 0$, then 
\[
\Psi\left(\frac{xy}{4}\right)\leq \Psi_1(x)\Psi(y).
\]
Now, taking  $x\geq \max\{8,\Psi_1^{-1}(\Phi_1(1))\}$ we get that $\Psi \in \Delta_2$ globally.
\end{proof}

The following lemma generalizes \cite[Lemma 5.2]{ABGMS2015}.

\begin{lem}\label{lem_coer}
Let $\Phi,\Psi$ be complementary $N$-functions and  $\Phi_1$ be any $N$-function satisfying \eqref{eq:caract_delta2}and suppose that $\Psi \in \Delta_2$ globally. Then
\begin{equation}\label{eq:coer_mod}
\lim\limits_{\|u\orlnor\to \infty}
\frac{\int_0^T \Phi(|u|)\,dt}{\Phi_0(\|u\orlnor)}=\infty,
\end{equation}


for every $\Phi_0$ with $\Phi_0=o(\Phi_1)$ at $\infty$ where.

Reciprocally if  \eqref{eq:coer_mod} holds for some $N$-function $\Phi_0$,  then $\Psi\in\Delta_2$ (at $\infty$). 
\end{lem}

\begin{proof}
By the assumptions on $\Phi$ and $\Phi_1$  and the inequality \eqref{amemiya-ine}, we have, for $r>1$,
\[
\frac{\int_0^T \Phi(|u|)\,dt}{\Phi_0(\|u\orlnor)}\geq
\Phi_1(r) \frac{\int_0^T \Phi(r^{-1}|u|)\,dt}{\Phi_0(\|u\orlnor)}\geq
\frac{\Phi_1(r)}{\Phi_0(\|u\orlnor)}\{r^{-1}\|u\orlnor-1\}.
\]
Now, we choose $r=\frac{\|u\orlnor}{2}$ and as $\|u\orlnor\to\infty$ we can assume $r>1$.
Next, we use the fact that $\Phi_1\in\Delta_2$ and
$\Phi_0=o(\Phi_1)$ at $\infty$, and  we get
\[
\lim\limits_{\|u\orlnor \to \infty} \frac{\int_0^T \Phi(|u|)\,dt}{\Phi_0(\|u\orlnor)}\geq
\lim\limits_{\|u\orlnor \to \infty} \frac{\Phi_1\left(\frac{\|u\orlnor}{2}\right)}{\Phi_0(\|u\orlnor)}
\geq
C \lim\limits_{\|u\orlnor \to \infty} \frac{\Phi_1(\|u\orlnor)}{\Phi_0(\|u\orlnor)}=\infty.
\]
The last assertion of the lemma follows from the fact that if $\Phi_0$ is an $N$-function, then $\Phi_0(u)\geq k|u|$ for  $k$ small enough and $|u1>1$. Therefore \eqref{eq:coer_mod} holds for $\Phi_0(u)=|u|$, then \cite[Lemma 5.2]{ABGMS2015}  implies  $\Psi\in\Delta_2$ at $\infty$.
\end{proof}


\begin{comentario}  We point out that this lemma can be applied to more cases than \cite[Lemma 5.2]{ABGMS2015}. For example, if $\Phi(u)=u^2$, $\Phi_1$ and $\Phi_0$ are  $N$-functions with principal parts equal to $u^2/\log u$ and $u^2/(\log u)^2$ respectively (see \cite[p. 16]{KR} and \cite[Section 7]{KR} for the definition and properties of principal part). Then  \eqref{eq:coer_mod} holds for $\Phi_0$, however $\Phi_0(u)$ is not dominated for any  power function $|u|^{\alpha}$ for every $\alpha<2$. 
\end{comentario}






\begin{defi}We define the  functionals $J_{C,\Phi_0}:\lphi\to (-\infty,+\infty]$ and $  H_{C,\Phi_0}:\rr^n\to \rr$, where $C>0$ and $\Phi_0$ is an $N$-function, by
\begin{equation}\label{func_phi}
  J_{C,\Phi_0}(u):= \rho_{\Phi}\left(u\right)-C\Phi_0\left(\|u\orlnor\right),
\end{equation}
 and

\begin{equation}\label{eq:functional_H-bis}
 H_{C,\Phi_0}(x):=\int_0^TF(t,x)dt-C\Phi_0(|x|),
\end{equation}
respectively.
\end{defi}











Like in \cite{ABGMS2015} we consider Lagrangians $\mathcal{L}$ which are lower bounded as follows 
\begin{equation}\label{cota_inf}
\mathcal{L}(t,x,y)\geq \alpha_0\Phi\left(\frac{|y|}{\Lambda}\right)+ F(t,x).
\end{equation}

In \cite{tang1998periodic} and \cite{tang2010periodic} was considered, for the $p$-laplacian case, potentials $F$ satisfying the inequality
\[ |\nabla F(t,x)|\leq b_1(t)|x|^{\alpha}+b_2(t),\]
where  $b_1,b_2 \in L^1_1$ and $\alpha$ is any power less than $p$. Thus, they said $F$ is a sublinear nonlinearity. In this paper, we will consider bounds on $\nabla F$ of a more general type.

\begin{defi} We said taht $\nabla F(t,x)$ satisfies a \emph{grow $\phi_0$-condition} if
\begin{equation}\label{holder_cont-mu}
  \left| \nabla F(t,x) \right|\leq b_1(t)\varphi_0(|x|)+b_2(t),
\end{equation}
where $\varphi_0=\Phi'_0$ with $\Phi_0$ an $N$-function.

\end{defi}
The employment of  $N$-functions instead of power functions in  inequalities like  \eqref{holder_cont-mu}  will allow us to extend some results of   \cite{tang1998periodic} and \cite{tang2010periodic} even in the $p$-laplacian case.


Based on \cite{mawhin2010critical} we say that $F$ satisfies the condition (A) if  $F(t,x)$ is a Carath\'eo\-dory function and  $F$ is continuously differentiable with respect to $x$. Moreover, the next inequality holds
\begin{equation}\label{condA2}|F(t,x)|+ |D_{x}F(t,x)|\leq a(|x|)b_0(t),\quad\text{for a.e. }t\in [0,T], \forall x\in\rr^d.
\end{equation}

The following theorem establishes  coercivity of $I$ assuming sublinear conditions on the nonlinearity  $\nabla F$. 












\begin{thm}\label{coercitividad-r}
Let  $\mathcal{L}$ be a lagrangian function satisfying \eqref{cotaL}, \eqref{cotaDxL}, \eqref{cotaDyL}, \eqref{cota_inf}  and suppose that $F$ satisfies condition (A). We assume the following conditions:
\begin{enumerate}
\item $\Psi\in\Delta_2$.
\item Inequality \eqref{holder_cont-mu} with $b_1,b_2 \in L^1_1$,  $\varphi_0=\Phi'_0$ where $\Phi_0$ is a differentiable $N$-function that satisfies the $\Delta_2$-condition globally such that
$\Phi_0=o(\Phi_1)$ at $\infty$ and $\Phi_1$ verifies \eqref{eq:caract_delta2}.
\item 
\begin{equation}\label{eq:propiedad-coercividad-phi0}
\lim_{|x|\to\infty}\frac{\int_{0}^{T}F(t,x)\ dt}{\Phi_0(|x|)}=+\infty.
\end{equation}
\end{enumerate}
Then  the action integral $I$ is coercive.
\end{thm}

\begin{proof}
By the decomposition $u=\overline{u}+\tilde{u}$,   Cauchy-Schwarz's inequality
and \eqref{holder_cont-mu}, we have
\begin{equation}\label{cota-diferencia-F}
\begin{split}
&\left|\int_0^T F(t,u)-F(t,\overline{u})\,dt\right|=
\left|\int_0^T \int_0^1 \nabla F(t,\overline{u}+s\tilde{u}(t))\ccdot \tilde{u}(t) \,ds \,dt\right|
\\
&\leq \int_0^T \int_0^1 b_1(t)\varphi_0(|\overline{u}+s\tilde{u}(t)|)|\tilde{u}(t)|\,ds\,dt+
\int_0^T \int_0^1 b_2(t)|\tilde{u}(t)|\,ds\,dt
\\
&=I_1+I_2.
\end{split}
\end{equation}
On the one hand, by H\"older's and Sobolev's inequalities, we estimate $I_2$ as follows
\begin{equation}\label{cota-i2}
I_2\leq \|b_2\|_{L^1} \|\tilde{u}\|_{L^{\infty}}\leq
C_1\|\dot{u}\orlnor,
\end{equation}
 where $C_1=C_1(\|b_2\|_{L^1}, T)$. 

On the other hand, since $\Phi_0 \in \Delta_2$ globally, then $\varphi_0 \in \Delta_2$ globally and 
consequently $\varphi_0$ is a quasi-subadditive function, i.e. there exists $C(\varphi_0)>0$ such that 
$\varphi_0(a+b)\leq C(\varphi_0)(\varphi_0(a)+\varphi_0(b))$ for every $a,b\geq 0$.
In this way, we have
\begin{equation}\label{pot-suma}
\varphi_0(|\overline{u}+s\tilde{u}(t)|)\leq
C(\varphi_0)[\varphi_0(|\overline{u}|)+\varphi_0(\|\tilde{u}\|_{L^{\infty}})],
\end{equation}
for every $s \in [0,1]$. 

Now,  inequality \eqref{pot-suma}, H\"older's and Sobolev's inequalities,
 the monotonicity, the subadditivity and  the $\Delta_2$-condition on $\varphi_0$, imply that
\begin{equation}\label{cota-i1}
\begin{split}
I_1&
\leq C(\varphi_0)\bigg\{ \varphi_0(|\overline{u}|) \|b_1\|_{L^1} \|\tilde{u}\|_{L^{\infty}}+
 \|b_1\|_{L^1}\varphi_0(\|\tilde{u}\|_{L^\infty})\|\tilde{u}\|_{L^\infty}\bigg\}
\\
&\leq C_2 \bigg\{ \varphi_0(|\overline{u}|) \|u'\orlnor
+\varphi_0(\|\dot{u}\orlnor) \|\dot{u}\orlnor\bigg\},
%\\
%&=
%C_2 \bigg\{ f(|\overline{u}|) \|u'\orlnor
%+\varepsilon(\|u'\orlnor)\|u'\orlnor^{\alpha_{\Phi}}
%%f(\|\dot{u}\orlnor) \|\dot{u}\orlnor\bigg\}
\end{split}
\end{equation}
where $C_2=C_2(\varphi_0,T, \|b_1\|_{L^1} )$. 

Next, by Young's inequality with complementary functions $\Phi_0$ and $\Psi_0$ and the fact that 
$\Phi_0 \in \Delta_2$ globally, Young's equality \cite[Eq. 2.7-2.8]{KR} and \cite[Th. 3-(ii), p. 23]{rao1991theory}, we get
\begin{equation}\label{cota-i1-parcial}
 \begin{split}
\varphi_0(|\overline{u}|) \|u'\orlnor
&\leq 
\Psi_0(\varphi_0(|\overline{u}|))+
\Phi_0(\|u'\orlnor)
\\
&\leq 
|\overline{u}|\varphi_0(|\overline{u}|)
+\Phi_0(\|u'\orlnor)
\\
&\leq C(\Phi_0)
\Phi_0(|\overline{u}|)
+\Phi_0(\|u'\orlnor)
\end{split}
\end{equation}
and 
\begin{equation}\label{cota-i1-parcial-segunda}
\varphi_0(\|\dot{u}\orlnor) \|\dot{u}\orlnor
\leq 
C(\Phi_0) \Phi_0(\|\dot{u}\orlnor),
\end{equation}
with $C(\Phi_0)$ the constant that comes from the $\Delta_2$-condition on $\Phi_0$.

From \eqref{cota-i1}, \eqref{cota-i1-parcial}, \eqref{cota-i1-parcial-segunda} and \eqref{cota-i2},
%and the inequality $x^{r_1}\leq x^{r_2}+1$, for any $x\geq 0$ and $r_1\leq r_2$, 
we have
\begin{equation}\label{cota-i1-i2}
\begin{split}
I_1+I_2
&
\leq C_3
\bigg\{ 
\Phi_0(|\overline{u}|)
+\Phi_0(\|u'\orlnor)
+\|u'\orlnor
\bigg\}\\
&
\leq C_4
\bigg\{ 
\Phi_0(|\overline{u}|)
+\Phi_0(\|u'\orlnor)
+1
\bigg\},
\end{split}
\end{equation}
with $C_3$ and $C_4$ depending on $\Phi_0,T, \|b_1\|_{L^1}$ and $\|b_2\|_{L^1} $. The last inequality follows from the fact that $\Phi_0$ is an $N$-function, then there exists $C>0$ such that $\Phi_0(x)\geq Cx$ for every $x\geq 1$. Thus $x\leq C\Phi_0(x)+1$ for every $x\geq 0$.


In the subsequent estimates, we use  \eqref{cota_inf}, \eqref{cota-diferencia-F},
\eqref{cota-i1-i2}, the fact that $\Phi_0 \in \Delta_2$ and we get
\begin{equation}\label{cota_inf_I}
\begin{split}
I(u)&\geq\alpha_0\rho_{\Phi}\left( \frac{u'}{\Lambda}\right)+\int_0^TF(t,u)\ dt
\\ 
&=\alpha_0\rho_{\Phi}\left( \frac{u'}{\Lambda}\right)+ \int_0^T \left[F(t,u)-F(t,\overline{u})\right]\ dt
+  \int_0^TF(t,\overline{u})\ dt
\\
&\geq \alpha_0\rho_{\Phi}\left( \frac{u'}{\Lambda}\right)
-C_4 \Phi_0(\|\dot{u}\orlnor)
+\int_0^TF(t,\overline{u})\ dt-
C_4 \Phi_0(|\overline{u}|)-
C_4 
\\
&\geq
\alpha_0\rho_{\Phi}\left( \frac{u'}{\Lambda}\right)
-C_4 \Phi_0(\|\dot{u}\orlnor)
+H_{C_4, \Phi_0}(\overline{u})
-C_4 
\\&\geq
\alpha_0\rho_{\Phi}\left( \frac{u'}{\Lambda}\right)
-C_5 \Phi_0\left(\frac{\|\dot{u}\orlnor}{\Lambda} \right)
+H_{C_4, \Phi_0}(\overline{u})
-C_4 
\\&=
\alpha_0J_{C_6,\Phi_0}\left(\frac{\dot{u}}{\Lambda}\right)
+H_{C_4, \Phi_0}(\overline{u})
-C_4, 
\end{split}
\end{equation}
where $C_5=C_5(\Phi_0,\Lambda,C_4)$ and $C_6=\frac{C_5}{\alpha_0}$.



Let $u_n$ be  a sequence in $\domi$ with
$\|u_n\sobnor\to\infty$ and we have to prove that $I(u_n)\to\infty$.
On the contrary, suppose  that for a subsequence, 
still denoted by $u_n$, $I(u_n)$ is upper bounded, i.e., there exists $M>0$ such that $|I(u_{n})|\leq M$.
As $\|u_n\sobnor\to\infty$, from Lemma \ref{infinito-a-prom-upunto},  we have $|\overline{u}_n|+\|u'_n\orlnor\to \infty$. Passing to a subsequence, still denoted $u_n$, we can assume that $|\overline{u}_n|\to \infty$ or $\|u'_n\orlnor\to \infty$.
Now, Lemma \ref{lem_coer} implies that the functional $J_{C_6,\Phi_0}(\frac{\dot{u}}{\Lambda})$ is coercive;
and, by \eqref{eq:propiedad-coercividad-phi0},
the functional $H_{C_4,\Phi_0}(\overline{u})$ is also coercive, then
$J_{C_6,\Phi_0}(\frac{\dot{u}_n}{\Lambda}) \to \infty$ or $H_{C_4,\Phi_0}(\overline{u}_n)\to \infty$.
From \eqref{condA2}, we have that on a bounded set the functional $H_{C_4,\Phi_0}(\overline{u}_n)$ is lower bounded and also $J_{C_6,\Phi_0}(\frac{\dot{u}_n}{\Lambda})\geq 0$.
Therefore,  $I(u_n)\to\infty$ as $\|u_n\sobnor\to\infty$ which contradicts the initial assumption on the behavior of $I(u_n)$.
\end{proof}













% 
% 
% 
% 
% \subsection{versi\'on potencias por si acaso...} 
% \begin{thm}\label{coercitividad-r}
% Let  $\mathcal{L}$ be a lagrangian function satisfying \eqref{cotaL}, \eqref{cotaDxL}, \eqref{cotaDyL}, \eqref{cota_inf}  and $F$ satisfies condition (A). We assume the following conditions:
% \begin{enumerate}
% \item $\Psi\in\Delta_2$.
% \item There exist  non negative functions  $b_1,b_2 \in L^1_1$ and a constant $1<\mu<\alpha_{\Phi}$  such that 
% for any $x\in\rr^d$ and a.e. $t\in [0,T]$
% \begin{equation}\label{holder_cont-mu}
%   \left| \nabla F(t,x) \right|\leq b_1(t)|x|^{\mu-1}+b_2(t).
% \end{equation}
% \item There exists a real positive number $\sigma$ such that $\sigma>(\mu-1)\beta_{\Psi}$ and
% \begin{equation}\label{propiedad-coercividad-sigma}
% |x|^{\sigma}=o\left(\int_{0}^{T}F(t,x)\ dt\right)\;\;\mbox{as}\;\;|x|\to \infty.
% \end{equation}
% \end{enumerate}
% Then  the action integral $I$ is coercive.
% \end{thm}
% 
% 
% 
% 
% \begin{proof} 
% By the decomposition $u=\overline{u}+\tilde{u}$,  Mean Value Theorem, Cauchy-Schwarz's inequality
% and \eqref{holder_cont-mu}, we have
% \begin{equation}\label{cota-diferencia-F}
% \begin{split}
% &\left|\int_0^T F(t,u)-F(t,\overline{u})\,dt\right|=
% \left|\int_0^T \int_0^1 \nabla F(t,\overline{u}+s\tilde{u}(t))\ccdot \tilde{u}(t) \,ds \,dt\right|
% \\
% &\leq \int_0^T \int_0^1 b_1(t)|\overline{u}+s\tilde{u}(t)|^{\mu-1}|\tilde{u}(t)|\,ds\,dt+
% \int_0^T \int_0^1 b_2(t)|\tilde{u}(t)|\,ds\,dt
% \\
% &=I_1+I_2.
% \end{split}
% \end{equation}
% On the one hand, by H\"older's inequality and Sobolev's inequality, we estimate $I_2$ as follows
% \begin{equation}\label{cota-i2}
% I_2\leq \|b_2\|_{L^1} \|\tilde{u}\|_{L^{\infty}}\leq
% C_1\|\dot{u}\orlnor.
% \end{equation}
%  where $C_1=C_1(\|b_2\|_{L^1}, T)$. On the other hand, as $s\in [0,1]$, we have
% \begin{equation}\label{pot-suma}
% |\overline{u}+s\tilde{u}(t)|^{\mu-1}\leq
% C(\mu)(|\overline{u}|^{\mu-1}+\|\tilde{u}\|_{L^{\infty}}^{\mu-1}).
% \end{equation}
% where $C(\mu)=2^{\mu-2}$, for $\mu\geq 2$ and $C(\mu)=1$, for $1<\mu<2$. Now,  inequality \eqref{pot-suma}, H\"older's inequality and Sobolev's inequality imply that
% \begin{equation}\label{cota-i1}
% \begin{split}
% I_1&\leq 
% C(\mu)\left(|\overline{u}|^{\mu-1} \int_0^T b_1(t) |\tilde{u}(t)|\,dt+
% \|\tilde{u}\|^{\mu-1}_{L^{\infty}} \int_0^T b_1(t)|\tilde{u}(t)| \,dt\right)
% \\
% &\leq C(\mu)\bigg\{ |\overline{u}|^{\mu-1} \|b_1\|_{L^1} \|\tilde{u}\|_{L^{\infty}}+
%  \|b_1\|_{L^1}\|\tilde{u}\|^{\mu}_{L^\infty}\bigg\}
% \\
% &\leq C_2 \bigg\{ |\overline{u}|^{\mu-1} \|u'\orlnor+ \|\dot{u}\orlnor^{\mu}\bigg\},
% \end{split}
% \end{equation}
% where $C_2=C_2(\mu,T, \|b_1\|_{L^1} )$. Let $\mu'$ be a positive constant such that $1<\mu\leq \mu'<\alpha_{\Phi}$. 
% Next, using Young's inequality with conjugate exponents $\mu'$ and $\frac{\mu'}{\mu'-1}$ 
%  we get
% \begin{equation}\label{cota-i1-parcial}
% |\overline{u}|^{\mu-1}   \|u'\orlnor
% \leq \frac{(\mu'-1)}{\mu'}|\b{\overline{u}|^{\sigma}}
% +\frac{1}{\mu'} \|u'\orlnor^{\mu'}
% \end{equation}
% where $\sigma=\frac{(\mu-1) \mu'}{\mu'-1}$. We point out that $\sigma$ is an arbitrary positive constant bigger than $(\mu-1)b_{\Psi}$.
% 
% From \eqref{cota-i1}, \eqref{cota-i1-parcial}, \eqref{cota-i2} and the inequality $x^{r_1}\leq x^{r_2}+1$, for any $x\geq 0$ and $r_1\leq r_2$, we have
% \begin{equation}\label{cota-i1-i2}
% \begin{split}
% I_1+I_2
% &\leq C_3\bigg\{ |\overline{u}|^{\sigma}
% + \|\dot{u}\orlnor^{\mu'}
% + \|\dot{u}\orlnor^{\mu}
% +\|\dot{u}\orlnor\bigg\}\\
% &\leq C_3\bigg\{ |\overline{u}|^{\sigma}
% + \|\dot{u}\orlnor^{\mu'}
% +1\bigg\}
% \end{split}
% \end{equation}
% with $C_3= C_3(\mu,T, \|b_1\|_{L^1},\mu' )$. In the subsequent estimates, we use the decomposition $u=\overline{u}+\tilde{u}$, \eqref{cota_inf}, \eqref{cota-diferencia-F},
% \eqref{cota-i1-i2} and we get
% \begin{equation}\label{cota_inf_I}
% \begin{split}
% I(u)&\geq\alpha_0\rho_{\Phi}\left( \frac{u'}{\Lambda}\right)+\int_0^TF(t,u)\ dt
% \\ 
% &=\alpha_0\rho_{\Phi}\left( \frac{u'}{\Lambda}\right)+ \int_0^T \left[F(t,u)-F(t,\overline{u})\right]\ dt
% +  \int_0^TF(t,\overline{u})\ dt
% \\
% &\geq \alpha_0\rho_{\Phi}\left( \frac{u'}{\Lambda}\right)
% -C_3 \|\dot{u}\orlnor^{\mu'}
% +\int_0^TF(t,\overline{u})\ dt-
% C_3 |\overline{u}|^{\sigma}-C_3\\
% &=\alpha_0J_{C_4,\mu'}\left(\frac{\dot{u}}{\Lambda}\right)
% + H_{C_3,\sigma}(\overline{u})-C_3,
% \end{split}
% \end{equation}
% where $C_4=\Lambda^{\mu'}C_3/\alpha_0$.
% 
% 
% Let $u_n$ be  a sequence in $\domi$ with
% $\|u_n\sobnor\to\infty$ and we have to prove that $I(u_n)\to\infty$.
% On the contrary, suppose  that for a subsequence, 
% still denoted by $u_n$, $I(u_n)$ is upper bounded, i.e., there exists $M>0$ such that $|I(u_{n})|\leq M$.
% As $\|u_n\sobnor\to\infty$, from Lemma \ref{infinito-a-prom-upunto},  we have $|\overline{u}_n|+\|u'_n\orlnor\to \infty$.
% Then, there exists a subsequence of $\{u_n\}$, still denoted by $u_n$, which is not bounded.
% Then, 
% $|\overline{u}_n|\to \infty$ or $\|u'_n\orlnor\to \infty$.
% Now, Lemma \ref{lem_coer} implies that the functional $J_{C_4,\mu'}(\frac{\dot{u}}{\Lambda})$ is coercive,
% and, by \eqref{propiedad-coercividad-sigma},
% the functional $H(\overline{u})$ is also coercive, then
% $J_{C_4,\mu'}(\frac{\dot{u}_n}{\Lambda}) \to \infty$ or $H(\overline{u}_n)\to \infty$.
% From \eqref{condA2}, we have that on a bounded set the functional $H(\overline{u}_n)$ is lower bounded; and, $J_{C_4,\mu'}(\frac{\dot{u}_n}{\Lambda})$ is also lower bounded  because the modular $\rho_{\Phi}\left(\frac{\dot{u}}{\Lambda}\right)$ is always bigger than zero.
% Therefore,  $I(u_n)\to\infty$ as $\|u_n\sobnor\to\infty$ which contradicts the initial assumption on the behavior of $I(u_n)$.
% \end{proof}
% 
% {\bf Leer y ver si es coherente lo anterior,  si conviene trabajar siempre con $u_n$ o habr\'ia que usar la notaci\'on de subsucesiones expl\'icita!!!}
% 
% {\bf Falta leer y corregir la  secci\'on que sigue del caso l\'imite!!!}



\section{Main result}\label{sec:main}



In order to find conditions for the lower semicontinuity of  $I$, 
we perform a little adaptation of  a result of \cite{ekeland1999convex}. 


\begin{lem}\label{semicontinf}
Let $\mathcal{L}(t,x,y)$ be a  differentiable Carath\'eodory function. Suppose that  $F$ satisfies the condition (A) and the inequality
\begin{equation}\label{cota_inf_2}
\mathcal{L}(t,x,y)\geq \Phi\left(|y|\right)+ F(t,x),
\end{equation}
where $\Phi$ is an $N$-function. 
In addition, suppose that  $\mathcal{L}(t,x,\cdot)$ is convex in $\rr^d$ for each $(t,x)\in [0,T]\times\rr^d$.  Let $\{u_n\}\subset\wphi$ be a sequence such that $u_n$ converges  uniformly  to a function $u\in\wphi$ and $u'_n$ converges in the weak topology of $L^1_d$ to $u'$.   Then
\begin{equation}\label{liminf0}I(u)\leq \liminf_{n\to\infty}I(u_n).
\end{equation}

\end{lem}

\begin{proof} First, we point out that \eqref{cota_inf_2} and \eqref{condA2} imply that $I$ is defined on $\wphi$ taking values on the interval $(-\infty,+\infty]$. 
Let $\{u_n\}$ be a sequence  satisfying the assumptions of the theorem.   We define the differentiable Carath\'eodory function $\mathcal{\hat{L}}=\mathcal{L}-F$ and we denote by $\hat{I}$ its  associated action integral. Using  \cite[Thm. 2.1, p. 243]{ekeland1999convex}, we get
\begin{equation}\label{liminf1}
\int_0^T\mathcal{\hat{L}}(t,u,u')\ dt\leq \liminf_{n\to\infty}\int_0^T\mathcal{\hat{L}}(t,u_n,u'_n)\ dt.
\end{equation}
Taking account of the uniform convergence of $u_n$ and the fact that  $F$  is a  Carath\'eodory function,  we obtain that $F(t,u_n(t))\to F(t,u(t))$ a.e. $t\in[0,T]$.  Since the sequence $u_n$ is uniformly bounded, from \eqref{condA2} follows that there exists $g\in L_1^1([0,T])$ such that $|F(t,u_n(t))|\leq g(t)$. Now, by the Dominated Convergence Theorem, we have that
\begin{equation}\label{liminf2}
\lim_{n\to\infty}\int_0^TF(t,u_n(t))\ dt=\int_0^TF(t,u(t))\ dt.
\end{equation}
Finally, as a consequence of  \eqref{liminf1} and  \eqref{liminf2}, we obtain \eqref{liminf0}.
\end{proof}


\begin{lem}\label{lem:deb*cerrado}
$\ephi$ is weak* closed in $\lphi$.
\end{lem}


\begin{proof}
From \cite[Thm. 7, p. 110]{rao1991theory} we have that $\lphi=\left[\epsi_d\right]^*
$.
Then, $\lphi$ is a dual and therefore we are allowed to speak about the weak* topology of $\lphi$.
Besides, $\ephi
$ is separable (see \cite[Thm. 1, p. 87]{rao1991theory}).
Let $S=\ephi\cap \{u \in \lphi|\|u\orlnor\leq 1\}$, then $S$ is closed in the norm $\|\cdot\orlnor$. Now, according to \cite[Cor. 5, p. 148]{rao1991theory} $S$ is weak* sequentially compact. Thus, $S$ is weak* sequentially closed because
is $u_n\in S$ and
$u_n \overset{*}{\rightharpoonup}u \in \lphi$ then  the weak* sequentially compactness implies the existence of $v \in S$ and a subsequence $u_{n_k}$ such that 
$u_{n_k}\overset{*}{\rightharpoonup}v$. Finally, by the uniqueness of   the limit, we get
$u=v\in S$.
As $\epsi_d$ is separable and $\lphi=\left[\epsi_d\right]^*$, the ball of $\lphi$ $\{u \in \lphi | \|u\orlnor\leq 1\}$ is  weak* metrizable (see \cite[Thm. 5.1, p. 138]{Conway1977}).
Thus, $S$ is closed respect to  the weak* topology. Now, by the Krein-Smulian Theorem, \cite[Cor. 12.6, p. 165]{Conway1977} implies that $\ephi$ is weak* closed.
\end{proof}

Gathering our previous results we obtain existence of solutions.

Let $\wphiet=\wphi_T \cap \wphie_d$.


\begin{thm} 
Let $\Phi$ and $\Psi$ be complementary $N$-functions. 
Suppose that the differentiable Carath\'eodory function $\mathcal{L}(t,x,y)$ is strictly convex at $y$, $D_{y}\mathcal{L}$ is $T$-periodic with respect to $t$. In addition, assume the same hypothesis than Theorem \ref{coercitividad-r}. Then, problem \eqref{ProbPrin} has a solution.
\end{thm}

\begin{proof}



Let $\{u_n\}\subset \wphiet$  be a  minimizing sequence for the problem  $\inf\{I(u)|u\in\wphiet\}$.
Since  $I(u_n)$, $n=1,2,\ldots$  is upper bounded, Theorem \ref{coercitividad-r}  implies that $\{u_n\}$ is norm bounded in $\wphie_d$. Hence, in virtue of Corollary \cite[Corollary 2.2]{ABGMS2015}, we can assume, taking a subsequence if necessary, that $u_n$ converges uniformly to a $T$-periodic continuous function $u$.
Then, $u$ is bounded and $u \in \ephi$.

As $u'_n \in \ephi\subset \lphi$,
%The space  $\lphi$ is a predual space, concretely $\lphi=\left[\epsi_d\right]^*$. Thus, by \cite[Cor. 5, p. 148]{rao1991theory} and since $u'_n$ is bounded in $\lphi$,
there exists a subsequence (again denoted by $u'_n$) such that $u'_n$ converges to a function $v\in\lphi$ in the weak* topology of $\lphi$.
Since $\ephi$ is weak* closed, by Lemma \ref{lem:deb*cerrado}, $v\in \ephi$.

From this fact and the uniform convergence of $u_n$ to $u$, we obtain that
\[
\int_0^T\b{\dot{\xi}}\cdot u\ dt=\lim_{n\to\infty}\int_0^T\b{\dot{\xi}}\cdot u_n \ dt=-\lim_{n\to\infty}\int_0^T\b{\xi}\cdot u'_n\ dt=-\int_0^T\b{\xi}\cdot v\ dt
\]
for every $T$-periodic function $\b{\xi}\in C^{\infty}([0,T],\rr^d)\subset\epsi_d$.
Thus $v=u'$ a.e. $t\in [0,T]$ (see \cite[p. 6]{mawhin2010critical}) and $u\in\ephi_T$.

Now, taking into account the relations $\left[L^1_d\right]^*=L^{\infty}_d\subset  \epsi_d$ and $\lphi\subset L^1_d$, we have that $u'_n$ converges to $u'$ in the weak topology of $L^1_d$. Consequently,  Lemma \ref{semicontinf} applied to the $N$-function $\alpha_0\Phi\left(|\ccdot|/\Lambda\right)$ implies that
\[I(u)\leq  \liminf_{n\to\infty}I(u_n)=\inf\limits_{u\in\wphie_T}I(u).\]

As $u\in \wphiet\subset \domi$ then $I(u)>-\infty$, hence, $u$ is a minimun and therefore  $I'(u)\in (\wphiet)^{\perp}$. Finally,
invoking Theorem \ref{critpoint},  the proof concludes.\end{proof}



 \section{Limit case $\mu=\alpha_{\Phi}$}
Assuming $\|b_1\|_{L^1}$  small enough, in  \cite{zhao2004periodic, tang2010periodic} 
coercivity  was obtained even  for the limit value $\mu=p$ in inequality \eqref{holder_cont-mu}.  

{\bf OJO que $\mu$ no aparece en \eqref{holder_cont-mu}!!!!. Quiz\'as deber\'ia decir $\varphi_0(x)=x^p$. O, mecionarse la ecuaci\'on anterior donde aparece $\alpha<p$, no $\mu$.} 

This result leans on the  fact that
\begin{equation}
 \|u\orlnor^{\alpha_{\Phi}}=O\left(\int_0^T \Phi(|u|)\,dt\right)\quad\text{for } \|u\orlnor\to\infty,
\end{equation}
when  $\Phi(u)=|u|^p$.
Nevertheless, it is no longer the case  for any $N$-function $\Phi$ as the following example shows.

In this section, from now on we will suppose that
\[\Phi(u)=
\left\{
\begin{array}{ll}
\frac{p-1}{p}u^p&u\leq e
\\
\frac{u^p}{\log u}-\frac{e^p}{p}&u>e
\end{array}
\right.\]
with $p>1$. Next, we will establish some properties of this function $\Phi$.

\begin{thm}
If $p\geq \frac{1+\sqrt 2}{2}$, then $\Phi$ is an $N$-function.
\end{thm}


\begin{proof}
We have
\[\varphi(u)=\Phi'(u)=\left\{
\begin{array}{cccc}
(p-1)u^{p-1}&:=&\varphi_1(u)& \mbox{if}\;u\leq e
\\
\frac{u^{p-1}}{\log u}(p-\frac{1}{\log u})&:=&\varphi_2(u)&\mbox{if}\; u\geq e
\end{array}
\right.
\]

First let us see that $\Phi'$ is increasing when $p\geq \frac{1+\sqrt {2}}{2}$.
For this purpose, since $\varphi_1(e)=\varphi_2(e)$, it is enough to see that $\varphi_1$ is increasing  on $[0,e]$ and $\varphi_2$ is increasing on
$[e,\infty)$ for every $p\geq \frac{1+\sqrt {2}}{2}$. Clearly
$\varphi_1$ is an increasing function for $p>1$.  On the other hand, an elementary analysis of the function shows that
$\varphi_2'(u)>0$ on $[e,\infty)$ if and only if
 $p \notin(\frac{1-\sqrt2}{2},\frac{1+\sqrt2}{2})$.  Therefore $\varphi_2$ is an icreasing function when $p\geq \frac{1+\sqrt2}{2}$.

 Besides $\varphi_2(u)\to \infty$ and  $\varphi_1(u)\to 0$  as $u \to  \infty$ and $u\to 0$  respectively, provided that $p>1$. Hence, $\Phi$ is an $N$-function.
\end{proof}


\begin{thm} For every $\varepsilon>0$, there exists a positive constant $C=C(p,\varepsilon)$  such that
\begin{equation}\label{cota-sup-indices}
C^{-1}t^{p-\varepsilon}\Phi(u)\leq \Phi(tu) \leq Ct^p\Phi(u)\quad t\geq 1, u>0,
\end{equation}
\end{thm}

\begin{proof} If $u\leq tu\leq e$, then $\Phi(tu)=t^p\Phi(u)$ and \eqref{cota-sup-indices} holds with $C=1$.

If $u\leq e\leq tu$, as $\frac{e^p}{p}>0$  and $\log(tu)\geq 1$, we have
$\Phi(tu)\leq t^pu^p= \frac{p}{p-1}t^p\Phi(u)$. Thus, the second inequality of  \eqref{cota-sup-indices} holds with $C=\frac{p}{p-1}$. On the other hand, as $f(t)=\frac{t}{\log t}$ is increasing on $[e,\infty)$, then $f((tu)^p)\geq  f(e^p)=e^p/p$.
Now,
\[
\begin{split}
\Phi(tu)&=\frac{p(tu)^p}{\log (tu)^p}-\frac{e^p}{p}\\
&= \frac{(p-1)(tu)^p}{\log(tu)^p}+\frac{(tu)^p}{\log (tu)^p}-\frac{e^p}{p}
\\
&\geq \frac{p-1}{p}\frac{(tu)^p}{\log(tu)}\\
&\geq
\frac{p-1}{p}\frac{t^{\varepsilon}}{\log t+1}t^{p-\varepsilon}u^p.
\end{split}
\]
Since $\varepsilon e^{1-\varepsilon}$ is the minimum value of $t\mapsto\frac{t^{\varepsilon}}{\log t+1}$  on the interval $[1,+\infty)$ then
\[
\Phi(tu)\geq \frac{p-1}{p}\varepsilon e^{1-\varepsilon}t^{p-\varepsilon}u^p,
\]
which is the first inequality of \eqref{cota-sup-indices} with $C=\frac{p}{p-1}\varepsilon^{-1} e^{-1+\varepsilon}$.


If $e\leq u\leq tu$, then
\begin{equation}\label{Phi-de-u-a-v}
\Phi(tu)\leq \frac{t^pu^p}{\log(tu)}\leq \frac{t^pu^p}{\log(u)}=\frac{pt^pv}{\log v},
\end{equation} 
where $v:=u^p$ and $v\geq e^p$.  
If $\alpha>0$, the function $x\mapsto \frac{x}{x-\alpha}$ is decreasing on $(\alpha,\infty)$
and the function $v\mapsto \frac{pv}{\log v}$ is increasing  on $[e^p,\infty)$.
Therefore,  we have
\[
\frac{\frac{pv}{\log v}}{\frac{pv}{\log v}-\frac{e^p}{p}}\leq
\frac{e^p}{e^p-\frac{e^p}{p}}=\frac{p}{p-1}
\]
for every $v \geq e^p$. In this way, from \eqref{Phi-de-u-a-v}, we have
\[
\Phi(tu)\leq \frac{pt^p}{p-1}\left(\frac{pv}{\log v}-\frac{e^p}{p}\right)=
 \frac{pt^p}{p-1}\left(\frac{u^p}{\log u}-\frac{e^p}{p}\right)
\]
and the second inequality of  \eqref{cota-sup-indices} holds with $C=\frac{p}{p-1}$. For the first inequality we have, as it was proved previously,

\[
  \Phi(tu)
  \geq
  \frac{p-1}{p}\frac{(tu)^p}{\log(tu)}
  =
  \frac{p-1}{p}
  \frac{t^{\varepsilon} \log u^{\varepsilon}}{\log(t^{\varepsilon}u^{\varepsilon})}
  \frac{t^{p-\varepsilon}u^p}{\log u}
\]
Let $f(s)=\frac{sA}{\log s+A}$ with $s\geq 1$ and $A\geq \varepsilon$.  If $A\leq 1$,  the function $f$ attains a minimum on $[1,\infty)$ at $s=e^{1-A}$ and the minimum value is $f(e^{1-A})=Ae^{1-A}\geq \varepsilon$. If $A> 1$, $f$ is increasing  on $[1,\infty)$ and its minimum value is $f(1)=1$. Then, $f(s)\geq \varepsilon$ in any case,   therefore
\[
\Phi(tu)\geq \frac{p-1}{p}\varepsilon \frac{t^{p-\varepsilon}u^p}{\log u}\geq
\frac{p-1}{p}\varepsilon t^{p-\varepsilon}\Phi(u).
\]
Therefore, \eqref{cota-sup-indices} holds with $C=\frac{p}{\varepsilon (p-1)}$, because this $C$ is the biggest constant that we have obtained in each case under consideration.
\end{proof}



\begin{comentario}
The inequality
\[
\Phi(tu)\geq Ct^p\Phi(u)
\]
is false for every $C$ because for every $u\geq e$ we have
\[
\lim\limits_{t \to \infty}\frac{\Phi(tu)}{t^p\Phi(u)}=0
\]
\end{comentario}






\begin{thm}
$\alpha_{\Phi}=\beta_{\Phi}=p$
\end{thm}

\begin{proof}
From \eqref{MO_indices} and \eqref{cota-sup-indices}, we get
\[
\beta_{\Phi}=\lim\limits_{t \to \infty} \frac{\log\left[\sup\limits_{u>0} \frac{\Phi(tu)}{\Phi(u)}\right]}{\log t}
\leq
\lim \limits_{t \to \infty} \frac{\log C+p\log t}{\log t}=p.
\]
On the other hand, employing \eqref{MO_indices} and performing some elementary calculations, we obtain
\[
\alpha_{\Phi}=
\lim\limits_{t \to 0^+} \frac{\log\left[\sup\limits_{u>0} \frac{\Phi(tu)}{\Phi(u)}\right]}{\log t}=
\lim\limits_{s \to \infty} \frac{\log\left[\sup\limits_{v>0} \frac{\Phi(v)}{\Phi(sv)}\right]^{-1}}{\log s}=
\lim\limits_{s \to \infty} \frac{\log\left[\inf\limits_{v>0} \frac{\Phi(sv)}{\Phi(v)}\right]}{\log s}
\]
where $v:=tu$ and $s:=\frac{1}{t}$.
Then, using \eqref{cota-sup-indices},  for every $\varepsilon>0$ we have
\[
\alpha_{\Phi}=
\lim\limits_{s \to \infty} \frac{\log\left[\inf\limits_{v>0} \frac{\Phi(sv)}{\Phi(v)}\right]}{\log s}\geq
\lim\limits_{s \to \infty} \frac{\log C+(p-\varepsilon)\log s}{\log s}\geq p-\varepsilon,
\]
therefore $\alpha_{\Phi}\geq p$.

Finally, as $\alpha_{\Phi}\leq \beta_{\Phi}\leq p$, we get
$\alpha_{\Phi}=\beta_{\Phi}=p$.
\end{proof}



Now, we are able to see that
\[
\rho_{\Phi}(u)=\int_0^T \Phi(|u|)\,dx\geq C\|u\orlnor^{\alpha_{\Phi}}=C\|u\orlnor^p
\]
is false.

In fact, if we take $u\equiv t>0$, then $\|u\orlnor^p=C_1t^p$ where $C_1=\|1\orlnor$ and
$\int_0^T \Phi(|u|)\,dx=C_2\Phi(t)$ with $C_2=T$.
Then, if $\rho_{\Phi}(u)\geq C\|u\orlnor^p$ were true, then $\Phi(t)\geq C t^p$ would also be true; however, this
last inequality is false.
 

\section*{Acknowledgments}
The authors are partially supported by a UNRC grant number 18/C417. The first author is  partially supported by a  UNSL grant number 22/F223. 


 \bibliographystyle{elsarticle-num} 
 \bibliography{biblio}


\end{document}
