\documentclass[twoside]{article}


\NeedsTeXFormat{LaTeX2e}
\ProvidesPackage{mathscinet}[2002/04/17 v1.05]
\RequirePackage{textcmds}\relax
\ProvideTextCommandDefault{\cprime}{\tprime}



%\usepackage{hyperref}
\usepackage{amssymb,amsthm}
\usepackage{amsmath}
\usepackage{color}
\usepackage{ esint }

\usepackage{fancyhdr}
\usepackage{times}

\usepackage[latin1]{inputenc}

\usepackage{comment}
\usepackage{url}
\usepackage{xcolor}
\usepackage{adjustbox}
\newtheorem{thm}{Theorem}[section]
\newtheorem{cor}[thm]{Corollary}
\newtheorem{lem}[thm]{Lemma}
\newtheorem{rem}[thm]{Remark}
\newtheorem{defi}[thm]{Definition}
\newtheorem{prop}[thm]{Proposition}
\theoremstyle{remark}
\newtheorem{comentario}{Remark}


\title{Periodic solutions of 
Euler-Lagrange equations with ``sublinear nonlinearity'' in an Orlicz-Sobolev space setting}
\author{Sonia Acinas \thanks{SECyT-UNRC, UNSL and CONICET}\\
Instituto de Matem\'atica Aplicada San Luis (CONICET-UNSL)\\
(5700) San Luis, Argentina\\
Universidad Nacional de La Pampa\\
(6300) Santa Rosa, La Pampa, Argentina\\
\url{sonia.acinas@gmail.com}\\[3mm]
Fernando D. Mazzone \thanks{SECyT-UNRC and CONICET}\\
Dpto. de Matem\'atica, Facultad de Ciencias Exactas, F\'{\i}sico-Qu\'{\i}micas y Naturales\\
Universidad Nacional de R\'{i}o Cuarto\\
(5800) R\'{\i}o Cuarto, C\'ordoba, Argentina,\\
\url{fmazzone@exa.unrc.edu.ar}
}

\date{}

\newcommand{\orlnor}{\|_{L^{\Phi}}}
\newcommand{\lurnor}{\|^{*}_{L^{\Phi}}}
\newcommand{\linf}{\|_{L^{\infty}}}
\newcommand{\lphi}{L^{\Phi}}
\newcommand{\lpsi}{L^{\Psi}}
\newcommand{\ephi}{E^{\Phi}}
\newcommand{\claseor}{C^{\Phi}}
\newcommand{\wphi}{W^{1}\lphi}
\newcommand{\sobnor}{\|_{W^{1}\lphi}}
\newcommand{\domi}{\mathcal{E}^{\Phi}_d(\lambda)}
\renewcommand{\b}[1]{\boldsymbol{#1}}
\newcommand{\rr}{\mathbb{R}}
\newcommand{\nn}{\mathbb{N}}
\newcommand{\ccdot}{\b{\cdot}}
\renewcommand{\leq}{\leqslant} 
\renewcommand{\geq}{\geqslant} 
\newcommand{\epsi}{E^{\Psi}}

\begin{document}



\maketitle
%
\begingroup%Locallizing the change to `thefootnote'.
    \renewcommand{\thefootnote}{}%Removing the footnote symbol.
    %
    \footnotetext{%
    %   2010 Mathematics Subject Classification
    %   http://www.ams.org/msc/
    \textbf{2010  AMS Subject Classification.} Primary: .
    Secondary: .
    }%
        \footnotetext{%
    \textbf{Keywords and phrases.}  .
    }%
    \endgroup
%
%
%
%

\begin{abstract}
In this paper we....
\end{abstract}




\pagestyle{fancy} \headheight 35pt \fancyhead{} \fancyfoot{}

\fancyfoot[C]{\thepage} \fancyhead[CE]{\nouppercase{S. Acinas and F.D. Mazzone }} \fancyhead[CO]{\nouppercase{\section}}

\fancyhead[CO]{\nouppercase{\leftmark}}


%\tableofcontents




\section{Introduction}
This paper is concerned with the existence of periodic solutions of the problem
\begin{equation}\label{ProbPrin}
    \left\{%
\begin{array}{ll}
   \frac{d}{dt} D_{y}\mathcal{L}(t,\b{u}(t),\b{\dot{u}}(t))= D_{\b{x}}\mathcal{L}(t,\b{u}(t),\b{\dot{u}}(t)) \quad \hbox{a.e.}\ t \in (0,T)\\
    \b{u}(0)-\b{u}(T)=\b{\dot{u}}(0)-\b{\dot{u}}(T)=0
\end{array}%
\right.
\end{equation}
where $T>0$, $\b{u}:[0,T]\to\rr^d$ is absolutely continuous and the \emph{Lagrangian} $\mathcal{L}:[0,T]\times\rr^d\times\rr^d\to\rr$ is a Carath\'eodory function satisfying the conditions
\begin{eqnarray}
|\mathcal{L}(t,\b{x},\b{y})| &\leq a(|\b{x}|)\left(b(t)+ \Phi\left(\frac{|\b{y}|}{\lambda}+f(t) \right)\right),\label{cotaL}\\
|D_{\b{x}}\mathcal{L}(t,\b{x},\b{y})| &\leq a(|\b{x}|)\left(b(t)+ \Phi\left(\frac{|\b{y}|}{\lambda}+f(t) \right)\right),\label{cotaDxL}\\
|D_{\b{y}}\mathcal{L}(t,\b{x},\b{y})| &\leq a(|\b{x}|)\left(c(t)+ \varphi\left(\frac{|\b{y}|}{\lambda}+f(t)\right)  \right).\label{cotaDyL}
\end{eqnarray}
In these inequalities we assume that  $a\in C(\mathbb{R}^+,\mathbb{R}^+)$, $\lambda>0$, $\Phi$ is an $N$-function (see section  Preliminaries  for definitions), $\varphi$ is the right continuous derivative of $\Phi$. The non negative functions $b,c$ and $f$ satisfy that  $b\in L^1_1([0,T]) $,  $c\in\lpsi_1([0,T])$ and  $f\in \ephi_1([0,T])$, where  the Banach spaces $ L^1_1([0,T]), \lpsi_1([0,T])$ and  $\ephi_1([0,T])$  will be defined later.


It is well known that problem \eqref{ProbPrin} comes from a variational one, that is,  a solution of \eqref{ProbPrin}  
is a critical point of the \emph{action integral}
\begin{equation}\label{integral_accion}
I(\b{u})=\int_{0}^T \mathcal{L}(t,\b{u}(t),\b{\dot{u}}(t))\ dt.
\end{equation}




\section{Preliminaries}\label{preliminares}

For reader convenience, we give a short introduction to Orlicz and Orlicz-Sobolev spaces of vector valued functions and a  list  of results that we will use throughout the article. 
Classic references for Orlicz spaces of real valued functions are \cite{adams_sobolev,KR,rao1991theory}.
For  Orlicz spaces of vector valued functions, see \cite{Orliczvectorial2005} and the references therein.

Hereafter we denote  by $\mathbb{R}^+$  the set of all non negative real numbers. A function $\Phi:\mathbb{R}^+\to \mathbb{R}^+ $ is called an \emph{$N$-function} if $\Phi$ is given by 
\[
\Phi(t)=\int_{0}^t \varphi(\tau)\ d\tau,\quad\hbox{for } t\geq 0,
\]
where $\varphi:\mathbb{R}^+\rightarrow \mathbb{R}^+$ is a right continuous non decreasing function  satisfying   $\varphi(0)=0$, $\varphi(t)>0$ for $t>0$ and
$\lim_{t\rightarrow \infty}\varphi(t)=+\infty$.

Given a function $\varphi$ as above, we  consider the so-called right inverse function $\psi$ of $\varphi$ which is 
defined by $\psi(s)=\sup_{\varphi(t)\leq s}t$.
The function $\psi$ satisfies the same properties as the function $\varphi$, therefore we have an $N$-function $\Psi$ such that $\Psi'=\psi$ .
 The function $\Psi$ is called the \emph{complementary function} of $\Phi$.


We say that $\Phi$ satisfies the  \emph{$\Delta_2$-condition}, denoted by $\Phi \in \Delta_2$, 
if there exist  constants $K>0$ and  $t_0\geq 0$ such that 
\begin{equation}\label{delta2defi}\Phi(2t)\leq K\Phi(t)
\end{equation}
for every $t\geq t_0$. 
If $t_0=0$,  we say that $\Phi$ satisfies the \emph{$\Delta_2$-condition globally} ($\Phi \in \Delta_2$ globally).  

% and plain symbols indicate scalars.

Let $d$ be a positive integer. We denote by $\mathcal{M}_d:=\mathcal{M}_d([0,T])$ the set of all measurable functions defined on $[0,T]$ with values on $\mathbb{R}^d$ and  we write $\b{u}=(u_1,\dots,u_d)$ for  $\b{u}\in \mathcal{M}_d$.
In this paper we adopt the convention that bold symbols denote points in $\mathbb{R}^d$.


Given  an $N$-function $\Phi$ we define the \emph{modular function} 
$\rho_{\Phi}:\mathcal{M}_d\to \mathbb{R}^+\cup\{+\infty\}$ by
\[\rho_{\Phi}(\b{u}):= \int_0^T \Phi(|\b{u}|)\ dt.\]
Here $|\cdot|$ is the euclidean norm of $\mathbb{R}^d$.
The \emph{Orlicz class} $C_d^{\Phi}=C_d^{\Phi}([0,T])$  is given  by
\begin{equation}\label{claseOrlicz}
  C^{\Phi}_d:=\left\{\b{u}\in \mathcal{M}_d | \rho_{\Phi}(\b{u})< \infty \right\}.
\end{equation}
The \emph{Orlicz space} $\lphi_d=L^{\Phi}_d([0,T])$ is the linear hull of $\claseor_d$;
equivalently,
\begin{equation}\label{espacioOrlicz}
\lphi_d:=\left\{ \b{u}\in \mathcal{M}_d | \exists \lambda>0: \rho_{\Phi}(\lambda \b{u}) < \infty   \right\}.
\end{equation}
  The Orlicz space $\lphi_d$ equipped with the \emph{Orlicz norm}
\[
\|  \b{u}  \orlnor:=\sup \left\{  \int_0^T \b{u}\b{\cdot} \b{v}\ dt \big| \rho_{\Psi}(\b{v})\leq 1\right\},
\]
is a Banach space. By $\b{u}\b{\cdot} \b{v}$ we denote the usual dot product in $\mathbb{R}^{d}$ between $\b{u}$ and $\b{v}$.  
The following alternative expression for the norm, known as \emph{Amemiya norm},     will  be useful (see \cite[Thm. 10.5]{KR} and \cite{hudzik2000amemiya}). For every $\b{u}\in\lphi$,

\begin{equation}\label{amemiya}
\|\b{u}\orlnor=\inf\limits_{k>0}\frac{1}{k}\left\{1+\rho_{\Phi}(k\b{u})\right\}.
\end{equation}



The subspace $\ephi_d=\ephi_d([0,T])$ is defined as the closure in $\lphi_d$ of the subspace $L^{\infty}_d$ of all $\mathbb{R}^d$-valued essentially bounded functions. It is shown that  $\ephi_d$ is the only one maximal subspace contained in the Orlicz class $\claseor_d$, i.e. 
$\b{u}\in\ephi_d$ if and only if $\rho_{\Phi}(\lambda \b{u})<\infty$ for any $\lambda>0$.  

A generalized version of \emph{H\"older's inequality} holds in Orlicz spaces (see \cite[Th. 9.3]{KR}). Namely, if $\b{u}\in\lphi_d$ and $\b{v}\in\lpsi_d$ then $\b{u}\ccdot\b{v}\in L_1^1$ and
\begin{equation}\label{holder}
\int_0^T\b{v}\ccdot\b{u}\ dt\leq \|\b{u}\orlnor\|\b{v}\|_{L^{\Psi}}.
\end{equation}




If $X$ and $Y$ are  Banach spaces such that  $Y\subset X^*$, we denote by $\langle\cdot,\cdot\rangle:Y\times X\to\mathbb{R}$ the bilinear pairing  map given by $\langle x^*,x\rangle=x^*(x)$. H\"older's inequality shows that $\lpsi_d\subset \left[\lphi_d\right]^*$, where the pairing  
$\langle \b{v}, \b{u}\rangle$
is defined by 
\begin{equation}\label{pairing}
  \langle \b{v},\b{u}\rangle=\int_0^T\b{v}\ccdot\b{u}\ dt
\end{equation}
with  $\b{u}\in\lphi_d$ and $\b{v}\in\lpsi_d$.
 Unless $\Phi \in \Delta_2$, the relation $\lpsi_d= \left[\lphi_d\right]^*$ will not hold. In general, it is true  that  $\left[\ephi_d\right]^*=\lpsi_d$.


Like in \cite{KR}, we will consider the subset $\Pi(\ephi_d,r)$ of $\lphi_d$ given by
\[\Pi(\ephi_d,r):=\{\b{u}\in\lphi_d| d(\b{u},\ephi_d)<r\}.\]
This set is related to the Orlicz class $\claseor_d$ by means of inclusions, namely,
\begin{equation}\label{inclusiones}\Pi(\ephi_d, r )\subset r \claseor_d\subset\overline{\Pi(\ephi_d,r)}
\end{equation}
for any positive $r$.
If $\Phi \in \Delta_2$,  then the sets $\lphi_d$, $\ephi_d$, $\Pi(\ephi_d,r)$ and $\claseor_d$ are equal.



We define the \emph{Sobolev-Orlicz space} $\wphi_d$ (see \cite{adams_sobolev}) by
\[\wphi_d:=\{\b{u}| \b{u} \hbox{ is absolutely continuous and } \b{\dot{u}}\in \lphi_d\}.\]
$\wphi_d$ is a Banach space when equipped with the norm
\begin{equation}\label{def-norma-orlicz-sob}
\|  \b{u}  \|_{\wphi}= \|  \b{u}  \|_{\lphi} + \|\b{\dot{u}}\orlnor.
\end{equation}



For a  function $\b{u}\in L^1_d([0,T])$, we write $\b{u}=\overline{\b{u}}+\widetilde{\b{u}}$ where $\overline{\b{u}} =\frac1T\int_0^T \b{u}(t)\ dt$ and $\widetilde{\b{u}}=\b{u}-\overline{\b{u}}$.

As usual, if $(X,\|\cdot\|_X)$ is a Banach space and $(Y,\|\cdot \|_Y)$ is a subspace of $X$,  we write $Y\hookrightarrow X$ and we say that $Y$ is \emph{embedded} in $X$  when the restricted identity map $i_Y:Y\to X$ is bounded. That is, there exists $C>0$ such that  for any $y\in Y$ we have $\|y\|_X\leq C\|y\|_Y$.  With this notation, H\"older's inequality states that  $\lpsi_d\hookrightarrow  \left[\lphi_d\right]^*$; and, it is easy to see that for every $N$-function $\Phi$ we have that $L^{\infty}_d\hookrightarrow\lphi_d \hookrightarrow L^1_d$.


 Recall that a function   $w:\mathbb{R}^+\to \mathbb{R}^+$ is called  a \emph{modulus of continuity} if $w$ is a continuous increasing function which satisfies $w(0)=0$. For example, it can be easily shown that $w(s)=s\Phi^{-1}(1/s)$ is a modulus of  continuity for every $N$-function $\Phi$.  We say that $\b{u}:[0,T]\to\rr^d$  has modulus of continuity $w$  when there exists a constant $C>0$ such that 
\begin{equation}\label{w-holder}|\b{u}(t)-\b{u}(s)|\leq Cw(|t-s|).
\end{equation}


We denote by $C^w([0,T],\rr^d)$  the space of  $w$-H\"older continuous functions. This is the space of all functions satisfying \eqref{w-holder} for some $C>0$ and it is a Banach space with norm
\[\|\b{u}\|_{  C^w([0,T],\rr^d) }  :=\|\b{u}\|_{L^{\infty}}+\sup\limits_{t\neq s}\frac{|\b{u}(t)-\b{u}(s)|}{w(|t-s|)}.\]





 An important aspect of the theory of Sobolev spaces is related to embedding theorems. There is an extensive literature on this question in the  Orlicz-Sobolev space setting, see for example
 \cite{cianchi2000fully,cianchi1999some,claverooptimal,edmunds2000optimal,kerman2006optimal}.
The next simple lemma, whose proof can be found in \cite{ABGMS2015}, will be used systematically.




\begin{lem}\label{inclusion orlicz} Let  $w(s):= s\Phi^{-1}(1/s)$. Then, the following statements hold:
\begin{enumerate}
\item\label{inclusion orlicz_item1} $\wphi\hookrightarrow C^w([0,T],\rr^d) $ and for every $\b{u}\in\wphi$
\begin{align}
 &\left|\b{u}(t)-\b{u}(s) \right| \leq  \|\b{\dot{u}}\orlnor w(| t-s|),&\label{in-sob-cont}
\\
& \|\b{u}\|_{L^{\infty}} \leq\Phi^{-1}\left(\frac{1}{T}\right)\max\{1,T\}\|\b{u}\sobnor&\label{sobolev}
\end{align}
\item For every $\b{u}\in\wphi$ we have $\widetilde{\b{u}}\in L^{\infty}_d$ and 
\begin{align}
& \|\widetilde{\b{u}}\|_{L^{\infty}} \leq T\Phi^{-1}\left(\frac{1}{T}\right)\|\b{\dot u}\orlnor&\text{  (Sobolev's inequality).}\label{wirtinger}
\end{align}




\end{enumerate}
\end{lem}


The following result is analogous to some lemmata in $W^1L^p_d$, see \cite{xu2007some}.
\begin{lem}\label{infinito-a-prom-upunto}
If $\|\b{u}\sobnor\to \infty$, then $(|\b{\overline u}|+\|\b{\dot u}\orlnor)\to \infty$.
\end{lem}

\begin{proof}
By the decomposition $\b{u}=\b{\overline u}+\b{\tilde{u}}$ and some elementary operations, 
we get
\begin{equation}\label{cota-u-lphi}
\|\b{u}\orlnor=
\|\b{\overline u}+\b{\tilde{u}}\orlnor\leq 
\|\b{\overline u}\orlnor+\|\b{\tilde{u}}\orlnor=
|\b{\overline u}|\|1\orlnor+\|\b{\tilde{u}}\orlnor.
\end{equation}
It is known that $L^{\infty}_d\hookrightarrow\lphi_d$, i.e.,
there exists $C_1=C_1(T)>0$ such that for any $\b{\tilde{u}}\in L^{\infty}_d$ we have
\[
\|\b{\tilde{u}}\orlnor
\leq 
C_1 \|\b{\tilde{u}}\|_{L^{\infty}};
\]
and, applying  Sobolev's inequality,  we obtain the Wirtinger's inequality, that is there exists $C_2=C_2(T)>0$ suvh that 
\begin{equation}\label{cota-u-tilde}
\|\b{\tilde{u}}\orlnor
\leq 
C_2\|\b{\dot{u}}\orlnor.
\end{equation}

Therefore, from \eqref{cota-u-lphi}, \eqref{cota-u-tilde} and \eqref{def-norma-orlicz-sob}, 
we get
\[
\|\b{u}\sobnor\leq
C_3(|\b{\overline u}|+\|\b{\dot{u}}\orlnor)
\]
where $C_3=C_3(T)$. Finally, as $\|\b{u}\sobnor\to \infty$ we conclude that   
$(|\b{\overline u}|+\|\b{\dot{u}}\orlnor)\to \infty$.
\end{proof}


We present a definition that will be useful later.
 
\begin{defi} A function $\mathcal{L}:[0,T]\times \mathbb{R}^d \times \mathbb{R}^d \rightarrow \mathbb{R}$ is a \emph{Carath\'eodory} function if for fixed $(\b{x},\b{y})$
the map $t \mapsto \mathcal{L}(t, \b{x},\b{y})$ is measurable  and for fixed $t$ the map  $(\b{x},\b{y}) \mapsto \mathcal{L}(t, \b{x}, \b{y})$ is continuous  for almost everywhere $t\in [0,T]$. We say that 
$\mathcal{L}(t, \b{x},\b{y})$ is  \emph{differentiable Carath\'eodory} if in addition $\mathcal{L}(t, \b{x},\b{y})$ is
continuously differentiable with respect to $\b{x}$ and $\b{y}$  for almost everywhere $t\in [0,T]$.

\end{defi}


In \cite{ABGMS2015} we proved the next results.

\begin{thm}\label{teorema_acotacion}
Let $\mathcal{L}$ be a differentiable Carath\'eodory function satisfying \eqref{cotaL}, \eqref{cotaDxL} and \eqref{cotaDyL}. 
Then the following statements hold:
\begin{enumerate}
\item \label{T1item1} \label{A1} The action integral given by \eqref{integral_accion}
is finitely defined on $\domi:=W^{1}\lphi_d\cap\{\b{u}|\b{\dot{u}}\in\Pi(\ephi_d,\lambda)\}$.

\item\label{T1item3} The function  $I$ is G\^ateaux differentiable on $\domi$ and  its derivative $I'$ is demicontinuous from $\domi$  into $\left[\wphi_d \right]^*$. Moreover, $I'$ is given by the following expression
\begin{equation}\label{DerAccion}
\langle  I'(\b{u}),\b{v}\rangle= \int_0^T \left\{D_{\b{x}}\mathcal{L}\big(t,\b{u},\b{\dot{u}}\big)\ccdot \b{v}+ D_{\b{y}}\mathcal{L}\big(t,\b{u},\b{\dot{u}}\big)\ccdot\b{\dot{v}}\right\} \ dt.
\end{equation}

\item\label{T1item4}  If  $\Psi \in \Delta_2$ then 
  $I'$ is continuous from $\domi$ into $\left[\wphi_d\right]^*$ when both spaces are equipped with the strong topology.
\end{enumerate}
\end{thm}





In \cite{ABGMS2015} we derive the Euler-Lagrange equations associated to critical points of action integrals on the subspace of $T$-periodic functions.  
We denote by $\wphi_T$ the subspace of $\wphi_d$ containing all  $T$-periodic functions. As usual, when $Y$ is a subspace of
the Banach space $X$, we denote by $Y^{\perp}$ the \emph{annihilator subspace} of $X^*$, i.e. the subspace
that consists of all  bounded linear functions which are identically zero on $Y$.

We recall that  a function $f: \mathbb{R}^d \to \mathbb{R}$ is called \emph{strictly convex} if 
$f\left(\tfrac{\b{x}+\b{y}}{2}\right)< \tfrac{1}{2} \left(f\left(
\b{x}\right)+f\left( \b{y}\right)\right)$ for  $\b{x}\neq\b{y}$.  
It is  well known that if $f$ is a strictly convex and differentiable function, then
$D_{\b{x}}f:\mathbb{R}^d\to\mathbb{R}^d$ is a one-to-one map  (see, e.g. \cite[Thm. 12.17]{rockafellar2009variational}).


\begin{thm}\label{critpoint} Let $\b{u}\in\domi$ be  a $T$-periodic function. The following statements are equivalent:
\begin{enumerate}
 \item $I'(\b{u})\in\left( \wphi_T\right)^{\perp}$.
 \item  $D_{\b{y}}\mathcal{L}(t,\b{u}(t),\b{\dot{u}}(t))$ is an absolutely continuous function and $\b{u}$ solves the following boundary value problem
 \begin{equation}\label{ecualagran2}
    \left\{%
\begin{array}{ll}
   \frac{d}{dt} D_{y}\mathcal{L}(t,\b{u}(t),\b{\dot{u}}(t))= D_{\b{x}}\mathcal{L}(t,\b{u}(t),\b{\dot{u}}(t)) \quad \hbox{a.e.}\ t \in (0,T)\\
    \b{u}(0)-\b{u}(T)=D_{\b{y}}\mathcal{L}(0,\b{u}(0),\b{\dot{u}}(0))-D_{\b{y}}\mathcal{L}(T,\b{u}(T),\b{\dot{u}}(T))=0.
\end{array}%
\right.
\end{equation}
\end{enumerate}
Moreover if $D_{\b{y}}\mathcal{L}(t,x,y)$ is $T$-periodic with respect to the variable $t$ and strictly convex with respect to $\b{y}$, then
$D_{\b{y}}\mathcal{L}(0,\b{u}(0),\b{\b{\dot{\b{u}}}}(0))-D_{\b{y}}\mathcal{L}(T,\b{u}(T),\b{\dot{u}}(T))=0$ is equivalent to $\b{\dot{u}}(0)=\b{\dot{u}}(T)$.
\end{thm}


{\bf Habr\'ia que ver si el lugar de los \'indices es el adecuado. Copi\'e lo que ten\'iamos en el 
primer trabajo.}


Next, we enumerate some definitions and results from the theory of convex functions. 
We suggest \cite{FK97, GP77, KR, M, rao1991theory} for definitions, proofs and additional details.

We denote by $\alpha_{\varphi}$ and $\beta_{\varphi}$ the so called  \emph{Matuszewska-Orlicz indices} of the function $\varphi$, which are defined next. Given
an increasing, unbounded, continuous function  \linebreak $\varphi:[ 0,+\infty)\to [0,+\infty)$ such that $\varphi(0)=0$ we define
\begin{equation}\label{MO_indices}
    \alpha_{\varphi}:=\lim\limits_{t\to 0^{+}}\frac{\log \left (\sup\limits_{u>0}\frac{\varphi(t u)}{\varphi(u)} \right ) }{\log(t)},\quad
    \beta_{\varphi}:=\lim\limits_{t\to +\infty}\frac{\log \left  (\sup\limits_{u>0}\frac{\varphi(t u)}{\varphi(u)}\right )}{\log(t)}.
\end{equation}
We have that $0\leq \alpha_{\varphi}\leq \beta_{\varphi}\leq +\infty$. The relation $\beta_{\varphi}<\infty$ holds true if and only if $\varphi$ is a
$\Delta_2$-function. If $\varphi$ is a homeomorphism  we have that
\begin{equation}\label{inv_indices}
    \alpha_{\varphi^{-1}}=\frac{1}{\beta_{\varphi}}.
\end{equation}
Moreover $\varphi\in\mathcal{F}$ implies  $\alpha_{\varphi}\geq 1$. As a consequence, $\varphi^{-1}$ is a $\Delta_2$-function.

 It is well known   that if $\varphi$ is an increasing $\Delta_2$-function, $\varphi$ is controlled by above and below 
 by power functions.  More concretely, for every $\epsilon>0$ there exists a
constant $K=K(\varphi,\epsilon)$ such that, for every $t,u\geq 0$,
\begin{equation}\label{delta2-potencias}
    K^{-1}\min\big\{t^{\beta_{\varphi}+\epsilon},t^{\alpha_{\varphi}-\epsilon} \big\}\varphi(u)\leq \varphi(t u)\leq
    K\max\big\{t^{\beta_{\varphi}+\epsilon},t^{\alpha_{\varphi}-\epsilon} \big\}\varphi(u).
\end{equation}


We define the following  functionals $J_{C,\mu}:\lphi\to (-\infty,+\infty]$ and $  H_{C,\sigma}:\rr^n\to \rr$, with $C,\nu,\sigma>0$, by
\begin{equation}\label{func_phi}
  J_{C,\nu}(\b{u}):= \rho_{\Phi}\left(\b{u}\right)-C\|\b{u}\orlnor^{\nu},
\end{equation}
 and

\begin{equation}\label{eq:functional_H}
 H_{C,\sigma}(\b{x})=\int_0^TF(t,\b{x})dt-C|\b{x}|^{\sigma},
\end{equation}
respectively.



We will use the classic notations big $O$ and little $o$ \cite{erdelyi}. The  following lemma, was essentially  proved in \cite{ABGMS2015}.



\begin{lem}\label{lem_coer} Let $\Phi$ and $\Psi$ be complementary $N$-functions and let $\b{u}\in\lphi$. Then:
\begin{enumerate}
  \item\label{th:coer_item1} $\|\b{u}\orlnor=O\left( \rho_{\Phi}\left(\b{u} \right)  \right)$.
  
  \item\label{th:coer_item2} If $\Psi \in \Delta_2$ globally, then there exists a constant $\alpha_{\Phi}>1$ such that, for any $0<\mu<\alpha_{\Phi}$,
\begin{equation}\label{coer_modular} \|\b{u}\orlnor^{\mu} =o\left(\rho_{\Phi}\left(\b{u}\right)\right)
\end{equation}
as $\|\b{u}\orlnor\to \infty$.

\item\label{th:coer_item3} If \eqref{coer_modular} holds for $\mu\geq 1$ then $\Psi \in \Delta_2$.  
\end{enumerate}
\end{lem}

\begin{proof} \ref{th:coer_item1}. In virtue of \cite[Lemma 5.2(1)]{ABGMS2015} we have that $\lim_{\|\b{u}\orlnor \to+\infty }J_{C,1}(\b{u})=+\infty$ for $C<1$. Then for sufficiently large $ \|\b{u}\orlnor$ we have that $J_{C,1}(\b{u})>0$. This implies $\rho_{\Phi}(\b{u})\leq C^{-1}\|\b{u}\orlnor$. 

\ref{th:coer_item2}. Taking account of \cite[Lemma 5.2(2)]{ABGMS2015} we have that $\lim_{\|\b{u}\orlnor \to+\infty }J_{C,1}(\b{u})=+\infty$ for every $C>0$. Therefore for any $n\in \nn$ there exists $k_n>0$ such that for $  \|\b{u}\orlnor>k_n$ we  have  $J_{n,\mu}(\b{u})>1$. Then $\rho_{\Phi}(\b{u})/ \|\b{u}\orlnor^{\mu}>n$. 

\ref{th:coer_item3}. The statement \ref{th:coer_item3} represents a parcial reciprocal of  \ref{th:coer_item2}. The difference rests on the fact that the $\Delta_2$ condition for $\Psi$  in item  \ref{th:coer_item2} is global, while in item   \ref{th:coer_item3} is for large values. We can suppose $\mu=1$. By \eqref{coer_modular} we obtain for any $C>1$ that
\[\lim_{   \|\b{u}\orlnor\to+\infty} J_{C,1}(\b{u})=\lim_{   \|\b{u}\orlnor\to+\infty}
 \|\b{u}\orlnor\left(\frac{\rho_{\Phi}(\b{u})}{ \|\b{u}\orlnor}-C\right)=+\infty.\]
 Therefore, using \cite[Lemma 5.2(3)]{ABGMS2015}, we conclude $\Psi\in\Delta_2$. 

\end{proof}










\section{Lagrangians with sublinear ``nonlinearity''????}


Like in \cite{ABGMS2015} we consider Lagrangians $\mathcal{L}$ which are lower bounded as follows 
\begin{equation}\label{cota_inf}
\mathcal{L}(t,\b{x},\b{y})\geq \alpha_0\Phi\left(\frac{|\b{y}|}{\Lambda}\right)+ F(t,\b{x}).
\end{equation}

Based on \cite{mawhin2010critical} we say that $F$ satisfies the condition (A) if  $F(t,\b{x})$ is a Carath\'eo\-dory function and  $F$ is continuously differentiable with respect to $\b{x}$. Moreover, the next inequality holds 
\begin{equation}\label{condA2}|F(t,\b{x})|+ |D_{\b{x}}F(t,\b{x})|\leq a(|\b{x}|)b_0(t),\quad\text{for a.e. }t\in [0,T], \forall\b{x}\in\rr^d.
\end{equation}

Now, we have another result about coercivity of $I$ assuming some conditions on the nonlinearity  $\nabla F$. 

\begin{thm}\label{coercitividad-r}
Let  $\mathcal{L}$ be a lagrangian function satisfying \eqref{cotaL}, \eqref{cotaDxL}, \eqref{cotaDyL}, \eqref{cota_inf}  and $F$ satisfies condition (A). We assume the following conditions:
\begin{enumerate}
\item $\Psi\in\Delta_2$.
\item There exist  non negative functions  $b_1,b_2 \in L^1_1$ and a constant $1<\mu<\alpha_{\Phi}$  such that 
for any $\b{x}\in\rr^d$ and a.e. $t\in [0,T]$
\begin{equation}\label{holder_cont-mu}
  \left| \nabla F(t,\b{x}) \right|\leq b_1(t)|\b{x}|^{\mu-1}+b_2(t).
\end{equation}
\item There exists a real positive number $\sigma$ such that $\sigma>(\mu-1)\beta_{\Psi}$ and
\begin{equation}\label{propiedad-coercividad-sigma}
|\b{x}|^{\sigma}=o\left(\int_{0}^{T}F(t,\b{x})\ dt\right)\;\;\mbox{as}\;\;|\b{x}|\to \infty.
\end{equation}
\end{enumerate}
Then  the action integral $I$ is coercive.
\end{thm}




\begin{proof} 
By the decomposition $u=\overline{u}+\b{\tilde{u}}$,  Mean Value Theorem, Cauchy-Schwarz's inequality 
and \eqref{holder_cont-mu}, we have
\begin{equation}\label{cota-diferencia-F}
\begin{split}
&\left|\int_0^T F(t,\b{u})-F(t,\b{\overline{u}})\,dt\right|=
\left|\int_0^T \int_0^1 \nabla F(t,\b{\overline{u}}+s\b{\tilde{u}}(t))\ccdot \b{\tilde{u}}(t) \,ds \,dt\right|
\\
&\leq \int_0^T \int_0^1 b_1(t)|\b{\overline{u}}+s\b{\tilde{u}}(t)|^{\mu-1}|\b{\tilde{u}}(t)|\,ds\,dt+
\int_0^T \int_0^1 b_2(t)|\b{\tilde{u}}(t)|\,ds\,dt
\\
&=I_1+I_2.
\end{split}
\end{equation}
On the one hand, by H\"older's inequality and Sobolev's inequality, we estimate $I_2$ as follows
\begin{equation}\label{cota-i2}
I_2\leq \|b_2\|_{L^1} \|\b{\tilde{u}}\|_{L^{\infty}}\leq
C_1\|\b{\dot u}\orlnor.
\end{equation}
 where $C_1=C_1(\|b_2\|_{L^1}, T)$. On the other hand, as $s\in [0,1]$, we have
\begin{equation}\label{pot-suma}
|\b{\overline{u}}+s\b{\tilde{u}}(t)|^{\mu-1}\leq
C(\mu)(|\b{\overline{u}}|^{\mu-1}+\|\b{\tilde{u}}\|_{L^{\infty}}^{\mu-1}).
\end{equation}
where $C(\mu)=2^{\mu-2}$, for $\mu\geq 2$ and $C(\mu)=1$, for $1<\mu<2$. Now,  inequality \eqref{pot-suma}, H\"older's inequality and Sobolev's inequality imply that
\begin{equation}\label{cota-i1}
\begin{split}
I_1&\leq 
C(\mu)\left(|\b{\overline{u}}|^{\mu-1} \int_0^T b_1(t) |\b{\tilde{u}}(t)|\,dt+
\|\b{\tilde{u}}\|^{\mu-1}_{L^{\infty}} \int_0^T b_1(t)|\b{\tilde{u}}(t)| \,dt\right)
\\
&\leq C(\mu)\bigg\{ |\b{\overline{u}}|^{\mu-1} \|b_1\|_{L^1} \|\b{\tilde{u}}\|_{L^{\infty}}+
 \|b_1\|_{L^1}\|\b{\tilde{u}}\|^{\mu}_{L^\infty}\bigg\}
\\
&\leq C_2 \bigg\{ |\b{\overline{u}}|^{\mu-1} \|\b{\dot{u}}\orlnor+ \|\b{\dot u}\orlnor^{\mu}\bigg\},
\end{split}
\end{equation}
where $C_2=C_2(\mu,T, \|b_1\|_{L^1} )$. Let $\mu'$ be a positive constant such that $1<\mu\leq \mu'<\alpha_{\Phi}$. 
Next, using Young's inequality with conjugate exponents $\mu'$ and $\frac{\mu'}{\mu'-1}$ 
 we get
\begin{equation}\label{cota-i1-parcial}
|\b{\overline{u}}|^{\mu-1}   \|\b{\dot{u}}\orlnor
\leq \frac{(\mu'-1)}{\mu'}|\b{\overline{u}|^{\sigma}}
+\frac{1}{\mu'} \|\b{\dot{u}}\orlnor^{\mu'}
\end{equation}
where $\sigma=\frac{(\mu-1) \mu'}{\mu'-1}$. We point out that $\sigma$ is an arbitrary positive constant bigger than $(\mu-1)b_{\Psi}$.

From \eqref{cota-i1}, \eqref{cota-i1-parcial}, \eqref{cota-i2} and the inequality $x^{r_1}\leq x^{r_2}+1$, for any $x\geq 0$ and $r_1\leq r_2$, we have
\begin{equation}\label{cota-i1-i2}
\begin{split}
I_1+I_2
&\leq C_3\bigg\{ |\b{\overline{u}}|^{\sigma}
+ \|\b{\dot u}\orlnor^{\mu'}
+ \|\b{\dot u}\orlnor^{\mu}
+\|\b{\dot u}\orlnor\bigg\}\\
&\leq C_3\bigg\{ |\b{\overline{u}}|^{\sigma}
+ \|\b{\dot u}\orlnor^{\mu'}
+1\bigg\}
\end{split}
\end{equation}
with $C_3= C_3(\mu,T, \|b_1\|_{L^1},\mu' )$. In the subsequent estimates, we use the decomposition $u=\overline{u}+\b{\tilde{u}}$, \eqref{cota_inf}, \eqref{cota-diferencia-F},
\eqref{cota-i1-i2} and we get
\begin{equation}\label{cota_inf_I}
\begin{split}
I(\b{u})&\geq\alpha_0\rho_{\Phi}\left( \frac{\b{\dot{u}}}{\Lambda}\right)+\int_0^TF(t,\b{u})\ dt
\\ 
&=\alpha_0\rho_{\Phi}\left( \frac{\b{\dot{u}}}{\Lambda}\right)+ \int_0^T \left[F(t,\b{u})-F(t,\b{\overline{u}})\right]\ dt 
+  \int_0^TF(t,\b{\overline{u}})\ dt
\\
&\geq \alpha_0\rho_{\Phi}\left( \frac{\b{\dot{u}}}{\Lambda}\right)
-C_3 \|\b{\dot u}\orlnor^{\mu'}
+\int_0^TF(t,\b{\overline{u}})\ dt-
C_3 |\b{\overline{u}}|^{\sigma}-C_3\\
&=\alpha_0J_{C_4,\mu'}\left(\frac{\b{\dot u}}{\Lambda}\right)
+ H_{C_3,\sigma}(\b{\overline{u}})-C_3,
\end{split}
\end{equation}
where $C_4=\Lambda^{\mu'}C_3/\alpha_0$.


Let $\b{u}_n$ be  a sequence in $\domi$ with 
$\|\b{u}_n\sobnor\to\infty$ and we have to prove that $I(\b{u}_n)\to\infty$. 
On the contrary, suppose  that for a subsequence, 
still denoted by $\b{u}_n$, $I(\b{u}_n)$ is upper bounded, i.e., there exists $M>0$ such that $|I(\b{u}_{n})|\leq M$. 
As $\|\b{u}_n\sobnor\to\infty$, from Lemma \ref{infinito-a-prom-upunto},  we have $|\overline{\b{u}}_n|+\|\b{\dot{u}}_n\orlnor\to \infty$.
Then, there exists a subsequence of $\{\b{u}_n\}$, still denoted by $\b{u}_n$, which is not bounded.
Then, 
$|\b{\overline u}_n|\to \infty$ or $\|\b{\dot{u}}_n\orlnor\to \infty$.
Now, Lemma \ref{lem_coer} implies that the functional $J_{C_4,\mu'}(\frac{\b{\dot u}}{\Lambda})$ is coercive,
and, by \eqref{propiedad-coercividad-sigma},
the functional $H(\b{\overline{u}})$ is also coercive, then 
$J_{C_4,\mu'}(\frac{\b{\dot u}_n}{\Lambda}) \to \infty$ or $H(\b{\overline{u}}_n)\to \infty$.
From \eqref{condA2}, we have that on a bounded set the functional $H(\b{\overline{u}}_n)$ is lower bounded; and, $J_{C_4,\mu'}(\frac{\b{\dot u}_n}{\Lambda})$ is also lower bounded  because the modular $\rho_{\Phi}\left(\frac{\b{\dot u}}{\Lambda}\right)$ is always bigger than zero. 
Therefore,  $I(\b{u}_n)\to\infty$ as $\|\b{u}_n\sobnor\to\infty$ which contradicts the initial assumption on the behavior of $I(\b{u}_n)$. 
\end{proof}

{\bf Leer y ver si es coherente lo anterior,  si conviene trabajar siempre con $u_n$ o habr\'ia que usar la notaci\'on de subsucesiones expl\'icita!!!}

{\bf Falta leer y corregir la  secci\'on que sigue del caso l\'imite!!!}

\section{Limit case $\mu=\alpha_{\Phi}$}
Assuming $\|b_1\|_{L^1}$  small enough, in  \cite{zhao2004periodic, tang2010periodic} even  was obtained coercitivity for the limit value $\mu=p$ in inequality \eqref{holder_cont-mu}.  This result lean on the  fact that when  $\Phi(u)=|u|^p$,

\begin{equation}
 \|u\orlnor^{\alpha_{\Phi}}=O\left(\int_0^T \Phi(|u|)\,dt\right),\quad\text{for } \|u\orlnor\to\infty.
\end{equation}
However, it is no longer the case  for any $N$-function $\Phi$ as the example below shows

From now on in this section, we will suppose that 
\[\Phi(u)=
\left\{
\begin{array}{ll}
\frac{p-1}{p}u^p&u\leq e
\\
\frac{u^p}{\log u}-\frac{e^p}{p}&u>e
\end{array}
\right.\]
with $p>1$. Next, we will establish some properties of this $\Phi$. 

\begin{thm}
If $p\geq \frac{1+\sqrt 2}{2}$, then $\Phi$ is an $N$-function.
\end{thm}


\begin{proof}
We have 
\[\varphi(u)=\Phi'(u)=\left\{
\begin{array}{cccc}
(p-1)u^{p-1}&:=&\varphi_1(u)& \mbox{if}\;u\leq e
\\
\frac{u^{p-1}}{\log u}(p-\frac{1}{\log u})&:=&\varphi_2(u)&\mbox{if}\; u\geq e
\end{array}
\right.
\]

First let us see that $\Phi'$ is increasing when $p\geq \frac{1+\sqrt {2}}{2}$.
For this purpose, since $\varphi_1(e)=\varphi_2(e)$, it is enough see that $\varphi_1$ is increasing  on $[0,e]$ and $\varphi_2$ is increasing on
$[e,\infty)$ for every $p\geq \frac{1+\sqrt {2}}{2}$. Clearly
$\varphi_1$ is an increasing function for $p>1$.  On the other hand, an elementary function analysis shows that
$\varphi_2'(u)>0$ on $[e,\infty)$ if and only if
 $p \notin(\frac{1-\sqrt2}{2},\frac{1+\sqrt2}{2})$.  Therefore $\varphi_2$ is an icreasing function when $p\geq \frac{1+\sqrt2}{2}$.

 Besides $\varphi_2(u)\to \infty$ and  $\varphi_1(u)\to 0$  as $u \to  \infty$ and $u\to 0$  respectively, provided that $p>1$. Hence $\Phi$ is $N$-function.

\end{proof}


\begin{thm} For every $\varepsilon>0$, there exists a positive constant $C=C(p,\varepsilon)$  such that
\begin{equation}\label{cota-sup-indices}
C^{-1}t^{p-\varepsilon}\Phi(u)\leq \Phi(tu) \leq Ct^p\Phi(u)\quad t\geq 1, u>0.
\end{equation}
\end{thm}

\begin{proof} If $u\leq tu\leq e$, then $\Phi(tu)=t^p\Phi(u)$ and \eqref{cota-sup-indices} holds with $C=1$.

If $u\leq e\leq tu$, as $\frac{e^p}{p}>0$  and $\log(tu)\geq 1$, we have 
$\Phi(tu)\leq t^pu^p= \frac{p}{p-1}t^p\Phi(u)$. Thus, the last inequality in  \eqref{cota-sup-indices} holds with $C=\frac{p}{p-1}$. On the other hand, as $f(t)=\frac{t}{\log t}$ is increasing on $[e,\infty)$, then $f((tu)^p)\geq  f(e^p)=e^p/p$.
Now, 
\[
\begin{split}
\Phi(tu)&=\frac{p(tu)^p}{\log (tu)^p}-\frac{e^p}{p}\\
&= \frac{(p-1)(tu)^p}{\log(tu)^p}+\frac{(tu)^p}{\log (tu)^p}-\frac{e^p}{p}
\\
&\geq \frac{p-1}{p}\frac{(tu)^p}{\log(tu)}\\
&\geq
\frac{p-1}{p}\frac{t^{\varepsilon}}{\log t+1}t^{p-\varepsilon}u^p.
\end{split}
\]
Since $\varepsilon e^{1-\varepsilon}$ is the minimum value of $t\mapsto\frac{t^{\varepsilon}}{\log t+1}$  on the interval $[1,+\infty)$ then  
\[
\Phi(tu)\geq \frac{p-1}{p}\varepsilon e^{1-\varepsilon}t^{p-\varepsilon}u^p,
\]
which is the first inequality of \eqref{cota-sup-indices} with $C=\frac{p}{p-1}\varepsilon^{-1} e^{-1+\varepsilon}$.  


If $e\leq u\leq tu$, then 
\begin{equation}\label{Phi-de-u-a-v}
\Phi(tu)\leq \frac{t^pu^p}{\log(tu)}\leq \frac{t^pu^p}{\log(u)}=\frac{pt^pv}{\log v}
\end{equation} where $v:=u^p$ and $v\geq e^p$.  If $\alpha>0$, the function $x\mapsto \frac{x}{x-\alpha}$ is decreasing on $(\alpha,\infty)$ 
and the function $v\mapsto \frac{pv}{\log v}$ is increasing  on $[e^p,\infty)$. 
Therefore,  we have 
\[
\frac{\frac{pv}{\log v}}{\frac{pv}{\log v}-\frac{e^p}{p}}\leq 
\frac{e^p}{e^p-\frac{e^p}{p}}=\frac{p}{p-1}
\]
for every $v \geq e^p$. In this way, from \eqref{Phi-de-u-a-v}, we have
\[
\Phi(tu)\leq \frac{pt^p}{p-1}\left(\frac{pv}{\log v}-\frac{e^p}{p}\right)=
 \frac{pt^p}{p-1}\left(\frac{u^p}{\log u}-\frac{e^p}{p}\right)
\]
and the second inequality of  \eqref{cota-sup-indices} holds with $C=\frac{p}{p-1}$. For the first inequality we have, as it was proved previously,

\[
  \Phi(tu)
  \geq
  \frac{p-1}{p}\frac{(tu)^p}{\log(tu)}
  =
  \frac{p-1}{p}
  \frac{t^{\varepsilon} \log u^{\varepsilon}}{\log(t^{\varepsilon}u^{\varepsilon})}
  \frac{t^{p-\varepsilon}u^p}{\log u}
\]
Let $f(s)=\frac{sA}{\log s+A}$ with $s\geq 1$ and $A\geq \varepsilon$.  If $A\leq 1$, then, the function $f$ attains a minimum on $[1,\infty)$ at $s=e^{1-A}$ and the minimum value is $f(e^{1-A})=Ae^{1-A}\geq \varepsilon$. If $A> 1$, $f$ is increasing  on $[1,\infty)$ and its minimum value is $f(1)=1$. Then, $f(s)\geq \varepsilon$ in any case,   therefore
\[
\Phi(tu)\geq \frac{p-1}{p}\varepsilon \frac{t^{p-\varepsilon}u^p}{\log u}\geq
\frac{p-1}{p}\varepsilon t^{p-\varepsilon}\Phi(u).
\]
Therefore \eqref{cota-sup-indices} holds with $C=\frac{p}{\varepsilon (p-1)}$, because this $C$ is the biggest constant that we have obtained in each case under consideration.
\end{proof}



\begin{rem}
The inequality 
\[
\Phi(tu)\geq Ct^p\Phi(u)
\] 
is false for every $C$ because for every $u\geq e$ we have 
\[
\lim\limits_{t \to \infty}\frac{\Phi(tu)}{t^p\Phi(u)}=0
\]
\end{rem}





\begin{thm}
$\alpha_{\Phi}=\beta_{\Phi}=p$
\end{thm}

\begin{proof}
From \eqref{MO_indices} and \eqref{cota-sup-indices}, we get 
\[
\beta_{\Phi}=\lim\limits_{t \to \infty} \frac{\log\left[\sup\limits_{u>0} \frac{\Phi(tu)}{\Phi(u)}\right]}{\log t}
\leq
\lim \limits_{t \to \infty} \frac{\log C+p\log t}{\log t}=p.
\]
On the other hand, employing \eqref{MO_indices} and performing some elementary calculations, we obtain
\[
\alpha_{\Phi}=
\lim\limits_{t \to 0^+} \frac{\log\left[\sup\limits_{u>0} \frac{\Phi(tu)}{\Phi(u)}\right]}{\log t}=
\lim\limits_{s \to \infty} \frac{\log\left[\sup\limits_{v>0} \frac{\Phi(v)}{\Phi(sv)}\right]^{-1}}{\log s}=
\lim\limits_{s \to \infty} \frac{\log\left[\inf\limits_{v>0} \frac{\Phi(sv)}{\Phi(v)}\right]}{\log s}
\]
where $v:=tu$ and $s:=\frac{1}{t}$.
Then, using \eqref{cota-inf-indices},  for every $\varepsilon>0$ we have
\[
\alpha_{\Phi}=
\lim\limits_{s \to \infty} \frac{\log\left[\inf\limits_{v>0} \frac{\Phi(sv)}{\Phi(v)}\right]}{\log s}\geq
\lim\limits_{s \to \infty} \frac{\log C+(p-\varepsilon)\log s}{\log s}\geq p-\varepsilon,
\]
therefore $\alpha_{\Phi}\geq p$. 

Finally, as $\alpha_{\Phi}\leq \beta_{\Phi}\leq p$, we get  
$\alpha_{\Phi}=\beta_{\Phi}=p$.
\end{proof}



Now, we are able to see that 
\[
\rho_{\Phi}(u)=\int_0^T \Phi(|u|)\,dx\geq C\|u\orlnor^{\alpha_{\Phi}}=C\|u\orlnor^p
\]
is false.

In fact, if we take $u\equiv t>0$, then $\|u\orlnor^p=C_1t^p$ where $C_1=\|1\orlnor$ and
$\int_0^T \Phi(|u|)\,dx=C_2\Phi(t)$ with $C_2=T$. 
Then, if $\rho_{\Phi}(u)\geq C\|u\orlnor^p$ were true, then $\Phi(t)\geq C t^p$ would also be true; however, this
last inequality is false.
 


\section{Bounding by power-behaviour functions}

\begin{thm}\label{coercitividad-r}
Let  $\mathcal{L}$ be a lagrangian function satisfying \eqref{cotaL}, \eqref{cotaDxL}, \eqref{cotaDyL}, \eqref{cota_inf}  and $F$ satisfies condition (A). We assume the following conditions:
\begin{enumerate}
\item $\Psi\in\Delta_2$.
\item There exist  non negative functions  $b_1,b_2 \in L^1_1$ and 
$f(\b{x})=\varepsilon(\b{x})|\b{x}|^{\alpha_{\Phi}-1}$ with $\varepsilon(\b{x})\to 0$ for $|x|\to\infty$,  which is  non-decreasing, sub additive and  such that
for any $\b{x}\in\rr^d$ and a.e. $t\in [0,T]$
\begin{equation}\label{holder_cont-mu}
  \left| \nabla F(t,\b{x}) \right|\leq b_1(t)f(|\b{x}|)+b_2(t).
\end{equation}
\item There exists a real positive number $\sigma$ such that $\sigma>(\alpha_{\Phi}-1)\beta_{\Psi}$ and
\begin{equation}\label{propiedad-coercividad-sigma}
|\b{x}|^{\sigma}=o\left(\int_{0}^{T}F(t,\b{x})\ dt\right)\;\;\mbox{as}\;\;|\b{x}|\to \infty.
\end{equation}
{\bf En las cuentas, aparece la funci\'on $\varepsilon$!!! 
?`sirve a\'un la f\'ormula anterior???}
\end{enumerate}
Then  the action integral $I$ is coercive.
\end{thm}

\begin{proof}
By the decomposition $u=\overline{u}+\b{\tilde{u}}$,  Mean Value Theorem, Cauchy-Schwarz's inequality 
and \eqref{holder_cont-mu}, we have
\begin{equation}\label{cota-diferencia-F}
\begin{split}
&\left|\int_0^T F(t,\b{u})-F(t,\b{\overline{u}})\,dt\right|=
\left|\int_0^T \int_0^1 \nabla F(t,\b{\overline{u}}+s\b{\tilde{u}}(t))\ccdot \b{\tilde{u}}(t) \,ds \,dt\right|
\\
&\leq \int_0^T \int_0^1 b_1(t)f(|\b{\overline{u}}+s\b{\tilde{u}}(t)|)|\b{\tilde{u}}(t)|\,ds\,dt+
\int_0^T \int_0^1 b_2(t)|\b{\tilde{u}}(t)|\,ds\,dt
\\
&=I_1+I_2.
\end{split}
\end{equation}
On the one hand, by H\"older's inequality and Sobolev's inequality, we estimate $I_2$ as follows
\begin{equation}\label{cota-i2}
I_2\leq \|b_2\|_{L^1} \|\b{\tilde{u}}\|_{L^{\infty}}\leq
C_1\|\b{\dot u}\orlnor.
\end{equation}
 where $C_1=C_1(\|b_2\|_{L^1}, T)$. On the other hand, as $s\in [0,1]$, we have
\begin{equation}\label{pot-suma}
f(|\b{\overline{u}}+s\b{\tilde{u}}(t)|)\leq
C(f)(f(|\b{\overline{u}}|)+f(\|\b{\tilde{u}}\|_{L^{\infty}})).
\end{equation}
where $C(f)$ springs from the subadditivity of $f$. Now,  inequality \eqref{pot-suma}, H\"older's inequality, Sobolev's inequality and {\bf the properties of $f$( no decrecimiento y subaditividad )} imply that
\begin{equation}\label{cota-i1}
\begin{split}
I_1&
\leq C(f)\bigg\{ f(|\b{\overline{u}}|) \|b_1\|_{L^1} \|\b{\tilde{u}}\|_{L^{\infty}}+
 \|b_1\|_{L^1}f(\|\b{\tilde{u}}\|_{L^\infty})\|\b{\tilde{u}}\|_{L^\infty}\bigg\}
\\
&\leq C_2 \bigg\{ f(|\b{\overline{u}}|) \|\b{\dot{u}}\orlnor
+f(\|\b{\dot u}\orlnor) \|\b{\dot u}\orlnor\bigg\}
\\
&=
C_2 \bigg\{ f(|\b{\overline{u}}|) \|\b{\dot{u}}\orlnor
+\varepsilon(\|\b{\dot{u}}\orlnor)\|\b{\dot{u}}\orlnor^{\alpha_{\Phi}}
%f(\|\b{\dot u}\orlnor) \|\b{\dot u}\orlnor
\bigg\}
\end{split}
\end{equation}
where $C_2=C_2(f,T, \|b_1\|_{L^1} )$. 


Let $\mu$ be a positive constant such that $1<\mu<\alpha_{\Phi}$. 
Next, using Young's inequality with conjugate exponents $\mu$ and $\frac{\mu}{\mu-1}$ 
 we get
\begin{equation}\label{cota-i1-parcial}
 \begin{split}
f(|\b{\overline{u}}|) \|\b{\dot{u}}\orlnor
&=
[\varepsilon(|\b{\overline{u}}|)
|\b{\overline{u}}|^{\alpha_{\Phi}-1}]   \|\b{\dot{u}}\orlnor
\\
&\leq 
\frac{(\mu-1)}{\mu}
[\varepsilon(|\b{\overline{u}}|)]^{\frac{\mu}{\mu-1}}
|\b{\overline{u}}|^{(\alpha_{\Phi}-1)\frac{\mu}{\mu-1}}
+\frac{1}{\mu} \|\b{\dot{u}}\orlnor^{\mu}
\\
&=
\frac{1}{\gamma}
[\varepsilon(|\b{\overline{u}}|)]^{\gamma}
|\b{\overline{u}}|^{(\alpha_{\Phi}-1)\gamma}
+\frac{1}{\mu} \|\b{\dot{u}}\orlnor^{\mu}
\\
&=
\frac{1}{\gamma}
[\varepsilon(|\b{\overline{u}}|)]^{\gamma}
|\b{\overline{u}}|^{\sigma}
+\frac{1}{\mu} \|\b{\dot{u}}\orlnor^{\mu}
\end{split}
\end{equation}
where $\gamma=\frac{\mu}{\mu-1}$ and $\sigma=(\alpha_{\Phi}-1)\gamma$. 
We point out that $\sigma=(\alpha_{\Phi}-1)\gamma$ is an arbitrary positive constant bigger than $(\alpha_{\Phi}-1)b_{\Psi}=\alpha_{\Phi}$ ???.

($\mu<\alpha_{\Phi}$ then $\sigma=\frac{\mu}{\mu-1}>b_{\Psi}$, because
$\frac{1}{\mu}+\frac{1}{b_{\Psi}}>\frac{1}{\alpha_{\Phi}}+\frac{1}{b_{\Psi}}=1$, then
$\frac{1}{b_{\Psi}}>\frac{\mu-1}{\mu}=\frac{1}{\gamma}$)

From \eqref{cota-i1}, \eqref{cota-i1-parcial} and \eqref{cota-i2},
%and the inequality $x^{r_1}\leq x^{r_2}+1$, for any $x\geq 0$ and $r_1\leq r_2$, 
we have
\begin{equation}\label{cota-i1-i2}
\begin{split}
I_1+I_2
&
\leq C_3
\bigg\{ 
[\varepsilon(|\b{\overline{u}}|)]^{\gamma}
|\b{\overline{u}}|^{\sigma}
+ \|\b{\dot{u}}\orlnor^{\mu}
+ \varepsilon(\|\b{\dot{u}}\orlnor)\|\b{\dot{u}}\orlnor^{\alpha_{\Phi}}
\bigg\}
%\\
%&\leq C_3\bigg\{ |\b{\overline{u}}|^{\sigma}
%+ \|\b{\dot {u}}\orlnor^{\mu'}
%+1\bigg\}
\end{split}
\end{equation}
with $C_3= C_3(\mu,T, \|b_1\|_{L^1} )$, $\mu<\alpha_{\Phi}$ and $\sigma>b_{\Psi}(\alpha_{\Phi}-1)$.




In the subsequent estimates, we use the decomposition $u=\overline{u}+\b{\tilde{u}}$, \eqref{cota_inf}, \eqref{cota-diferencia-F},
\eqref{cota-i1-i2} and we get
\begin{equation}\label{cota_inf_I}
\begin{split}
I(\b{u})&\geq\alpha_0\rho_{\Phi}\left( \frac{\b{\dot{u}}}{\Lambda}\right)+\int_0^TF(t,\b{u})\ dt
\\ 
&=\alpha_0\rho_{\Phi}\left( \frac{\b{\dot{u}}}{\Lambda}\right)+ \int_0^T \left[F(t,\b{u})-F(t,\b{\overline{u}})\right]\ dt 
+  \int_0^TF(t,\b{\overline{u}})\ dt
\\
&\geq \alpha_0\rho_{\Phi}\left( \frac{\b{\dot{u}}}{\Lambda}\right)
-C_3 \|\b{\dot u}\orlnor^{\mu}
+\int_0^TF(t,\b{\overline{u}})\ dt-
C_3 [\varepsilon(|\b{\overline{u}}|)]^{\gamma}|\b{\overline{u}}|^{\sigma}-C_3
\varepsilon(\|\b{\dot{u}}\orlnor)\|\b{\dot{u}}\orlnor^{\alpha_{\Phi}}
\\
&=\alpha_0J_{C_4,\mu}\left(\frac{\b{\dot u}}{\Lambda}\right)
+\int_0^TF(t,\b{\overline{u}})\ dt-
C_3 [\varepsilon(|\b{\overline{u}}|)]^{\gamma}|\b{\overline{u}}|^{\sigma}-C_3
\varepsilon(\|\b{\dot{u}}\orlnor)\|\b{\dot{u}}\orlnor^{\alpha_{\Phi}},
\end{split}
\end{equation}
where $C_4=\Lambda^{\mu'}C_3/\alpha_0$.

{\bf De ac\'a en adelante, es copia del  final de la otra demo. Debe adaptarse a la f\'ormula anterior, si es que ella logra sobrevivir....}  

Let $\b{u}_n$ be  a sequence in $\domi$ with 
$\|\b{u}_n\sobnor\to\infty$ and we have to prove that $I(\b{u}_n)\to\infty$. 
On the contrary, suppose  that for a subsequence, 
still denoted by $\b{u}_n$, $I(\b{u}_n)$ is upper bounded, i.e., there exists $M>0$ such that $|I(\b{u}_{n})|\leq M$. 
As $\|\b{u}_n\sobnor\to\infty$, from Lemma \ref{infinito-a-prom-upunto},  we have $|\overline{\b{u}}_n|+\|\b{\dot{u}}_n\orlnor\to \infty$.
Then, there exists a subsequence of $\{\b{u}_n\}$, still denoted by $\b{u}_n$, which is not bounded.
Then, 
$|\b{\overline u}_n|\to \infty$ or $\|\b{\dot{u}}_n\orlnor\to \infty$.
Now, Lemma \ref{lem_coer} implies that the functional $J_{C_4,\mu'}(\frac{\b{\dot u}}{\Lambda})$ is coercive,
and, by \eqref{propiedad-coercividad-sigma},
the functional $H(\b{\overline{u}})$ is also coercive, then 
$J_{C_4,\mu'}(\frac{\b{\dot u}_n}{\Lambda}) \to \infty$ or $H(\b{\overline{u}}_n)\to \infty$.
From \eqref{condA2}, we have that on a bounded set the functional $H(\b{\overline{u}}_n)$ is lower bounded; and, $J_{C_4,\mu'}(\frac{\b{\dot u}_n}{\Lambda})$ is also lower bounded  because the modular $\rho_{\Phi}\left(\frac{\b{\dot u}}{\Lambda}\right)$ is always bigger than zero. 
Therefore,  $I(\b{u}_n)\to\infty$ as $\|\b{u}_n\sobnor\to\infty$ which contradicts the initial assumption on the behavior of $I(\b{u}_n)$. 
\end{proof}
probando




\section*{Acknowledgments}
The authors are partially supported by a UNRC grant number 18/C417. The first author is  partially supported by a  UNSL grant number 22/F223. 


 \bibliographystyle{elsarticle-num} 
 \bibliography{biblio}


\end{document}
