\documentclass[twoside]{article}


\NeedsTeXFormat{LaTeX2e}
\ProvidesPackage{mathscinet}[2002/04/17 v1.05]
\RequirePackage{textcmds}\relax
\ProvideTextCommandDefault{\cprime}{\tprime}



%\usepackage{hyperref}
\usepackage{amssymb,amsthm}
\usepackage{amsmath}
\usepackage{color}
\usepackage{ esint }

\usepackage{fancyhdr}
\usepackage{times}

\usepackage[latin1]{inputenc}

\usepackage{comment}
\usepackage{url}
\usepackage{xcolor}
\usepackage{adjustbox}
\usepackage{hyperref}

\newtheorem{thm}{Theorem}[section]
\newtheorem{cor}[thm]{Corollary}
\newtheorem{lem}[thm]{Lemma}

\newtheorem{defi}[thm]{Definition}
\newtheorem{prop}[thm]{Proposition}
\theoremstyle{remark}
\newtheorem{comentario}{Remark}



\title{Periodic solutions of 
Euler-Lagrange equations with ``sublinear nonlinearity'' in an Orlicz-Sobolev space setting}
\author{Sonia Acinas \thanks{SECyT-UNRC, UNSL and CONICET}\\
Instituto de Matem\'atica Aplicada San Luis (IMASL)\\ 
Universidad Nacional de San Luis and CONICET\\
Ej\'ercito de los Andes 950,
(D5700HDW) San Luis, Argentina\\
Universidad Nacional de La Pampa\\
(L6300CLB) Santa Rosa, La Pampa, Argentina\\
\url{sonia.acinas@gmail.com}\\[3mm]
Fernando D. Mazzone \thanks{SECyT-UNRC and CONICET}\\
Dpto. de Matem\'atica, Facultad de Ciencias Exactas, F\'{\i}sico-Qu\'{\i}micas y Naturales\\
Universidad Nacional de R\'{i}o Cuarto\\
(5800) R\'{\i}o Cuarto, C\'ordoba, Argentina,\\
\url{fmazzone@exa.unrc.edu.ar}
}

\date{}

\newcommand{\orlnor}{\|_{L^{\Phi}}}
\newcommand{\lurnor}{\|^{*}_{L^{\Phi}}}
\newcommand{\linf}{\|_{L^{\infty}}}
\newcommand{\lphi}{L^{\Phi}}
\newcommand{\lpsi}{L^{\Psi}}
\newcommand{\ephi}{E^{\Phi}}
\newcommand{\claseor}{C^{\Phi}}
\newcommand{\wphi}{W^{1}\lphi}
\newcommand{\wphiet}{W^{1}\ephi_T}
\newcommand{\wphie}{W^{1}\ephi}
\newcommand{\sobnor}{\|_{W^{1}\lphi}}
\newcommand{\domi}{\mathcal{E}^{\Phi}_d(\lambda)}
\renewcommand{\b}[1]{\boldsymbol{#1}}
\newcommand{\rr}{\mathbb{R}}
\newcommand{\nn}{\mathbb{N}}
\newcommand{\ccdot}{\b{\cdot}}
\renewcommand{\leq}{\leqslant} 
\renewcommand{\geq}{\geqslant} 
\newcommand{\epsi}{E^{\Psi}}

\begin{document}



\maketitle
%
\begingroup%Locallizing the change to `thefootnote'.
    \renewcommand{\thefootnote}{}%Removing the footnote symbol.
    %
    \footnotetext{%
    %   2010 Mathematics Subject Classification
    %   http://www.ams.org/msc/
    \textbf{2010  AMS Subject Classification.} Primary: .
    Secondary: .
    }%
        \footnotetext{%
    \textbf{Keywords and phrases.}  .
    }%
    \endgroup
%
%
%
%

\begin{abstract}
In this paper we....

ALGO AS\'I HAB\'IA QUE ESCRIBIR AC\'A O EN LA INTRO....

The results of this paper improve on the classic ones obtained in \cite{tang1998periodic} and \cite{tang2010periodic} with $|\nabla F(t,\b{x})|\leq f(t) \b{x}^{\alpha}+g(t)$
for $\alpha<1$ and $\alpha<p$ respectively. Here, the bounds of the sublinearity are bigger than those considered in the aforementioned works.
\end{abstract}




\pagestyle{fancy} \headheight 35pt \fancyhead{} \fancyfoot{}

\fancyfoot[C]{\thepage} \fancyhead[CE]{\nouppercase{S. Acinas and F.D. Mazzone }} \fancyhead[CO]{\nouppercase{\section}}

\fancyhead[CO]{\nouppercase{\leftmark}}


%\tableofcontents




\section{Introduction}
This paper is concerned with the existence of periodic solutions of the problem
\begin{equation}\label{ProbPrin}
    \left\{%
\begin{array}{ll}
   \frac{d}{dt} D_{y}\mathcal{L}(t,\b{u}(t),\b{\dot{u}}(t))= D_{\b{x}}\mathcal{L}(t,\b{u}(t),\b{\dot{u}}(t)) \quad \hbox{a.e.}\ t \in (0,T)\\
    \b{u}(0)-\b{u}(T)=\b{\dot{u}}(0)-\b{\dot{u}}(T)=0
\end{array}%
\right.
\end{equation}
where $T>0$, $\b{u}:[0,T]\to\rr^d$ is absolutely continuous and the \emph{Lagrangian} $\mathcal{L}:[0,T]\times\rr^d\times\rr^d\to\rr$ is a Carath\'eodory function, which is continously differentiable with respect to $\b{x}$ and $\b{y}$ for a.e. $t\in [0,T]$,   satisfying the conditions
\begin{eqnarray}
|\mathcal{L}(t,\b{x},\b{y})| &\leq a(|\b{x}|)\left(b(t)+ \Phi\left(\frac{|\b{y}|}{\lambda}+f(t) \right)\right), \text{ a.e. } t\in [0,T],\label{cotaL}\\
|D_{\b{x}}\mathcal{L}(t,\b{x},\b{y})| &\leq a(|\b{x}|)\left(b(t)+ \Phi\left(\frac{|\b{y}|}{\lambda}+f(t) \right)\right), \text{ a.e. } t\in [0,T],\label{cotaDxL}\\
|D_{\b{y}}\mathcal{L}(t,\b{x},\b{y})| &\leq a(|\b{x}|)\left(c(t)+ \varphi\left(\frac{|\b{y}|}{\lambda}+f(t)\right)  \right),  \text{ a.e. } t\in [0,T].\label{cotaDyL}
\end{eqnarray}
In these inequalities we assume that  $a\in C(\mathbb{R}^+,\mathbb{R}^+)$, $\lambda>0$, $\Phi$ is an $N$-function (see section  Preliminaries  for definitions), $\varphi$ is the right continuous derivative of $\Phi$. The non negative functions $b,c$ and $f$ satisfy that  $b\in L^1_1([0,T]) $,  $c\in\lpsi_1([0,T])$ and  $f\in \ephi_1([0,T])$, where  the Banach spaces $ L^1_1([0,T]), \lpsi_1([0,T])$ and  $\ephi_1([0,T])$  will be defined later.


It is well known that problem \eqref{ProbPrin} comes from a variational one, that is,  a solution of \eqref{ProbPrin}  
is a critical point of the \emph{action integral}
\begin{equation}\label{integral_accion}
I(\b{u})=\int_{0}^T \mathcal{L}(t,\b{u}(t),\b{\dot{u}}(t))\ dt.
\end{equation}




\section{Preliminaries}\label{preliminares}

For reader convenience, we give a short introduction to Orlicz and Orlicz-Sobolev spaces of vector valued functions and a  list  of results that we will use throughout the article. 
Classic references for Orlicz spaces of real valued functions are \cite{adams_sobolev,KR,rao1991theory}.
For  Orlicz spaces of vector valued functions, see \cite{Orliczvectorial2005} and the references therein.

Hereafter we denote  by $\mathbb{R}^+$  the set of all non negative real numbers. A function $\Phi:\mathbb{R}^+\to \mathbb{R}^+ $ is called an \emph{$N$-function} if $\Phi$ is given by 
\[
\Phi(t)=\int_{0}^t \varphi(\tau)\ d\tau,\quad\hbox{for } t\geq 0,
\]
where $\varphi:\mathbb{R}^+\rightarrow \mathbb{R}^+$ is a right continuous non decreasing function  satisfying   $\varphi(0)=0$, $\varphi(t)>0$ for $t>0$ and
$\lim_{t\rightarrow \infty}\varphi(t)=+\infty$.

Given a function $\varphi$ as above, we  consider the so-called right inverse function $\psi$ of $\varphi$ which is 
defined by $\psi(s)=\sup_{\varphi(t)\leq s}t$.
The function $\psi$ satisfies the same properties as the function $\varphi$, therefore we have an $N$-function $\Psi$ such that $\Psi'=\psi$ .
 The function $\Psi$ is called the \emph{complementary function} of $\Phi$.


We say that $\Phi$ satisfies the  \emph{$\Delta_2$-condition}, denoted by $\Phi \in \Delta_2$, 
if there exist  constants $K>0$ and  $t_0\geq 0$ such that 
\begin{equation}\label{delta2defi}\Phi(2t)\leq K\Phi(t)
\end{equation}
for every $t\geq t_0$. 
If $t_0=0$,  we say that $\Phi$ satisfies the \emph{$\Delta_2$-condition globally} ($\Phi \in \Delta_2$ globally).  

% and plain symbols indicate scalars.

Let $d$ be a positive integer. We denote by $\mathcal{M}_d:=\mathcal{M}_d([0,T])$ the set of all measurable functions defined on $[0,T]$ with values on $\mathbb{R}^d$ and  we write $\b{u}=(u_1,\dots,u_d)$ for  $\b{u}\in \mathcal{M}_d$.
In this paper we adopt the convention that bold symbols denote points in $\mathbb{R}^d$.


Given  an $N$-function $\Phi$ we define the \emph{modular function} 
$\rho_{\Phi}:\mathcal{M}_d\to \mathbb{R}^+\cup\{+\infty\}$ by
\[\rho_{\Phi}(\b{u}):= \int_0^T \Phi(|\b{u}|)\ dt.\]
Here $|\cdot|$ is the euclidean norm of $\mathbb{R}^d$.
The \emph{Orlicz class} $C_d^{\Phi}=C_d^{\Phi}([0,T])$  is given  by
\begin{equation}\label{claseOrlicz}
  C^{\Phi}_d:=\left\{\b{u}\in \mathcal{M}_d | \rho_{\Phi}(\b{u})< \infty \right\}.
\end{equation}
The \emph{Orlicz space} $\lphi_d=L^{\Phi}_d([0,T])$ is the linear hull of $\claseor_d$;
equivalently,
\begin{equation}\label{espacioOrlicz}
\lphi_d:=\left\{ \b{u}\in \mathcal{M}_d | \exists \lambda>0: \rho_{\Phi}(\lambda \b{u}) < \infty   \right\}.
\end{equation}
  The Orlicz space $\lphi_d$ equipped with the \emph{Orlicz norm}
\[
\|  \b{u}  \orlnor:=\sup \left\{  \int_0^T \b{u}\b{\cdot} \b{v}\ dt \big| \rho_{\Psi}(\b{v})\leq 1\right\},
\]
is a Banach space. By $\b{u}\b{\cdot} \b{v}$ we denote the usual dot product in $\mathbb{R}^{d}$ between $\b{u}$ and $\b{v}$.  
The following alternative expression for the norm, known as \emph{Amemiya norm},     will  be useful (see \cite[Thm. 10.5]{KR} and \cite{hudzik2000amemiya}). For every $\b{u}\in\lphi$,

\begin{equation}\label{amemiya}
\|\b{u}\orlnor=\inf\limits_{k>0}\frac{1}{k}\left\{1+\rho_{\Phi}(k\b{u})\right\}.
\end{equation}



The subspace $\ephi_d=\ephi_d([0,T])$ is defined as the closure in $\lphi_d$ of the subspace $L^{\infty}_d$ of all $\mathbb{R}^d$-valued essentially bounded functions. It is shown that  $\ephi_d$ is the only one maximal subspace contained in the Orlicz class $\claseor_d$, i.e. 
$\b{u}\in\ephi_d$ if and only if $\rho_{\Phi}(\lambda \b{u})<\infty$ for any $\lambda>0$.  

A generalized version of \emph{H\"older's inequality} holds in Orlicz spaces (see \cite[Th. 9.3]{KR}). Namely, if $\b{u}\in\lphi_d$ and $\b{v}\in\lpsi_d$ then $\b{u}\ccdot\b{v}\in L_1^1$ and
\begin{equation}\label{holder}
\int_0^T\b{v}\ccdot\b{u}\ dt\leq \|\b{u}\orlnor\|\b{v}\|_{L^{\Psi}}.
\end{equation}




If $X$ and $Y$ are  Banach spaces such that  $Y\subset X^*$, we denote by $\langle\cdot,\cdot\rangle:Y\times X\to\mathbb{R}$ the bilinear pairing  map given by $\langle x^*,x\rangle=x^*(x)$. H\"older's inequality shows that $\lpsi_d\subset \left[\lphi_d\right]^*$, where the pairing  
$\langle \b{v}, \b{u}\rangle$
is defined by 
\begin{equation}\label{pairing}
  \langle \b{v},\b{u}\rangle=\int_0^T\b{v}\ccdot\b{u}\ dt
\end{equation}
with  $\b{u}\in\lphi_d$ and $\b{v}\in\lpsi_d$.
 Unless $\Phi \in \Delta_2$, the relation $\lpsi_d= \left[\lphi_d\right]^*$ will not hold. In general, it is true  that  $\left[\ephi_d\right]^*=\lpsi_d$.


Like in \cite{KR}, we will consider the subset $\Pi(\ephi_d,r)$ of $\lphi_d$ given by
\[\Pi(\ephi_d,r):=\{\b{u}\in\lphi_d| d(\b{u},\ephi_d)<r\}.\]
This set is related to the Orlicz class $\claseor_d$ by means of inclusions, namely,
\begin{equation}\label{inclusiones}\Pi(\ephi_d, r )\subset r \claseor_d\subset\overline{\Pi(\ephi_d,r)}
\end{equation}
for any positive $r$.
If $\Phi \in \Delta_2$,  then the sets $\lphi_d$, $\ephi_d$, $\Pi(\ephi_d,r)$ and $\claseor_d$ are equal.



We define the \emph{Sobolev-Orlicz space} $\wphi_d$ (see \cite{adams_sobolev}) by
\[\wphi_d:=\{\b{u}| \b{u} \hbox{ is absolutely continuous and } \b{\dot{u}}\in \lphi_d\}.\]
$\wphi_d$ is a Banach space when equipped with the norm
\begin{equation}\label{def-norma-orlicz-sob}
\|  \b{u}  \|_{\wphi}= \|  \b{u}  \|_{\lphi} + \|\b{\dot{u}}\orlnor.
\end{equation}



For a  function $\b{u}\in L^1_d([0,T])$, we write $\b{u}=\overline{\b{u}}+\widetilde{\b{u}}$ where $\overline{\b{u}} =\frac1T\int_0^T \b{u}(t)\ dt$ and $\widetilde{\b{u}}=\b{u}-\overline{\b{u}}$.

As usual, if $(X,\|\cdot\|_X)$ is a Banach space and $(Y,\|\cdot \|_Y)$ is a subspace of $X$,  we write $Y\hookrightarrow X$ and we say that $Y$ is \emph{embedded} in $X$  when the restricted identity map $i_Y:Y\to X$ is bounded. That is, there exists $C>0$ such that  for any $y\in Y$ we have $\|y\|_X\leq C\|y\|_Y$.  With this notation, H\"older's inequality states that  $\lpsi_d\hookrightarrow  \left[\lphi_d\right]^*$; and, it is easy to see that for every $N$-function $\Phi$ we have that $L^{\infty}_d\hookrightarrow\lphi_d \hookrightarrow L^1_d$.


 Recall that a function   $w:\mathbb{R}^+\to \mathbb{R}^+$ is called  a \emph{modulus of continuity} if $w$ is a continuous increasing function which satisfies $w(0)=0$. For example, it can be easily shown that $w(s)=s\Phi^{-1}(1/s)$ is a modulus of  continuity for every $N$-function $\Phi$.  We say that $\b{u}:[0,T]\to\rr^d$  has modulus of continuity $w$  when there exists a constant $C>0$ such that 
\begin{equation}\label{w-holder}|\b{u}(t)-\b{u}(s)|\leq Cw(|t-s|).
\end{equation}


We denote by $C^w([0,T],\rr^d)$  the space of  $w$-H\"older continuous functions. This is the space of all functions satisfying \eqref{w-holder} for some $C>0$ and it is a Banach space with norm
\[\|\b{u}\|_{  C^w([0,T],\rr^d) }  :=\|\b{u}\|_{L^{\infty}}+\sup\limits_{t\neq s}\frac{|\b{u}(t)-\b{u}(s)|}{w(|t-s|)}.\]





 An important aspect of the theory of Sobolev spaces is related to embedding theorems. There is an extensive literature on this question in the  Orlicz-Sobolev space setting, see for example
 \cite{cianchi2000fully,cianchi1999some,claverooptimal,edmunds2000optimal,kerman2006optimal}.
The next simple lemma, whose proof can be found in \cite{ABGMS2015}, will be used systematically.




\begin{lem}\label{inclusion orlicz} Let  $w(s):= s\Phi^{-1}(1/s)$. Then, the following statements hold:
\begin{enumerate}
\item\label{inclusion orlicz_item1} $\wphi\hookrightarrow C^w([0,T],\rr^d) $ and for every $\b{u}\in\wphi$
\begin{align}
 &\left|\b{u}(t)-\b{u}(s) \right| \leq  \|\b{\dot{u}}\orlnor w(| t-s|),&\label{in-sob-cont}
\\
& \|\b{u}\|_{L^{\infty}} \leq\Phi^{-1}\left(\frac{1}{T}\right)\max\{1,T\}\|\b{u}\sobnor&\label{sobolev}
\end{align}
\item For every $\b{u}\in\wphi$ we have $\widetilde{\b{u}}\in L^{\infty}_d$ and 
\begin{align}
& \|\widetilde{\b{u}}\|_{L^{\infty}} \leq T\Phi^{-1}\left(\frac{1}{T}\right)\|\b{\dot u}\orlnor&\text{  (Sobolev's inequality).}\label{wirtinger}
\end{align}




\end{enumerate}
\end{lem}


The following result is analogous to some lemmata in $W^1L^p_d$, see \cite{xu2007some}.
\begin{lem}\label{infinito-a-prom-upunto}
If $\|\b{u}\sobnor\to \infty$, then $(|\b{\overline u}|+\|\b{\dot u}\orlnor)\to \infty$.
\end{lem}

\begin{proof}
By the decomposition $\b{u}=\b{\overline u}+\b{\tilde{u}}$ and some elementary operations, 
we get
\begin{equation}\label{cota-u-lphi}
\|\b{u}\orlnor=
\|\b{\overline u}+\b{\tilde{u}}\orlnor\leq 
\|\b{\overline u}\orlnor+\|\b{\tilde{u}}\orlnor=
|\b{\overline u}|\|1\orlnor+\|\b{\tilde{u}}\orlnor.
\end{equation}
It is known that $L^{\infty}_d\hookrightarrow\lphi_d$, i.e.
there exists $C_1=C_1(T)>0$ such that for any $\b{\tilde{u}}\in L^{\infty}_d$ we have
\[
\|\b{\tilde{u}}\orlnor
\leq 
C_1 \|\b{\tilde{u}}\|_{L^{\infty}};
\]
and, applying  Sobolev's inequality,  we obtain Wirtinger's inequality, that is there exists $C_2=C_2(T)>0$ such that 
\begin{equation}\label{cota-u-tilde}
\|\b{\tilde{u}}\orlnor
\leq 
C_2\|\b{\dot{u}}\orlnor.
\end{equation}

Therefore, from \eqref{cota-u-lphi}, \eqref{cota-u-tilde} and \eqref{def-norma-orlicz-sob}, 
we get
\[
\|\b{u}\sobnor\leq
C_3(|\b{\overline u}|+\|\b{\dot{u}}\orlnor)
\]
where $C_3=C_3(T)$. Finally, as $\|\b{u}\sobnor\to \infty$ we conclude that   
$(|\b{\overline u}|+\|\b{\dot{u}}\orlnor)\to \infty$.
\end{proof}


We present a definition that will be useful later.
 
\begin{defi} A function $\mathcal{L}:[0,T]\times \mathbb{R}^d \times \mathbb{R}^d \rightarrow \mathbb{R}$ is a \emph{Carath\'eodory} function if for fixed $(\b{x},\b{y})$
the map $t \mapsto \mathcal{L}(t, \b{x},\b{y})$ is measurable  and for fixed $t$ the map  $(\b{x},\b{y}) \mapsto \mathcal{L}(t, \b{x}, \b{y})$ is continuous  for almost everywhere $t\in [0,T]$. We say that 
$\mathcal{L}(t, \b{x},\b{y})$ is  \emph{differentiable Carath\'eodory} if in addition $\mathcal{L}(t, \b{x},\b{y})$ is
continuously differentiable with respect to $\b{x}$ and $\b{y}$  for almost everywhere $t\in [0,T]$.

\end{defi}


In \cite{ABGMS2015} we proved the next results.

\begin{thm}\label{teorema_acotacion}
Let $\mathcal{L}$ be a differentiable Carath\'eodory function satisfying \eqref{cotaL}, \eqref{cotaDxL} and \eqref{cotaDyL}. 
Then the following statements hold:
\begin{enumerate}
\item \label{T1item1} \label{A1} The action integral given by \eqref{integral_accion}
is finitely defined on $\domi:=W^{1}\lphi_d\cap\{\b{u}|\b{\dot{u}}\in\Pi(\ephi_d,\lambda)\}$.

\item\label{T1item3} The function  $I$ is G\^ateaux differentiable on $\domi$ and  its derivative $I'$ is demicontinuous from $\domi$  into $\left[\wphi_d \right]^*$. Moreover, $I'$ is given by the following expression
\begin{equation}\label{DerAccion}
\langle  I'(\b{u}),\b{v}\rangle= \int_0^T \left\{D_{\b{x}}\mathcal{L}\big(t,\b{u},\b{\dot{u}}\big)\ccdot \b{v}+ D_{\b{y}}\mathcal{L}\big(t,\b{u},\b{\dot{u}}\big)\ccdot\b{\dot{v}}\right\} \ dt.
\end{equation}

\item\label{T1item4}  If  $\Psi \in \Delta_2$ then 
  $I'$ is continuous from $\domi$ into $\left[\wphi_d\right]^*$ when both spaces are equipped with the strong topology.
\end{enumerate}
\end{thm}





In \cite{ABGMS2015} we derive the Euler-Lagrange equations associated to critical points of action integrals on the subspace of $T$-periodic functions.  
We denote by $\wphi_T$ the subspace of $\wphi_d$ containing all  $T$-periodic functions. As usual, when $Y$ is a subspace of
the Banach space $X$, we denote by $Y^{\perp}$ the \emph{annihilator subspace} of $X^*$, i.e. the subspace
that consists of all  bounded linear functions which are identically zero on $Y$.

We recall that  a function $f: \mathbb{R}^d \to \mathbb{R}$ is called \emph{strictly convex} if 
$f\left(\tfrac{\b{x}+\b{y}}{2}\right)< \tfrac{1}{2} \left(f\left(
\b{x}\right)+f\left( \b{y}\right)\right)$ for  $\b{x}\neq\b{y}$.  
It is  well known that if $f$ is a strictly convex and differentiable function, then
$D_{\b{x}}f:\mathbb{R}^d\to\mathbb{R}^d$ is a one-to-one map  (see, e.g. \cite[Thm. 12.17]{rockafellar2009variational}).

The following theorem is a slight modification of  \cite[Th. 4.1]{ABGMS2015} and it is proved in a similar way.

\begin{thm}\label{critpoint} Let $\b{u}\in\wphiet$. The following statements are equivalent:
\begin{enumerate}
 \item $I'(\b{u})\in\left( \wphiet\right)^{\perp}$.
 \item  $D_{\b{y}}\mathcal{L}(t,\b{u}(t),\b{\dot{u}}(t))$ is an absolutely continuous function and $\b{u}$ solves the following boundary value problem
 \begin{equation}\label{ecualagran2}
    \left\{%
\begin{array}{ll}
   \frac{d}{dt} D_{y}\mathcal{L}(t,\b{u}(t),\b{\dot{u}}(t))= D_{\b{x}}\mathcal{L}(t,\b{u}(t),\b{\dot{u}}(t)) \quad \hbox{a.e.}\ t \in (0,T)\\
    \b{u}(0)-\b{u}(T)=D_{\b{y}}\mathcal{L}(0,\b{u}(0),\b{\dot{u}}(0))-D_{\b{y}}\mathcal{L}(T,\b{u}(T),\b{\dot{u}}(T))=0.
\end{array}%
\right.
\end{equation}
\end{enumerate}
Moreover if $D_{\b{y}}\mathcal{L}(t,x,y)$ is $T$-periodic with respect to the variable $t$ and strictly convex with respect to $\b{y}$, then
$D_{\b{y}}\mathcal{L}(0,\b{u}(0),\b{\b{\dot{\b{u}}}}(0))-D_{\b{y}}\mathcal{L}(T,\b{u}(T),\b{\dot{u}}(T))=0$ is equivalent to $\b{\dot{u}}(0)=\b{\dot{u}}(T)$.
\end{thm}

HABR\'IA QUE ARREGLAR EL TEOREMA ANTERIOR CAMBIANDO $\wphi_T$ por $\wphiet$??????

{\bf Habr\'ia que ver si el lugar de los \'indices es el adecuado. Copi\'e lo que ten\'iamos en el 
primer trabajo.}


Next, we enumerate some definitions and results from the theory of convex functions. 
We suggest \cite{FK97, GP77, KR, M, rao1991theory} for definitions, proofs and additional details.

We denote by $\alpha_{\varphi}$ and $\beta_{\varphi}$ the so-called  \emph{Matuszewska-Orlicz indices} of the function $\varphi$, which are defined next. Given
an increasing, unbounded, continuous function  \linebreak $\varphi:[ 0,+\infty)\to [0,+\infty)$ such that $\varphi(0)=0$ we define
\begin{equation}\label{MO_indices}
    \alpha_{\varphi}:=\lim\limits_{t\to 0^{+}}\frac{\log \left (\sup\limits_{u>0}\frac{\varphi(t u)}{\varphi(u)} \right ) }{\log(t)},\quad
    \beta_{\varphi}:=\lim\limits_{t\to +\infty}\frac{\log \left  (\sup\limits_{u>0}\frac{\varphi(t u)}{\varphi(u)}\right )}{\log(t)}.
\end{equation}
We have that $0\leq \alpha_{\varphi}\leq \beta_{\varphi}\leq +\infty$. The relation $\beta_{\varphi}<\infty$ holds true if and only if $\varphi$ satisfies the $\Delta_2$-condition. If $\varphi$ is a homeomorphism  we have that
\begin{equation}\label{inv_indices}
    \alpha_{\varphi^{-1}}=\frac{1}{\beta_{\varphi}}.
\end{equation}
Moreover $\varphi\in\mathcal{F}$ implies  $\alpha_{\varphi}\geq 1$. As a consequence, $\varphi^{-1}$ satisfies the $\Delta_2$-function.

 It is well known   that if $\varphi$ is an increasing function that satisfies the $\Delta_2$-condition, $\varphi$ is controlled by above and below 
 by power functions.  More concretely, for every $\epsilon>0$ there exists a
constant $K=K(\varphi,\epsilon)$ such that, for every $t,u\geq 0$,
\begin{equation}\label{delta2-potencias}
    K^{-1}\min\big\{t^{\beta_{\varphi}+\epsilon},t^{\alpha_{\varphi}-\epsilon} \big\}\varphi(u)\leq \varphi(t u)\leq
    K\max\big\{t^{\beta_{\varphi}+\epsilon},t^{\alpha_{\varphi}-\epsilon} \big\}\varphi(u).
\end{equation}



\section{Lagrangians satisfying  sublinear nonlinearity type conditions}
\begin{lem}
Let $\Phi,\Psi$ complementary functions.
The next statements are equivalent:
\begin{enumerate}
\item\label{item1} $\Psi \in \Delta_2$ globally.
\item\label{item2} There exists an $N$-function $\Phi_1 \in \Delta_2$ such that
\begin{equation}\label{eq:caract_delta2}
\Phi(rs)\geq \Phi_1(r)\Phi(s)\;\;\mbox{for every}\;\;r\geq1,\;\;s\geq 0.
\end{equation}
\end{enumerate}
\end{lem}

\begin{proof}
\ref{item1})$\Rightarrow$\ref{item2}) As $\Psi \in  \Delta_2$ globally, there exist $k>0$ and $\nu>1$ such that
\[
\Phi(rs)\geq k r^\nu \Phi(s)\;\;r\geq 1,\;s>0,
\]
which is \eqref{eq:caract_delta2} with $\Phi_1(r)=kr^\nu$ that is an $N$-function satisfying the $\Delta_2$-condition.

\ref{item2})$\Rightarrow$\ref{item1})
Next, we follow  \cite[p. 32, Prop. 13]{rao1991theory} and \cite[p. 29, Prop. 9]{rao1991theory}.
Assume that 
\[
\Phi_1(r)\Phi(s)\leq \Phi(rs)\;\;r>1,\;s\geq 0.
\]
Let $u=\Phi_1(r)\geq \Phi_1(1)$ and $v=\Phi(s)\geq 0$. By a well known inequality \cite[p. 13, Prop. 1]{rao1991theory} and \eqref{eq:caract_delta2},  we have  for $u\geq \Phi_1(1)$ and $v\geq 0$
\[
\frac{uv}{\Psi^{-1}(uv)}\leq \Phi^{-1}(uv)\leq\Phi_1^{-1}(u)\Phi^{-1}(v)\leq
\frac{4uv}{\Psi_1^{-1}(u)\Psi^{-1}(v)},
\]
then 
\[
\Psi^{-1}_1(u)\Psi^{-1}(v)\leq 4 \Psi^{-1}(uv).
\]
If we take $x=\Psi^{-1}_1(u)\geq \Psi^{-1}_1(\Phi_1(1))$ and $y=\Psi^{-1}(v)\geq 0$, then 
\[
\Psi\left(\frac{xy}{4}\right)\leq \Psi_1(x)\Psi(y).
\]
Now, taking  $x\geq \max\{8,\Psi_1^{-1}(\Phi_1(1))\}$ we get that $\Psi \in \Delta_2$ globally.
\end{proof}

The following lemma generalizes \cite[Lemma 5.2]{ABGMS2015}.

\begin{lem}\label{lem_coer}
Let $\Phi,\Psi$ be $N$-functions and suppose that $\Psi \in \Delta_2$ globally. Then
\begin{equation}\label{eq:coer_mod}
\lim\limits_{\|u\orlnor\to \infty}
\frac{\int_0^T \Phi(|u|)\,dt}{\Phi_0(\|u\orlnor)}=\infty,
\end{equation}


for every $\Phi_0$ with $\Phi_0=o(\Phi_1)$ at $\infty$ where $\Phi_1$ is any $N$-function satisfying \eqref{eq:caract_delta2}.

Reciprocally if  \eqref{eq:coer_mod} holds for some $N$-function $\Phi_0$,  then $\Psi\in\Delta_2$ (at $\infty$). 
\end{lem}

\begin{proof}
By the assumptions on $\Phi$ and $\Phi_1$  and the identity \eqref{amemiya}, we have
\[
\frac{\int_0^T \Phi(|u|)\,dt}{\Phi_0(\|u\orlnor)}\geq
\Phi_1(r) \frac{\int_0^T \Phi(r^{-1}|u|)\,dt}{\Phi_0(\|u\orlnor)}\geq
\frac{\Phi_1(r)}{\Phi_0(\|u\orlnor)}\{r^{-1}\|u\orlnor-1\}.
\]
Now, we choose $r=\frac{\|u\orlnor}{2}$ and as $\|u\orlnor\to\infty$ we can assume $r>1$.
Next, we use the fact that $\Phi_1\in\Delta_2$ and
$\Phi_0=o(\Phi_1)$ at $\infty$, and  we get
\[
\lim\limits_{\|u\orlnor \to \infty} \frac{\int_0^T \Phi(|u|)\,dt}{\Phi_0(\|u\orlnor)}\geq
\lim\limits_{\|u\orlnor \to \infty} \frac{\Phi_1\left(\frac{\|u\orlnor}{2}\right)}{\Phi_0(\|u\orlnor)}
\geq
C \lim\limits_{\|u\orlnor \to \infty} \frac{\Phi_1(\|u\orlnor)}{\Phi_0(\|u\orlnor)}=\infty.
\]
The last assertion of the lemma follows from the fact that if $\Phi_0$ is an $N$-function, then $\Phi_0(u)\geq ku$ for  $k$ small enough. Therefore \eqref{eq:coer_mod} holds for $\Phi_0(u)=|u|$, then \cite[Lemma 5.2]{ABGMS2015}  implies  $\Psi\in\Delta_2$ at $\infty$.
\end{proof}


\begin{comentario}  We point out that this lemma can be applied to more cases than \cite[Lemma 5.2]{ABGMS2015}. For example, if $\Phi(u)=u^2$, $\Phi_1$ and $\Phi_0$ are  $N$-functions with principal parts equal to $u^2/\log u$ and $u^2/(\log u)^2$ respectively (see \cite[p. 16]{KR} and \cite[Section 7]{KR} for the definition and properties of principal part). Then  \eqref{eq:coer_mod} holds for $\Phi_0$, however $\Phi_0(u)$ is not dominated for any  power function $|u|^{\alpha}$ for every $\alpha<2$. 
\end{comentario}






We define the following  functionals $J_{C,\Phi_0}:\lphi\to (-\infty,+\infty]$ and $  H_{C,\Phi_0}:\rr^n\to \rr$, where $C>0$ and $\Phi_0$ is an $N$-function, by
\begin{equation}\label{func_phi}
  J_{C,\Phi_0}(\b{u}):= \rho_{\Phi}\left(\b{u}\right)-C\Phi_0\left(\|\b{u}\orlnor\right),
\end{equation}
 and

\begin{equation}\label{eq:functional_H-bis}
 H_{C,\Phi_0}(\b{x}):=\int_0^TF(t,\b{x})dt-C\Phi_0(|\b{x}|),
\end{equation}
respectively.











Like in \cite{ABGMS2015} we consider Lagrangians $\mathcal{L}$ which are lower bounded as follows 
\begin{equation}\label{cota_inf}
\mathcal{L}(t,\b{x},\b{y})\geq \alpha_0\Phi\left(\frac{|\b{y}|}{\Lambda}\right)+ F(t,\b{x}).
\end{equation}

If $\mathcal{L}$ is given by the right hand side in \eqref{cota_inf} and $\Phi(u)=|u|^2$, then the ODE $\ddot{\b{u}}=\nabla F(t,\b{u}(t))$ in \eqref{ProbPrin}  is quasilinear,  being  $\nabla F(t,\b{u}(t))$ the nonlinearity. Following the literature, we refer to $\nabla F$ as the non linearity even when we assume in \eqref{cota_inf} just the inequality. In \cite{tang1998periodic} and \cite{tang2010periodic} the authors considered, for the $p$-laplacian case, non linearities satisfying the inequality
\[ |\nabla F(t,\b{x})|\leq b_1(t)|\b{x}|^{\alpha}+b_2(t),\]
where  $b_1,b_2 \in L^1_1$ and $\alpha$ is any power less than $p$. Thus, they said $F$ is a sublinear nonlinearity. In this paper, we consider the following type of bounds for the nonlinearity 
\begin{equation}\label{holder_cont-mu}
  \left| \nabla F(t,\b{x}) \right|\leq b_1(t)\varphi_0(|\b{x}|)+b_2(t),
\end{equation}
where $\varphi_0=\Phi'_0$ with $\Phi_0$ an $N$-function. The employment of  $N$-functions instead of power functions in  inequalities like  \eqref{holder_cont-mu}  will allow us to extend some results of   \cite{tang1998periodic} and \cite{tang2010periodic} even in the $p$-laplacian case.


Based on \cite{mawhin2010critical} we say that $F$ satisfies the condition (A) if  $F(t,\b{x})$ is a Carath\'eo\-dory function and  $F$ is continuously differentiable with respect to $\b{x}$. Moreover, the next inequality holds 
\begin{equation}\label{condA2}|F(t,\b{x})|+ |D_{\b{x}}F(t,\b{x})|\leq a(|\b{x}|)b_0(t),\quad\text{for a.e. }t\in [0,T], \forall\b{x}\in\rr^d.
\end{equation}

The following theorem establishes  coercivity of $I$ assuming sublinear conditions on the nonlinearity  $\nabla F$. 












\begin{thm}\label{coercitividad-r}
Let  $\mathcal{L}$ be a lagrangian function satisfying \eqref{cotaL}, \eqref{cotaDxL}, \eqref{cotaDyL}, \eqref{cota_inf}  and suppose that $F$ satisfies condition (A). We assume the following conditions:
\begin{enumerate}
\item $\Psi\in\Delta_2$.
\item Inequality \eqref{holder_cont-mu} with $b_1,b_2 \in L^1_1$,  $\varphi_0=\Phi'_0$ where $\Phi_0$ is a differentiable $N$-function that satisfies the $\Delta_2$-condition globally such that
$\Phi_0=o(\Phi_1)$ at $\infty$ and $\Phi_1$ verifies \eqref{eq:caract_delta2}.
\item 
\begin{equation}\label{eq:propiedad-coercividad-phi0}
\lim_{|x|\to\infty}\frac{\int_{0}^{T}F(t,\b{x})\ dt}{\Phi_0(|\b{x}|)}=+\infty.
\end{equation}
\end{enumerate}
Then  the action integral $I$ is coercive.
\end{thm}

\begin{proof}
By the decomposition $\b{u}=\overline{\b{u}}+\b{\tilde{u}}$,   Cauchy-Schwarz's inequality 
and \eqref{holder_cont-mu}, we have
\begin{equation}\label{cota-diferencia-F}
\begin{split}
&\left|\int_0^T F(t,\b{u})-F(t,\b{\overline{u}})\,dt\right|=
\left|\int_0^T \int_0^1 \nabla F(t,\b{\overline{u}}+s\b{\tilde{u}}(t))\ccdot \b{\tilde{u}}(t) \,ds \,dt\right|
\\
&\leq \int_0^T \int_0^1 b_1(t)\varphi_0(|\b{\overline{u}}+s\b{\tilde{u}}(t)|)|\b{\tilde{u}}(t)|\,ds\,dt+
\int_0^T \int_0^1 b_2(t)|\b{\tilde{u}}(t)|\,ds\,dt
\\
&=I_1+I_2.
\end{split}
\end{equation}
On the one hand, by H\"older's and Sobolev's inequalities, we estimate $I_2$ as follows
\begin{equation}\label{cota-i2}
I_2\leq \|b_2\|_{L^1} \|\b{\tilde{u}}\|_{L^{\infty}}\leq
C_1\|\b{\dot u}\orlnor,
\end{equation}
 where $C_1=C_1(\|b_2\|_{L^1}, T)$. 

On the other hand, since $\Phi_0 \in \Delta_2$ globally, then $\varphi_0 \in \Delta_2$ globally and 
consequently $\varphi_0$ is a quasi-subadditive function, i.e. there exists $C(\varphi_0)>0$ such that 
$\varphi_0(a+b)\leq C(\varphi_0)(\varphi_0(a)+\varphi_0(b))$ for every $a,b\geq 0$.
In this way, we have
\begin{equation}\label{pot-suma}
\varphi_0(|\b{\overline{u}}+s\b{\tilde{u}}(t)|)\leq
C(\varphi_0)[\varphi_0(|\b{\overline{u}}|)+\varphi_0(\|\b{\tilde{u}}\|_{L^{\infty}})],
\end{equation}
for every $s \in [0,1]$. 

Now,  inequality \eqref{pot-suma}, H\"older's and Sobolev's inequalities,
 the monotonicity, the subadditivity and  the $\Delta_2$-condition on $\varphi_0$, imply that
\begin{equation}\label{cota-i1}
\begin{split}
I_1&
\leq C(\varphi_0)\bigg\{ \varphi_0(|\b{\overline{u}}|) \|b_1\|_{L^1} \|\b{\tilde{u}}\|_{L^{\infty}}+
 \|b_1\|_{L^1}\varphi_0(\|\b{\tilde{u}}\|_{L^\infty})\|\b{\tilde{u}}\|_{L^\infty}\bigg\}
\\
&\leq C_2 \bigg\{ \varphi_0(|\b{\overline{u}}|) \|\b{\dot{u}}\orlnor
+\varphi_0(\|\b{\dot u}\orlnor) \|\b{\dot u}\orlnor\bigg\},
%\\
%&=
%C_2 \bigg\{ f(|\b{\overline{u}}|) \|\b{\dot{u}}\orlnor
%+\varepsilon(\|\b{\dot{u}}\orlnor)\|\b{\dot{u}}\orlnor^{\alpha_{\Phi}}
%%f(\|\b{\dot u}\orlnor) \|\b{\dot u}\orlnor\bigg\}
\end{split}
\end{equation}
where $C_2=C_2(\varphi_0,T, \|b_1\|_{L^1} )$. 

Next, by Young's inequality with complementary functions $\Phi_0$ and $\Psi_0$ and the fact that 
$\Phi_0 \in \Delta_2$ globally, Young's equality \cite[Eq. 2.7-2.8]{KR} and \cite[Th. 3-(ii), p. 23]{rao1991theory}, we get
\begin{equation}\label{cota-i1-parcial}
 \begin{split}
\varphi_0(|\b{\overline{u}}|) \|\b{\dot{u}}\orlnor
&\leq 
\Psi_0(\varphi_0(|\b{\overline{u}}|))+
\Phi_0(\|\b{\dot{u}}\orlnor)
\\
&\leq 
|\b{\overline{u}}|\varphi_0(|\b{\overline{u}}|)
+\Phi_0(\|\b{\dot{u}}\orlnor)
\\
&\leq C(\Phi_0)
\Phi_0(|\b{\overline{u}}|)
+\Phi_0(\|\b{\dot{u}}\orlnor)
\end{split}
\end{equation}
and 
\begin{equation}\label{cota-i1-parcial-segunda}
\varphi_0(\|\b{\dot u}\orlnor) \|\b{\dot u}\orlnor
\leq 
C(\Phi_0) \Phi_0(\|\b{\dot u}\orlnor),
\end{equation}
with $C(\Phi_0)$ the constant that comes from the $\Delta_2$-condition on $\Phi_0$.

From \eqref{cota-i1}, \eqref{cota-i1-parcial}, \eqref{cota-i1-parcial-segunda} and \eqref{cota-i2},
%and the inequality $x^{r_1}\leq x^{r_2}+1$, for any $x\geq 0$ and $r_1\leq r_2$, 
we have
\begin{equation}\label{cota-i1-i2}
\begin{split}
I_1+I_2
&
\leq C_3
\bigg\{ 
\Phi_0(|\b{\overline{u}}|)
+\Phi_0(\|\b{\dot{u}}\orlnor)
+\|\b{\dot{u}}\orlnor
\bigg\}\\
&
\leq C_4
\bigg\{ 
\Phi_0(|\b{\overline{u}}|)
+\Phi_0(\|\b{\dot{u}}\orlnor)
+1
\bigg\},
\end{split}
\end{equation}
with $C_3$ and $C_4$ depending on $\Phi_0,T, \|b_1\|_{L^1}$ and $\|b_2\|_{L^1} $. The last inequality follows from the fact that $\Phi_0$ is an $N$-function, then there exists $C>0$ such that $\Phi_0(x)\geq Cx$ for every $x\geq 1$. Thus $x\leq C\Phi_0(x)+1$ for every $x\geq 0$.


In the subsequent estimates, we use  \eqref{cota_inf}, \eqref{cota-diferencia-F},
\eqref{cota-i1-i2}, the fact that $\Phi_0 \in \Delta_2$ and we get
\begin{equation}\label{cota_inf_I}
\begin{split}
I(\b{u})&\geq\alpha_0\rho_{\Phi}\left( \frac{\b{\dot{u}}}{\Lambda}\right)+\int_0^TF(t,\b{u})\ dt
\\ 
&=\alpha_0\rho_{\Phi}\left( \frac{\b{\dot{u}}}{\Lambda}\right)+ \int_0^T \left[F(t,\b{u})-F(t,\b{\overline{u}})\right]\ dt 
+  \int_0^TF(t,\b{\overline{u}})\ dt
\\
&\geq \alpha_0\rho_{\Phi}\left( \frac{\b{\dot{u}}}{\Lambda}\right)
-C_4 \Phi_0(\|\b{\dot u}\orlnor)
+\int_0^TF(t,\b{\overline{u}})\ dt-
C_4 \Phi_0(|\b{\overline{u}}|)-
C_4 
\\
&\geq
\alpha_0\rho_{\Phi}\left( \frac{\b{\dot{u}}}{\Lambda}\right)
-C_4 \Phi_0(\|\b{\dot u}\orlnor)
+H_{C_4, \Phi_0}(\b{\overline{u}})
-C_4 
\\&\geq
\alpha_0\rho_{\Phi}\left( \frac{\b{\dot{u}}}{\Lambda}\right)
-C_5 \Phi_0\left(\frac{\|\b{\dot u}\orlnor}{\Lambda} \right)
+H_{C_4, \Phi_0}(\b{\overline{u}})
-C_4 
\\&=
\alpha_0J_{C_6,\Phi_0}\left(\frac{\b{\dot u}}{\Lambda}\right)
+H_{C_4, \Phi_0}(\b{\overline{u}})
-C_4, 
\end{split}
\end{equation}
where $C_5=C_5(\Phi_0,\Lambda,C_4)$ and $C_6=\frac{C_5}{\alpha_0}$.



Let $\b{u}_n$ be  a sequence in $\domi$ with 
$\|\b{u}_n\sobnor\to\infty$ and we have to prove that $I(\b{u}_n)\to\infty$. 
On the contrary, suppose  that for a subsequence, 
still denoted by $\b{u}_n$, $I(\b{u}_n)$ is upper bounded, i.e., there exists $M>0$ such that $|I(\b{u}_{n})|\leq M$. 
As $\|\b{u}_n\sobnor\to\infty$, from Lemma \ref{infinito-a-prom-upunto},  we have $|\overline{\b{u}}_n|+\|\b{\dot{u}}_n\orlnor\to \infty$. Passing to a subsequence, still denoted $\b{u}_n$, we can assume that $|\b{\overline u}_n|\to \infty$ or $\|\b{\dot{u}}_n\orlnor\to \infty$.
Now, Lemma \ref{lem_coer} implies that the functional $J_{C_6,\Phi_0}(\frac{\b{\dot u}}{\Lambda})$ is coercive;
and, by \eqref{eq:propiedad-coercividad-phi0},
the functional $H_{C_4,\Phi_0}(\b{\overline{u}})$ is also coercive, then 
$J_{C_6,\Phi_0}(\frac{\b{\dot u}_n}{\Lambda}) \to \infty$ or $H_{C_4,\Phi_0}(\b{\overline{u}}_n)\to \infty$.
From \eqref{condA2}, we have that on a bounded set the functional $H_{C_4,\Phi_0}(\b{\overline{u}}_n)$ is lower bounded and also $J_{C_6,\Phi_0}(\frac{\b{\dot u}_n}{\Lambda})\geq 0$. 
Therefore,  $I(\b{u}_n)\to\infty$ as $\|\b{u}_n\sobnor\to\infty$ which contradicts the initial assumption on the behavior of $I(\b{u}_n)$. 
\end{proof}













% 
% 
% 
% 
% \subsection{versi\'on potencias por si acaso...} 
% \begin{thm}\label{coercitividad-r}
% Let  $\mathcal{L}$ be a lagrangian function satisfying \eqref{cotaL}, \eqref{cotaDxL}, \eqref{cotaDyL}, \eqref{cota_inf}  and $F$ satisfies condition (A). We assume the following conditions:
% \begin{enumerate}
% \item $\Psi\in\Delta_2$.
% \item There exist  non negative functions  $b_1,b_2 \in L^1_1$ and a constant $1<\mu<\alpha_{\Phi}$  such that 
% for any $\b{x}\in\rr^d$ and a.e. $t\in [0,T]$
% \begin{equation}\label{holder_cont-mu}
%   \left| \nabla F(t,\b{x}) \right|\leq b_1(t)|\b{x}|^{\mu-1}+b_2(t).
% \end{equation}
% \item There exists a real positive number $\sigma$ such that $\sigma>(\mu-1)\beta_{\Psi}$ and
% \begin{equation}\label{propiedad-coercividad-sigma}
% |\b{x}|^{\sigma}=o\left(\int_{0}^{T}F(t,\b{x})\ dt\right)\;\;\mbox{as}\;\;|\b{x}|\to \infty.
% \end{equation}
% \end{enumerate}
% Then  the action integral $I$ is coercive.
% \end{thm}
% 
% 
% 
% 
% \begin{proof} 
% By the decomposition $u=\overline{u}+\b{\tilde{u}}$,  Mean Value Theorem, Cauchy-Schwarz's inequality 
% and \eqref{holder_cont-mu}, we have
% \begin{equation}\label{cota-diferencia-F}
% \begin{split}
% &\left|\int_0^T F(t,\b{u})-F(t,\b{\overline{u}})\,dt\right|=
% \left|\int_0^T \int_0^1 \nabla F(t,\b{\overline{u}}+s\b{\tilde{u}}(t))\ccdot \b{\tilde{u}}(t) \,ds \,dt\right|
% \\
% &\leq \int_0^T \int_0^1 b_1(t)|\b{\overline{u}}+s\b{\tilde{u}}(t)|^{\mu-1}|\b{\tilde{u}}(t)|\,ds\,dt+
% \int_0^T \int_0^1 b_2(t)|\b{\tilde{u}}(t)|\,ds\,dt
% \\
% &=I_1+I_2.
% \end{split}
% \end{equation}
% On the one hand, by H\"older's inequality and Sobolev's inequality, we estimate $I_2$ as follows
% \begin{equation}\label{cota-i2}
% I_2\leq \|b_2\|_{L^1} \|\b{\tilde{u}}\|_{L^{\infty}}\leq
% C_1\|\b{\dot u}\orlnor.
% \end{equation}
%  where $C_1=C_1(\|b_2\|_{L^1}, T)$. On the other hand, as $s\in [0,1]$, we have
% \begin{equation}\label{pot-suma}
% |\b{\overline{u}}+s\b{\tilde{u}}(t)|^{\mu-1}\leq
% C(\mu)(|\b{\overline{u}}|^{\mu-1}+\|\b{\tilde{u}}\|_{L^{\infty}}^{\mu-1}).
% \end{equation}
% where $C(\mu)=2^{\mu-2}$, for $\mu\geq 2$ and $C(\mu)=1$, for $1<\mu<2$. Now,  inequality \eqref{pot-suma}, H\"older's inequality and Sobolev's inequality imply that
% \begin{equation}\label{cota-i1}
% \begin{split}
% I_1&\leq 
% C(\mu)\left(|\b{\overline{u}}|^{\mu-1} \int_0^T b_1(t) |\b{\tilde{u}}(t)|\,dt+
% \|\b{\tilde{u}}\|^{\mu-1}_{L^{\infty}} \int_0^T b_1(t)|\b{\tilde{u}}(t)| \,dt\right)
% \\
% &\leq C(\mu)\bigg\{ |\b{\overline{u}}|^{\mu-1} \|b_1\|_{L^1} \|\b{\tilde{u}}\|_{L^{\infty}}+
%  \|b_1\|_{L^1}\|\b{\tilde{u}}\|^{\mu}_{L^\infty}\bigg\}
% \\
% &\leq C_2 \bigg\{ |\b{\overline{u}}|^{\mu-1} \|\b{\dot{u}}\orlnor+ \|\b{\dot u}\orlnor^{\mu}\bigg\},
% \end{split}
% \end{equation}
% where $C_2=C_2(\mu,T, \|b_1\|_{L^1} )$. Let $\mu'$ be a positive constant such that $1<\mu\leq \mu'<\alpha_{\Phi}$. 
% Next, using Young's inequality with conjugate exponents $\mu'$ and $\frac{\mu'}{\mu'-1}$ 
%  we get
% \begin{equation}\label{cota-i1-parcial}
% |\b{\overline{u}}|^{\mu-1}   \|\b{\dot{u}}\orlnor
% \leq \frac{(\mu'-1)}{\mu'}|\b{\overline{u}|^{\sigma}}
% +\frac{1}{\mu'} \|\b{\dot{u}}\orlnor^{\mu'}
% \end{equation}
% where $\sigma=\frac{(\mu-1) \mu'}{\mu'-1}$. We point out that $\sigma$ is an arbitrary positive constant bigger than $(\mu-1)b_{\Psi}$.
% 
% From \eqref{cota-i1}, \eqref{cota-i1-parcial}, \eqref{cota-i2} and the inequality $x^{r_1}\leq x^{r_2}+1$, for any $x\geq 0$ and $r_1\leq r_2$, we have
% \begin{equation}\label{cota-i1-i2}
% \begin{split}
% I_1+I_2
% &\leq C_3\bigg\{ |\b{\overline{u}}|^{\sigma}
% + \|\b{\dot u}\orlnor^{\mu'}
% + \|\b{\dot u}\orlnor^{\mu}
% +\|\b{\dot u}\orlnor\bigg\}\\
% &\leq C_3\bigg\{ |\b{\overline{u}}|^{\sigma}
% + \|\b{\dot u}\orlnor^{\mu'}
% +1\bigg\}
% \end{split}
% \end{equation}
% with $C_3= C_3(\mu,T, \|b_1\|_{L^1},\mu' )$. In the subsequent estimates, we use the decomposition $u=\overline{u}+\b{\tilde{u}}$, \eqref{cota_inf}, \eqref{cota-diferencia-F},
% \eqref{cota-i1-i2} and we get
% \begin{equation}\label{cota_inf_I}
% \begin{split}
% I(\b{u})&\geq\alpha_0\rho_{\Phi}\left( \frac{\b{\dot{u}}}{\Lambda}\right)+\int_0^TF(t,\b{u})\ dt
% \\ 
% &=\alpha_0\rho_{\Phi}\left( \frac{\b{\dot{u}}}{\Lambda}\right)+ \int_0^T \left[F(t,\b{u})-F(t,\b{\overline{u}})\right]\ dt 
% +  \int_0^TF(t,\b{\overline{u}})\ dt
% \\
% &\geq \alpha_0\rho_{\Phi}\left( \frac{\b{\dot{u}}}{\Lambda}\right)
% -C_3 \|\b{\dot u}\orlnor^{\mu'}
% +\int_0^TF(t,\b{\overline{u}})\ dt-
% C_3 |\b{\overline{u}}|^{\sigma}-C_3\\
% &=\alpha_0J_{C_4,\mu'}\left(\frac{\b{\dot u}}{\Lambda}\right)
% + H_{C_3,\sigma}(\b{\overline{u}})-C_3,
% \end{split}
% \end{equation}
% where $C_4=\Lambda^{\mu'}C_3/\alpha_0$.
% 
% 
% Let $\b{u}_n$ be  a sequence in $\domi$ with 
% $\|\b{u}_n\sobnor\to\infty$ and we have to prove that $I(\b{u}_n)\to\infty$. 
% On the contrary, suppose  that for a subsequence, 
% still denoted by $\b{u}_n$, $I(\b{u}_n)$ is upper bounded, i.e., there exists $M>0$ such that $|I(\b{u}_{n})|\leq M$. 
% As $\|\b{u}_n\sobnor\to\infty$, from Lemma \ref{infinito-a-prom-upunto},  we have $|\overline{\b{u}}_n|+\|\b{\dot{u}}_n\orlnor\to \infty$.
% Then, there exists a subsequence of $\{\b{u}_n\}$, still denoted by $\b{u}_n$, which is not bounded.
% Then, 
% $|\b{\overline u}_n|\to \infty$ or $\|\b{\dot{u}}_n\orlnor\to \infty$.
% Now, Lemma \ref{lem_coer} implies that the functional $J_{C_4,\mu'}(\frac{\b{\dot u}}{\Lambda})$ is coercive,
% and, by \eqref{propiedad-coercividad-sigma},
% the functional $H(\b{\overline{u}})$ is also coercive, then 
% $J_{C_4,\mu'}(\frac{\b{\dot u}_n}{\Lambda}) \to \infty$ or $H(\b{\overline{u}}_n)\to \infty$.
% From \eqref{condA2}, we have that on a bounded set the functional $H(\b{\overline{u}}_n)$ is lower bounded; and, $J_{C_4,\mu'}(\frac{\b{\dot u}_n}{\Lambda})$ is also lower bounded  because the modular $\rho_{\Phi}\left(\frac{\b{\dot u}}{\Lambda}\right)$ is always bigger than zero. 
% Therefore,  $I(\b{u}_n)\to\infty$ as $\|\b{u}_n\sobnor\to\infty$ which contradicts the initial assumption on the behavior of $I(\b{u}_n)$. 
% \end{proof}
% 
% {\bf Leer y ver si es coherente lo anterior,  si conviene trabajar siempre con $u_n$ o habr\'ia que usar la notaci\'on de subsucesiones expl\'icita!!!}
% 
% {\bf Falta leer y corregir la  secci\'on que sigue del caso l\'imite!!!}



\section{Main result}\label{sec:main}



In order to find conditions for the lower semicontinuity of  $I$, 
we perform a little adaptation of  a result of \cite{ekeland1999convex}. 


\begin{lem}\label{semicontinf}
Let $\mathcal{L}(t,\b{x},\b{y})$ be a  differentiable Carath\'eodory function. Suppose that  $F$ satisfies the condition (A) and the inequality
\begin{equation}\label{cota_inf_2}
\mathcal{L}(t,\b{x},\b{y})\geq \Phi\left(|\b{y}|\right)+ F(t,\b{x}),
\end{equation}
where $\Phi$ is an $N$-function. 
In addition, suppose that  $\mathcal{L}(t,\b{x},\cdot)$ is convex in $\rr^d$ for each $(t,\b{x})\in [0,T]\times\rr^d$.  Let $\{\b{u}_n\}\subset\wphi$ be a sequence such that $\b{u}_n$ converges  uniformly  to a function $\b{u}\in\wphi$ and $\b{\dot{u}}_n$ converges in the weak topology of $L^1_d$ to $\b{\dot{u}}$.   Then
\begin{equation}\label{liminf0}I(\b{u})\leq \liminf_{n\to\infty}I(\b{u}_n).
\end{equation}

\end{lem}

\begin{proof} First, we point out that \eqref{cota_inf_2} and \eqref{condA2} imply that $I$ is defined on $\wphi$ taking values on the interval $(-\infty,+\infty]$. 
Let $\{\b{u}_n\}$ be a sequence  satisfying the assumptions of the theorem.   We define the differentiable Carath\'eodory function $\mathcal{\hat{L}}=\mathcal{L}-F$ and we denote by $\hat{I}$ its  associated action integral. Using  \cite[Thm. 2.1, p. 243]{ekeland1999convex}, we get
\begin{equation}\label{liminf1}
\int_0^T\mathcal{\hat{L}}(t,\b{u},\b{\dot{u}})\ dt\leq \liminf_{n\to\infty}\int_0^T\mathcal{\hat{L}}(t,\b{u}_n,\b{\dot{u}}_n)\ dt.
\end{equation}
Taking account of the uniform convergence of $\b{u}_n$ and the fact that  $F$  is a  Carath\'eodory function,  we obtain that $F(t,\b{u}_n(t))\to F(t,\b{u}(t))$ a.e. $t\in[0,T]$.  Since the sequence $\b{u}_n$ is uniformly bounded, from \eqref{condA2} follows that there exists $g\in L_1^1([0,T])$ such that $|F(t,\b{u}_n(t))|\leq g(t)$. Now, by the Dominated Convergence Theorem, we have that 
\begin{equation}\label{liminf2}
\lim_{n\to\infty}\int_0^TF(t,\b{u}_n(t))\ dt=\int_0^TF(t,\b{u}(t))\ dt.
\end{equation}
Finally, as a consequence of  \eqref{liminf1} and  \eqref{liminf2}, we obtain \eqref{liminf0}.
\end{proof}


\begin{lem}\label{lem:deb*cerrado}
$\ephi_d$ is weak* closed in $\lphi_d$.
\end{lem}


\begin{proof}
From \cite[Thm. 7, p. 110]{rao1991theory} we have that $\lphi_d=\left[\epsi_d\right]^*
$.
Then, $\lphi_d$ is a dual and therefore we are allowed to speak about the weak* topology of $\lphi_d$.
Besides, $\ephi_d
$ is separable (see \cite[Thm. 1, p. 87]{rao1991theory}).
Let $S=\ephi_d\cap \{u \in \lphi_d|\|u\orlnor\leq 1\}$, then $S$ is closed in the norm $\|\cdot\orlnor$. Now, according to \cite[Cor. 5, p. 148]{rao1991theory} $S$ is weak* sequentially compact. Thus, $S$ is weak* sequentially closed because 
is $u_n\in S$ and
$u_n \overset{*}{\rightharpoonup}u \in \lphi$ then  the weak* sequentially compactness implies the existence of $v \in S$ and a subsequence $u_{n_k}$ such that 
$u_{n_k}\overset{*}{\rightharpoonup}v$. Finally, by the uniqueness of   the limit, we get
$u=v\in S$.
As $\epsi_d$ is separable and $\lphi_d=\left[\epsi_d\right]^*$, the ball of $\lphi$ $\{u \in \lphi | \|u\orlnor\leq 1\}$ is  weak* metrizable (see \cite[Thm. 5.1, p. 138]{Conway1977}).
Thus, $S$ is closed respect to  the weak* topology. Now, by the Krein-Smulian Theorem, \cite[Cor. 12.6, p. 165]{Conway1977} implies that $\ephi_d$ is weak* closed.
\end{proof}

Gathering our previous results we obtain existence of solutions.

Let $\wphiet=\wphi_T \cap \wphie_d$.


\begin{thm} 
Let $\Phi$ and $\Psi$ be complementary $N$-functions. 
Suppose that the differentiable Carath\'eodory function $\mathcal{L}(t,\b{x},\b{y})$ is strictly convex at $\b{y}$, $D_{\b{y}}\mathcal{L}$ is $T$-periodic with respect to $t$. In addition, assume the same hypothesis than Theorem \ref{coercitividad-r}. Then, problem \eqref{ProbPrin} has a solution.
\end{thm}

\begin{proof}



Let $\{\b{u}_n\}\subset \wphiet$  be a  minimizing sequence for the problem  $\inf\{I(\b{u})|\b{u}\in\wphiet\}$.
Since  $I(\b{u}_n)$, $n=1,2,\ldots$  is upper bounded, Theorem \ref{coercitividad-r}  implies that $\{\b{u}_n\}$ is norm bounded in $\wphie_d$. Hence, in virtue of Corollary \cite[Corollary 2.2]{ABGMS2015}, we can assume, taking a subsequence if necessary, that $\b{u}_n$ converges uniformly to a $T$-periodic continuous function $\b{u}$.
Then, $\b{u}$ is bounded and $\b{u} \in \ephi_d$. 

As $\b{\dot{u}}_n \in \ephi_d\subset \lphi_d$, 
%The space  $\lphi_d$ is a predual space, concretely $\lphi_d=\left[\epsi_d\right]^*$. Thus, by \cite[Cor. 5, p. 148]{rao1991theory} and since $\b{\dot{u}}_n$ is bounded in $\lphi_d$,  
there exists a subsequence (again denoted by $\b{\dot{u}}_n$) such that $\b{\dot{u}}_n$ converges to a function $\b{v}\in\lphi_d$ in the weak* topology of $\lphi_d$. 
Since $\ephi_d$ is weak* closed, by Lemma \ref{lem:deb*cerrado}, $\b{v}\in \ephi_d$.

From this fact and the uniform convergence of $\b{u}_n$ to $\b{u}$, we obtain that
\[
\int_0^T\b{\dot{\xi}}\ccdot\b{u}\ dt=\lim_{n\to\infty}\int_0^T\b{\dot{\xi}}\ccdot\b{u}_n \ dt=-\lim_{n\to\infty}\int_0^T\b{\xi}\ccdot\b{\dot{u}}_n\ dt=-\int_0^T\b{\xi}\ccdot\b{v}\ dt
\]
for every $T$-periodic function $\b{\xi}\in C^{\infty}([0,T],\rr^d)\subset\epsi_d$.
Thus $\b{v}=\b{\dot{u}}$ a.e. $t\in [0,T]$ (see \cite[p. 6]{mawhin2010critical}) and $\b{u}\in\ephi_T$.

Now, taking into account the relations $\left[L^1_d\right]^*=L^{\infty}_d\subset  \epsi_d$ and $\lphi_d\subset L^1_d$, we have that $\b{\dot{u}}_n$ converges to $\b{\dot{u}}$ in the weak topology of $L^1_d$. Consequently,  Lemma \ref{semicontinf} applied to the $N$-function $\alpha_0\Phi\left(|\ccdot|/\Lambda\right)$ implies that
\[I(\b{u})\leq  \liminf_{n\to\infty}I(\b{u}_n)=\inf\limits_{\b{u}\in\wphie_T}I(\b{u}).\]

As $\b{u}\in \wphiet\subset \domi$ then $I(\b{u})>-\infty$, hence, $\b{u}$ is a minimun and therefore  $I'(\b{u})\in (\wphiet)^{\perp}$. Finally,
invoking Theorem \ref{critpoint},  the proof concludes.\end{proof}



 \section{Limit case $\mu=\alpha_{\Phi}$}
Assuming $\|b_1\|_{L^1}$  small enough, in  \cite{zhao2004periodic, tang2010periodic} 
coercivity  was obtained even  for the limit value $\mu=p$ in inequality \eqref{holder_cont-mu}.  

{\bf OJO que $\mu$ no aparece en \eqref{holder_cont-mu}!!!!. Quiz\'as deber\'ia decir $\varphi_0(x)=x^p$. O, mecionarse la ecuaci\'on anterior donde aparece $\alpha<p$, no $\mu$.} 

This result leans on the  fact that
\begin{equation}
 \|u\orlnor^{\alpha_{\Phi}}=O\left(\int_0^T \Phi(|u|)\,dt\right)\quad\text{for } \|u\orlnor\to\infty,
\end{equation}
when  $\Phi(u)=|u|^p$.
Nevertheless, it is no longer the case  for any $N$-function $\Phi$ as the following example shows.

In this section, from now on we will suppose that
\[\Phi(u)=
\left\{
\begin{array}{ll}
\frac{p-1}{p}u^p&u\leq e
\\
\frac{u^p}{\log u}-\frac{e^p}{p}&u>e
\end{array}
\right.\]
with $p>1$. Next, we will establish some properties of this function $\Phi$.

\begin{thm}
If $p\geq \frac{1+\sqrt 2}{2}$, then $\Phi$ is an $N$-function.
\end{thm}


\begin{proof}
We have
\[\varphi(u)=\Phi'(u)=\left\{
\begin{array}{cccc}
(p-1)u^{p-1}&:=&\varphi_1(u)& \mbox{if}\;u\leq e
\\
\frac{u^{p-1}}{\log u}(p-\frac{1}{\log u})&:=&\varphi_2(u)&\mbox{if}\; u\geq e
\end{array}
\right.
\]

First let us see that $\Phi'$ is increasing when $p\geq \frac{1+\sqrt {2}}{2}$.
For this purpose, since $\varphi_1(e)=\varphi_2(e)$, it is enough to see that $\varphi_1$ is increasing  on $[0,e]$ and $\varphi_2$ is increasing on
$[e,\infty)$ for every $p\geq \frac{1+\sqrt {2}}{2}$. Clearly
$\varphi_1$ is an increasing function for $p>1$.  On the other hand, an elementary analysis of the function shows that
$\varphi_2'(u)>0$ on $[e,\infty)$ if and only if
 $p \notin(\frac{1-\sqrt2}{2},\frac{1+\sqrt2}{2})$.  Therefore $\varphi_2$ is an icreasing function when $p\geq \frac{1+\sqrt2}{2}$.

 Besides $\varphi_2(u)\to \infty$ and  $\varphi_1(u)\to 0$  as $u \to  \infty$ and $u\to 0$  respectively, provided that $p>1$. Hence, $\Phi$ is an $N$-function.
\end{proof}


\begin{thm} For every $\varepsilon>0$, there exists a positive constant $C=C(p,\varepsilon)$  such that
\begin{equation}\label{cota-sup-indices}
C^{-1}t^{p-\varepsilon}\Phi(u)\leq \Phi(tu) \leq Ct^p\Phi(u)\quad t\geq 1, u>0,
\end{equation}
\end{thm}

\begin{proof} If $u\leq tu\leq e$, then $\Phi(tu)=t^p\Phi(u)$ and \eqref{cota-sup-indices} holds with $C=1$.

If $u\leq e\leq tu$, as $\frac{e^p}{p}>0$  and $\log(tu)\geq 1$, we have
$\Phi(tu)\leq t^pu^p= \frac{p}{p-1}t^p\Phi(u)$. Thus, the second inequality of  \eqref{cota-sup-indices} holds with $C=\frac{p}{p-1}$. On the other hand, as $f(t)=\frac{t}{\log t}$ is increasing on $[e,\infty)$, then $f((tu)^p)\geq  f(e^p)=e^p/p$.
Now,
\[
\begin{split}
\Phi(tu)&=\frac{p(tu)^p}{\log (tu)^p}-\frac{e^p}{p}\\
&= \frac{(p-1)(tu)^p}{\log(tu)^p}+\frac{(tu)^p}{\log (tu)^p}-\frac{e^p}{p}
\\
&\geq \frac{p-1}{p}\frac{(tu)^p}{\log(tu)}\\
&\geq
\frac{p-1}{p}\frac{t^{\varepsilon}}{\log t+1}t^{p-\varepsilon}u^p.
\end{split}
\]
Since $\varepsilon e^{1-\varepsilon}$ is the minimum value of $t\mapsto\frac{t^{\varepsilon}}{\log t+1}$  on the interval $[1,+\infty)$ then
\[
\Phi(tu)\geq \frac{p-1}{p}\varepsilon e^{1-\varepsilon}t^{p-\varepsilon}u^p,
\]
which is the first inequality of \eqref{cota-sup-indices} with $C=\frac{p}{p-1}\varepsilon^{-1} e^{-1+\varepsilon}$.


If $e\leq u\leq tu$, then
\begin{equation}\label{Phi-de-u-a-v}
\Phi(tu)\leq \frac{t^pu^p}{\log(tu)}\leq \frac{t^pu^p}{\log(u)}=\frac{pt^pv}{\log v},
\end{equation} 
where $v:=u^p$ and $v\geq e^p$.  
If $\alpha>0$, the function $x\mapsto \frac{x}{x-\alpha}$ is decreasing on $(\alpha,\infty)$
and the function $v\mapsto \frac{pv}{\log v}$ is increasing  on $[e^p,\infty)$.
Therefore,  we have
\[
\frac{\frac{pv}{\log v}}{\frac{pv}{\log v}-\frac{e^p}{p}}\leq
\frac{e^p}{e^p-\frac{e^p}{p}}=\frac{p}{p-1}
\]
for every $v \geq e^p$. In this way, from \eqref{Phi-de-u-a-v}, we have
\[
\Phi(tu)\leq \frac{pt^p}{p-1}\left(\frac{pv}{\log v}-\frac{e^p}{p}\right)=
 \frac{pt^p}{p-1}\left(\frac{u^p}{\log u}-\frac{e^p}{p}\right)
\]
and the second inequality of  \eqref{cota-sup-indices} holds with $C=\frac{p}{p-1}$. For the first inequality we have, as it was proved previously,

\[
  \Phi(tu)
  \geq
  \frac{p-1}{p}\frac{(tu)^p}{\log(tu)}
  =
  \frac{p-1}{p}
  \frac{t^{\varepsilon} \log u^{\varepsilon}}{\log(t^{\varepsilon}u^{\varepsilon})}
  \frac{t^{p-\varepsilon}u^p}{\log u}
\]
Let $f(s)=\frac{sA}{\log s+A}$ with $s\geq 1$ and $A\geq \varepsilon$.  If $A\leq 1$,  the function $f$ attains a minimum on $[1,\infty)$ at $s=e^{1-A}$ and the minimum value is $f(e^{1-A})=Ae^{1-A}\geq \varepsilon$. If $A> 1$, $f$ is increasing  on $[1,\infty)$ and its minimum value is $f(1)=1$. Then, $f(s)\geq \varepsilon$ in any case,   therefore
\[
\Phi(tu)\geq \frac{p-1}{p}\varepsilon \frac{t^{p-\varepsilon}u^p}{\log u}\geq
\frac{p-1}{p}\varepsilon t^{p-\varepsilon}\Phi(u).
\]
Therefore, \eqref{cota-sup-indices} holds with $C=\frac{p}{\varepsilon (p-1)}$, because this $C$ is the biggest constant that we have obtained in each case under consideration.
\end{proof}



\begin{comentario}
The inequality
\[
\Phi(tu)\geq Ct^p\Phi(u)
\]
is false for every $C$ because for every $u\geq e$ we have
\[
\lim\limits_{t \to \infty}\frac{\Phi(tu)}{t^p\Phi(u)}=0
\]
\end{comentario}






\begin{thm}
$\alpha_{\Phi}=\beta_{\Phi}=p$
\end{thm}

\begin{proof}
From \eqref{MO_indices} and \eqref{cota-sup-indices}, we get
\[
\beta_{\Phi}=\lim\limits_{t \to \infty} \frac{\log\left[\sup\limits_{u>0} \frac{\Phi(tu)}{\Phi(u)}\right]}{\log t}
\leq
\lim \limits_{t \to \infty} \frac{\log C+p\log t}{\log t}=p.
\]
On the other hand, employing \eqref{MO_indices} and performing some elementary calculations, we obtain
\[
\alpha_{\Phi}=
\lim\limits_{t \to 0^+} \frac{\log\left[\sup\limits_{u>0} \frac{\Phi(tu)}{\Phi(u)}\right]}{\log t}=
\lim\limits_{s \to \infty} \frac{\log\left[\sup\limits_{v>0} \frac{\Phi(v)}{\Phi(sv)}\right]^{-1}}{\log s}=
\lim\limits_{s \to \infty} \frac{\log\left[\inf\limits_{v>0} \frac{\Phi(sv)}{\Phi(v)}\right]}{\log s}
\]
where $v:=tu$ and $s:=\frac{1}{t}$.
Then, using \eqref{cota-sup-indices},  for every $\varepsilon>0$ we have
\[
\alpha_{\Phi}=
\lim\limits_{s \to \infty} \frac{\log\left[\inf\limits_{v>0} \frac{\Phi(sv)}{\Phi(v)}\right]}{\log s}\geq
\lim\limits_{s \to \infty} \frac{\log C+(p-\varepsilon)\log s}{\log s}\geq p-\varepsilon,
\]
therefore $\alpha_{\Phi}\geq p$.

Finally, as $\alpha_{\Phi}\leq \beta_{\Phi}\leq p$, we get
$\alpha_{\Phi}=\beta_{\Phi}=p$.
\end{proof}



Now, we are able to see that
\[
\rho_{\Phi}(u)=\int_0^T \Phi(|u|)\,dx\geq C\|u\orlnor^{\alpha_{\Phi}}=C\|u\orlnor^p
\]
is false.

In fact, if we take $u\equiv t>0$, then $\|u\orlnor^p=C_1t^p$ where $C_1=\|1\orlnor$ and
$\int_0^T \Phi(|u|)\,dx=C_2\Phi(t)$ with $C_2=T$.
Then, if $\rho_{\Phi}(u)\geq C\|u\orlnor^p$ were true, then $\Phi(t)\geq C t^p$ would also be true; however, this
last inequality is false.
 

\section*{Acknowledgments}
The authors are partially supported by a UNRC grant number 18/C417. The first author is  partially supported by a  UNSL grant number 22/F223. 


 \bibliographystyle{elsarticle-num} 
 \bibliography{biblio}


\end{document}
