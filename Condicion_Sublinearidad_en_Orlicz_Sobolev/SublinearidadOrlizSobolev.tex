\documentclass[twoside]{article}


\NeedsTeXFormat{LaTeX2e}
\ProvidesPackage{mathscinet}[2002/04/17 v1.05]
\RequirePackage{textcmds}\relax
\ProvideTextCommandDefault{\cprime}{\tprime}



%\usepackage{hyperref}
\usepackage{amssymb,amsthm}
\usepackage{amsmath}
\usepackage{color}
\usepackage{ esint }
\usepackage{mathabx}
\usepackage{fancyhdr}
\usepackage{times}

\usepackage[latin1]{inputenc}

\usepackage{comment}
\usepackage{url}
\usepackage{xcolor}
\usepackage{adjustbox}
\usepackage{hyperref}

\newtheorem{thm}{Theorem}[section]
\newtheorem{cor}[thm]{Corollary}
\newtheorem{lem}[thm]{Lemma}

\newtheorem{defi}[thm]{Definition}
\newtheorem{prop}[thm]{Proposition}
\theoremstyle{remark}
\newtheorem{comentario}{Remark}



\title{Periodic solutions of 
Euler-Lagrange equations with ``sublinear nonlinearity'' in an Orlicz-Sobolev space setting}
\author{Sonia Acinas \thanks{SECyT-UNRC, UNSL and CONICET}\\
Instituto de Matem\'atica Aplicada San Luis (IMASL)\\ 
Universidad Nacional de San Luis and CONICET\\
Ej\'ercito de los Andes 950,
(D5700HDW) San Luis, Argentina\\
Universidad Nacional de La Pampa\\
(L6300CLB) Santa Rosa, La Pampa, Argentina\\
\url{sonia.acinas@gmail.com}\\[3mm]
Fernando D. Mazzone \thanks{SECyT-UNRC and CONICET}\\
Dpto. de Matem\'atica, Facultad de Ciencias Exactas, F\'{\i}sico-Qu\'{\i}micas y Naturales\\
Universidad Nacional de R\'{i}o Cuarto\\
(5800) R\'{\i}o Cuarto, C\'ordoba, Argentina,\\
\url{fmazzone@exa.unrc.edu.ar}
}

\date{}

\newcommand{\orlnor}{\|_{L^{\Phi}}}
\newcommand{\lurnor}{\|^{*}_{L^{\Phi}}}
\newcommand{\linf}{\|_{L^{\infty}}}
\newcommand{\lphi}{L^{\Phi}}
\newcommand{\lpsi}{L^{\Psi}}
\newcommand{\ephi}{E^{\Phi}}
\newcommand{\claseor}{C^{\Phi}}
\newcommand{\wphi}{W^{1}\lphi}
\newcommand{\wphiet}{W^{1}\ephi_T}
\newcommand{\wphie}{W^{1}\ephi}
\newcommand{\sobnor}{\|_{W^{1}\lphi}}
\newcommand{\domi}{\mathcal{E}^{\Phi}_d(\lambda)}
\renewcommand{\b}[1]{\boldsymbol{#1}}
\newcommand{\rr}{\mathbb{R}}
\newcommand{\nn}{\mathbb{N}}
\newcommand{\ccdot}{\b{\cdot}}
\renewcommand{\leq}{\leqslant} 
\renewcommand{\geq}{\geqslant} 
\newcommand{\epsi}{E^{\Psi}}

\begin{document}



\maketitle
%
\begingroup%Locallizing the change to `thefootnote'.
    \renewcommand{\thefootnote}{}%Removing the footnote symbol.
    %
    \footnotetext{%
    %   2010 Mathematics Subject Classification
    %   http://www.ams.org/msc/
    \textbf{2010  AMS Subject Classification.} Primary: .
    Secondary: .
    }%
        \footnotetext{%
    \textbf{Keywords and phrases.}  .
    }%
    \endgroup
%
%
%
%

\begin{abstract}

In this paper we obtain existence of periodic solutions, in the Orlicz-Sobolev space $\wphi([0,T])$, of hamiltonian systems with a potential  function $F$ satisfying the inequality  $|\nabla F(t,x)|\leq b_1(t) \Phi_0'(|x|)+b_2(t)$, with    $b_1, b_2\in L^1$ and for certain $N$-functions $\Phi_0$.

\end{abstract}




\pagestyle{fancy} \headheight 35pt \fancyhead{} \fancyfoot{}

\fancyfoot[C]{\thepage} \fancyhead[CE]{\nouppercase{S. Acinas and F.D. Mazzone }} \fancyhead[CO]{\nouppercase{\section}}

\fancyhead[CO]{\nouppercase{\leftmark}}


%\tableofcontents




\section{Introduction}
The purpose of this paper is to study the existence  of periodic solution for the
following non-autonomous second-order system:

\begin{equation}\label{ProbPrin}
    \left\{%
\begin{array}{ll}
   \frac{d}{dt}\left(u'(t)\frac{\Phi'(|u'|)}{|u'|}\right) = \nabla F(t,u(t)) \quad \hbox{a.e.}\ t \in (0,T)\\
    u(0)-u(T)=u'(0)-u'(T)=0
\end{array}%
\right.
\end{equation}
where $T>0$, $u:[0,T]\to\rr^d$ is absolutely continuous and  $\Phi$ is a differentiable  $N$-function (see section Preliminaries for definitions). Furthermore, the \emph{potential} $F:[0,T]\times\rr^d\to\rr$  satisfies the following conditions:
\begin{description}

 \item[(C)]\label{item:condicion_c} $F$ and its gradient $\nabla F$ are  Carath\'eodory functions, i.e. they are measurable functions with respect to $t\in [0,T]$, for every  $x\in\rr^d$, and   continuous functions with  respect to  $x\in\rr^d$ for a.e. $t \in [0,T]$.

 \item[(A)]\label{item:condicion_a}  For   a.e. $t\in [0,T]$, it holds that
\begin{equation}
|F(t,x)| + |\nabla F(t,x)|  \leq a(|x|)b(t).
\end{equation}
In this inequality we assume that the function  $a:[0,+\infty)\to [0,+\infty)$ is continuous and nondecreasing and $0\leq b\in L^1([0,T],\rr)$.


\end{description}

We will call the differential operator
\[L_{\Phi}[u]=\frac{d}{dt}\left(u'(t)\frac{\Phi'(|u'|)}{|u'|}\right) \]
the
\emph{$\Phi$-laplacian operator}. If $\Phi(x)=|x|^p/p$, $1<p<\infty$, $L_{\Phi}$ is the well known $p$-laplacian operator. In this case, we have the \emph{Dirichlet problem} for the $p$-laplacian

\begin{equation}\label{ProbP-lapla}
    \left\{%
\begin{array}{ll}
   \frac{d}{dt}\left(u'(t)|u'|^{p-2}\right) = \nabla F(t,u(t)) \quad \hbox{a.e.}\ t \in (0,T)\\
    u(0)-u(T)=u'(0)-u'(T)=0
\end{array}%
\right.
\end{equation}



The problem \eqref{ProbPrin} comes from a variational one, that is,  the equation in  \eqref{ProbPrin}  is the Euler-Lagrange equation associated to the \emph{action integral}
\begin{equation}\label{integral_accion}
I(u)=\int_{0}^T \Phi(|u'(t)|)+F(t,u(t))\ dt.
\end{equation}

PARA MEJORAR Y AMPLIAR!!!

The main result of this article is Theorem \ref{coercitividad-r} which establishes conditions to guarantee existence of 
solutions of the problem \eqref{ProbPrin} by minimization of functional \eqref{integral_accion}. 
We point out that the hypothesis of
Theorem \ref{coercitividad-r} are generalizations of those given in 
\cite{wu1999periodic,zhao2004periodic,zhao2005existence,tang2010periodic} about the sublinearity.


\section{Preliminaries}\label{preliminares}

For reader convenience, we give a short introduction to Orlicz and Orlicz-Sobolev spaces of vector valued functions and a  list  of results that we will use throughout the article. 
Classic references for Orlicz spaces of real valued functions are \cite{adams_sobolev,KR,rao1991theory}.
For  Orlicz spaces of vector valued functions, see \cite{Orliczvectorial2005} and the references therein.

Hereafter we denote  by $\mathbb{R}^+$  the set of all non negative real numbers. A function $\Phi:\mathbb{R}^+\to \mathbb{R}^+ $ is called an \emph{$N$-function} if $\Phi$ is convex and satisfies that
\[
\lim_{t\to+\infty}\frac{\Phi(t)}{t}=+\infty\quad\text{and}\quad \lim_{t\to 0}\frac{\Phi(t)}{t}=0
\]
In addition,  in this paper  we assume that $\Phi$ is differentiable and we call $\varphi$  the derivative of $\Phi$. 
On these assumptions, $\varphi:\mathbb{R}^+\rightarrow \mathbb{R}^+$ is a homeomorphism whose inverse is $\psi$. 
We denote by $\Psi$ the primitive of $\psi$ that satisfies $\Psi(0)=0$. Then, $\Psi$ is an $N$-function which  is called the \emph{complementary function} of $\Phi$.


There exist several order relations between $N$-functions (see \cite[Sec. 2.2]{rao1991theory}). 
Following \cite[Def. 1, p. 15]{rao1991theory} we say that the $N$-function $\Phi_2$ is \emph{essentially stronger} than the $N$-function  $\Phi_1$  ($\Phi_1\llcurly\Phi_2$) if and only if there exists $x_0\geq 0$ such that $\Phi_1(x)\leq \Phi_2(ax)$, for every $a>0$ and $x\geq x_0$.



We also say that a function $\eta:\mathbb{R}^+\rightarrow \mathbb{R}^+$ satisfies the  \emph{$\Delta_2$-condition}, denoted by $\eta \in \Delta_2$,
if there exist  constants $K>0$ and  $t_0\geq 0$ such that 
\begin{equation}\label{delta2defi}\eta(2t)\leq K\eta(t),
\end{equation}
for every $t\geq t_0$. 
If $t_0=0$,  a function   $\eta:\mathbb{R}^+\rightarrow \mathbb{R}^+$ is said to satisfy the \emph{$\Delta_2$-condition globally} ($\eta \in \Delta_2$ globally).


Let $d$ be a positive integer. We denote by $\mathcal{M}:=\mathcal{M}([0,T],\rr^d)$ the set of all measurable functions defined on $[0,T]$ with values on $\mathbb{R}^d$ and  we write $u=(u_1,\dots,u_d)$ for  $u\in \mathcal{M}$.



Given  an $N$-function $\Phi$ we define the \emph{modular function} 
$\rho_{\Phi}:\mathcal{M}\to \mathbb{R}^+\cup\{+\infty\}$ by
\[\rho_{\Phi}(u):= \int_0^T \Phi(|u|)\ dt.\]
Here $|\cdot|$ is the euclidean norm of $\mathbb{R}^d$.
The \emph{Orlicz class} $C^{\Phi}=C^{\Phi}([0,T],\rr^d)$  is defined  by
\begin{equation}\label{claseOrlicz}
  C^{\Phi}_d:=\left\{u\in \mathcal{M} | \rho_{\Phi}(u)< \infty \right\}.
\end{equation}
The \emph{Orlicz space} $\lphi=L^{\Phi}([0,T],\rr^d)$ is the linear hull of $\claseor$;
equivalently,
\begin{equation}\label{espacioOrlicz}
\lphi:=\left\{ u\in \mathcal{M}| \exists \lambda>0: \rho_{\Phi}(\lambda u) < \infty   \right\}.
\end{equation}
  The Orlicz space $\lphi$ equipped with the \emph{Orlicz norm}
\[
\|  u  \orlnor:=\sup \left\{  \int_0^T u\b{\cdot} v\ dt \big| \rho_{\Psi}(v)\leq 1\right\},
\]
is a Banach space. By $u\b{\cdot} v$ we denote the usual dot product in $\mathbb{R}^{d}$ between $u$ and $v$.
The following alternative expression for the norm, known as \emph{Amemiya norm},     will  be useful (see \cite[Thm. 10.5]{KR} and \cite{hudzik2000amemiya}). For every $u\in\lphi$,

\begin{equation}\label{amemiya}
\|u\orlnor=\inf\limits_{k>0}\frac{1}{k}\left\{1+\rho_{\Phi}(ku)\right\}.
\end{equation}
In particular
\begin{equation}\label{amemiya-ine}
\|u\orlnor\leq \frac{1}{k}\left\{1+\rho_{\Phi}(ku)\right\},\quad\text{for every } k>0.
\end{equation}


The subspace $\ephi=\ephi([0,T],\rr^d)$ is defined as the closure in $\lphi$ of the subspace $L^{\infty}_d([0,T],\rr^d)$ of all $\mathbb{R}^d$-valued essentially bounded functions. It is shown that  $\ephi$ is the only one maximal subspace contained in the Orlicz class $\claseor$, i.e.
$u\in\ephi$ if and only if $\rho_{\Phi}(\lambda u)<\infty$ for any $\lambda>0$.

A generalized version of \emph{H\"older's inequality} holds in Orlicz spaces (see \cite[Thm. 9.3]{KR}). Namely, if $u\in\lphi$ and $v\in\lpsi$ then $u\cdot v\in L^1$ and
\begin{equation}\label{holder}
\int_0^Tv\cdot u\ dt\leq \|u\orlnor\|v\|_{L^{\Psi}}.
\end{equation}




If $X$ and $Y$ are  Banach spaces such that  $Y\subset X^*$, we denote by $\langle\cdot,\cdot\rangle:Y\times X\to\mathbb{R}$ the bilinear pairing  map given by $\langle x^*,x\rangle=x^*(x)$. H\"older's inequality shows that $\lpsi\subset \left[\lphi\right]^*$, where the pairing
$\langle v, u\rangle$
is defined by 
\begin{equation}\label{pairing}
  \langle v,u\rangle=\int_0^Tv\cdot u\ dt,
\end{equation}
with  $u\in\lphi$ and $v\in\lpsi$.
 Unless $\Phi \in \Delta_2$, the relation $\lpsi= \left[\lphi\right]^*$ will not hold. In general, it is true  that  $\left[\ephi\right]^*=\lpsi$.



We define the \emph{Sobolev-Orlicz space} $\wphi$ (see \cite{adams_sobolev}) by
\[\wphi:=\{u| u \hbox{ is absolutely continuous on $[0,T]$ and } u'\in \lphi\}.\]
$\wphi$ is a Banach space when equipped with the norm
\begin{equation}\label{def-norma-orlicz-sob}
\|  u  \|_{\wphi}= \|  u  \|_{\lphi} + \|u'\orlnor.
\end{equation}

Moreover, we introduce the following subspaces of $\wphi$
%%
\begin{equation}\label{def-esp-orlicz-sob-per}
\begin{split}
\wphie&=\{u\in\wphi|u'\in\ephi\},\\
\wphie_T&=\{u\in\wphie|u(0)=u(T)\}.
\end{split}
\end{equation}



For a  function $u\in L^1_d([0,T])$, we write $u=\overline{u}+\widetilde{u}$ where $\overline{u} =\frac1T\int_0^T u(t)\ dt$ and $\widetilde{u}=u-\overline{u}$.

As usual, if $(X,\|\cdot\|_X)$ is a Banach space and $(Y,\|\cdot \|_Y)$ is a subspace of $X$,  we write $Y\hookrightarrow X$ and we say that $Y$ is \emph{embedded} in $X$  when the restricted identity map $i_Y:Y\to X$ is bounded. That is, there exists $C>0$ such that  for any $y\in Y$ we have $\|y\|_X\leq C\|y\|_Y$.  With this notation, H\"older's inequality states that  $\lpsi\hookrightarrow  \left[\lphi\right]^*$; and, it is easy to see that for every $N$-function $\Phi$ we have that $L^{\infty}_d\hookrightarrow\lphi \hookrightarrow L^1_d$.


 Recall that a function   $w:\mathbb{R}^+\to \mathbb{R}^+$ is called  a \emph{modulus of continuity} if $w$ is a continuous increasing function which satisfies $w(0)=0$. For example, it can be easily shown that $w(s)=s\Phi^{-1}(1/s)$ is a modulus of  continuity for every $N$-function $\Phi$.  We say that $u:[0,T]\to\rr^d$  has modulus of continuity $w$  when there exists a constant $C>0$ such that
\begin{equation}\label{w-holder}|u(t)-u(s)|\leq Cw(|t-s|).
\end{equation}


We denote by $C^w([0,T],\rr^d)$  the space of  $w$-H\"older continuous functions. This is the space of all functions satisfying \eqref{w-holder} for some $C>0$ and it is a Banach space with norm
\[\|u\|_{  C^w([0,T],\rr^d) }  :=\|u\|_{L^{\infty}}+\sup\limits_{t\neq s}\frac{|u(t)-u(s)|}{w(|t-s|)}.\]





 An important aspect of the theory of Sobolev spaces is related to embedding theorems. There is an extensive literature on this question in the  Orlicz-Sobolev space setting, see for example
 \cite{cianchi2000fully,cianchi1999some,claverooptimal,edmunds2000optimal,kerman2006optimal}.
The next simple lemma, whose proof can be found in \cite{ABGMS2015}, will be used systematically.




\begin{lem}\label{inclusion orlicz} Let  $w(s):= s\Phi^{-1}(1/s)$. Then, the following statements hold:
\begin{enumerate}
\item\label{inclusion orlicz_item1} $\wphi\hookrightarrow C^w([0,T],\rr^d) $ and for every $u\in\wphi$
\begin{align}
 &\left|u(t)-u(s) \right| \leq  \|u'\orlnor w(| t-s|)&\text{  (Morrey's inequality),}\label{in-sob-cont}
\\
& \|u\|_{L^{\infty}} \leq\Phi^{-1}\left(\frac{1}{T}\right)\max\{1,T\}\|u\sobnor&\text{  (Sobolev's inequality).}\label{sobolev}
\end{align}
\item For every $u\in\wphi$ we have $\widetilde{u}\in L^{\infty}_d$ and
\begin{align}
& \|\widetilde{u}\|_{L^{\infty}} \leq T\Phi^{-1}\left(\frac{1}{T}\right)\|u'\orlnor&
\text{  (Sobolev-Wirtinger's inequality).}\label{wirtinger}
\end{align}




\end{enumerate}
\end{lem}






\section{Lagrangians satisfying  sublinear nonlinearity type conditions}



\begin{lem}\label{lem:deb*cerrado}
$\ephi$ is weak* closed in $\lphi$.
\end{lem}


\begin{proof}
From \cite[Thm. 7, p. 110]{rao1991theory} we have that $\lphi=\left[\epsi\right]^*
$.
Then, $\lphi$ is a dual and therefore we are allowed to speak about the weak* topology of $\lphi$.
Besides, $\ephi
$ is separable (see \cite[Thm. 1, p. 87]{rao1991theory}).
Let $S=\ephi\cap \{u \in \lphi|\|u\orlnor\leq 1\}$, then $S$ is closed in the norm $\|\cdot\orlnor$. 
Now, according to \cite[Cor. 5, p. 148]{rao1991theory} $S$ is weak* sequentially compact. 
Thus, $S$ is weak* sequentially closed because if $u_n\in S$ and
$u_n \overset{*}{\rightharpoonup}u \in \lphi$ then  the weak* sequentially compactness implies the existence of $v \in S$ and a subsequence $u_{n_k}$ such that
$u_{n_k}\overset{*}{\rightharpoonup}v$. Finally, by the uniqueness of   the limit, we get
$u=v\in S$.
As $\epsi$ is separable and $\lphi=\left[\epsi\right]^*$, the ball of $\lphi$ $\{u \in \lphi | \|u\orlnor\leq 1\}$ is  weak* metrizable (see \cite[Thm. 5.1, p. 138]{Conway1977}).
Thus, $S$ is closed with respect to  the weak* topology. Now, by  Krein-Smulian theorem, \cite[Cor. 12.6, p. 165]{Conway1977} implies that $\ephi$ is weak* closed.
\end{proof}

The following result is analogous to some lemmata in $W^{1,p}$, see \cite{xu2007some}.
\begin{lem}\label{infinito-a-prom-upunto}
If $\|u\sobnor\to \infty$, then $(|\overline{u}|+\|u'\orlnor)\to \infty$.
\end{lem}

\begin{proof}
By the decomposition $u=\overline{u}+\tilde{u}$ and some elementary operations,
we get
\begin{equation}\label{cota-u-lphi}
\|u\orlnor=
\|\overline{u}+\tilde{u}\orlnor\leq
\|\overline{u}\orlnor+\|\tilde{u}\orlnor=
|\overline{u}|\|1\orlnor+\|\tilde{u}\orlnor.
\end{equation}
It is known that $L^{\infty}_d\hookrightarrow\lphi$, i.e.
there exists $C_1=C_1(T)>0$ such that for any $\tilde{u}\in L^{\infty}_d$ we have
\[
\|\tilde{u}\orlnor
\leq
C_1 \|\tilde{u}\|_{L^{\infty}};
\]
and, applying  Sobolev's inequality,  we obtain Wirtinger's inequality, 
that is there exists $C_2=C_2(T)>0$ such that
\begin{equation}\label{cota-u-tilde}
\|\tilde{u}\orlnor
\leq
C_2\|u'\orlnor.
\end{equation}

Therefore, from \eqref{cota-u-lphi}, \eqref{cota-u-tilde} and \eqref{def-norma-orlicz-sob},
we get
\[
\|u\sobnor\leq
C_3(|\overline{u}|+\|u'\orlnor)
\]
where $C_3=C_3(T)$. Finally, as $\|u\sobnor\to \infty$ we conclude that
$(|\overline{u}|+\|u'\orlnor)\to \infty$.
\end{proof}

\begin{lem}\label{lem:submultipliativa}
Let $\Phi,\Psi$ be complementary functions.
The next statements are equivalent:
\begin{enumerate}
\item\label{item1} $\Psi \in \Delta_2$ globally.
\item\label{item2} There exists an $N$-function $\Phi_1$ such that
\begin{equation}\label{eq:caract_delta2}
\Phi(rs)\geq \Phi_1(r)\Phi(s)\;\;\mbox{for every}\;\;r\geq1,\;\;s\geq 0.
\end{equation}
\end{enumerate}
\end{lem}

\begin{proof}
\ref{item1})$\Rightarrow$\ref{item2})
By virtue of the $\Delta_2$-condition on $\Psi$, \cite[Thm. 11.7]{M} and \cite[Cor. 11.6]{M} (see also  \cite[Eq. (2.8)]{AGMS}), we get constants $K>0$ and $\alpha_{\Phi}>1$ such that
\begin{equation}\label{delta2-consecuencia}
\Phi(r s)\geq Kr^{\nu}\Phi(s),
\end{equation}
for any $1<\nu<\alpha_{\Phi}$,  $s\geq 0$ and $r>1$. This proves  \eqref{eq:caract_delta2} with $\Phi_1(r)=kr^\nu$, which is an $N$-function.

\ref{item2})$\Rightarrow$\ref{item1})
Next, we follow  \cite[p. 32, Prop. 13]{rao1991theory} and \cite[p. 29, Prop. 9]{rao1991theory}.
Assume that 
\[
\Phi_1(r)\Phi(s)\leq \Phi(rs)\;\;r>1,\;s\geq 0.
\]
Let $u=\Phi_1(r)\geq \Phi_1(1)$ and $v=\Phi(s)\geq 0$. By a well known inequality \cite[p. 13, Prop. 1]{rao1991theory} and \eqref{eq:caract_delta2},  we have  for $u\geq \Phi_1(1)$ and $v> 0$
\[
\frac{uv}{\Psi^{-1}(uv)}\leq \Phi^{-1}(uv)\leq\Phi_1^{-1}(u)\Phi^{-1}(v)\leq
\frac{4uv}{\Psi_1^{-1}(u)\Psi^{-1}(v)},
\]
then 
\[
\Psi^{-1}_1(u)\Psi^{-1}(v)\leq 4 \Psi^{-1}(uv).
\]
If we take $x=\Psi^{-1}_1(u)\geq \Psi^{-1}_1(\Phi_1(1))$ and $y=\Psi^{-1}(v)\geq 0$, then 
\[
\Psi\left(\frac{xy}{4}\right)\leq \Psi_1(x)\Psi(y).
\]
Now, taking  $x\geq \max\{8,\Psi_1^{-1}(\Phi_1(1))\}$ we get that $\Psi \in \Delta_2$ globally.
\end{proof}

The following lemma generalizes \cite[Lemma 5.2]{ABGMS2015}.

\begin{lem}\label{lem_coer}
Let $\Phi,\Psi$ be complementary $N$-functions with $\Psi \in \Delta_2$ globally. Let $\Phi_1$ be any $N$-function satisfying \eqref{eq:caract_delta2}. Then
\begin{equation}\label{eq:coer_mod}
\lim\limits_{\|u\orlnor\to \infty}
\frac{\int_0^T \Phi(|u|)\,dt}{\Phi_0(k\|u\orlnor)}=\infty,
\end{equation}


for every $\Phi_0$ with $\Phi_0\llcurly\Phi_1$ and $k>0$. If  \eqref{eq:coer_mod} holds for some $N$-function $\Phi_0$,  then $\Psi\in\Delta_2$ (at $\infty$).
\end{lem}

\begin{proof}
By the assumptions on $\Phi$ and $\Phi_1$  and  inequality \eqref{amemiya-ine}, for $r>1$ we have 
\[
\int_0^T \Phi(|u|)\,dt\geq
\Phi_1(r) \int_0^T \Phi(r^{-1}|u|)\,dt\geq
\Phi_1(r)\{r^{-1}\|u\orlnor-1\}.
\]
Now, we choose $r=\frac{\|u\orlnor}{2}$ and as $\|u\orlnor\to\infty$ we can assume $r>1$ and by \cite[Thm. 2 (b)(v), p. 16]{rao1991theory}.
\[
\lim\limits_{\|u\orlnor \to \infty} \frac{\int_0^T \Phi(|u|)\,dt}{\Phi_0(k\|u\orlnor)}\geq
\lim\limits_{\|u\orlnor \to \infty} \frac{\Phi_1\left(\frac{\|u\orlnor}{2}\right)}{\Phi_0(k\|u\orlnor)}
=\infty.
\]




Finally, if $\Phi_0$ is an $N$-function, then $\Phi_0(x)\geq \alpha |x|$ for  $\alpha$ small enough and $|x|>1$.
Therefore \eqref{eq:coer_mod} holds for $\Phi_0(x)=|x|$, then \cite[Lemma 5.2]{ABGMS2015}  implies  $\Psi\in\Delta_2$ at $\infty$.
\end{proof}


\begin{comentario}  We point out that this lemma can be applied to more cases than \cite[Lemma 5.2]{ABGMS2015}. For example, if $\Phi(u)=u^2$, $\Phi_1$ and $\Phi_0$ are  $N$-functions with principal parts equal to $u^2/\log u$ and $u^2/(\log u)^2$ respectively (see \cite[p. 16]{KR} and \cite[Sec. 7]{KR} for the definition and properties of principal part), then  \eqref{eq:coer_mod} holds for $\Phi_0$.
However, $\Phi_0(u)$ is not dominated for any  power function $|u|^{\alpha}$ for every $\alpha<2$. 
\end{comentario}






\begin{defi}We define the  functionals $J_{C,\varphi}:\lphi\to (-\infty,+\infty]$ and $  H_{C,\varphi}:\rr^n\to \rr$, where $C>0$ and $\varphi:[0,+\infty)\to [0,+\infty)$ is an by
\begin{equation}\label{func_phi}
  J_{C,\varphi}(u):= \rho_{\Phi}\left(u\right)-C\varphi\left(\|u\orlnor\right),
\end{equation}
 and

\begin{equation}\label{eq:functional_H-bis}
 H_{C,\varphi}(x):=\int_0^TF(t,x)dt-C\varphi(2|x|),
\end{equation}
respectively.
\end{defi}

In \cite{tang1998periodic} and \cite{tang2010periodic} the authors  considered, for the $p$-laplacian case, potentials $F$ satisfying the inequality
\begin{equation}\label{eq:cota_pot} |\nabla F(t,x)|\leq b_1(t)|x|^{\alpha}+b_2(t),
 \end{equation}


where  $b_1,b_2 \in L^1_1$ and $\alpha<p$. Thus, they called $F$  a sublinear nonlinearity. 
In this paper, we will consider bounds on $\nabla F$ of a more general type.

\begin{defi} Let $\Phi_0$ be a differentiable $N$-function. We say that $G:[0,T]\times\rr^n\to\rr$  satisfies a $\Phi_0$-\emph{grow condition} if
\begin{equation}\label{holder_cont-mu}
  \left| G(t,x) \right|\leq b_1(t)\Phi_0'(|x|)+b_2(t),
\end{equation}
with $b_1,b_2 \in L^1([0,T],\rr)$.

\end{defi}



\begin{thm}\label{coercitividad-r} Let $\Phi$ be an $N$-function whose complementary function $\Psi$ satisfies the $\Delta_2$ condition globally. 
Assume that the $N$-function $\Phi_1$ satisfies \eqref{eq:caract_delta2}, $F$ satisfies (C) and
(A), and  $\nabla F$ satisfies a $\Phi_0$-grow condition for some 
 $N$-function  $\Phi_0$ such that $\Phi_0\llcurly\Phi_1$.
Furthermore, we suppose that
\begin{equation}\label{eq:propiedad-coercividad-phi0}
\lim_{|x|\to\infty}\frac{\int_{0}^{T}F(t,x)\ dt}{\Psi_2(\Phi_0'(2|x|))}=+\infty.
\end{equation}
for some $N$-function $\Psi_2$ whose complementary function $\Phi_2$ satisfies $\Phi_0\llcurly \Phi_2\llcurly \Phi_1$.
Then,  the problem \eqref{ProbPrin} has at least a solution which minimizes the action integral $I$ on $\wphie_T$.
\end{thm}

\begin{proof}
By the decomposition $u=\overline{u}+\tilde{u}$,   Cauchy-Schwarz's inequality
and \eqref{holder_cont-mu}, we have
\begin{equation}\label{cota-diferencia-F}
\begin{split}
&\left|\int_0^T F(t,u)-F(t,\overline{u})\,dt\right|=
\left|\int_0^T \int_0^1 \nabla F(t,\overline{u}+s\tilde{u}(t))\ccdot \tilde{u}(t) \,ds \,dt\right|
\\
&\leq \int_0^T \int_0^1 b_1(t)\Phi_0'(|\overline{u}+s\tilde{u}(t)|)|\tilde{u}(t)|\,ds\,dt+
\int_0^T \int_0^1 b_2(t)|\tilde{u}(t)|\,ds\,dt
\\
&=:I_1+I_2.
\end{split}
\end{equation}
On the one hand, by H\"older's and Sobolev-Wirtinger's inequalities we estimate $I_2$ as follows
\begin{equation}\label{cota-i2}
I_2\leq \|b_2\|_{L^1} \|\tilde{u}\|_{L^{\infty}}\leq
C_1\|u'\orlnor,
\end{equation}
 where $C_1=C_1(\|b_2\|_{L^1}, T)$. 

We note that, since $\Phi_0'$ is increasing function  and $\Phi_0'(x)\geq 0$ for $x\geq 0$, then $\Phi'_0(a+b)\leq \Phi'_0(2a)+\Phi'_0(2b)$ for every $a,b\geq 0$.
In this way, we have
\begin{equation}\label{pot-suma}
\Phi'_0(|\overline{u}+s\tilde{u}(t)|)\leq
\Phi'_0(2|\overline{u}|)+\Phi'_0(2\|\tilde{u}\|_{L^{\infty}}),
\end{equation}
for every $s \in [0,1]$.  Now,  inequality \eqref{pot-suma}, H\"older's and Sobolev-Wirtinger's inequalities imply that
\begin{equation}\label{cota-i1}
\begin{split}
I_1&
\leq \Phi'_0(2|\overline{u}|) \|b_1\|_{L^1} \|\tilde{u}\|_{L^{\infty}}+\Phi'_0(2\|\tilde{u}\|_{L^\infty})
 \|b_1\|_{L^1}\|\tilde{u}\|_{L^\infty}
\\
&\leq C_2 \bigg\{ \Phi'_0(2|\overline{u}|) \|u'\orlnor
+\Phi'_0(C_3\|u'\orlnor) \|u'\orlnor\bigg\},
\end{split}
\end{equation}
where $C_2=C_2(T, \|b_1\|_{L^1} )$ and $C_3=C_3(T)$. Next, by Young's inequality with complementary functions $\Phi_2$ and $\Psi_2$
\begin{equation}\label{cota-i1-parcial}
 \begin{split}
\Phi_0'(2|\overline{u}|) \|u'\orlnor
&\leq 
\Psi_2(\Phi_0'(2|\overline{u}|))+
\Phi_2(\|u'\orlnor).
\end{split}
\end{equation}
We have that any $N$-function $\Phi_0$ satisfies the inequality $x\Phi_0'(x)\leq \Phi_0(2x)$ (see \cite[p. 17]{rao1991theory} ). Moreover, since $\Phi_0\llcurly\Phi_2$ there exists $x_0=x_0(\Phi_0,\Phi_2,T)\geq 0$ such that $\Phi_0(2C_3x)\leq \Phi_2(x)$, for every $x\geq x_0$. Therefore, $\Phi_0(2C_3x)\leq \Phi_2(x)+C_4$, with $C_4=\Phi_0(2x_0)$. The previous observations imply
\begin{equation}\label{cota-i1-parcial-segunda}
\Phi_0'(C_3\|u'\orlnor) \|u'\orlnor
\leq 
C_3^{-1}(\Phi_2(\|u'\orlnor)+C_4).
\end{equation}
From \eqref{cota-i1}, \eqref{cota-i1-parcial}, \eqref{cota-i1-parcial-segunda} and \eqref{cota-i2}, we have
\begin{equation}\label{cota-i1-i2}
\begin{split}
I_1+I_2
&
\leq C_5
\bigg\{ 
\Psi_2(\Phi_0'(2|\overline{u}|))
+\Phi_2(\|u'\orlnor)
+\|u'\orlnor +1
\bigg\}\\
\end{split}
\end{equation}
with $C_5$ depending on $\Phi_0, \Phi_2, T, \|b_1\|_{L^1}$ and $\|b_2\|_{L^1} $.

In the subsequent estimates, we use  \eqref{cota-diferencia-F},
\eqref{cota-i1-i2}, we get
\begin{equation}\label{cota_inf_I}
\begin{split}
I(u)&
=\rho_{\Phi}(u')+\int_0^TF(t,u)\ dt
\\ 
&=\rho_{\Phi}( u')+ \int_0^T \left[F(t,u)-F(t,\overline{u})\right]\ dt
+  \int_0^TF(t,\overline{u})\ dt
\\
&\geq \rho_{\Phi}( u')
-C_5 \Phi_2(\|u'\orlnor)
+\int_0^TF(t,\overline{u})\ dt-
C_5 \Psi_2(\Phi_0'(2|\overline{u}|))-
C_5
\\
&\geq 
\rho_{\Phi}( u')
-C_5 \Phi_2(\|u'\orlnor)
+H_{C_5, \Psi_2\circ\Phi_0}(\overline{u})
-C_5
\\
&=
J_{C_5,\Phi_0}(u')
+H_{C_5, \Psi_2\circ\Phi_0}(\overline{u})
-C_5.
\end{split}
\end{equation}



Let $u_n$ be  a sequence in $\wphi$ with
$\|u_n\sobnor\to\infty$ and we have to prove that $I(u_n)\to\infty$.
On the contrary, suppose  that for a subsequence,
still denoted by $u_n$, $I(u_n)$ is upper bounded, i.e. there exists $M>0$ such that $|I(u_{n})|\leq M$.
As $\|u_n\sobnor\to\infty$, from Lemma \ref{infinito-a-prom-upunto},  we have $|\overline{u}_n|+\|u'_n\orlnor\to \infty$. Passing to a subsequence is necessary, still denoted $u_n$, we can assume that $|\overline{u}_n|\to \infty$ or $\|u'_n\orlnor\to \infty$.
Now, Lemma \ref{lem_coer} implies that the functional $J_{C_5,\Phi_0}(u')$ is coercive;
and, by \eqref{eq:propiedad-coercividad-phi0},
the functional $H_{C_5,\Phi_0}(\overline{u})$ is also coercive, then
$J_{C_5,\Phi_0}(u'_n) \to \infty$ or $H_{C_5,\Phi_0}(\overline{u}_n)\to \infty$.
From the condition (A) on $F$, we have that on a bounded set the functional $H_{C_5,\Phi_0}(\overline{u}_n)$ is lower bounded and
also $J_{C_5,\Phi_0}(u'_n)\geq 0$.
Therefore,  $I(u_n)\to\infty$ as $\|u_n\sobnor\to\infty$ which contradicts the initial assumption on the behavior of $I(u_n)$.

Let $\{u_n\}\subset \wphiet$  be a  minimizing sequence for the problem  $\inf\{I(u)|u\in\wphiet\}$.
Since  $I(u_n)$, $n=1,2,\ldots$,  is upper bounded, the previous part of the proof shows that $\{u_n\}$ is norm bounded in $\wphie$. Hence, by virtue of  \cite[Cor. 2.2]{ABGMS2015}, we can assume, taking a subsequence if necessary, that $u_n$ converges uniformly to a $T$-periodic continuous (therefore in $\ephi_T$)  function $u$. As $u'_n \in \ephi$ is a norm bounded sequence in $\lphi$,
there exists a subsequence (again denoted by $u'_n$) such that $u'_n$ converges to a function $v\in\lphi$ in the weak* topology of $\lphi$.
Since $\ephi$ is weak* closed, by Lemma \ref{lem:deb*cerrado}, $v\in \ephi$. From this fact and the uniform convergence of $u_n$ to $u$, we obtain that
\[
\int_0^T\xi'\cdot u\ dt=\lim_{n\to\infty}\int_0^T\xi'\cdot u_n \ dt=
-\lim_{n\to\infty}\int_0^T\xi\cdot u'_n\ dt=-\int_0^T\xi\cdot v\ dt
\]
for every $T$-periodic function $\xi\in C^{\infty}([0,T],\rr^d)\subset\epsi$.
Thus $v=u'$ a.e. $t\in [0,T]$ (see \cite[p. 6]{mawhin2010critical}) and $u\in\wphiet$.

Now, taking into account the relations $\left[L^1\right]^*=L^{\infty}\subset  \epsi$ and $\lphi\subset L^1$, we have that $u'_n$ converges to $u'$ in the weak topology of $L^1$. Consequently,  from the semicontinuity of $I$ (see \cite[Lemma 6.1]{ABGMS2015})  we get 
\[I(u)\leq  \liminf_{n\to\infty}I(u_n)=\inf\limits_{v\in\wphie_T}I(v).\]

Hence $u\in \wphiet$ is a minimun and, since $I$ is G\^ateaux differentiable on $\wphie$ (see  \cite[Thm. 3.2]{ABGMS2015}), 
therefore $I'(u)\in (\wphiet)^{\perp}$. Thus,
\[\int_0^T \frac{\Phi'(|u'(t)|)}{|u'(t)|}u'(t)\cdot v'(t)dt =-\int_0^T \nabla F(t,u(t))\cdot v(t)dt,\]
for every  $v\in \wphiet$.  

From \cite[Lemma 2.4]{ABGMS2015} we have 
 $u'(t)\Phi'(|u'(t)|)/|u'(t)|\in \lpsi([0,T],\rr^n)\hookrightarrow L^1([0,T],\rr^n)$; 
and,  from condition (A) and the fact that $u\in L^{\infty}$, it follows that $\nabla F(t,u(t))\in L^1([0,T],\rr^n)$. 
Consequently, from \cite[p. 6]{mawhin2010critical} we obtain that the differential equations in \eqref{ProbPrin} 
are verified and $u'(0)\Phi'(|u'(0)|)/|u'(0)|=u'(T)\Phi'(|u'(T)|)/|u'(T)|$ holds. Thus $u'(0)=u'(T)$.
\end{proof}



In the article \cite{ABGMS2015} we have considered more general lagrangian functions 
$\mathcal{L}:[0,T]\times\rr^d\times\rr^d\to\rr$ 
satisfying the conditions
\begin{align*}
|\mathcal{L}(t,x,y)| &\leq a(|x|)\left(b(t)+ \Phi\left(\frac{|y|}{\lambda}+f(t) \right)\right)&\text{(A1),}
\\
|D_{x}\mathcal{L}(t,x,y)| &\leq a(|x|)\left(b(t)+ \Phi\left(\frac{|y|}{\lambda}+f(t) \right)\right)&\text{(A2),}
\\
|D_{y}\mathcal{L}(t,x,y)| &\leq a(|x|)\left(c(t)+ \varphi\left(\frac{|y|}{\lambda}+f(t)\right)\right)
&\text{(A3),}
\end{align*}
where $a\in C(\mathbb{R}^+,\mathbb{R}^+)$, $\lambda>0$, $\Phi$ is an $N$-function, 
$\varphi$ is the right continuous derivative of $\Phi$,  
$b\in L^1_1([0,T])$,  $c\in\lpsi_1([0,T])$ and  $f\in \ephi_1([0,T])$.

In  \cite[Thm. 6.2]{ABGMS2015} we obtained existence results of solutions of the problem 
$\inf\{I(u):u \in \wphie \}$ 
where the action integral was given by $I(u)=\int_{0}^T \mathcal{L}(t,u(t),u'(t))\ dt$. 

Unfortunately, we made a mistake in the proof of \cite[Thm. 4.1]{ABGMS2015}, 
because  the minimum of the functional  might be out of the domain of differentiability.

Now, we can fix the aforementioned mistake by the process of minimization developed 
in the last part of the proof of Theorem \ref{coercitividad-r}.

Furthermore, based on Theorem \ref{coercitividad-r},
we can get another existence result under different hypothesis  than those assumed in \cite[Thm. 6.2]{ABGMS2015}, as follows.


\begin{cor}
Let $\Phi,\Psi$ be complementary $N$-functions with $\Psi \in \Delta_2$ globally. 
Suppose that  $\mathcal{L}(t,x,y)$ is a differentiable Carath\'eodory function such that (A1),(A2) and (A3) hold. 
Assume also that $\mathcal{L}(t,x,y)$ is strictly convex at $y$ and 
\begin{equation}
\mathcal{L}(t,x,y) \geq \Phi\left(|y|\right)+ F(t,x),
\end{equation}
where the function $F(t,x)$ satisfies conditions (A) and (C).
Then, the problem
\begin{equation}\label{ProbPrin-gral}
    \left\{%
\begin{array}{ll}
  \frac{d}{dt} D_{y}\mathcal{L}(t,u(t),u'(t))= D_{x}\mathcal{L}(t,u(t),u'(t)) \quad \hbox{a.e.}\ t \in (0,T)\\
    u(0)-u(T)=u'(0)-u'(T)=0
\end{array}%
\right.
\end{equation}
has at least one solution $u:[0,T]\to\rr^d$ absolutely continuous, 
which minimizes the action integral
\begin{equation}\label{integral_accion}
I(u)=\int_{0}^T \mathcal{L}(t,u(t),u'(t))\ dt.
\end{equation}
\end{cor}


\begin{proof}
In Theorem \ref{coercitividad-r} we have just seen that the action integral 
$\int_0^T \Phi(|u'(t)|)+F(t,u(t))\,dt$
is coercive, then the functional $\mathcal L$ does so. 

Let $\{u_n\}\subset \wphiet$ be   a  minimizing sequence for the problem  $\inf\{I(u)|u\in\wphiet\}$.
Now, $\{u_n\}$ is bounded due to the coercivity of $\mathcal{L}$. 

By \cite[Thm. 3.2, Lemma 6.1]{ABGMS2015} we have that $I$ is differentiable on $\wphie$ and lower semi-continuous, respectively.

Next, we find the minimum of $I$ by means of a similar argument to the one developed 
on the last part of the proof of Theorem \ref{coercitividad-r}, 
changing  $u'(t)\Phi'(|u'(t)|)/|u'(t)|$ by $D_{y}\mathcal{L}(t,u(t),u'(t))$  which also belongs to $\lpsi$  
(see \cite[Eq. (26)]{ABGMS2015}). 

Finally, the strict convexity of $\mathcal{L}(t,x,y)$ at $y$ and the injectivity of 
the function $D_y\mathcal{L}(T,u(T),\cdot)$ imply that $u'(0)=u'(T)$.
\end{proof}




 \section{Applications}

 In this section we developed some applications of our main results
so that the reader can appreciate the innovations that bring.

One of the main novelties of our work is that we obtain existence of solutions for lagrangian functions $\mathcal{L}(t,\b{x},\b{y})$ where the  nonlinearity is not necessarily  a power function.
And, in virtue of the fact that we have not supposed that the $N$-functions $\Phi$ satisfy the $\Delta_2$ condition, the nonlinearity is not even bounded by power functions.


We can applied Theorem \ref{coercitividad-r} to Lagrangians $\mathcal{L}=\mathcal{L}(t,x,y)$ with an exponential, or super-exponential grow in $y$ like
  \[\mathcal{L}(t,x,y)=e^{|y|^2}+F(t,x),\]
In this case, considering  the $N$-function $\Phi(x)=e^{x^2}-1$,   whose complementary
function  $\Psi(x)$, satisfies the $\Delta_2$-condition (see \cite[p. 28]{KR}).

when $F$  satisfies some of the alternatives 1-4 in the main theorem.
In this case, considering  the $N$-function $\Phi(x)=e^x-x-1$,  conditions (2),(3) and (4) hold. Moreover, the complementary
function of $\Phi$, i.e.  $\Psi(x)=(1+x)\ln(1+x)-x$, satisfies the $\Delta_2$-condition. We can apply our main theorem even if $\Psi\notin\Delta_2$, although in this case we need an additional condition on the size of certain constants.



%====================================================================================================


 The employment of  $N$-functions instead of power functions in  inequalities like  \eqref{holder_cont-mu}  will allow us to extend some results of   \cite{tang1998periodic} and \cite{tang2010periodic}  to $\Phi$-laplacian operators with $N$-functions $\Phi$ which grow faster than power functions, for example with an exponential grow.


 Furthermore, we want to emphasize that, even in the case of $p$-laplacian operator \eqref{ProbP-lapla}, our results  extend  previous ones (see \cite{tang1998periodic, tang2010periodic}), because
 we get bounds that may be  sharper than those in \cite{tang1998periodic,tang2010periodic}. 
More precisely, in  \cite[Th. 2.1]{tang2010periodic} X. Tang and X. Zhang obtained existences of solutions of \eqref{ProbP-lapla} 
under the assumption \eqref{eq:cota_pot}
 for any $\alpha\in (0,p-1)$. Meanwhile, our Theorem \ref{coercitividad-r} implies existence for the potential

 \[F_0(t,x)=|x|^p/\ln(2+|x|)^2.\]

 We note that this $F$ does not satisfy \eqref{eq:cota_pot} for any $\alpha<p-1$.  
Next we will show  an $N$-function $\Phi_0$ satisfying the hypothesis of Theorem \ref{coercitividad-r} for this potential $F_0$.


We define
\[\Phi_0(u)=
\left\{
\begin{array}{ll}
\frac{p-1}{p}u^p&u\leq e
\\
\frac{u^p}{\log u}-\frac{e^p}{p}&u>e
\end{array}
\right.\]
with $p>1$. Next, we will establish some properties of this function $\Phi_0$.

\begin{thm}
If $p\geq \frac{1+\sqrt 2}{2}$, then $\Phi_0$ is a differentiable $N$-function. 
The $N$-function $\Phi_0$ satisfies that for every $\varepsilon>0$, there exists a positive constant $C=C(p,\varepsilon)$  such that
\begin{equation}\label{cota-sup-indices}
C^{-1}t^{p-\varepsilon}\Phi_0(u)\leq \Phi_0(tu) \leq Ct^p\Phi_0(u)\quad t\geq 1, u>0,
\end{equation}

\end{thm}


\begin{proof}
We have
\[\varphi(u)=\Phi_0'(u)=\left\{
\begin{array}{cccc}
(p-1)u^{p-1}&:=&\varphi_1(u)& \mbox{if}\;u\leq e
\\
\frac{u^{p-1}}{\log u}(p-\frac{1}{\log u})&:=&\varphi_2(u)&\mbox{if}\; u\geq e
\end{array}
\right.
\]

First let us see that $\Phi_0'$ is increasing when $p\geq \frac{1+\sqrt {2}}{2}$.
For this purpose, since $\varphi_1(e)=\varphi_2(e)$, it is enough to see that $\varphi_1$ is increasing  on $[0,e]$ and $\varphi_2$ is increasing on
$[e,\infty)$ for every $p\geq \frac{1+\sqrt {2}}{2}$. Clearly
$\varphi_1$ is an increasing function for $p>1$.  On the other hand, an elementary analysis of the function shows that
$\varphi_2'(u)>0$ on $[e,\infty)$ if and only if
 $p \notin(\frac{1-\sqrt2}{2},\frac{1+\sqrt2}{2})$.  Therefore $\varphi_2$ is an increasing function when $p\geq \frac{1+\sqrt2}{2}$.

Moreover $\varphi_2(u)\to \infty$ and  $\varphi_1(u)\to 0$  as $u \to  \infty$ and $u\to 0$  respectively, provided that $p>1$. Hence, $\Phi_0$ is an $N$-function.



Next we will prove \eqref{cota-sup-indices}. If $u\leq tu\leq e$, then $\Phi_0(tu)=t^p\Phi_0(u)$ and \eqref{cota-sup-indices} holds with $C=1$. If $u\leq e\leq tu$, as $\frac{e^p}{p}>0$  and $\log(tu)\geq 1$, we have
$\Phi_0(tu)\leq t^pu^p= \frac{p}{p-1}t^p\Phi_0(u)$. Thus, the second inequality of  \eqref{cota-sup-indices} holds with $C=\frac{p}{p-1}$. On the other hand, as $f(t)=\frac{t}{\log t}$ is increasing on $[e,\infty)$, then $f((tu)^p)\geq  f(e^p)=e^p/p$.
Now,
\[
\begin{split}
\Phi_0(tu)&=\frac{p(tu)^p}{\log (tu)^p}-\frac{e^p}{p}\\
&= \frac{(p-1)(tu)^p}{\log(tu)^p}+\frac{(tu)^p}{\log (tu)^p}-\frac{e^p}{p}
\\
&\geq \frac{p-1}{p}\frac{(tu)^p}{\log(tu)}\\
&\geq
\frac{p-1}{p}\frac{t^{\varepsilon}}{\log t+1}t^{p-\varepsilon}u^p.
\end{split}
\]
Since $\varepsilon e^{1-\varepsilon}$ is the minimum value of $t\mapsto\frac{t^{\varepsilon}}{\log t+1}$  on the interval $[1,+\infty)$ then
\[
\Phi_0(tu)\geq \frac{p-1}{p}\varepsilon e^{1-\varepsilon}t^{p-\varepsilon}u^p,
\]
which is the first inequality of \eqref{cota-sup-indices} with $C=\frac{p}{p-1}\varepsilon^{-1} e^{-1+\varepsilon}$.


If $e\leq u\leq tu$, then
\begin{equation}\label{Phi-de-u-a-v}
\Phi_0(tu)\leq \frac{t^pu^p}{\log(tu)}\leq \frac{t^pu^p}{\log(u)}=\frac{pt^pv}{\log v},
\end{equation} 
where $v:=u^p$ and $v\geq e^p$.  
If $\alpha>0$, the function $x\mapsto \frac{x}{x-\alpha}$ is decreasing on $(\alpha,\infty)$
and the function $v\mapsto \frac{pv}{\log v}$ is increasing  on $[e^p,\infty)$.
Therefore,  we have
\[
\frac{\frac{pv}{\log v}}{\frac{pv}{\log v}-\frac{e^p}{p}}\leq
\frac{e^p}{e^p-\frac{e^p}{p}}=\frac{p}{p-1}
\]
for every $v \geq e^p$. In this way, from \eqref{Phi-de-u-a-v}, we have
\[
\Phi_0(tu)\leq \frac{pt^p}{p-1}\left(\frac{pv}{\log v}-\frac{e^p}{p}\right)=
 \frac{pt^p}{p-1}\left(\frac{u^p}{\log u}-\frac{e^p}{p}\right)
\]
and the second inequality of  \eqref{cota-sup-indices} holds with $C=\frac{p}{p-1}$. For the first inequality we have, as it was proved previously,

\[
  \Phi_0(tu)
  \geq
  \frac{p-1}{p}\frac{(tu)^p}{\log(tu)}
  =
  \frac{p-1}{p}
  \frac{t^{\varepsilon} \log u^{\varepsilon}}{\log(t^{\varepsilon}u^{\varepsilon})}
  \frac{t^{p-\varepsilon}u^p}{\log u}
\]
Let $f(s)=\frac{sA}{\log s+A}$ with $s\geq 1$ and $A\geq \varepsilon$.  If $A\leq 1$,  the function $f$ attains a minimum on $[1,\infty)$ at $s=e^{1-A}$ and the minimum value is $f(e^{1-A})=Ae^{1-A}\geq \varepsilon$. If $A> 1$, $f$ is increasing  on $[1,\infty)$ and its minimum value is $f(1)=1$. Then, $f(s)\geq \varepsilon$ in any case,   therefore
\[
\Phi_0(tu)\geq \frac{p-1}{p}\varepsilon \frac{t^{p-\varepsilon}u^p}{\log u}\geq
\frac{p-1}{p}\varepsilon t^{p-\varepsilon}\Phi_0(u).
\]
Finally, \eqref{cota-sup-indices} holds with $C=\frac{p}{\varepsilon (p-1)}$, because this $C$ is the biggest constant that we have obtained in each case under consideration.
\end{proof}



\begin{comentario}
The inequality
\[
\Phi_0(tu)\geq Ct^p\Phi_0(u)
\]
is false for every $C$ because for every $u\geq e$ we have
\[
\lim\limits_{t \to \infty}\frac{\Phi_0(tu)}{t^p\Phi_0(u)}=0
\]
\end{comentario}

We note that $\Phi_0$ and $F_0$ satisfy \eqref{eq:propiedad-coercividad-phi0}. For the $p$-laplacian operator we have that $\Phi(|u|)=|u|^p/p$. Then we can take $\Phi_1=\Phi$ in \eqref{eq:caract_delta2}. Clearly $\Phi_0\llcurly\Phi_1$.


% <<<<<<< HEAD
% 
% 
% 
% \begin{thm}
% $\alpha_{\Phi}=\beta_{\Phi}=p$
% \end{thm}
% 
% \begin{proof}
% From \eqref{MO_indices} and \eqref{cota-sup-indices}, we get
% \[
% \beta_{\Phi}=\lim\limits_{t \to \infty} \frac{\log\left[\sup\limits_{u>0} \frac{\Phi(tu)}{\Phi(u)}\right]}{\log t}
% \leq
% \lim \limits_{t \to \infty} \frac{\log C+p\log t}{\log t}=p.
% \]
% On the other hand, employing \eqref{MO_indices} and performing some elementary calculations, we obtain
% \[
% \alpha_{\Phi}=
% \lim\limits_{t \to 0^+} \frac{\log\left[\sup\limits_{u>0} \frac{\Phi(tu)}{\Phi(u)}\right]}{\log t}=
% \lim\limits_{s \to \infty} \frac{\log\left[\sup\limits_{v>0} \frac{\Phi(v)}{\Phi(sv)}\right]^{-1}}{\log s}=
% \lim\limits_{s \to \infty} \frac{\log\left[\inf\limits_{v>0} \frac{\Phi(sv)}{\Phi(v)}\right]}{\log s}
% \]
% where $v:=tu$ and $s:=\frac{1}{t}$.
% Then, using \eqref{cota-sup-indices},  for every $\varepsilon>0$ we have
% \[
% \alpha_{\Phi}=
% \lim\limits_{s \to \infty} \frac{\log\left[\inf\limits_{v>0} \frac{\Phi(sv)}{\Phi(v)}\right]}{\log s}\geq
% \lim\limits_{s \to \infty} \frac{\log C+(p-\varepsilon)\log s}{\log s}\geq p-\varepsilon,
% \]
% therefore $\alpha_{\Phi}\geq p$.
% 
% Finally, as $\alpha_{\Phi}\leq \beta_{\Phi}\leq p$, we get
% $\alpha_{\Phi}=\beta_{\Phi}=p$.
% \end{proof}
% 
% 
% 
% Now, we are able to see that
% \[
% \rho_{\Phi}(u)=\int_0^T \Phi(|u|)\,dx\geq C\|u\orlnor^{\alpha_{\Phi}}=C\|u\orlnor^p
% \]
% is false.
% 
% In fact, if we take $u\equiv t>0$, then $\|u\orlnor^p=C_1t^p$ where $C_1=\|1\orlnor$ and
% $\int_0^T \Phi(|u|)\,dx=C_2\Phi(t)$ with $C_2=T$.
% Then, if $\rho_{\Phi}(u)\geq C\|u\orlnor^p$ were true, then $\Phi(t)\geq C t^p$ would also be true; however, this
% last inequality is false.
%  
% 
% 
%  Assuming $\|b_1\|_{L^1}$  small enough, in  \cite{zhao2004periodic, tang2010periodic}
% coercivity  was obtained even  for the limit value $\alpha=p-1$ in inequality \eqref{holder_cont-mu}.
% 
% 
% 
% 
% This result leans on the  fact that
% \begin{equation}
%  \|u\orlnor^{\alpha_{\Phi}}=O\left(\int_0^T \Phi(|u|)\,dt\right)\quad\text{for } \|u\orlnor\to\infty,
% \end{equation}
% when  $\Phi(u)=|u|^p$.
% Nevertheless, it is no longer the case  for any $N$-function $\Phi$ as the following example shows.

\section*{Acknowledgments}
The authors are partially supported by a UNRC grant number 18/C417. The first author is  partially supported by a  UNSL grant number 22/F223. 



%
% \begin{thm}
% $\alpha_{\Phi}=\beta_{\Phi}=p$
% \end{thm}
%
% \begin{proof}
% From \eqref{MO_indices} and \eqref{cota-sup-indices}, we get
% \[
% \beta_{\Phi}=\lim\limits_{t \to \infty} \frac{\log\left[\sup\limits_{u>0} \frac{\Phi(tu)}{\Phi(u)}\right]}{\log t}
% \leq
% \lim \limits_{t \to \infty} \frac{\log C+p\log t}{\log t}=p.
% \]
% On the other hand, employing \eqref{MO_indices} and performing some elementary calculations, we obtain
% \[
% \alpha_{\Phi}=
% \lim\limits_{t \to 0^+} \frac{\log\left[\sup\limits_{u>0} \frac{\Phi(tu)}{\Phi(u)}\right]}{\log t}=
% \lim\limits_{s \to \infty} \frac{\log\left[\sup\limits_{v>0} \frac{\Phi(v)}{\Phi(sv)}\right]^{-1}}{\log s}=
% \lim\limits_{s \to \infty} \frac{\log\left[\inf\limits_{v>0} \frac{\Phi(sv)}{\Phi(v)}\right]}{\log s}
% \]
% where $v:=tu$ and $s:=\frac{1}{t}$.
% Then, using \eqref{cota-sup-indices},  for every $\varepsilon>0$ we have
% \[
% \alpha_{\Phi}=
% \lim\limits_{s \to \infty} \frac{\log\left[\inf\limits_{v>0} \frac{\Phi(sv)}{\Phi(v)}\right]}{\log s}\geq
% \lim\limits_{s \to \infty} \frac{\log C+(p-\varepsilon)\log s}{\log s}\geq p-\varepsilon,
% \]
% therefore $\alpha_{\Phi}\geq p$.
%
% Finally, as $\alpha_{\Phi}\leq \beta_{\Phi}\leq p$, we get
% $\alpha_{\Phi}=\beta_{\Phi}=p$.
% \end{proof}
%
%
%
% Now, we are able to see that
% \[
% \rho_{\Phi}(u)=\int_0^T \Phi(|u|)\,dx\geq C\|u\orlnor^{\alpha_{\Phi}}=C\|u\orlnor^p
% \]
% is false.
%
% In fact, if we take $u\equiv t>0$, then $\|u\orlnor^p=C_1t^p$ where $C_1=\|1\orlnor$ and
% $\int_0^T \Phi(|u|)\,dx=C_2\Phi(t)$ with $C_2=T$.
% Then, if $\rho_{\Phi}(u)\geq C\|u\orlnor^p$ were true, then $\Phi(t)\geq C t^p$ would also be true; however, this
% last inequality is false.
%
%
%
% As it has seen in the proof of Lemma \ref{lem:submultipliativa} and in \cite[Lemma 5.2]{ABGMS2015},
% we may assume that the function $\Phi_0$
% in Lemma \ref{lem:submultipliativa} ???no ser\'ia \ref{lem_coer}????
% is given by $\Phi_0(x)=|x|^{\mu}$ with $0<\mu<\alpha_{\Phi}$ and
% where $\alpha_{\Phi}$  is a Matuszewska-Orlicz index (see \cite[Ch. 11]{M}).
% These indices are defined by
% \begin{equation}\label{MO_indices}
%     \alpha_{\Phi}:=\lim\limits_{t\to 0^{+}}\frac{\log \left (\sup\limits_{u>0}\frac{\Phi(t u)}{\Phi(u)} \right ) }{\log(t)},\quad
%     \beta_{\Phi}:=\lim\limits_{t\to +\infty}\frac{\log \left  (\sup\limits_{u>0}\frac{\Phi(t u)}{\Phi(u)}\right )}{\log(t)}.
% \end{equation}
% Hence, following the same lines  as the proof of Theorem \ref{coercitividad-r},  using \cite[Lemma 5.2]{ABGMS2015} instead of our Lemma \ref{lem:submultipliativa}, we can assume that $\Phi_0(x)=|x|^{\mu}$ with $0<\mu<\alpha_{\Phi}$ in  Theorem \ref{coercitividad-r}.
% {\bf (Se repite el Thm 3.7 y los dos p\'arrafos est\'an un poco enredados!!!)}.
%
% Assuming $\|b_1\|_{L^1}$  small enough, in  \cite{zhao2004periodic, tang2010periodic}
% coercivity  was obtained even  for the limit value $\alpha=p-1$ in inequality \eqref{holder_cont-mu}.
%
% \section*{Acknowledgments}
% The authors are partially supported by a UNRC grant number 18/C417. The first author is  partially supported by a  UNSL grant number 22/F223.
%
%
% This result leans on the  fact that
% \begin{equation}
%  \|u\orlnor^{\alpha_{\Phi}}=O\left(\int_0^T \Phi(|u|)\,dt\right)\quad\text{for } \|u\orlnor\to\infty,
% \end{equation}
% when  $\Phi(u)=|u|^p$.
% Nevertheless, it is no longer the case  for any $N$-function $\Phi$ as the following example shows.

 \bibliographystyle{elsarticle-num} 
 \bibliography{biblio}


\end{document}
