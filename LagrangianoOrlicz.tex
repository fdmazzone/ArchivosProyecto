%%% Preambulo%%%%%%%%%%%%%%%%%%%%%%%%%

\documentclass[twoside]{article}
%%Paquetes


\usepackage{amsmath,amssymb,amsthm}
\usepackage{color}
\usepackage{ esint }
%\usepackage{graphicx}
%\usepackage{wrapfig}
%\usepackage{subfigure}
\usepackage{fancyhdr}
\usepackage{times}
%\usepackage{theorem}
\usepackage[latin1]{inputenc}
%\usepackage{showkeys}
\usepackage{comment}
\usepackage{url}
\usepackage{xcolor}
\usepackage{adjustbox}
%Teorema y similes

\definecolor{rosa}{rgb}{1,0.3,0.9}
\definecolor{violeta1}{rgb}{0.5,0.3,0.5}
\definecolor{violeta}{rgb}{0.5,0.1,0.5}
\definecolor{negro}{rgb}{0.5,0.2,0.4}
\definecolor{celeste}{rgb}{0.1,0.4,1}
\definecolor{naranja}{rgb}{1,0.5,0}
\definecolor{color_nota_fer}{HTML}{DEBFDB}


\newenvironment{colbox}[2]{%
    \begin{adjustbox}{minipage={\linewidth},margin=1ex,bgcolor=#1,env=center}
        #2}{%
    \end{adjustbox}%
}
\newcounter{nota_fer_cont}
\newenvironment{nota_fer}[1]{\refstepcounter{nota_fer_cont}\begin{colbox}{color_nota_fer}{\textbf{Comentario Leo-Graciela-Fernando \arabic{nota_fer_cont}.} #1}}{\end{colbox}}


\newtheorem{thm}{Theorem}[section]
\newtheorem{cor}[thm]{Corollary}
\newtheorem{lem}[thm]{Lemma}
\newtheorem{rem}[thm]{Remark}
\newtheorem{defi}[thm]{Definition}
\newtheorem{prop}[thm]{Proposition}
\theoremstyle{remark}
\newtheorem{comentario}{Remark}


\title{Euler-Lagragian equations in an Orlicz-Sobolev space setting}
\author{Sonia Acinas \thanks{SECyT-UNRC}\\
Dpto. de Matem\'atica, Facultad de Ciencias Exactas y Naturales\\
Universidad Nacional de La Pampa\\
(6300) Santa Rosa, La Pampa, Argentina\\
\url{sonia.acinas@gmail.com}\\[3mm]
Leopoldo Buri \thanks{SECyT-UNRC}\\
Dpto. de Matem\'atica, Facultad de Ciencias Exactas, F\'{\i}sico-Qu\'{\i}micas y Naturales\\
Universidad Nacional de R\'{i}o Cuarto\\
(5800) R\'{\i}o Cuarto, C\'ordoba, Argentina,\\
\url{lburi@exa.unrc.edu.ar}\\[3mm]
Graciela Giubergia \thanks{SECyT-UNRC and CONICET}\\
Dpto. de Matem\'atica, Facultad de Ciencias Exactas, F\'{\i}sico-Qu\'{\i}micas y Naturales\\
Universidad Nacional de R\'{i}o Cuarto\\
(5800) R\'{\i}o Cuarto, C\'ordoba, Argentina,\\
\url{ggiubergia@exa.unrc.edu.ar}\\[3mm]
Fernando D. Mazzone \thanks{SECyT-UNRC and CONICET}\\
Dpto. de Matem\'atica, Facultad de Ciencias Exactas, F\'{\i}sico-Qu\'{\i}micas y Naturales\\
Universidad Nacional de R\'{i}o Cuarto\\
(5800) R\'{\i}o Cuarto, C\'ordoba, Argentina,\\
\url{fmazzone@exa.unrc.edu.ar}\\[3mm]
Erica L. Schwindt\thanks{ANR. AVENTURES - ANR-12-BLAN-BS01-0001-01}\\
Universit\'{e} d'{O}rl\'{e}ans, Laboratoire MAPMO, CNRS, UMR 7349, \\
F\'ed\'eration Denis Poisson, FR 2964,\\
B\^{a}timent de Math\'{e}matiques, BP 6759, 45067 Orl\'{e}ans Cedex 2, France,\\
\url{leris98@gmail.com}}

\date{}

\newcommand{\orlnor}{\|_{L^{\Phi}}}
\newcommand{\lurnor}{\|^{*}_{L^{\Phi}}}
\newcommand{\linf}{\|_{L^{\infty}}}
\newcommand{\lphi}{L^{\Phi}}
\newcommand{\lpsi}{L^{\Psi}}
\newcommand{\ephi}{E^{\Phi}}
\newcommand{\claseor}{\widetilde{L}^{\Phi}}
\newcommand{\wphi}{W^{1}\lphi}
\newcommand{\sobnor}{\|_{W^{1}\lphi}}
\newcommand{\domi}{W^{1}\left(\lphi,\Pi\left(\ephi,1\right)\right)}
\renewcommand{\b}[1]{\boldsymbol{#1}}
\newcommand{\rr}{\mathbb{R}}
\newcommand{\nn}{\mathbb{N}}



\begin{document}



\maketitle
%
\begingroup%Locallizing the change to `thefootnote'.
    \renewcommand{\thefootnote}{}%Removing the footnote symbol.
    %
    \footnotetext{%
    %   2010 Mathematics Subject Classification
    %   http://www.ams.org/msc/
    \textbf{2010  AMS Subject Classification.} Primary: .
    Secondary: .
    }%
        \footnotetext{%
    \textbf{Keywords and phrases.}  .
    }%
    \endgroup
%
%
%
%

\begin{abstract}
...
\end{abstract}




\pagestyle{fancy} \headheight 35pt \fancyhead{} \fancyfoot{}

\fancyfoot[C]{\thepage} \fancyhead[CE]{\nouppercase{S. Acinas, L. Buri, G. Giubergia, F. Mazzone and E. Schwindt}} \fancyhead[CO]{\nouppercase{\section}}

\fancyhead[CO]{\nouppercase{\leftmark}}


%\tableofcontents

\section{Preliminaries}

%\subsection{$N$-functions}
For reader convenience, a short introduction to Orlicz and Orlicz Sobolev spaces is given, we refer to \cite{adams_sobolev,KR} for additional details and proofs.

A function $\Phi:[0,+\infty)\to [0,+\infty)$ is called an \emph{$N$-function} if it has the form
\[
\Phi(t)=\int_{0}^t \varphi(\tau)\ d\tau,\quad\hbox{for } u\geq 0,
\]
where $\varphi: [0, \infty)\rightarrow [0, \infty)$ is a right continuous nondecreasing function  satisfying   $\varphi(0)=0$, $\varphi(t)>0$ for $t>0$ and
$\lim_{t\rightarrow \infty}\varphi(t)=+\infty$.

Given a function $\varphi$ as above, we also consider the so-called right inverse function $\psi$ of $\varphi$ which is defined $\psi(s)=\sup_{\varphi(t)\leq s}t$.
The function $\psi$ satisfies the same properties that function $\varphi$, therefore we have an $N$-function $\Psi$ associated to $\psi$. We say that $\Psi$ is the
\emph{complementary function} of $\Phi$.

Following  \cite{M}, we say that a $N$-function $\Phi$ is in the \emph{$\Delta_2$ class in infinity} ($\Phi\in\Delta^{\infty}_2$) when there exists a constant $K>0$  such that 
\begin{equation}\label{eq:delta2}\Phi(2x)\leq K\Phi(x),\end{equation} 
for all  $x\geq 1$. If \eqref{eq:delta2}  is satisfied for all $x\leq 1$ instead $x\geq 1$ we say that $\Phi$ is in  \emph{$\Delta_2$ class in $0$} ($\Phi\in\Delta_2^0$).   We write $\Phi\in\Delta^{a}_2$ when \eqref{eq:delta2}  holds for all $x\geq 0$. We define tha $\nabla_2$ class by duality with the $\Delta_2$ class, that is $\Phi\in \nabla_2^i$ if and only if $\Psi\in\Delta_2^i$, with $i=\infty,0,a$. Functions in the $\Delta_2$ and  $\nabla_2$ classes are controled by powers function. More precisely if $\Phi\in \Delta_2^i$ then there exists $\alpha>0$ such that $\Phi(tx)\leq t^{\alpha}\Phi(x)$ for every $    






Throughout this article we will try with spaces of $\mathbb{R}^n$ valued functions defined on an interval $[0,T]\subset\mathbb{R}$. Given an $N$-function $\Phi$ we
define the \emph{Orlicz class} $\widetilde{L}^{\Phi}([0,T],\mathbb{R}^n)$ by
\begin{equation}\label{claseOrlicz}
  \widetilde{L}^{\Phi}([0,T],\mathbb{R}^n):=\left\{\b{u}: [0,T] \rightarrow\mathbb{R}^n,\b{u}\ \hbox{mesurable},\ \int_{[0,T]} \Phi(|\b{u}|)\ dx < \infty \right\}.
\end{equation}
here $|\cdot|$ is the euclidean norm of $\mathbb{R}^n$. When clear from the context, we will omit the domain and codomain in the notation of function spaces and
classes ($\claseor=\widetilde{L}^{\Phi}([0,T],\mathbb{R}^n)$). The \emph{Orlicz space} $\lphi=L^{\Phi}([0,T],\mathbb{R}^n)$ is defined as the linear hull of $\claseor$.
Equivalently
\begin{equation}\label{espacioOrlicz}
\lphi:=\left\{ \b{u}: [0,T] \rightarrow \mathbb{R}^n \bigg| \b{u}\ \hbox{is mesurable and},\ \int_{[0,T]} \Phi(\alpha|\b{u}|)\ dx < \infty \ \hbox{for some}\ \alpha >0   \right\}.
\end{equation}
 We adopt the convention of to use bold symbols for denote $\mathbb{R}^n$-valued fuction and plain symbols for scalar ones. The Orlicz space $\lphi$ equipped with the Orlicz norm
\[
\|  \b{u}  \orlnor:=\sup \left\{  \int_0^T\langle \b{u}, \b{v}\rangle dt: \int_0^T\Psi(|\b{v}|)\leq 1\right\},
\]
is a Banach space. Here by $\langle \b{u}, \b{v}\rangle$ we denote the usual dot product in $\mathbb{R}^{n}$ between $\b{u}$ and $\b{v}$.

%Alternatively  the Luxemburg norm
%\[
%\|  u  \lurnor =\inf \left\{ k>0:  \int_{0}^T \Phi \left(\frac{|u|}{k}\right)\ dx \leq 1 \right\},
%\]
%defines an equivalent norm.



The space $\ephi=\ephi([0,T],\mathbb{R}^n)$ is defined as the closure in $\lphi$ of the subspace $L^{\infty}$. The space $\ephi$ is the maximal subspace of the
Orlicz class $\claseor$ and, it is known that, $\left[\ephi\right]^*=\lpsi$.


Likes in \cite{KR}, we will consider the subset $\Pi(\ephi,r)=\Pi\left(\ephi\left([0,T],\mathbb{R}^n\right),r\right)$ of $\lphi([0,T],\mathbb{R}^n)$ defined by
\[\Pi(\ephi,r):=\{\b{u}\in\lphi: d(\b{u},\ephi)<r\}.\]
This set is related to the Orlicz class $\claseor$ by means of inclusions
\begin{equation}\label{inclusiones}\Pi(\ephi,1)\subset \claseor \subset\overline{\Pi(\ephi,1)}.\end{equation}
The proof of this fact, and similar ones, is given by real valued function in \cite{KR},
the extension to $\mathbb{R}^n$-valued functions does not involve any difficulty. When the function $\Phi$ is of the $\Delta_2$ class then the four sets $\lphi$, $\ephi$ $\Pi(\ephi,1)$ and $\claseor$ are equal.

Let $X$ and $Y$ be subsets of certain vector spaces of $\mathbb{R}^n$-valued measurable functions defined in $[0,T]$.   We denote by  $W^1(X,Y)$ to the set defined by
\[W^1(X,Y):=\{\b{u}| \b{u} \hbox{ is absolutely continuous and } \b{u}\in X, \b{\dot{u}}\in Y\}.\]
If $X=Y$ we simply write $W^1(X,X)=W^1X$. In this paper $X$ and $Y$ will be some subset of an Orlicz space.  When $X=Y=\lphi$ we have the usual Sobolev-Orlicz space
 $\wphi$ (see \cite{adams_sobolev}) , which is a Banach space  equipped with the norm
\[
\|  \b{u}  \|_{\wphi}= \|  \b{u}  \|_{\lphi} + \|\b{\dot{u}}\orlnor.
\]

 An important aspect of the theory of Sobolev spaces is related to embedding theorems. There is an extensive literature on this question in the setting of Orlicz-Sobolev spaces, see for example
 \cite{cianchi1999some,cianchi2000fully,claverooptimal,edmunds2000optimal,kerman2006optimal}.
For this reason the following simple  Lemma, which we will use systematically, it is well known. We include a brief proof for sake of completeness. As is usual, if $X$ and $Y$ are normed spaces, with $X\subset Y$,  we write $X\hookrightarrow Y$ when the identity map is an bounded operator between $X$ and $Y$.


\begin{lem}\label{inclusion orlicz}$\wphi\hookrightarrow L^{\infty}$.
\end{lem}
\begin{proof}
Let $\b{u}\in \wphi$. From the mean value theorem there exists $\tau$ such that
$\b{u}(\tau)=\frac{1}{T}\int\limits_{0}^{T}\b{u}(s)ds$, thus
\begin{equation}\label{desigualdad1}\begin{split}
|\b{u}(t)|\leqslant |\b{u}(\tau)|+\int\limits_{\tau}^{t}|\b{\dot{u}}(s)|ds
%&\leq |\b{u}(\tau)|+\int\limits_{0}^{T}|\b{\dot{u}}(s)|ds\\
\leq |\b{u}(\tau)|+\|\b{\dot{u}}\|_{L^{\Phi}}\|1\|_{L^{\Psi}}.
\end{split}
\end{equation}
Moreover
\begin{equation}\label{desigualdad2}\begin{split}
|\b{u}(\tau)|\leq \frac{1}{T}\int\limits_{0}^{T}|\b{u}(s)|ds\leq \frac{1}{T}\|\b{u}\|_{L^{\Phi}}\|1\|_{L^{\Psi}}.
\end{split}
\end{equation}
From \eqref{desigualdad1} and \eqref{desigualdad2} we obtain
\begin{equation}\label{estimacion}
\|\b{u}\|_{L^{\infty}}\leq C(T)\|\b{u}\|_{\wphi}.
\end{equation}
\end{proof}


\begin{comentario}
If $\frac{1}{T}\int\limits_{0}^{T}\b{u}(s)ds=0$, from \eqref{desigualdad1} we have
\begin{equation}\label{estimacion2}
\|\b{u}\|_{L^{\infty}}\leq C(T)\|\dot{\b{u}}\|_{\lphi}.
\end{equation}
with $C(T)=\|1\|_{L^{\Psi}}$
\end{comentario}


We will use repeatedly the following elementary consequence of the previous theorem. Hereafter we denote  by $\mathbb{R}^+$ to the set of all non negative real numbers.
\begin{cor}\label{a_bound} Let $a\in C(\mathbb{R}^+,\mathbb{R}^+)$ and we define the composition operator $\b{a}$ by $\b{a}(\b{u})(t)= a(|\b{u}(t)|)$. Then $\b{a}:\wphi\to L^{\infty}([0,T],\mathbb{R})$ is bounded, which is there exists a non decreasing function $c:\mathbb{R}^+\to\mathbb{R}^+$ such that
 $\|\b{a}(\b{u})\|_{L^{\infty}([0,T])}\leq c(\|\b{u}\|_{\wphi})$.
\end{cor}
\begin{proof}  If $\b{u}\in \wphi$ then, by Lemma \ref{inclusion orlicz} ,
 $\b{u}\in L^{\infty}$ and $\|\b{u}\|_{L^{\infty}}\leq C(T)\|\b{u}\|_{\wphi}=:A$.   By hypothesis
 $a:[0,A]\rightarrow\mathbb{R}$ is bounded, hence the supremum $c(A):=\sup_{[0,A]}a$ is well defined. Clearly $c$ satisfies the statement of Corollary.
\end{proof}


The following lemma is an easy consequence of more general principles  related to  operators of Nemitskii type, see \cite[�17]{KR}.

\begin{lem}\label{phi_comp}   Let $\boldsymbol{\varphi}$ be the  composition operator
 defined by $\boldsymbol{\varphi}(\b{u})(t)= \varphi(|\b{u}(t)|)$. Then  $\boldsymbol{\varphi}$ acts from $\Pi(\ephi,1)$ into $\tilde{L}^{\Psi}$.
\end{lem}
\begin{proof}
  As consequence of \cite[Lemma 9.1]{KR} we have that  $\boldsymbol{\varphi}\left(B_{\lphi}(0,1)\right)\subset \tilde{L}^{\Psi}$, where
$B_{\lphi}(\b{u}_0,r)$ is the open ball with center $\b{u}_0$ and radius $r>0$. Therefore, applying \cite[Lemma 17.1]{KR} we deduce that $\boldsymbol{\varphi}$ acts from $\Pi(\ephi,1)$ into $\tilde{L}^{\Psi}$.
\end{proof}

We need also the following technical lemma.
\begin{lem}\label{segundo lema}
Let $\{\b{u}_n\}_{n\in \mathbb{N}}$ a sequence of  functions in $\Pi(\ephi,1)$, and $\b{u}\in \lphi$ such that $\b{u}_n\rightarrow \b{u}$ in $\lphi$. Then there exist a subsequence
$\b{u}_{n_k}$ and a real valued function $h\in\Pi\left(\ephi\left([0,T],\mathbb{R}\right),1\right)$ such that $\b{u}_{n_k}\rightarrow \b{u} \quad\text{a.e.}$ and $|\b{u}_{n_k}|\leq h\quad\text{a.e.}$.
\end{lem}



\begin{proof}
Let $r:=d(\b{u},\ephi)$, $r<1$. Because $\b{u}_n$ converges to $\b{u}$, there exists a subsequence $(n_k)$ such that
\[\|\b{u}_{n_k}-\b{u}\|_{\Phi}<\frac{1-r}{2}\quad \text{ and }\quad \|\b{u}_{n_k}-\b{u}_{n_{k+1}}\|_{\Phi}<2^{-k}(1-r)\]
Let $h:[0,T]\rightarrow\mathbb{R}$ defined by
\begin{equation}\label{serie} h(x)=|\b{u}_{n_1}(x)|+\sum_{k=2}^{\infty}|\b{u}_{n_k}(x)-\b{u}_{n_{k-1}}(x)|.
\end{equation}
Since
\[d(|\b{u}_{n_1}|,\ephi)\leq d(\b{u}_{n_1},\ephi)\leq d(\b{u}_{n_1},\b{u})+d(\b{u},\ephi)<\frac{1+r}{2},\]
we have
\[d(h,\ephi)\leq d(h,|\b{u}_{n_1}|)+d(|\b{u}_{n_1}|,\ephi)< 1.\]
Therefore, $h\in\Pi(\ephi,1)$.  In particular,  $|h|<\infty$ a.e. We conclude that la serie $\b{u}_1(x)+\sum_{k=2}^{\infty}(\b{u}_{n_k}(x)-\b{u}_{n_{k-1}}(x))$
is absolutely convergent a.e.  This imply that $\b{u}_{n_k}\rightarrow \b{u} \quad\text{a.e.}$. The inequality $|\b{u}_{n_k}|\leq h$ is clear from the definition of $h$.
\end{proof}
The following simple result will be useful.
\begin{lem}\label{lema_conv_may}
Suppose that $\Phi$ is a $\Delta_2$ function.  If $u_n \in\lphi$ is a sequence such that $u_n\to 0$ a.e. and suppose that there exist $M\in\lphi$ with $|u_n|\leq M$
then $\|u_n\orlnor\to 0$.
\end{lem}
\begin{proof}
 According to \cite[Theorem 9.4]{KR} it is sufficient to prove that
 \[\int_0^T\Phi(|\b{u}_n|)dt\to 0,\quad\text{for }n\to\infty.\]
 This is an immediate consequence of $\Phi(|\b{u}_n|)\leq \Phi(M)\in L^1$ and the Dominated Convergence Theorem.
\end{proof}


If $X$ is a Banach spaces, we denote by $\langle\cdot,\cdot\rangle:X\times X^*\to\mathbb{R}$ the bilinear map given by the pairing between $X$ and its dual space $X^*$.  We recall the definition of Gate\^{a}ux derivative, see \cite{ambrosetti} for details. Given a function $I:U\to\mathbb{R}$ where $U$ is an open set of a Banach space $X$,
we say that $I$ has a G\^ateaux derivative en $\b{u} \in U$ if there exists $\b{u}^*\in X^*$ such that for every $\b{v} \in X$
\[
\lim_{s \rightarrow 0}\frac{I(\b{u}+s\b{v})-I(\b{u}) }{s}=\langle \b{u},\b{u}^*\rangle.
\]



\section{Differetiability of action integrals on Orlicz spaces}

\begin{defi} We said that a function $\mathcal{L}:[0,T]\times \mathbb{R}^n \times \mathbb{R}^n \rightarrow \mathbb{R}$ is a Caratheodory function if for fixed $(\b{x},\b{y})$
the map $t \mapsto \mathcal{L}(t, \b{x},\b{y})$ is measurable  and for fixed $t$ the map  $(\b{x},\b{y}) \mapsto \mathcal{L}(t, \b{x}, \b{y})$ is continuously differentiable for almost everywhere $t\in [0,T]$.

\end{defi}



\begin{thm}\label{teorema_acotacion}
Let $\mathcal{L}:[0,T]\times \mathbb{R}^n \times \mathbb{R}^n \rightarrow \mathbb{R}$ be a Caratheodory function and $\Phi,\Psi$ be complementary  $N$-functions. Suppose that there
exists $a \in C(\mathbb{R}^+, \mathbb{R}^+)$, $b \in L^1$, $c \in \lpsi$ such that

\begin{eqnarray}
|\mathcal{L}(t,\b{x},\b{y})| &\leq a(|\b{x}|)\left(b(t)+ \Phi(|\b{y}|)  \right),\label{cotaL}\\
|D_{\b{x}}\mathcal{L}(t,\b{x},\b{y})| &\leq a(|\b{x}|)\left(b(t)+ \Phi(|\b{y}|)  \right),\label{cotaDxL}\\
|D_{\b{y}}\mathcal{L}(t,\b{x},\b{y})| &\leq a(|\b{x}|)\left(c(t)+ \varphi(|\b{y}|)  \right).\label{cotaDyL}
\end{eqnarray}




Then the following statements hold
\begin{enumerate}
\item \label{T1item1} \label{A1} The \emph{action integral}  $I: W^{1}(\lphi,\claseor) \rightarrow \mathbb{R}$ defined by
\begin{equation}\label{integral_accion}
I(\b{u})=\int_{0}^T \mathcal{L}(t,\b{u}(t),\b{\dot{u}}(t))\ dt
\end{equation}
is finite for every $\b{u}\in W^{1}(\lphi,\claseor)$.

\item\label{T1item3} The function  $I$ is G\^ateaux differentiable on $W^{1}\left(\lphi,\Pi\left(\ephi,1\right)\right)$ and  its derivative $I'$ is continuous from $\domi$ with the strong topology into $\left[\wphi \right]^*$ equipped with the $w^*$-topology. Moreover $I'$ is given by the expression
\[
\langle v, I'(\b{u})\rangle= \int_0^T \bigg\{\bigg\langle D_{\b{x}}\mathcal{L}\big(t,\b{u}(t),\b{\dot{u}}(t)\big), \b{v}(t)\bigg\rangle+ \bigg\langle D_{y}\mathcal{L}\big(t,\b{u}(t),\b{\dot{u}}(t)\big),\dot{\b{v}}(t)\bigg\rangle\bigg\} \ dt.
\]

\item\label{T1item4}  If $\Phi$ and $\Psi$ are $\Delta_2$ functions,
  $I'$ is continuous from $\wphi$ into $\left[\wphi\right]^*$ when both spaces are equipped with the strong topology.


\end{enumerate}
\end{thm}
\begin{proof} From Corollary \ref{a_bound} we obtain a constant $c=c(\|\b{u}\sobnor )$ such that  $a(|\b{u}|)\leq c$.
 Thus,
 \[|\mathcal{L}(t,\b{u},\b{\dot{u}})| \leq M\left(b(t)+ \Phi(|\b{\dot{u}}|)  \right)\in
 L^1.\]

 We split the proof of  \ref{T1item3} in three steps.

\noindent\textbf{Step 1.} We prove that $\b{u} \mapsto D_{\b{x}}\mathcal{L}(t,\b{u},\b{\dot{u}})$ is continuous from $\domi$ into $L^{1}([0,T])$ whith the strong topology on both sets. We take   $\{\b{u}_n\}_{n\in \mathbb{N}}$ a sequence of  functions in $W^{1}(\lphi,\Pi(\ephi,1))$, and $\b{u}\in W^{1}(\lphi,\Pi(\ephi,1))$ such that $\b{u}_n\rightarrow \b{u}$ in $\wphi$.
Then $\b{u}_n\rightarrow \b{u}$ in $\lphi$ and $\b{\dot{u}}_n\rightarrow \b{\dot{u}}$ in $\lphi$. By Lemma \ref{segundo lema} there exist a subsequence $\b{u}_{n_k}$ and $h\in \Pi(\ephi,1))$
such that $\b{u}_{n_k}\rightarrow \b{u} \quad\text{a.e.}$, $\b{\dot{u}}_{n_k}\rightarrow \b{\dot{u}} \quad\text{a.e.}$ and $|\b{\dot{u}}_{n_k}|\leq h\quad\text{a.e.}$.  Since $\b{u}_{n_k}$, $k=1,2,\ldots$ is a strong convergent sequence in $\wphi$, it is a bounded sequence in $\wphi$. According to Lemmas \ref{inclusion orlicz} and Corollary \ref{a_bound} there exists $M>0$ such that $\|\b{a}(\b{u}_{n_k})\|_{L^{\infty}} \leq M$, $k=1,2,\ldots$.  From the previous facts and \eqref{cotaDxL} we get
\begin{equation}\label{DxL1}
|D_x\mathcal{L}(t,\b{u}_{n_k}(t),\b{\dot{u}}_{n_k}(t))|\leq M\left(b(t)+\Phi(|h|)\right) \in L^1.
\end{equation}
By the Caratheodory condition
\[D_x\mathcal{L}(t,\b{u}_{n_k}(t),\b{\dot{u}}_{n_k}(t))\to D_x\mathcal{L}(t,\b{u}(t),\b{\dot{u}}(t))\quad\hbox{ for a.e }t\in[0,T].\]
Applying the Dominated Convergence Theorem we conclude the proof of step 1.

\noindent\textbf{Step 2.} We will prove that the mapping  $\b{u}
 \mapsto  D_{y}\mathcal{L}(t,\b{u},\b{\dot{u}})$ is continuous from $\domi$ with the strong topology  into $\left[\lphi\right]^*$  with the weak$^*$ topology. Let $\b{u}\in W^{1}(\lphi,\Pi(\ephi,1))$.  It follows from Lemma \ref{phi_comp} and Corollary \ref{a_bound} that $\varphi(\b{u}(t))\in\tilde{L}^{\Psi}$ and $a(|\b{u}(t)|)\in L^{\infty}$ respectively. Therefore, in virtue of  \eqref{cotaDyL} we get
\begin{equation}\label{DyLpsi}
   \left\|D_y\mathcal{L}(t,\b{u}(t),\b{\dot{u}}(t))\right\|_{L^{\Psi}}\leq  c(\|\b{u}\|_{\wphi} )\left(\|c\|_{L^{\Psi}}+\|\boldsymbol{\varphi}(\b{u})\|_{L^{\Psi}} \right)
\end{equation}
Hence $D_y\mathcal{L}(t,\b{u}(t),\b{\dot{u}}(t))\in L^{\Psi}$.

Now, let us to prove the continuity of the map   $\b{u}\mapsto D_y\mathcal{L}(t,\b{u}(t),\b{\dot{u}}(t))$. We take $\b{u}_n,\b{u}\in W^{1}(\lphi,\Pi(\ephi,1))$ with $\b{u}_n\to \b{u}$ in the norm of $\wphi$. We must prove that  $D_y\mathcal{L}(t,\b{u}_n(t),\dot{\b{u}_n}(t))\rightharpoonup D_y\mathcal{L}(t,\b{u}(t),\b{\dot{u}}(t))$. Suppose, on the contrary, that there exists $\b{v}\in\lphi$, $\epsilon>0$ and a subsequence of $\{\b{u}_n\}$ (again denoted for simplicity $\{\b{u}_n\}$)  such that
\begin{equation}\label{cota_eps}
 \left| \langle D_y\mathcal{L}(t,\b{u}_n(t),\b{\dot{u}_n}(t)), \b{v} \rangle - \langle D_y\mathcal{L}(t,\b{u}(t),\b{\dot{u}}(t)), \b{v} \rangle\right|\geq \epsilon.
\end{equation}
We have $\b{u}_n\rightarrow \b{u}$ in $\lphi$ and
$\b{\dot{u}}_n\rightarrow \b{\dot{u}}$ in $\lphi$. By Lemma \ref{segundo lema}, there exist a subsequence $\b{u}_{n_k}$ and $h\in \Pi(\ephi,1)$ such that $\b{u}_{n_k}\rightarrow \b{u} \quad\text{a.e.}$, $\b{\dot{u}}_{n_k}\rightarrow \b{\dot{u}} \quad\text{a.e.}$ and $|\b{\dot{u}}_{n_k}|\leq h\quad\text{a.e.}$. As in the previous step, since $\b{u}_n$ is a convergent sequence, the Corrollary \ref{a_bound} implies that $a(|\b{u}_n(t)|)$ is uniformly bounded by certain constant $C$. Therefore, from \eqref{cotaDyL} we get
\[
  \left |\langle D_y\mathcal{L}(t,\b{u}_n(t),\b{\dot{u}}_n(t)) ,\b{ v}\rangle \right| \leq C\left(c(t)|\b{v}(t)|+\varphi(|h(t)|)|\b{v}(t)|\right).
\]
We have that $c(t),\varphi(h(t))\in\lpsi$ and $|\b{v}|\in\lphi$. Then,  H\"older inequality for Orlicz spaces implies that  $\langle D_y\mathcal{L}(t,\b{u}_{n_k}(t),\b{\dot{u}}_{n_k}(t)), \b{v}\rangle$ is dominated by a function in $L^1$. Thus, from the Lebesgue dominated convergence Theorem we deduce
\begin{equation}\label{conv_debil}\int_0^T \langle D_y\mathcal{L}(t,\b{u}_{n_k}(t),\b{\dot{u}}_{n_k}(t)),\b{ v}(t) \rangle dt \to \int_0^T\langle D_y\mathcal{L}(t,\b{u}(t),\b{\dot{u}}(t)),\b{ v}(t) \rangle dt \end{equation}
which contradict the inequality \eqref{cota_eps}. This completes the proof of step 2.

\textbf{Step 3.} Finally we prove \ref{T1item3}. The proof follows similar lines that \cite[Theorem 1.4]{mawhin2010critical}. For $\b{u}\in \domi$ and $\b{v}\in\wphi$ we define the function
\[f(s,t):=\mathcal{L}(t,\b{u}(t)+s\b{v}(t),\b{\dot{u}}(t)+s\b{\dot{v}}(t)).\]
We remark that by \ref{T1item1}
\[I(\b{u}+s\b{v})=\int_0^Tf(s,t)\ dt\]
is well defined and it is finite valued for $t\in [0,T]$ and  $|s|\leq s_0:=\left(1-d(\b{\dot{u}},\ephi)\right)/\|\b{v}\sobnor$ ($\b{v}\neq 0$). Using  Corollary \ref{a_bound} we have
where
\[ \|a(|\b{u}+s\b{v}|)\|_{L^{\infty}}\leq  c(\|\b{u}+s\b{v}\sobnor)\leq
 c(\|\b{u}\sobnor+s_0\|\b{v}\sobnor).
\]
Consequently, applying chain rule,  inequalities \eqref{cotaDxL}-\eqref{cotaDyL}, the previous inequality and using that $\varphi$ and $\Phi$ are non decreasing, we obtain
\begin{equation}\label{ctg}
\begin{split}
|D_s f(s,t)|&=\big|\langle D_x\mathcal{L}(t,\b{u}+s\b{v},\b{\dot{u}}+s\b{\dot{v}}), \b{v}\rangle + \langle D_y\mathcal{L}(t,\b{u}+s\b{v},\b{\dot{u}}+s\b{\dot{v}}), \b{\dot{v}}\rangle\big| \\
 & \leq c \big[(b(t)+\Phi(|\b{\dot{u}}|+s_0|\b{\dot{v}}|))|\b{v}|+ (c(t)+ \varphi(|\b{\dot{u}}|+s_0|\b{\dot{v}}|))|\b{\dot{v}}| \big]
\end{split}
\end{equation}
It is easy to show that $d(\b{w},\ephi)\leq d(|\b{w}|,\ephi)$ for every $\b{w} \in \lphi$. Then
\[
d \left(|\b{\dot{u}}|+s_0|\b{\dot{v}}|, \ephi \right) %d(|\b{\dot{u}}|,\ephi)+ d(|\b{\dot{u}}|+s_0|\b{\dot{v}}|, |\b{\dot{u}}|)
\leq d \left(|\b{\dot{u}}|,\ephi \right)+ s_0 \|\b{\dot{v}}\orlnor < 1.
\]
As a consequence $|\b{\dot{u}}|+s_0|\b{\dot{v}}| \in \Pi(\ephi,1) \subset \claseor$. Then $b+\Phi(|\b{\dot{u}}|+s_0|\b{\dot{v}}|) \in L^1$ and since $\b{v} \in L^{\infty}$ we have that
$(b+\Phi(|\b{\dot{u}}|+s_0|\b{\dot{v}}|))|\b{v}| \in L^1$. On the other hand, from Lemma \ref{phi_comp} we obtain $c(t)+ \varphi(|\b{\dot{u}}|+s_0|\b{\dot{v}}|) \in L^{\Psi}$ and since $\b{\dot{v}} \in L^{\Phi}$, applying the H\"older inequality
$(c(t)+ \varphi(|\b{\dot{u}}|+s_0|\b{\dot{v}}|))|\b{\dot{v}}| \in L^1$. Thus, from \eqref{ctg} and the above discussion there exists a function $g \in L^1([0,T], \mathbb{R}^{+})$
such that $|D_s f(s,t)| \leq g(t)$. Consequently, $I$ has a directional derivative and
\[
\langle v, I'(\b{u}) \rangle=\frac{d}{ds}I(\b{u}+sv)\big|_{s=0}=\int_0^T \left(\langle D_{x}\mathcal{L}(t,\b{u},\b{\dot{u}}), \b{v}\rangle+ \langle D_{y}\mathcal{L}(t,\b{u},\b{\dot{u}}),\b{\dot{v}}\rangle\right) \ dt.
\]
Moreover, from \eqref{DxL1}, \eqref{DyLpsi}, Lemma \ref{inclusion orlicz} and previous formula
\[
|\langle I'(\b{u}), \b{v} \rangle| \leq c \|v\linf + c \|\b{\dot{v}}\orlnor \leq c \|\b{v}\sobnor.
\]
This complete the proof of the G\^ateaux differentiability of $I$. Finally, the continuity of $I': \left(\domi, \|\cdot \sobnor\right) \to \left(\left[\wphi
\right]^*, w^* \right)$ is a consequence of the continuity of the mappings $\b{u} \mapsto D_{\b{x}}\mathcal{L}(t,\b{u},\b{\dot{u}})$ and $\b{u} \mapsto
D_{\b{y}}\mathcal{L}(t,\b{u},\b{\dot{u}})$. Indeed, we set $\b{u}_n,\b{u}\in W^{1}(\lphi,\Pi(\ephi,1))$ with $\b{u}_n\to \b{u}$ in the norm of $\wphi$ and $\b{v} \in
\wphi$ , then
\[
\begin{split}
\left\langle \b{v}, I'(\b{u}_{n}) \right\rangle &= \int_0^T \left\{\left\langle D_{\b{x}}\mathcal{L}\left(t,\b{u}_n(t),\b{\dot{u}}_n(t)\right),
\b{v}(t)\right\rangle+
\left\langle D_{y}\mathcal{L}\left(t,\b{u}_n(t),\b{\dot{u}}_n(t)\right),\dot{\b{v}}(t)\right\rangle\right\} \ dt\\
&\rightarrow \int_0^T \left\{\left\langle D_{\b{x}}\mathcal{L}\left(t,\b{u}(t),\b{\dot{u}}(t)\right), \b{v}(t)\right\rangle+ \left\langle
D_{y}\mathcal{L}\left(t,\b{u}(t),\b{\dot{u}}(t)\right),\dot{\b{v}}(t)\right\rangle\right\} \ dt\\
&=\left\langle \b{v}, I'(\b{u}) \right\rangle.
\end{split}
\]


In order to prove  \ref{T1item4}, let us see that the maps $\b{u}\mapsto D_{\b{x}}\mathcal{L}(\cdot,\b{u}(\cdot),\b{\dot{u}}(\cdot))$  and $\b{u}\mapsto D_{\b{y}}\mathcal{L}(\cdot,\b{u}(\cdot),\b{\dot{u}}(\cdot))$  are continuous
from $\left(\wphi, \|\cdot \sobnor\right) $ into $\left( L^1, \|\cdot \|_{L^1}\right)$ and
 $\left(\lpsi,\|\cdot\|_{L^{\Psi}}\right)$ respectively.  The continuity of the first map is an immediate consequence of the step 1 in the proof of item \ref{T1item3} and the fact that $\Pi(\ephi,1) =\lphi$  when $\Phi$ is of the $\Delta_2$ class. We will prove the continuity of the second map.  We consider $\b{u}_n$ and $\b{u}$ with $\|\b{u}_n- \b{u}\sobnor\to 0$.
By Lemma \ref{segundo lema}, there exist a subsequence $\b{u}_{n_k}$ and $h\in \lphi$ such that $\b{u}_{n_k}\rightarrow \b{u} \quad\text{a.e.}$, $\b{\dot{u}}_{n_k}\rightarrow \b{\dot{u}} \quad\text{a.e.}$ and $|\b{\dot{u}}_{n_k}|\leq h\quad\text{a.e.}$.
 Then  since $\mathcal{L}$ is a Caratheodory function
 we have $ D_{\b{y}}\mathcal{L}(t,\b{u}_{n_k}(t),\b{\dot{u}}_{n_k}(t))\to D_{\b{y}}\mathcal{L}(t,\b{u}(t),\b{\dot{u}}(t))$ a.e. $t\in [0,T]$.  From Corollary \ref{a_bound} and the fact that $\|\b{u}_{n_k}\sobnor$ are uniformly bounded, we get a constant $C>0$ such that  $\|\b{a}(\b{u}_{n_k})\|_{L^{\infty}}\leq C$.
 By using \eqref{cotaDyL} and as $\Psi$ is of the $\Delta_2$ class, we obtain
 \[\begin{split}
    |D_{\b{y}}\mathcal{L}(t,\b{u}_{n_k}(t),\b{\dot{u}}_{n_k}(t))| &\leq a(|\b{u}_{n_k}|)\left( c(t) + \varphi (|\b{\dot{u}}_{n_k}(t)|)\right)\\
    &\leq C\left( c(t) + \varphi (|h|)\right)\in \lpsi
   \end{split}
\]
Therefore, invoking  Lemma \ref{lema_conv_may}, we have proved that
  of all sequence $\b{u}_n$ which converge to $\b{u}$ in $\wphi$ we can
extract a subsequence with $D_{\b{y}}\mathcal{L}(t,\b{u}_{n_k},\b{\dot{u}}_{n_k})\to D_{\b{y}}\mathcal{L}(t,\b{u},\b{\dot{u}})$ in the strong topology. The desired result follows from a standard argument.

Now we are  in conditions of  to prove the continuity of $I'$. Let $\b{u}, \b{u_0},\b{v}\in\lphi$. Then
\[
  \begin{split}
   \langle I'(\b{u})-I'(\b{u}_0),\b{v} \rangle
      &= \int_0^T \left\{ \left(D_{\b{x}}\mathcal{L}(t,\b{u}(t),\b{\dot{u}}(t))-D_{\b{x}}\mathcal{L}(t,\b{u}_0(t),\dot{\b{u}_0}(t))\right)\cdot v(t)\right.\\
      &\quad\left.+
      \left(D_{\b{y}}\mathcal{L}(t,\b{u}(t),\b{\dot{u}}(t))-D_{\b{y}}\mathcal{L}(t,\b{u}_0(t),\dot{\b{u}_0}(t))\right)\cdot \b{\dot{v}}(t)\right\}dt \\
      &\leq \left\{\| D_{\b{x}}\mathcal{L}(\cdot,\b{u}(\cdot),\b{\dot{u}}(\cdot))-D_{\b{x}}\mathcal{L}(\cdot,\b{u}_0(\cdot),\dot{\b{u}_0}(\cdot))\|_{L^1}\|\b{v}\|_{L^{\infty}}\right.\\
      &\quad+\left.
       \| D_{\b{y}}\mathcal{L}(\cdot,\b{u}(\cdot),\b{\dot{u}}(\cdot))-D_{\b{y}}\mathcal{L}(\cdot,\b{u}_0(\cdot),\dot{\b{u}_0}(\cdot))\|_{\lpsi}\|\b{\dot{v}}\|_{\lphi}\right\}\\
  \end{split}
\]
Taking supremum on $\b{v}$ with $\|\b{v}\sobnor\leq 1$ in the previous inequality and using Lemma \ref{inclusion orlicz} we obtain
% \[\begin{split}
%    \|I'(\b{u})-I'(\b{u}_0)\|_{\wphi} &\leq  C\bigg(\| D_{\b{x}}\mathcal{L}(\cdot,\b{u}(\cdot),\b{\dot{u}}(\cdot))-D_{\b{x}}\mathcal{L}(\cdot,\b{u}_0(\cdot),\dot{\b{u}_0}(\cdot))\|_{L^1}\right.\\
%      & + \| D_{\b{y}}\mathcal{L}(\cdot,\b{u}(\cdot),\b{\dot{u}}(\cdot))-D_{\b{y}}\mathcal{L}(\cdot,\b{u}_0(\cdot),\dot{\b{u}_0}(\cdot))\|_{\lpsi}\bigg)
%   \end{split}
%   \]
Therefore the results follows  of the  previously established continuity for $D_{\b{x}}\mathcal{L}$ and $D_{\b{y}}\mathcal{L}$.
\end{proof}



\section{Critical points and Euler-Lagrange equations}


In this section we derive the Euler-Lagrange equations associated to critical points of action integrals.


% \begin{equation}\label{ecualagran}
%     \left\{%
% \begin{array}{ll}
%    \frac{d}{dt} D_{y}\mathcal{L}(t,\b{u}(t),\b{\dot{u}}(t))= D_{\b{x}}\mathcal{L}(t,\b{u}(t),\b{\dot{u}}(t)) \quad \hbox{a.e.}\ t \in (0,T)\\
%     \b{u}(0)-\b{u}(T)=\b{\dot{u}}(0)-\b{\dot{u}}(T)=0.
% \end{array}%
% \right.
% \end{equation}

We denote by $\wphi_T$ the subspace of $\wphi$ of all functions $T$-periodic. Similarly we consider the subspaces $\ephi_T$, $\lphi_T$. As is usual, when $Y$ is a subspace of
the Banach space $X$, we denote by $Y^{\perp}$ the subspace of $X^*$ of all the bounded linear functions which are identically zero on $Y$.

We recall that  a function $f: \mathbb{R}^n \to \mathbb{R}$ is called \emph{strictly convex} if $f\left(\tfrac{\b{x}+\b{y}}{2}\right)< \tfrac{1}{2} \left(f\left(
\b{x}\right)+f\left( \b{y}\right)\right)$.  It is a well known that if $f$ is a strictly convex and differentiable functions then
$D_{\b{x}}f:\mathbb{R}^n\to\mathbb{R}^n$ is a one-to-one map  (see, for instance \cite[Theorem 12.17]{rockafellar2009variational}).


\begin{thm} The following statements are equivalent
\begin{enumerate}
 \item $I'(\b{u})\in\left( \wphi_T\right)^{\perp}$
 \item  $D_{\b{y}}\mathcal{L}(t,\b{u}(t),\b{\dot{u}}(t))$ is an absolutely continuous function and $\b{u}$ solve the following boundary value problem
 \begin{equation}\label{ecualagran2}
    \left\{%
\begin{array}{ll}
   \frac{d}{dt} D_{y}\mathcal{L}(t,\b{u}(t),\b{\dot{u}}(t))= D_{\b{x}}\mathcal{L}(t,\b{u}(t),\b{\dot{u}}(t)) \quad \hbox{a.e.}\ t \in (0,T)\\
    \b{u}(0)-\b{u}(T)=D_{\b{y}}\mathcal{L}(0,\b{u}(0),\b{\dot{u}}(0))-D_{\b{y}}\mathcal{L}(T,\b{u}(T),\b{\dot{u}}(T))=0.
\end{array}%
\right.
\end{equation}
\end{enumerate}
Moreover if $D_{\b{y}}\mathcal{L}(t,x,y)$ is $T$-periodic with respect to the variable $t$ and strictly convex with respect to $\b{y}$, then
$D_{\b{y}}\mathcal{L}(0,\b{u}(0),\b{\b{\dot{\b{u}}}}(0))-D_{\b{y}}\mathcal{L}(T,\b{u}(T),\b{\dot{u}}(T))=0$ is equivalent to $\b{\dot{u}}(0)=\b{\dot{u}}(T)$.
\end{thm}

\begin{proof} The condition $I'(\b{u})\in\left( \wphi_T\right)^{\perp}$ means that for every $\b{v}\in \wphi_T$ we have $\langle \b{v}, I'(\b{u})\rangle=0$. According to Theorem
\ref{teorema_acotacion} we have

\[\int_0^T\langle D_{\b{y}} \mathcal{L}(t,\b{u}(t),\b{\dot{u}}(t)), \b{\dot{v}}(t)\rangle dt=-\int_0^T \langle D_{\b{x}}\mathcal{L}(t,\b{u}(t),\b{\dot{u}}(t)),\b{ v}(t)\rangle dt \]
Using \cite[pag. 6]{mawhin2010critical} we obtain that  $D_{\b{y}}\mathcal{L}(t,\b{u}(t),\b{\dot{u}}(t))$ is absolutely continuous and $T$-periodic, therefore it is differentiable a.e.on $[0,T]$ and the first equality of \eqref{ecualagran2} holds true.
This complete the proof  1. implies 2. The proof of 2.implies 1.  is still easier and so  we will omit it.

The last part of the Corollary is a consequence of that $D_{\b{y}}\mathcal{L}(T,\b{u}(T),\b{\dot{u}}(T))=D_{\b{y}}\mathcal{L}(0,\b{u}(0),\b{\dot{u}}(0))=D_{\b{y}}\mathcal{L}(T,u(T),\b{\dot{u}}(0))$ and the injectivity of $D_{\b{y}}\mathcal{L}(T,u(T),\cdot)$.
\end{proof}

\section{Weak lower semicontinuity of actions integrals}

 \begin{lem}\label{unif_conv}
If the sequence $\{\b{u}_{k}\}_{k \geq 1}$ converges weakly to $\b{u}$ in $\wphi$, then $\{\b{u}_{k}\}_{k\geq 1}$ converges uniformly to $\b{u}$ on $[0,T]$.
\end{lem}
\begin{proof}
By Lemma \ref{inclusion orlicz}, the injection of $\wphi$ in $L^{\infty}$ is continuous. Since $\b{u}_{k}\rightharpoonup \b{u}$ in $\wphi$ it follows that
$\b{u}_{k}\rightharpoonup \b{u}$ in $C(0,T;\mathbb{R}^n)$. Since $\b{u}_{k}\rightharpoonup \b{u}$ in $\wphi$, we know that $\{\b{u}_{k}\}_{k \geq 1}$ is bounded in
$\wphi$ and, hence by \eqref{estimacion} in $C(0,T;\mathbb{R}^n)$. Moreover, the sequence $\{\b{u}_{k}\}_{k \geq 1}$ is equi-uniformly continuous since, for $0 \leq
s\leq t \leq T $, we have
\[
\begin{split}
\left|\b{u}_{k}(t)-\b{u}_{k}(s) \right|&\leq \int_{s}^t \left| \dot{\b{u}}_{k}(\tau)\right|\ \ d\tau \leq \| t-s\|_{\lpsi}\|\dot{\b{u}}_{k}\|_{\lphi}\\
&\leq \| t-s\|_{\lpsi}\|\b{u}_{k}\|_{\wphi} \leq C \| t-s\|_{\lpsi}.
\end{split}
\]
By Arzela-Ascoli theorem, $\{\b{u}_{k}\}_{k \geq 1}$ is relatively compact in $C(0,T;\mathbb{R}^n)$. By the uniqueness of the weak limit in $C(0,T;\mathbb{R}^n)$,
every uniformly convergent subsequence of $\{\b{u}_{k}\}_{k \geq 1}$ converges to $\b{u}$. Thus, $\{\b{u}_{k}\}_{k \geq 1}$ converges uniformly on $[0,T]$.

\end{proof}

\begin{thm}
We suppose that $\mathcal{L}(t,\b{x},\b{y})$ is a Charateodory functions satisfying \eqref{cotaL}-\eqref{cotaDyL}.
Moreover we assume $\mathcal{L}(t,\b{x},\cdot)$ is convex for each $t,\b{x}$. We suppose that $\Phi,\Psi$ are $\Delta_2$ functions. Then the functional \eqref{integral_accion} is weakly lower semicontinuous (w.l.s.c.).
\end{thm}


\begin{proof} We fix any $\b{u}\in\wphi$. What we must prove that for any sequence $\{\b{u}_n\}$ with $\b{u}_n\rightharpoonup \b{u}$ in $\wphi$ we have that $I(\b{u})\leq \liminf_n I(\b{u}_n)$. We write
\[
\begin{split}I(\b{v})&=\int_0^T\mathcal{L}(t,\b{v}(t),\b{\dot{v}}(t))dt\\
 &=\int_0^T\mathcal{L}(t,\b{v}(t),\b{\dot{v}}(t))-\mathcal{L}(t,\b{u}(t),\b{\dot{v}}(t))dt +\int_0^T\mathcal{L}(t,\b{u}(t),\b{\dot{v}}(t))dt\\
 &=:J(\b{v})+H(\b{v}).
\end{split}
\]
 As $\{\b{u}_n\}$ is a weakly convergent sequence,  by the Lemma \ref{unif_conv}  we have that $\b{u}_n\to \b{u}$ in $L^{\infty}$. By the mean value theorem for derivatives, we obtain
 a function $\b{\xi_n}(t)$, with $\b{\xi_n}(t)$ belonging to line segment joining $\b{u}_n(t)$ and $\b{u}(t)$, such that
 \begin{equation}\label{opJ}
 \begin{split}
   &\left|  \mathcal{L}(t,\b{u}_n(t),\b{\dot{u}}_n(t))-\mathcal{L}(t,\b{u}(t),\b{\dot{u}}_n(t))\right|\\
  &\leq|D_{\b{x}}\mathcal{L}(t,\b{\xi_n}(t),\b{\dot{u}}_n(t))||\b{u_n}(t)-\b{u}(t)|.
  \end{split}
 \end{equation}
The functions $\b{u_n}$, and therefore the functions $\b{\xi_n}$, are uniformly bounded in $L^{\infty}$. Thus, there exists $C>0$ such that $a(|\b{\xi_n}(t)|)\leq C $. Then,
using \eqref{cotaDxL} we get
\begin{equation}\label{acot_Dx}
 \left|D_{\b{x}}\mathcal{L}(t,\b{\xi_n}(t),\b{\dot{u}}_n(t))\right|\leq C \left(b(t)+\Phi(|\b{\dot{u}}_n(t))|)\right)
\end{equation}
Since $\Phi$ is a function of the $\Delta_2$ class, we have that the operator $\b{v}\mapsto\Phi(|\b{v}|)$  acts from $\lphi$ in $L^1$. Therefore, by
\cite[Lemma 17.4]{KR} we have that $\{\Phi(|\b{v}|): \|\b{v}\|_{\lphi}\leq r\}$ is bounded in $ L^1$ for any $r>0$. Hence
there exists a constant $C>0$ such that $\|\Phi(|\b{\dot{u}}_n(t))|)\|_{L^1}\leq C$. Then, from
\eqref{opJ}, \eqref{acot_Dx}, H\"older inequality and since $\|\b{u}_n-\b{u}\|_{L^{\infty}}\to 0$ and $b\in L^1$ we get $J(\b{u}_n)\to 0    $.

Now we will prove that $H(\b{v})$ is w.l.s.c. Since $H(\b{v})$ is convex it is sufficient to prove that $H$ is l.s.c (see \cite[Proposition 4.26]{fonseca2007modern}).  We suppose that  $\|\b{v_n}- \b{v}\sobnor\to 0$.

There exists $s=s_{n,t}\in[0,1]$
such that
\[|\mathcal{L}(t,\b{u}(t),\b{\dot{v}}_n(t))-\mathcal{L}(t,\b{u}(t),\b{\dot{v}}(t))|\leq|D_{\b{y}}\mathcal{L}(t,\b{u}(t),(1-s)\b{\dot{v}}_n(t)+s \b{\dot{v}}(t))||\b{\dot{v}}_n-\b{\dot{v}}|.\]
Let $\mathfrak{G}_n$ be the set $\{|\b{\dot{v}}_n(t)|\geq |\b{\dot{v}}(t)|\}$. Then
\[|(1-s)\b{\dot{v}}_n(t)+ s\b{\dot{v}}(t)|\leq \max\{|\b{\dot{v}}_n(t)|,|\b{\dot{v}}(t)|\}=\chi_{\mathfrak{G}_n}(t)|\b{\dot{v}}_n(t)|+
 \chi_{\mathfrak{G}_n^c}(t)|\b{\dot{v}}(t)|
\]
Therefore, using \eqref{cotaDyL} and taking account that $a(|\b{u}(t)|)\in L^{\infty}$ we get
\[\begin{split}|\mathcal{L}(t,\b{u}(t),\b{\dot{v}}_n(t))-\mathcal{L}(t,\b{u}(t),\b{\dot{v}}(t))|&\leq C\left(c(t)+\varphi(\chi_{\mathfrak{G}}|\b{\dot{v}}_n(t)|+
 \chi_{\mathfrak{G}^c}|\b{\dot{v}}(t)|)\right)|\b{\dot{v}}_n-\b{\dot{v}}|\\
 &=C\left(c(t)+\varphi(\chi_{\mathfrak{G}}|\b{\dot{v}}_n(t)|)+\varphi(
 \chi_{\mathfrak{G}^c}|\b{\dot{v}}(t)|)\right)|\b{\dot{v}}_n-\b{\dot{v}}|\\
 &\leq C\left(c(t)+\varphi(|\b{\dot{v}}_n(t)|)+\varphi(
 |\b{\dot{v}}(t)|)\right)|\b{\dot{v}}_n-\b{\dot{v}}|.
 \end{split}
 \]

 Now, in virtue of \cite[Lemma 9.1]{KR}, \cite[Lemma 17.1]{KR}, \cite[Theorem 17.4]{KR} and the uniform boundedness of $\b{\dot{u}}_n$ in
 $\lphi$ we have
 \[|H(\b{v}_n)-H(\b{v})|\leq C\|\b{\dot{v}}_n-\b{\dot{v}}\orlnor\to 0.\]
 Which completes the proof.
\end{proof}

\section{Coercivity discussion}

\begin{thm} 
\end{thm}


%Sea $\mathcal{L}(t, \b{x}, \b{y})=F(t,\b{x})+\Phi(|\b{y}|)$.

We consider the problem   (introduced in \eqref{ecualagran2}):
\begin{equation}\label{ecualagran3}
\frac{d}{dt} D_{y}\mathcal{L}(t,\b{u}(t),\b{\dot{u}}(t))= D_{\b{x}}\mathcal{L}(t,\b{u}(t),\b{\dot{u}}(t)) \quad \hbox{a.e.}\ t \in (0,T).
\end{equation}

\begin{nota_fer}
Me parece que vamos a tener que presentar el resultado siguiente de otra forma. Como est� escrito, resultar� un caso particular de otros teoremas. Tener en cuenta que, al fin y al cabo, se termina pidiendo que $\Phi,\Psi\in\Delta_2$, con lo cual se aplicar�an los resultados que escribe Sonia m�s abajo y la funcional ser�a coercitiva sin necesidad de la hip�tesis \eqref{cota}.  Para poder mantener el resultado de abajo, habr�a que contextualizarlo en un teorema donde no se pida  $\Phi,\Psi\in\Delta_2$. Se me ocurre que un buen lugar ser�a despu�s del comentario que sigue a la estimaci�n \eqref{eq:no_coerciva}. Pues all� se habla del papel que juega la constante $C$ y me parece bueno relacionarlo con eso.

Todo el razonamiento hasta llegar a la estimaci�n \eqref{eq:est_abajo_I} se puede tomar para usarlo con lo que escribi� Sonia. Justamente lo que demostr� Sonia es que la cota inferior en  \eqref{eq:est_abajo_I} tiende a infinito cuando $\|\b{\dot{u}}\orlnor\to\infty$.\textbf{ Eso si, habr�a que intetar generalizar este razonamiento, sin pedir la condici�n $\mathcal{L}(t,\b{x},\b{y})=\Phi(|\b{y}|)+F(t,\b{x})$.} Algo habr� que pedirle a $\mathcal{L}$, intentar que sea alguna desigualdad, en lugar de adjudicarle una forma m�s expl�cita. 

\end{nota_fer}

\begin{thm}[Theorem 1.5 M-W]
Sea $\mathcal{L}(t,\b{x},\b{y})=\Phi(|\b{y}|)+F(t,\b{x})$. We suppose that $\Phi,\Psi$ are $\Delta_2$ functions and that there exists $f \in L^1$  such that
\[
\left|\nabla F(t,\b{x}) \right|\leq f(t)
\]
for a.e. $t \in [0,T]$ and all $\b{x}\in \mathbb{R}^n $ with $f$ satisfying
\begin{equation}\label{cota}
\|f\|_{L^1([0,T])}\leq\frac{1}{2\|1\|_{\lpsi([0,T])}}.
\end{equation}
If
\begin{equation}\label{propiedad1coercividad}
\int_{0}^{T}F(t,\b{x})\ dt \rightarrow \infty \quad \hbox{as} \quad |\b{x}|\rightarrow \infty,
\end{equation}
then problem \eqref{ecualagran3} has at least one solution which minimizes the functional $I$ given by \eqref{integral_accion} on $\wphi$.
\end{thm}

\begin{proof}
For $\b{w} \in \wphi$, we write $\b{w}=\overline{\b{w}}+\widetilde{\b{w}}$ where $\overline{\b{w}} =\frac1T\int_0^T \b{w}(t)\ dt$.


\begin{comment}
\textbf{Case 1}:
\[
\begin{split}
I(\b{u})=&\int_{0}^T \mathcal{L}(t,\b{u}(t),\dot{\b{u}}(t))\ dt\\
=& \int_{0}^T (\mathcal{L}(t,\b{u}(t),\dot{\b{u}}(t))- \mathcal{L}(t,\overline{\b{u}},\dot{\b{u}}(t)))\ dt + \int_{0}^T
(\mathcal{L}(t,\overline{\b{u}},\dot{\b{u}}(t))-\mathcal{L}(t,\overline{\b{u}},\overline{\dot{\b{u}}}))\ dt\\
&+ \int_0^T \mathcal{L}(t,\overline{\b{u}},\overline{\dot{\b{u}}})\ dt\\
=&\int_{0}^T \int_{0}^{1} \langle D_{\b{x}}\mathcal{L}(t,\overline{\b{u}}+s\widetilde{\b{u}}(t),\dot{\b{u}}(t)),\widetilde{\b{u}}(t)\rangle\ ds \ dt +
\int_{0}^T \int_{0}^{1} \langle D_{\b{y}}\mathcal{L}(t,\overline{\b{u}},\dot{\b{u}}(t)+s\widetilde{\dot{\b{u}}}(t)),\widetilde{\dot{\b{u}}}(t)\rangle\ ds \ dt\\
&+  \int_{0}^T \mathcal{L}(t,\overline{\b{u}},\overline{\dot{\b{u}}})\ dt  \\
\geq &-C_1\|f\|_{L^1}\|\widetilde{\b{u}}\|_{L^{\infty}}-C_2\|g\|_{\lpsi}\|\widetilde{\dot{\b{u}}}\|_{\lphi} +  \int_{0}^T \mathcal{L}(t,\overline{\b{u}},\overline{\dot{\b{u}}})\ dt  \\
\geq &-C_1\|f\|_{L^1}\|\widetilde{\b{u}}\sobnor-C_2\|g\|_{\lpsi}\|\widetilde{\dot{\b{u}}}\|_{\lphi} +  \int_{0}^T \mathcal{L}(t,\overline{\b{u}},\overline{\dot{\b{u}}})\ dt  \\
\geq & -C_1\|f\|_{L^1}\|\dot{\widetilde{\b{u}}}\|_{\lphi} -C_2\|g\|_{\lpsi}\|\widetilde{\dot{\b{u}}}\|_{\lphi} +  \int_{0}^T
\mathcal{L}(t,\overline{\b{u}},\overline{\dot{\b{u}}})\ dt\\
\geq & -C_1\|f\|_{L^1}\|\dot{\b{u}}\|_{\lphi} -C_2\|g\|_{\lpsi}\|\widetilde{\dot{\b{u}}}\|_{\lphi} + {\color{celeste}\int_{0}^T
\mathcal{L}(t,\overline{\b{u}},\overline{\dot{\b{u}}})\ dt}.
\end{split}
\]
\end{comment}




\[
\begin{split}
I(\b{u})=&\int_{0}^T \mathcal{L}(t,\b{u}(t),\dot{\b{u}}(t))\ dt\\
=& \int_{0}^T (\mathcal{L}(t,\b{u}(t),\dot{\b{u}}(t))- \mathcal{L}(t,\overline{\b{u}},\dot{\b{u}}(t)))\ dt + \int_{0}^T
\mathcal{L}(t,\overline{\b{u}},\dot{\b{u}}(t))\ dt\\
=&\int_{0}^T \int_{0}^{1} \langle D_{\b{x}}\mathcal{L}(t,\overline{\b{u}}+s\widetilde{\b{u}}(t),\dot{\b{u}}(t)),\widetilde{\b{u}}(t)\rangle\ ds \ dt + + \int_{0}^T
\mathcal{L}(t,\overline{\b{u}},\dot{\b{u}}(t))\ dt\\
\geq &-\|f\|_{L^1}\|\widetilde{\b{u}}\|_{L^{\infty}} + \int_{0}^T
\mathcal{L}(t,\overline{\b{u}},\dot{\b{u}}(t))\ dt\\
\geq & -C_1\|f\|_{L^1}\|\dot{\widetilde{\b{u}}}\|_{\lphi} + \int_{0}^T
\mathcal{L}(t,\overline{\b{u}},\dot{\b{u}}(t))\ dt\\
\geq & -C_1\|f\|_{L^1}\|\dot{\b{u}}\|_{\lphi} {\color{celeste}+ \int_{0}^T \mathcal{L}(t,\overline{\b{u}},\dot{\b{u}}(t))\ dt}.
\end{split}
\]

Cuando tomamos $\mathcal{L}$
\[
\mathcal{L}(t,\b{x},\b{y})=\Phi(|\b{y}|)+F(t,\b{x}).
\]

\begin{comment}
Para caso 1:
\[
I(\b{u})\geq-C_1\|f\|_{L^1}\|\dot{\b{u}}\|_{\lphi} -C_2\|g\|_{\lpsi}\|\widetilde{\dot{\b{u}}}\|_{\lphi}+\int_{0}^T
\Phi(|\overline{\dot{\b{u}}}|)+F(t,\overline{\b{u}}).
\]
\end{comment}

Tenemos
\begin{equation}\label{eq:est_abajo_I}
I(\b{u})\geq-C_1\|f\|_{L^1}\|\dot{\b{u}}\|_{\lphi}+\int_{0}^T \Phi(|\dot{\b{u}}|)+F(t,\overline{\b{u}}).
\end{equation}
Para $\b{w}$ denotemos por $\|| \b{w} \||_{\lphi}$ la norma de Luxembourg de $\b{w}$.

Es sabido que que si $\|| \b{w} \||_{\lphi} \geq 1$ entonces $\int_{0}^T \Phi(|\b{w}|) \geq \|| \b{w} \||_{\lphi}$ y que $\|| \b{w}\||_{\lphi} \leq
\|\b{w}\|_{\lphi}\leq 2\|| \b{w}\||_{\lphi}$. Entonces se tiene para $\||\b{u}\|| \geq 1$

\[
\begin{split}
I(\b{u})&\geq-C_1\|f\|_{L^1}\|\dot{\b{u}}\|_{\lphi}+\int_{0}^T \Phi(|\dot{\b{u}}|)+F(t,\overline{\b{u}})\\
&\geq -C_1\|f\|_{L^1}\|\dot{\b{u}}\|_{\lphi}+\|| \dot{\b{u}} \||_{\lphi}+\int_{0}^TF(t,\overline{\b{u}})\\
&\geq-C_1\|f\|_{L^1}\|\dot{\b{u}}\|_{\lphi}+\frac{1}{2}\| \dot{\b{u}} \|_{\lphi}+\int_0^TF(t,\overline{\b{u}})\\
&= \left(-C_1\|f\|_{L^1}+\frac{1}{2}\right)\|\dot{\b{u}}\|_{\lphi}+\int_{0}^{T}F(t,\overline{\b{u}}).
\end{split}
\]
Therefore, if $\|\b{u}\sobnor \rightarrow \infty$ then $(|\overline{\b{u}}|^2+ \|\dot{\b{u}}\|_{\lphi})\rightarrow \infty$. If $|\overline{\b{u}}|^2 \rightarrow
\infty$ by \eqref{propiedad1coercividad} we have $I(\b{u}) \rightarrow \infty$. If $\|\dot{\b{u}}\|_{\lphi}\rightarrow \infty$ usamos \eqref{cota}. En efecto,

Si $-C_1\|f\|_{L^1}+\frac{1}{2} \geq 0$, entonces $I(\b{u}) \rightarrow \infty$ cuando $\|\dot{\b{u}}\|_{\lphi}\rightarrow \infty$. Pero $-C_1\|f\|_{L^1}+\frac{1}{2}
\geq 0$ si y solamente si $\|f\|_{L^1([0,T])}\leq\frac{1}{2 C_1}=\frac{1}{2\|1\|_{\lpsi([0,T])}}$, lo cual es justamente \eqref{cota}.
 ($C_1=\|1\|_{\lpsi([0,T])}$ por \eqref{estimacion2}).
\end{proof}





In the sequel, we will discuss  the  conditions that guarantee the coercivity of  the functional $u \to \int_0^T \Phi(|u|)\,dx$ in $L^{\Phi}$.

\begin{nota_fer}
\textbf{Tenemos la conejetura que todo sale si pedimos $\Phi$ que satisface $\nabla_2$. Problema: probarla o refutarla} Un problema m�s amplio ser�a tratar de sacar hip�tesis ya sea la condici�n $\Delta_2$ de alguna de las funciones ($\Phi$ o $\Psi$) o usar estas condiciones para el infinito o a�n para cero(????).
\end{nota_fer}


We know that  $\Phi\in \Delta_2$ if and only if 
for every $\epsilon>0$ there exists $C_\epsilon>0$ such that
\begin{equation}\label{delta2-control-potencia}
C^{-1}_\epsilon \min\{\lambda^{\alpha-\epsilon},\lambda^{\beta+\epsilon}\} \Phi(x)
\leq \Phi(\lambda x)\leq
C_\epsilon \max\{\lambda^{\alpha-\epsilon},\lambda^{\beta+\epsilon}\} \Phi(x)
\end{equation}
for every $x,\lambda>0$.

Recall that by Theorem 11.11 in \cite{M} we have $p\leq \alpha\leq \beta\leq q$ where $\alpha,\beta$ are lower and upper Orlicz indices, and $p,q$ are lower and upper Simonenko indices.
We also have that $\Phi,\Psi\in \Delta_2$ implies that $\Phi(x)\sim x\varphi(x)\sim \Psi(\varphi(x))$.

By Theorem 4.3 in \cite{KR} or Corollary 4.4 in \cite{rao1991theory} we know that $p>1$ if and only if $\Phi \in \nabla_2$ and by  Theorem 11.7 in \cite{M} we have that $\beta<\infty$ if and only if $\Phi \in \Delta_2$.

Theorem 10.4 in \cite{KR} says that if there exists $k^*$ such that $\int \Psi[\varphi(k^*|u|)]\,dx=1$, then $\|u{\orlnor}=\int \varphi(k^*|u|)|u|\,dx$.

\begin{comentario}
If $\varphi$ is a continuous function, then there exists $k^*$ in Theorem 4.3 of \cite{KR}. See \cite[pages 89, 90]{KR}.
\end{comentario}












\begin{lem}\label{kn_0} Let $\Phi\in \nabla_2\cap\Delta_2$ and  $\{k_n\}$ such that $\int \Psi[\varphi(k_n|u_n|)]\,dx=1$. If
$\|u_n{\orlnor}\to\infty $ as $n \to \infty$, then $k_n \to 0$ as $n \to \infty$.
\end{lem}


\begin{proof}
Since $\Phi\in \nabla_2\cap\Delta_2$, then $\Psi \circ \varphi\in \Delta_2$ and \eqref{delta2-control-potencia} holds.

Now, for such a sequence $k_n$ we have
\begin{equation}
1=\int \Psi[\varphi(k_n|u_n|)]\,dx\geq 
C_{\epsilon}^{-1}\min\{k_n^{\alpha-\epsilon},k_n^{\beta+\epsilon}\}\int \Psi(\varphi(|u_n|))\,dx
\end{equation}
for every $\epsilon>0$. Due to $\Phi \in \Delta_2$, there exists $C_{\Lambda}>0$ such that 
\begin{equation}
C_{\epsilon}^{-1}\min\{k_n^{\alpha-\epsilon},k_n^{\beta+\epsilon}\}\int \Psi(\varphi(|u_n|))\,dx
\geq 
C_{\Lambda}C^{-1}_{\epsilon}\min\{k_n^{\alpha-\epsilon},k_n^{\beta+\epsilon}\}
\int \Phi(|u_n|)\,dx
\end{equation}
for every $\epsilon>0$. 
\\
As $\|u_n{\orlnor}\to\infty $ as $n \to \infty$, there exists $N\in \nn$ such that $\|u_n{\orlnor}\geq 1$ for every $n> N$
and consequently 
$\int \Phi(|u_n|)\,dx\geq \||u_n\||_{\lphi}$ for every $n> N$. 
\\
In that way, we have 
\begin{equation}
1 \geq C_{\Lambda}C_{\epsilon}^{-1}\min\{k_n^{\alpha-\epsilon},k_n^{\beta+\epsilon}\}
\||u_n\||_{\lphi}
\end{equation}
for every $\epsilon>0$ and for every $n>N$. 
\\
{\color{violeta} In addition, $\alpha>1$ because $\Phi \in \nabla_2$}, then $\alpha-	\epsilon>0$ for every $0<\epsilon<1$.
\\
Now, we have
\begin{equation}
\frac{1}{C_{\Lambda}C_{\epsilon}^{-1} \|| u_n\||_{ \lphi}}\geq \min\{k_n^{\alpha-\epsilon},k_n^{\beta+\epsilon}\}
\end{equation}
for every $n>N$ and provided that $\alpha-\epsilon>0$.
\\
Because of the 
equivalence between Orlicz and Luxemburg norms, we have 
$\||u_n\||_{\lphi}\to \infty$ as $n \to \infty$, 
then
\begin{equation}
\min\{k_n^{\alpha-\epsilon},k_n^{\beta+\epsilon}\}\to 0\;\;\mbox{as}\;\;n\to \infty
\end{equation}
where $\alpha-\epsilon>0$ and $\beta+\epsilon>0$; therefore,  $k_n \to  0$ as $n \to \infty$.
\end{proof}




{\bf El resultado que sigue es v\'alido para cualquier funci\'on u?????}
\begin{nota_fer}  Si me parece que habr�a que decir que entendemos por coercitivo en el teorema de abajo. Yo creo que la demostraci�n \textbf{Problema: verificarlo y escribarlo} implica el sentido m�s fuerte de coercitividad
\[\lim_{\|u\orlnor\to\infty}\frac{\int_0^T\Phi(|u_n|)dx}{\|u\orlnor}=\infty\]
\end{nota_fer}

\begin{thm}
If $\Phi\in \Delta_2\cap \nabla_2$, then 
the functional $u \to \int_0^t \Phi(|u|)\,dx$ is  coercive in $L^\Phi$.
\end{thm}

\begin{proof}
Assume $\|u_n{\orlnor}\to\infty $ when $n\to \infty$ and 
{\color{violeta} suppose that there exists $\{k_n\}$} such that 
%\linebreak 
$\int \Psi[\varphi(k_n|u_n|)]\,dx=1$.
\\
In that way, by Theorem 10.4 in \cite{KR}, we have
\begin{equation}
\int \Phi(|u_n|)\,dx-C\|u_n{\orlnor}=
\int \Phi(|u_n|)\,dx-C\int \varphi(k_n|u_n|)|u_n|\,dx
\end{equation}
As $\Phi \in \Delta_2$, there exists $\Lambda_{\Phi}>0$ such that
\begin{equation}
\int \Phi(|u_n|)\,dx-\frac{C}{k_n}\int \varphi(k_n|u_n|)|u_n|k_n\,dx\geq 
\int \Phi(|u_n|)\,dx-\frac{C\Lambda_{\Phi}}{k_n}\int \Phi(k_n|u_n|)\,dx\
\end{equation}
%%
{\color{violeta} Since $\Phi\in \nabla_2$, then $\alpha>1$ \cite{KR,rao1991theory};
now, we choose $\epsilon>0$ such that $\alpha-\epsilon>1$.}
\\
In addition,  by Lemma \ref{kn_0},   
there exists $N \in \nn$ such that $k_n<1$ for every $n>N$ and therefore
\begin{equation}
\Phi(k_n |u_n|) \leq C_\epsilon k_n^{\alpha-\epsilon}\Phi(|u_n|)
\end{equation}
for every $n>N$ and where $\alpha-\epsilon>1$.
\\
Now, 
\begin{equation}
\int \Phi(|u_n|)\,dx-\frac{C\Lambda_{\Phi}}{k_n}
\int \Phi(k_n|u_n|)\,dx
\geq
\left(1-C\Lambda_{\Phi}C_\epsilon k_n^{\alpha-\epsilon-1}\right)\int \Phi(|u_n|)\,dx
\end{equation}
with $\alpha-\epsilon-1>0$.
\\
As $\|u_n{\orlnor}\to\infty $, there exists $N_2\in \nn$ such  $\|u_n\orlnor>1$ for every $n>N_2$, then
\begin{equation}
\left(1-C\Lambda_{\Phi}C_\epsilon k_n^{\alpha-\epsilon-1}\right)\int \Phi(|u_n|)\,dx
\geq 
\left(1-C\Lambda_{\Phi}C_\epsilon k_n^{\alpha-\epsilon-1}\right)\|| u_n|\|_{\lphi}
\end{equation}
with $\alpha-\epsilon-1>0$.
\\
Finally, due to $\|u_n{\orlnor}\to\infty $, 
we have $k_n^{\alpha-\epsilon-1}\to 0$, $\|| u_n|\|_{\lphi}\to \infty$ and consequently
$\int \Phi(|u_n|)\,dx-C\|u_n{\orlnor}\to \infty$ for every $C>0$; that is, 
the functional $u \to \int_0^t \Phi(|u|)\,dx$ is coercive in $L^\Phi$.
\end{proof}

\begin{nota_fer}
Lamentablemente el libro de Rao pag. 43 dice que $\nabla'$ implica $\nabla_2$. Luego la primera parte de este terorema es consecuencia de la proposici�n anterior.
 El otro inciso si me parece que tiene sentido pues para ver que no es coercitiva basta mostrar contraejemplos. Este comentario no aporta problemas abiertos, solo es una observaci�n de la redacci�n
\end{nota_fer} 


{\color{red}
Sonia: Totalmente de acuerdo, por eso escrib\'i la observaci\'on al final de la prueba. Si el resultado usando $\nabla_2$ es correcto, el primer inciso sobra.}

\begin{prop}
{\color{violeta} Let $u$ be a characteristic function.
\\
If $\Phi \in \Delta_2\cap\nabla'$, then the functional $u \to \int_0^t \Phi(|u|)\,dx$ is coercive in $L^\Phi$.}
\\
If $\Phi \in \nabla_3$, then the functional (la suelen llamar {\color{violeta} modular}) $u \to \int_0^t \Phi(|u|)\,dx$ is not coercive in $L^\Phi$.
\end{prop}

\begin{proof}
Let  $u=\alpha \chi_A$ with $\alpha \in \rr$  and where $A$ is a subset of $[0,T]$.
\\
Assume that 
$\|u\orlnor \to \infty$ then $\|u\orlnor=\alpha\|\chi_A\orlnor=\alpha m(A) \Psi^{-1}\left(\frac{1}{m(A)}\right) \to \infty$ and consequently $m(A) \Psi^{-1}\left(\frac{1}{m(A)}\right)\to  \infty$ as {\color{violeta} $m(A)\to 0$}.
\\

On the other hand, we have
\begin{equation}
\frac{\int \Phi(|u|)\,dx}{\|u\orlnor}=\frac{\Phi(\alpha)m(A)}{\alpha m(A) \Phi^{-1}(\frac{1}{m(A)})}=\frac{\Phi(\alpha)}{\alpha \Psi^{-1}(\frac{1}{m(A)})}.
\end{equation}
Let $r=\frac{1}{m(A)}\geq \frac{1}{T}$, then $\alpha \frac{\Psi^{-1}(r)}{r}\to \infty$ as $\alpha, r\to \infty$ and
\begin{equation}
\frac{\int \Phi(|u|)\,dx}{\|u\orlnor}=\frac{\Phi(\alpha)}{\alpha \Psi^{-1}(r)}
\end{equation}
As $\Phi \in \nabla '$, there exists  $C_1>0$ such that $\Phi(xy)\geq C_1\Phi(x)\Phi(y)$ for every $x,y>0$.
Taking $y=\frac{\alpha}{x}$, we have 
$
\Phi(\alpha)\geq C_1\Phi(x)\Phi(\frac{\alpha}{x})$ and choosing
$x=\Phi^{-1}(r)$ we get 
\begin{equation}\label{Phi-r-1}
\frac{\Phi(\alpha)}{r}\geq C_1 \Phi\left(\frac{\alpha}{\Phi^{-1}(r)}\right).
\end{equation}
\\
As $\Phi \in \Delta_2$, there exists $C_2>0$ such that 
\begin{equation}\label{Phi-r-2}
\Phi\left(\frac{\alpha}{\Phi^{-1}(r)}\right)\geq C_2 \frac{\alpha}{\Phi^{-1}(r)}\varphi\left(\frac{\alpha}{\Phi^{-1}(r)}\right).
\end{equation}
We also have that $r \leq \Phi^{-1}(r)\Psi^{-1}(r)\leq 2r$ for every $r>0$, then 
\begin{equation}\label{Phi-r-3}
C_2 \frac{\alpha}{\Phi^{-1}(r)}\varphi\left(\frac{\alpha}{\Phi^{-1}(r)}\right)\geq 
C_2\frac{\alpha \Psi^{-1}(r)}{2r}\varphi\left(\frac{\alpha \Psi^{-1}(r)}{2r}\right).
\end{equation}
Thus, from \eqref{Phi-r-1}-\eqref{Phi-r-3},
\begin{equation}
\frac{\Phi(\alpha)}{r}\geq C_1 C_2\frac{\alpha \Psi^{-1}(r)}{2r}\varphi\left(\frac{\alpha \Psi^{-1}(r)}{2r}\right);
\end{equation}
and then ({\bf Ac\'a ten\'ia que despejar y no irme por la ramas!!!)}


\begin{equation}
\frac{\Phi(\alpha)}{\alpha\Psi^{-1}(r)}\geq \frac{C_1 C_2}{2}\varphi\left(\frac{\alpha \Psi^{-1}(r)}{2r}\right).
\end{equation}
{\color{violeta} Due to $\alpha\frac{\Psi^{-1}(r)}{r}\to \infty$ as $\alpha,r\to \infty$} and 
the fact that $\varphi(x) \to \infty$ as $x \to \infty$, we obtain
\begin{equation}
\frac{C_1 C_2}{2}\varphi\left(\frac{\alpha \Psi^{-1}(r)}{2r}\right)\to \infty
\end{equation}
and consequently
\begin{equation}
\frac{\Phi(\alpha)}{\alpha\Psi^{-1}(r)}\to  \infty\;\;\mbox{as}\;\;\alpha,r\to \infty;
\end{equation}
that is, 
\begin{equation}
\frac{\int \Phi(|u|)\,dx}{\|u\orlnor}\to\infty\;\;\mbox{as}\;\;\|u\orlnor\to \infty.
\end{equation}
Now, we will see that  $\nabla_3$ functions do not imply the wished coercivity
in case of $u$ being a characteristic function.
\begin{nota_fer} Me parece que es posible ver que no es coercitiva en el sentido m�s debil que
\[\int \Phi(|u_n|)\,dx-C\|u_n{\orlnor}\]
no tiende a infinito.\textbf{Demostrarlo}
Por otra parte, notar que al final de la demostraci�n se toma $\alpha=r$, como lo que estamos mostrando es un contraejemplo, no es necesario mantener la distinci�n entre $\alpha$ y $r$. Podr�amos haber tomado de movida 
\[u=u_r=r\chi_{[0,1/r]},\]
\end{nota_fer}
To do so, suppose there exists a function  $\Phi \in \nabla_3$ such that 
\begin{equation}
\frac{\int \Phi(|u|)\,dx}{\|u\orlnor}=\frac{\Phi(\alpha)}{\alpha \Psi^{-1}(r)}\to \infty\;\;\mbox{as}\;\;\|u\orlnor\to \infty,
\end{equation}
then 
\begin{equation}\label{Phi-r-nabla3-1}
\frac{\int \Phi(|u|)\,dx}{\|u\orlnor}=\frac{\Phi(\alpha)}{\alpha \Psi^{-1}(r)} 
\leq 
\frac{\Phi(\alpha)\Phi^{-1}(r)}{\alpha r}\to \infty \;\mbox{as}\;\;\|u\orlnor\to \infty
\end{equation} 
for any $\alpha>0$.
\\
However,  $\Phi \in \nabla_3$ if and only if there exists $C_3>0$ such that
 \begin{equation}
 \frac {\Phi(r)\Phi^{-1}(r)} {r^2} \leq C_3\;\;\mbox{for every}\;\;r>0
 \end{equation}
 which contradicts \eqref{Phi-r-nabla3-1} choosing $\alpha=r$.
 \\
 Therefore,  if $\Phi\in \nabla_3$ then the functional $u \to \int_0^t \Phi(|u|)\,dx$ 
is not coercive in $L^{\Phi}$.
\end{proof}





\begin{rem}
$\nabla'$ functions are a subset of $\nabla_2$ functions and 
$\nabla_3$ functions belong to  the set of  $\Delta_2$ functions \cite[Chapter 2]{ rao1991theory}.
\end{rem}

We have just proved that 
coercivity  in $L^\Phi$ for characteristic functions holds when
$\Phi$ belongs  to a subclass of $\Delta_2$ functions.
\\
Nevertheless,  $\Delta_2$ condition is not sufficient  to get the wished coercivity in $L^\Phi$;
in fact, $\nabla_3$ functions, which belong to the set of $\Delta_2$ functions, do not imply coercivity in $L^\Phi$.

We can also see that $\Delta_2$ condition is not sufficient  to have coercivity in $L^\Phi$  providing a counterexample.
\begin{nota_fer} Deje de entender. Al fin y al cabo, el de abajo ser�a un contraejemplo para mostrar la misma cosa que ya vimos en otro contraejemplo? 


Pero se nos plantea este problema, la $\Phi$ del contraejemlo de abajo no ser� $\nabla_3$? Si fuese as� lo que decimos abajo ya lo sabemos \textbf{Problema abierto: probar o refutar que la $\Phi$ es $\nabla_3$}
\end{nota_fer}

{\color{red}
Sonia: En efecto, quise decir que como $\nabla_3$ implica $\Delta_2$, el ejemplo que viene a continuaci\'on es redundante para probar que $\Delta_2$ no implica coercividad.
\\
Por otra parte, en la p\'agina 39 del Rao dice que la funci\'on $\Psi$ que  usamos es $ \Delta_3$, luego $\Phi\in\nabla_3$ (ver Teorema 3 p\'agina 38 de Rao). Conclusi\'on: O ponemos el enunciado general para $\nabla_3$ o el contraejemplo que viene a continuaci\''on, porque dicen lo mismo.}

{\color{green}  Hay una diferencia entre el enunciado general y el ejemplo de abajo. El ejemplo de abajo es un contraejemplo a la afirmaci�n
\begin{equation}\label{eq:paraborrar}
\lim_{\|u\orlnor\to\infty}\int \Phi(|u_n|)\,dx-C\|u_n{\orlnor}=\infty
\end{equation}
Mientras que el resultado general lo es de
\begin{equation}\label{eq:paraborrar2}\lim_{\|u\orlnor\to\infty}\frac{\int_0^T\Phi(|u_n|)dx}{\|u\orlnor}=\infty
\end{equation}
Como \eqref{eq:paraborrar2} implica \eqref{eq:paraborrar},  el contraejemplo de abajo no es consecuencia del resultado general. Adem�s nosotros lo que necesitamos es \eqref{eq:paraborrar2}. Todo esto nos lleva a:

\textbf{Problema abierto (jaja): ver si el enunciado general  sigue sirviendo con \eqref{eq:paraborrar}
  }}



 In fact, suppose that $\Phi(r)=(r+1)\log(r+1)-r$  then $\Phi\in\Delta_2$; and, we also have $\Psi(r)=e^r-r-1$
which is not a $\Delta_2$ function \cite[Chapter 1, pp 22]{rao1991theory}.
\\
Let $u_n=\frac{1}{m(A_n)}\chi_{A_n}$ where $A_n \subset[0,T]$ such that $m(A_n)\to 0$.
\\
If $r_n=\frac{1}{m(A_n)}$, we have
\begin{equation}
\int \Phi(|u_n|)\,dx-C\|u_n{\orlnor}=\frac{\Phi(r_n)}{r_n}-C\Psi^{-1}(r_n)
\end{equation}
then 
\begin{equation}
\begin{split}
  \frac{\Phi(r_n)}{r_n}-C\Psi^{-1}(r_n) \leq \left(1+\frac{1}{r_n}\right)\log(r_n+1)-C\log(r_n+1)=
	\\
  \left(1+\frac{1}{r_n}-C\right)\log (r_n+1) 
	\end{split}
  \end{equation}
Now, as $r_n$ goes to $\infty$, we get 
 \begin{equation}\label{eq:no_coerciva}
  \lim\limits_{r_n \to \infty}
	\left(\frac{\Phi(r_n)}{r_n}-C\Psi^{-1}(r_n)\right)  \leq  \left(1-C\right) \infty.
  \end{equation}
In that way, 
$\int \Phi(|u_n|)\,dx-C\|u_n {\orlnor} $ is an  upper bounded function provided that $C>1$, which means that the functional
 $u \to \int_0^t \Phi(|u|)\,dx$ is not coercive in $L^\Phi$ for such a particular $\Delta_2$ function
$\Phi$.

{\bf Porque para tener coercividad, la expresi\'on no deber\'ia estar acotada para $C$ grande, es as\'i????}





\bibliographystyle{plain}
\bibliography{biblio}

%\printbibliography

\end{document}
