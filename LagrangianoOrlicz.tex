%%% Preambulo%%%%%%%%%%%%%%%%%%%%%%%%%

\documentclass[twoside]{article}
%%Paquetes


\usepackage{amsmath,amssymb,amsthm}
\usepackage{color}
\usepackage{ esint }
\usepackage{graphicx}
\usepackage{wrapfig}
\usepackage{subfigure}
\usepackage{fancyhdr}
\usepackage{times}
%\usepackage{theorem}
\usepackage[latin1]{inputenc}
%\usepackage{showkeys}
\usepackage{comment}


%Teorema y similes



\newtheorem{thm}{Theorem}[section]
\newtheorem{cor}[thm]{Corollary}
\newtheorem{lem}[thm]{Lemma}
\newtheorem{rem}[thm]{Remark}
\newtheorem{defi}{Definition}
\newtheorem{prop}[thm]{Proposition}



\title{Propuesta de trabajo 07-08 de noviembre 2013}
\author{Sonia Acinas \thanks{SECyT-UNRC}\\
Dpto. de Matem\'atica, Facultad de Ciencias Exactas y Naturales\\
Universidad Nacional de La Pampa\\
(6300) Santa Rosa, La Pampa, Argentina\\
sonia.acinas@gmail.com\\[3mm]
Leopoldo Buri \thanks{SECyT-UNRC}\\
Dpto. de Matem\'atica, Facultad de Ciencias Exactas, F\'{\i}sico-Qu\'{\i}micas y Naturales\\
Universidad Nacional de R\'{i}o Cuarto\\
(5800) R\'{\i}o Cuarto, C\'ordoba, Argentina,\\
lburi@exa.unrc.edu.ar\\[3mm]
Graciela Giubergia \thanks{SECyT-UNRC and CONICET}\\
Dpto. de Matem\'atica, Facultad de Ciencias Exactas, F\'{\i}sico-Qu\'{\i}micas y Naturales\\
Universidad Nacional de R\'{i}o Cuarto\\
(5800) R\'{\i}o Cuarto, C\'ordoba, Argentina,\\
ggiubergia@exa.unrc.edu.ar\\[3mm]
Fernando D. Mazzone \thanks{SECyT-UNRC and CONICET}\\
Dpto. de Matem\'atica, Facultad de Ciencias Exactas, F\'{\i}sico-Qu\'{\i}micas y Naturales\\
Universidad Nacional de R\'{i}o Cuarto\\
(5800) R\'{\i}o Cuarto, C\'ordoba, Argentina,\\
fmazzone@exa.unrc.edu.ar\\[3mm]
Erica L. Schwindt\thanks{ANR. AVENTURES - ANR-12-BLAN-BS01-0001-01}\\
Universit\'{e} d'{O}rl\'{e}ans, Laboratoire MAPMO, CNRS, UMR 7349, \\
F\'ed\'eration Denis Poisson, FR 2964,\\
B\^{a}timent de Math\'{e}matiques, BP 6759, 45067 Orl\'{e}ans Cedex 2, France,\\
leris98@gmail.com}

\date{}

\newcommand{\orlnor}{\|_{L^{\Phi}}}
\newcommand{\lurnor}{\|^{*}_{L^{\Phi}}}
\newcommand{\linf}{\|_{L^{\infty}}}
\newcommand{\lphi}{L^{\Phi}}
\newcommand{\lpsi}{L^{\Psi}}
\newcommand{\ephi}{E^{\Phi}}
\newcommand{\claseor}{\widetilde{L}^{\Phi}}
\newcommand{\wphi}{W^{1}\lphi}
\newcommand{\sobnor}{\|_{W^{1}\lphi}}
\newcommand{\domi}{W^{1}(\lphi,\Pi(\ephi,1))}

\begin{document}



\maketitle
%
\begingroup%Locallizing the change to `thefootnote'.
    \renewcommand{\thefootnote}{}%Removing the footnote symbol.
    %
    \footnotetext{%
    %   2010 Mathematics Subject Classification
    %   http://www.ams.org/msc/
    \textbf{2010  AMS Subject Classification.} Primary: .
    Secondary: .
    }%
        \footnotetext{%
    \textbf{Keywords and phrases.}  .
    }%
    \endgroup
%
%
%
%

\begin{abstract}
...
\end{abstract}




\pagestyle{fancy} \headheight 35pt \fancyhead{} \fancyfoot{}

\fancyfoot[C]{\thepage} \fancyhead[CE]{\nouppercase{S. Acinas, L. Buri, G. Giubergia, F. Mazzone and E. Schwindt}} \fancyhead[CO]{\nouppercase{\section}}

\fancyhead[CO]{\nouppercase{\leftmark}}


%\tableofcontents

\section{Preliminaries}

%\subsection{$N$-functions}
For reader convenience, a short introduction to Orlicz and Orlicz Sobolev spaces is given, we refer to \cite{adams_sobolev,KR} for additional details and proofs.

A function $\Phi:[0,+\infty)\to [0,+\infty)$ is called an $N$-function if it has the form
\[
\Phi(u)=\int_{0}^u \varphi(\tau)\ d\tau,\quad\hbox{for } u\geq 0,
\]
where $\varphi: [0, \infty)\rightarrow [0, \infty)$ is a right continuous nondecreasing function  satisfying   $\varphi(0)=0$, $\varphi(t)>0$ for $t>0$ and
$\lim_{t\rightarrow \infty}\varphi(t)=+\infty$.

Given a function $\varphi$ as above, we also consider the so-called right inverse function $\psi$ of $\varphi$ which is defined $\psi(v)=\sup_{\varphi(u)\leq v}u$.
The function $\psi$ satisfies the same properties that function $\varphi$, therefore we have an $N$-function $\Psi$ associated to $\psi$. We say that $\Psi$ is the
complementary function of $\Phi$.

Throughout this article we will try with spaces of $\mathbb{R}^n$ valued functions defined on an interval $[0,T]\subset\mathbb{R}$. Given an $N$-function $\Phi$ we
define the Orlicz class $\widetilde{L}^{\Phi}([0,T],\mathbb{R}^n)$ by
\begin{equation}\label{claseOrlicz}
  \widetilde{L}^{\Phi}([0,T],\mathbb{R}^n):=\left\{ u: [0,T] \rightarrow\mathbb{R}^n, u\ \hbox{mesurable},\ \int_{[0,T]} \Phi(|u|)\ dx < \infty \right\}.
\end{equation}
here $|\cdot|$ is the euclidean norm of $\mathbb{R}^n$. When clear from the context, we will omit the domain and codomain in the notation of function spaces and
classes ($\claseor=\widetilde{L}^{\Phi}([0,T],\mathbb{R}^n)$). The Orlicz space $\lphi=L^{\Phi}([0,T],\mathbb{R}^n)$ is defined as the linear hull of $\claseor$.
Equivalently
\begin{equation}\label{espacioOrlicz}
\lphi:=\left\{ u: [0,T] \rightarrow \mathbb{R}^n | u\ \hbox{is mesurable and},\ \int_{[0,T]} \Phi(\alpha|u|)\ dx < \infty \ \hbox{for some}\ \alpha >0   \right\}.
\end{equation}



The Orlicz space $\lphi$ equipped with the Orlicz norm
\[
\|  u  \orlnor:=\sup \left\{  \int_0^Tu\cdot vdt: \int_0^T\Psi(|v|)\leq 1\right\},
\]
is a Banach space. Here by $u \cdot v$ we denote the usual dot product in $\mathbb{R}^{n}$.

%Alternatively  the Luxemburg norm
%\[
%\|  u  \lurnor =\inf \left\{ k>0:  \int_{0}^T \Phi \left(\frac{|u|}{k}\right)\ dx \leq 1 \right\},
%\]
%defines an equivalent norm.



The space $\ephi=\ephi([0,T],\mathbb{R}^n)$ is defined as the closure in $\lphi$ of the subspace $L^{\infty}$. The space $\ephi$ is the maximal subspace of the
Orlicz class $\claseor$. It is known that $\left[\ephi\right]^*=\lpsi$.


We will use the set
\[\Pi(\ephi,r):=\{u\in\lphi: d(u,\ephi)<r\}.\]
This set is related to the Orlicz class $\claseor$ by means of inclusions
\[\Pi(\ephi,1)\subset \claseor \subset\overline{\Pi(\ephi,1)}.\]

Let $X$ and $Y$ be subsets of certain vector spaces of $\mathbb{R}^n$-valued measurable functions defined in $[0,T]$.   We call $W^1(X,Y)$ to the set defined by
\[W^1(X,Y):=\{u| u \hbox{ is absolutely continuous and } u\in X, \dot{u}\in Y\}.\]
If $X=Y$ we simply write $W^1(X,X)=W^1X$. In this paper $X$ and $Y$ will be some subset of an Orlicz space.  When $X=Y=\lphi$ we have the usual Sobolev-Orlicz space
 $\wphi$ (see \cite{adams_sobolev}) , which is a Banach space  equipped with the norm
\[
\|  u  \|_{\wphi}= \|  u  \|_{\lphi} + \|\dot{u}\orlnor.
\]



\begin{lem}\label{inclusion orlicz}
\[\wphi\hookrightarrow L^{\infty}\]
\end{lem}
\begin{proof}
Let $u\in \wphi$, then $u$ is absolutely continuous and there exists $\dot{u}(t)$ a.e. $t\in[0,T]$. From the mean value theorem there exists $\tau$ such that
$u(\tau)=\frac{1}{T}\int\limits_{0}^{T}u(s)ds$, thus
\[
u(t)=u(\tau)+\int\limits_{\tau}^{t}\dot{u}(s)ds.
\]
Thus we have
\begin{equation}\label{desigualdad1}\begin{split}
|u(t)|&\leqslant |u(\tau)|+\int\limits_{\tau}^{t}|\dot{u}(s)|ds\\
&\leqslant |u(\tau)|+\int\limits_{0}^{T}|\dot{u}(s)|ds\\
&\leqslant |u(\tau)|+\|\dot{u}\|_{L^{\Phi}}\|1\|_{L^{\Psi}}.
\end{split}
\end{equation}
Moreover
\begin{equation}\label{desigualdad2}\begin{split}
|u(\tau)|&\leqslant \frac{1}{T}\int\limits_{0}^{T}|u(s)|ds\\
&\leqslant \frac{1}{T}\|u\|_{L^{\Phi}}\|1\|_{L^{\Psi}}.
\end{split}
\end{equation}
From \eqref{desigualdad1} and \eqref{desigualdad2} we obtain
\[
\|u\|_{L^{\infty}}\leqslant C(T)\|u\|_{\wphi}.
\]

\end{proof}

\begin{lem}\label{segundo lema}
Let $\{u_n\}_{n\in \mathbb{N}}$ a sequence of  functions in $\Pi(\ephi,1)$, and $u\in \lphi$ such that $u_n\rightarrow u$ in $\lphi$. Then there exist a subsequence
$u_{n_k}$ and {\color{red}$h\in\Pi(\ephi,1)$} such that
\[u_{n_k}\rightarrow u \quad\text{a.e.}\]
\[|u_{n_k}|\leq h\quad\text{a.e.}\]
\end{lem}
\begin{proof}
Let $r:=d(u_1,\ephi)$. Because $u_n$ is a Cauchy sequence, for
each $k\in\mathbb{N}$ there exists $n_k$ such that
\[\|u_m-u_s\|_{\Phi}<2^{-k-1}(1-r)\quad m,s\geq n_k\].
We consider the subsequence $u_{n_k}$ with $n_1=1$. Let
$S_k=u_1+(u_{n_2}-u_{n_1})+\ldots+(u_{n_k}-u_{n_{k-1}})$, and let
$h:[0,T]\rightarrow\mathbb{R}$ defined by
\[ h(x)=|u_1(x)|+\sum_{k=2}^{\infty}|u_{n_k}(x)-u_{n_{k-1}}(x)|.\]
By previous conditions, $h\in\Pi(\ephi,1)$. From
$\Pi(\ephi,1)\subset \claseor$, we can see that $\int_{[0,T]}
\Phi(h)\ dx < \infty$ and so, $h<\infty$ a.e. Moreover, from \[
|u_{n_k}|=|S_k|\leq h\]we conclude that
$u_1(x)+\sum_{k=2}^{\infty}(u_{n_k}(x)-u_{n_{k-1}}(x))$ converges
absolutely a.e. Then $u_{n_k}\rightarrow u \quad\text{a.e.}$
\end{proof}

We recall the definition of Gate\^{a}ux derivative, see \cite{ambrosetti} for details. Given a function $I:U\to\mathbb{R}$ where $U$ is an open set of a Banach space $X$,
we say that $I$ has a G\^ateaux derivative en $u \in U$ if there exists $A\in X^*$ such that for every $v \in X$
\[
\lim_{s \rightarrow 0}\frac{I(u+sv)-I(u) }{s}=Av.
\]



\section{Differentiability  of Lagrangian functions in Orlicz spaces}

\begin{defi} We said that a function $L:[0,T]\times \mathbb{R}^n \times \mathbb{R}^n \rightarrow \mathbb{R}$ is a Caratheodory function if for fixed $(x,y)$
the map $t \mapsto L(t, x,y)$ is measurable  and for fixed $t$ the map  $(x,y) \mapsto L(t, x, y)$ is continuously differentiable for almost everywhere $t\in [0,T]$.

\end{defi}



\begin{thm}\label{teorema_acotacion}
Let $L:[0,T]\times \mathbb{R}^n \times \mathbb{R}^n \rightarrow \mathbb{R}$ be a Caratheodory function and $\Phi,\Psi$ be complementary  $N$-functions. Suppose that there
exists $a \in C(\mathbb{R}^+, \mathbb{R}^+)$, $b \in L^1$, $c \in \lpsi$ such that

\begin{eqnarray}
|L(t,x,y)| &\leq a(|x|)\left(b(t)+ \Phi(|y|)  \right),\label{cotaL}\\
|D_{x}L(t,x,y)| &\leq a(|x|)\left(b(t)+ \Phi(|y|)  \right),\label{cotaDxL}\\
|D_{y}L(t,x,y)| &\leq a(|x|)\left(c(t)+ \varphi(|y|)  \right).\label{cotaDyL}
\end{eqnarray}




Then the following statements hold
\begin{enumerate}
\item \label{T1item1} \label{A1} The function $I: W^{1}(\lphi,\claseor) \rightarrow \mathbb{R}$ defined by
\begin{equation}\label{integral_accion}
I(u)=\int_{0}^T L(t,u(t),\dot{u}(t))\ dt
\end{equation}
is everywhere finite.
\item\label{T1item2}  The mapping
\[
u \mapsto D_{x}L(t,u,\dot{u})
\]
is continuous from $\left(\domi, \|\cdot \sobnor\right)$ into $\left(L^{1}([0,T]), \|\cdot\|_{L^1}\right)$ and the mapping
\[
u \mapsto  D_{y}L(t,u,\dot{u})
\]
is continuous from $\left(\domi, \|\cdot \sobnor\right)$ into $\left(\left[\lphi\right]^*,w^*\right)$. % with the weak$^*$ topology.



\item\label{T1item3} The function  $I$ is G\^ateaux differentiable on $W^{1}(\lphi,\Pi(\ephi,1))$ and  its derivative
$I': \left(\domi, \|\cdot \sobnor\right) \to \left(\left[\wphi \right]^*, w^* \right)$ is continuous. Furthermore the G\^ateaux derivative of $I$ is given by
\[
\langle I'(u), v \rangle= \int_0^T \left(D_{x}L(t,u(t),\dot{u}(t))\cdot v(t)+ D_{y}L(t,u(t),\dot{u}(t))\cdot\dot{v}(t)\right) \ dt.
\]
\end{enumerate}
\end{thm}
\begin{proof}

First we prove \ref{T1item1}. If $u\in W^{1}(\lphi,\claseor) \rightarrow \mathbb{R}$ then, by Lemma \ref{inclusion orlicz} ,
 $u\in L^{\infty}$ and

 \[\|u\|_{L^{\infty}}\leq C(T)\|u\|_{\wphi}=:A.\]
  By hypothesis
 $a:[0,A]\rightarrow\mathbb{R}$ is bounded, i.e. there exists
 $M>0$ such that $a(|u|)\leq M$.
 Thus,
 \[|L(t,u,\dot{u})| \leq M\left(b(t)+ \Phi(|\dot{u}|)  \right)\in
 L^1.\]

 In order to prove \ref{T1item2}, we take   $\{u_n\}_{n\in \mathbb{N}}$ a sequence of  functions in $W^{1}(\lphi,\Pi(\ephi,1))$, and $u\in W^{1}(\lphi,\Pi(\ephi,1))$ such that $u_n\rightarrow u$ in $\wphi$.
Then $u_n\rightarrow u$ in $\lphi$ and $\dot{u}_n\rightarrow \dot{u}$ in $\lphi$. By Lemma \ref{segundo lema} there exist a subsequence $u_{n_k}$ and $h\in \Pi(\ephi,1))$
such that

\[u_{n_k}\rightarrow u \quad\text{a.e.}\]

\[\dot{u}_{n_k}\rightarrow \dot{u} \quad\text{a.e.}\]

\[|\dot{u}_{n_k}|\leq h\quad\text{a.e.}\]

Because of $u_{n_k}\rightarrow u$ in $\wphi$, there exist $M>0$ such that \[\|u_{n_k}\|_{\wphi}\leq M.\]
and
\[\|u_{n_k}\|_{L^{\infty}} \leq  C(T)\|u_{n_k}\|_{\wphi}\leq C(T)M=:N.\]

By the Caratheodory condition
\[D_xL(t,u_{n_k}(t),\dot{u}_{n_k}(t))\to D_xL(t,u(t),\dot{u}(t))\quad\hbox{ for a.e }t\in[0,T].\]
Moreover from \eqref{cotaDxL} we get

\begin{equation}\label{DxL1}
|D_xL(t,u_{n_k}(t),\dot{u}_{n_k}(t))|\leq N\left(b(t)+\Phi(|h|)\right) \in L^1.
\end{equation}

  We need apply \cite[Lemma 9.1]{KR} and \cite[Lemma 17.1]{KR}. We note that these lemmas hold true for
Orlicz spaces of vector valued functions and that in the proof of
\cite[Lemma 17.1]{KR}   it is not required the continuity
hypothesis  on the function  $f$, which is assumed
 in general in \cite{KR} for operators of Nemitskii type.  We consider the composition operator
$\boldsymbol{\varphi}$ defined by $\boldsymbol{\varphi}(u)(t)= \varphi(|u(t)|)$. Since $\varphi$ is a Borel function, $\boldsymbol{\varphi}$ transform measurable
functions into measurable functions. Furthermore, in virtue of \cite[Lemma 9.1]{KR}, $\boldsymbol{\varphi}\left(B_{\lphi}(0,1)\right)\subset \tilde{L}^{\Psi}$, where
$B_{\lphi}(u_0,r)$ is the open ball with center $u_0$ and radius $r>0$.  Now we can apply \cite[Lemma 17.1]{KR} obtaining that $\boldsymbol{\varphi}$ acts from
$\Pi(\ephi,1)$ into {\color{red}$\tilde{L}^{\Psi}$}. Therefore if $u\in W^{1}(\lphi,\Pi(\ephi,1))$, take norm in \eqref{cotaDyL}

\begin{equation}\label{DyLpsi}
   \left\|D_yL(t,u(t),\dot{u}(t))\right\|_{L^{\Psi}}\leq  M(\|u\|_{\wphi} )\left(\|c\|_{L^{\Psi}}+\|\boldsymbol{\varphi}(u)\|_{L^{\Psi}} \right)
\end{equation}

Therefore $D_yL(t,u(t),\dot{u}(t))\in L^{\Psi}$.

Let us prove that the map   $u\mapsto D_yL(t,u(t),\dot{u}(t))$ is continuous from $W^{1}(\lphi,\Pi(\ephi,1))$, with the norm topology, in $\left[\lphi\right]^{*}$,
with the weak$^*$ topology. We take $u_n,u\in W^{1}(\lphi,\Pi(\ephi,1))$ with $u_n\to u$ in the norm of $\wphi$. Then we must prove that for every $v\in\lphi$

\begin{equation}\label{conv_debil} \langle D_yL(t,u_n(t),\dot{u}_n(t)), v \rangle \to \langle D_yL(t,u(t),\dot{u}(t)), v \rangle. \end{equation}

Suppose that there exists $v\in\lphi$ such that  \eqref{conv_debil} is not true. Therefore we can assume that there exists $\epsilon>0$ such that
\begin{equation}\label{cota_eps}
 \left| \langle D_yL(t,u_n(t),\dot{u_n}(t)), v \rangle - \langle D_yL(t,u(t),\dot{u}(t)), v \rangle\right|\geq \epsilon.
\end{equation}


Then we have $u_n\rightarrow u$ in $\lphi$ and
$\dot{u}_n\rightarrow \dot{u}$ in $\lphi$. By Lemma \ref{segundo lema}, there exist a subsequence $u_{n_k}$ and $h\in \Pi(\ephi,1)$ such that
\[u_{n_k}\rightarrow u \quad\text{a.e.}\]

\[\dot{u}_{n_k}\rightarrow \dot{u} \quad\text{a.e.}\]

\[|\dot{u}_{n_k}|\leq h\quad\text{a.e.}\]

We note that since $u_n$ is a convergent sequence in $\wphi$ we have that $\|u_n\|_{\wphi}$ is uniformly bounded.
Therefore the constant $M(\|u_n\|_{\wphi} )$ is uniformly bounded by certain constant $C$. Let $v\in\lphi$,  then by \eqref{cotaDyL}
\[
  \begin{split}
  \left|D_yL(t,u_n(t),\dot{u}_n(t))\cdot v\right| &\leq C\left(c(t)|v(t)|+\varphi(|h(t)|)|v(t)|\right)\\
  \end{split}
\]
and
\[\int_0^T\left(c(t)|v(t)|+\varphi(|h(t)|)|v(t)|\right)dt\leq \|c\|_{\lpsi} \|v\orlnor+\|\boldsymbol{\varphi}(h)\|_{\lpsi} \|v\orlnor. \]
We conclude that $D_yL(t,u_{n_k}(t),\dot{u}_{n_k}(t))\cdot v$ is dominated by a function in $L^1$. Hence, by Lebesgue dominated convergence Theorem we get
\eqref{conv_debil} for the subsequence $n_k$, which would contradict the inequality \eqref{cota_eps}.

Finally we prove \ref{T1item3}. The proof follows the similar lines that \cite[Theorem 1.4]{mawhin2010critical}. For $u\in \domi$ and $v\in\wphi$ we define the function
\[f(s,t)=L(t,u(t)+sv(t),\dot{u}(t)+s\dot{v}(t)).\]
We remark that by \ref{T1item1}
\[I(u+sv)=\int_0^Tf(s,t)\ dt\]
is well defined and it is finite valued for $t\in [0,T]$ and  $|s|\leq s_0:=\left(1-d(\dot{u},\ephi)\right)/\|v\sobnor$ ($v\neq 0$).

Applying the chain rule and the inequalities \eqref{cotaDxL}-\eqref{cotaDyL} we obtain
\begin{equation}\label{ctg}
\begin{split}
|D_s f(s,t)|&=|D_xL(t,u+sv,\dot{u}+s\dot{v})\cdot v + D_yL(t,u+sv,\dot{u}+s\dot{v})\cdot \dot{v}| \\
&\leq a(|u+sv|)\left[(b(t)+\Phi(|\dot{u}+s\dot{v}|))|v|+ (c(t)+ \varphi(|\dot{u}+s\dot{v}|))|\dot{v}|\right]\\
& \leq a_0 \left[(b(t)+\Phi(|\dot{u}|+s_0|\dot{v}|))|v|+ (c(t)+ \varphi(|\dot{u}|+s_0|\dot{v}|))|\dot{v}|  \right]
\end{split}
\end{equation}
where
\[
 a_0 = \max \left\{ a(\xi): \xi\in \left[0, \|u \linf + s_0 \|v \linf \right] \right\}.
\]

It is easy to show that $d(w,\ephi)\leq d(|w|,\ephi)$ for every $w \in \lphi$
\[
d \left(|\dot{u}|+s_0|\dot{v}|, \ephi \right) %d(|\dot{u}|,\ephi)+ d(|\dot{u}|+s_0|\dot{v}|, |\dot{u}|)
\leq d \left(|\dot{u}|,\ephi \right)+ s_0 \|\dot{v}\orlnor < 1.
\]
As a consequence $|\dot{u}|+s_0|\dot{v}| \in \Pi(\ephi,1) \subset \claseor$. Then $b+\Phi(|\dot{u}|+s_0|\dot{v}|) \in L^1$ and since $v \in L^{\infty}$ we have that
$(b+\Phi(|\dot{u}|+s_0|\dot{v}|))|v| \in L^1$. On the other hand, we recall that $\boldsymbol{\varphi}$ acts from  $\Pi(\ephi,1)$ into $\tilde{L}^{\Psi}$ therefore
$c(t)+ \varphi(|\dot{u}|+s_0|\dot{v}|) \in L^{\Psi}$ and since $\dot{v} \in L^{\Phi}$, applying the H\"older inequality
$(c(t)+ \varphi(|\dot{u}|+s_0|\dot{v}|))|\dot{v}| \in L^1$. Thus, from \eqref{ctg} and the above discussion there exists a function $g \in L^1([0,T], \mathbb{R}^{+})$
such that $|D_s f(s,t)| \leq g(t)$. Consequently, $I$ has a directional derivative and
\[
\langle I'(u), v \rangle=\frac{d}{ds}I(u+sv)\big|_{s=0}=\int_0^T \left(D_{x}L(t,u,\dot{u})\cdot v+ D_{y}L(t,u,\dot{u})\cdot\dot{v}\right) \ dt.
\]
Moreover, from \eqref{DxL1}, \eqref{DyLpsi}, Lemma \ref{inclusion orlicz} and previous formula
\[
|\langle I'(u), v \rangle| \leq c \|v\linf + c \|\dot{v}\orlnor \leq c \|v\sobnor.
\]
This complete the proof of the G\^ateaux differentiability of $I$. Finally, the continuity of $I': \left(\domi, \|\cdot \sobnor\right) \to \left(\left[\wphi
\right]^*, w^* \right)$ is a consequence of item \ref{T1item2}. \textcolor{red}{Erica desarrollara este punto}
\end{proof}

\begin{lem}\label{lema_conv_may}
Suppose that $\Phi$ is a $\Delta_2$ function.  If $u_n \in\lphi$ is a sequence such that $u_n\to 0$ a.e. and suppose that there exist $M\in\lphi$ with $|u_n|\leq M$
then $\|u_n\orlnor\to 0$.
\end{lem}
\begin{proof}
 According to \cite[Theorem 9.4]{KR} it is sufficient to prove that
 \[\int_0^T\Phi(|u_n|)dt\to 0,\quad\text{for }n\to\infty.\]
 This is an immediate consequence of $\Phi(|u_n|)\leq \Phi(M)\in L^1$ and the Dominated Convergence Theorem.
\end{proof}


\begin{cor}
  In the particular case that $\Phi$ and $\Psi$ are $\Delta_2$ functions we have
  $I'$ is continuous from $(\wphi,\|\cdot\sobnor)$ into $(\left[\wphi\right]^*,\|\cdot \|_{\left[\wphi\right]^*})$.
\end{cor}

\begin{proof}
Let us see that the maps $u\mapsto D_xL(\cdot,u(\cdot),\dot{u}(\cdot))$  and $u\mapsto D_yL(\cdot,u(\cdot),\dot{u}(\cdot))$  are continuous
from $\left(\wphi, \|\cdot \sobnor\right) $ into $\left( L^1, \|\cdot \|_{L^1}\right)$ and
 $\left(\lpsi,\|\cdot\|_{L^{\Psi}}\right)$ respectively.

 The continuity of the first map is an immediate consequence of Theorem \ref{teorema_acotacion}(item \ref{T1item2}) and the fact that $\Pi(\ephi,1) =\lphi$
 when $\Phi\in\Delta_2$.

 We consider $u_n$ and $u$ with $\|u_n- u\sobnor\to 0$.  Let $h$ and $u_{n_k}$, $k=1,2,\ldots$ be as in Lemma \ref{segundo lema}. Then  since $L$ is a Caratheodory function
 we have $ D_yL(t,u_{n_k}(t),\dot{u}_{n_k}(t))\to D_yL(t,u(t),\dot{u}(t))$ a.e. $t\in [0,T]$.  From Lemma \ref{inclusion orlicz} and the fact that $\|u_{n_k}\sobnor$ are uniformly bounded, we get that there exists $N>0$ such that
 $\|u_{n_k}\|_{L^{\infty}}\leq N$. Therefore
  \[a(|u_{n_k}(t)|)\leq \sup_{0\leq s\leq N} a(s)=:C<\infty.\]
 By using \eqref{cotaDyL} and $\Psi\in\Delta_2$, we get 
 \[\begin{split}
    |D_yL(t,u_{n_k}(t),\dot{u}_{n_k}(t))| &\leq a(|u_{n_k}|)\left( c(t) + \varphi (|\dot{u}_{n_k}(t)|)\right)\\
    &\leq C\left( c(t) + \varphi (|h|)\right)\in \lpsi
   \end{split}
\]
Therefore, invoking  Lemma \ref{lema_conv_may}, we have proved that
  of all sequence $u_n$ which converge to $u$ in $\wphi$ we can
extract a subsequence with $\|D_yL(t,u_{n_k},\dot{u}_{n_k})-D_yL(t,u,\dot{u})\|_{\lpsi}\to 0$. The desired result follows from a standard argument.

\[
  \begin{split}
    \|I'(u)-I'(u_0)\|_{\left[\wphi\right]^*}&=\sup_{v\in \wphi,\|v\sobnor\leq 1} \langle I'(u)-I'(u_0),v \rangle\\
      &=\sup_{v\in \wphi, \|v\sobnor\leq 1} \int_0^T \left\{ \left(D_xL(t,u(t),\dot{u}(t))-D_xL(t,u_0(t),\dot{u_0}(t))\right)\cdot v(t)\right.\\
      &\quad\left.+
      \left(D_yL(t,u(t),\dot{u}(t))-D_yL(t,u_0(t),\dot{u_0}(t))\right)\cdot \dot{v}(t)\right\}dt \\
      &\leq \sup_{v\in \wphi, \|v\sobnor\leq 1} \left\{\| D_xL(\cdot,u(\cdot),\dot{u}(\cdot))-D_xL(\cdot,u_0(\cdot),\dot{u_0}(\cdot))\|_{L^1}\|v\|_{L^{\infty}}\right.\\
      &\quad+\left.
       \| D_yL(\cdot,u(\cdot),\dot{u}(\cdot))-D_yL(\cdot,u_0(\cdot),\dot{u_0}(\cdot))\|_{\lpsi}\|\dot{v}\|_{\lphi}\right\}\\
       &\leq C\left(\| D_xL(\cdot,u(\cdot),\dot{u}(\cdot))-D_xL(\cdot,u_0(\cdot),\dot{u_0}(\cdot))\|_{L^1}\right.\\
       &\left. \quad +
       \| D_yL(\cdot,u(\cdot),\dot{u}(\cdot))-D_yL(\cdot,u_0(\cdot),\dot{u_0}(\cdot))\|_{\lpsi}\right)
  \end{split}
\]
Therefore the results follows  of the  previously established continuity for $D_xL$ and $D_yL$.

\end{proof}






We recall that the definition of a strictly convex function, that is a function $f: \mathbb{R}^n \to \mathbb{R}$ such that
\[
f\left(\frac{x+y}{2}\right)< \frac{1}{2} \left(f\left( x\right)+f\left( y\right)\right).
\]


In this paper, we will consider the following problem


\begin{equation}\label{ecualagran}
    \left\{%
\begin{array}{ll}
   \frac{d}{dt} D_{y}L(t,u(t),\dot{u}(t))= D_{x}L(t,u(t),\dot{u}(t)) \quad \hbox{a.e.}\ t \in (0,T)\\
    u(0)-u(T)=\dot{u}(0)-\dot{u}(T)=0.
\end{array}%
\right.
\end{equation}

We denote by $\wphi_T$ the subspace of $\wphi$ of all functions $T$-periodic. Similarly we consider the subspaces $\ephi_T$, $\lphi_T$. As is usual, when $Y$ is a subspace of
the Banach space $X$, we denote by $Y^{\perp}$ the subspace of $X^*$ of all the functions which are identically zero on $Y$.

\begin{cor} The following statements are equivalent
\begin{enumerate}
 \item $I'(u)\in\left( \wphi_T\right)^{\perp}$
 \item  $D_yL(t,u(t),\dot{u}(t))dt$ is an absolutely continuous function and $u$ solve the following boundary value problem
 \begin{equation}\label{ecualagran2}
    \left\{%
\begin{array}{ll}
   \frac{d}{dt} D_{y}L(t,u(t),\dot{u}(t))= D_{x}L(t,u(t),\dot{u}(t)) \quad \hbox{a.e.}\ t \in (0,T)\\
    u(0)-u(T)=D_yL(0,u(0),\dot{u}(0))-D_yL(T,u(T),\dot{u}(T))=0.
\end{array}%
\right.
\end{equation}

\end{enumerate}
Moreover if $D_yL(t,x,y)$ is $T$-periodic with respect to the variable $t$ and strictly convex with respect to $y$, then
$D_yL(0,u(0),\dot{u}(0))-D_yL(T,u(T),\dot{u}(T))=0$ is equivalent to $\dot{u}(0)=\dot{u}(T)$.

\end{cor}

\begin{proof} The condition $I'(u)\in\left( \wphi_T\right)^{\perp}$ means that for every $v\in \wphi_T$ we have $\langle I'(u),v\rangle=0$. According to Theorem
\ref{teorema_acotacion} we have

\[\int_0^TD_yL(t,u(t),\dot{u}(t))\cdot \dot{v}(t)dt=-\int_0^TD_xL(t,u(t),\dot{u}(t))\cdot v(t)dt \]
Using \cite[pag. 6]{mawhin2010critical} we obtain that $D_yL(t,u(t),\dot{u}(t))$ is absolutely continuous, therefore it is differentiable a.e.on $[0,T]$ and that the first equality
of \ref{ecualagran} holds true. Moreover the same reference says that $D_yL(t,u(t),\dot{u}(t))$ is $T$-periodic.
This complete the proof of 1. implies 2. The converse is still easier than the previous one, we omit the proof of it.

The last part of the Corollary is a consequence of that $D_yL(T,u(T),\dot{u}(T))=D_yL(0,u(0),\dot{u}(0))=D_yL(T,u(T),\dot{u}(0))$ and the well known fact strictly
convexity implies injectivity (see, for instance \cite[Theorem 12.17]{rockafellar2009variational}).
\end{proof}

\begin{lem}
We suppose that $L(t,x,y)=\Phi(|y|)+F(t,x)$, where $F(t,x)$ is a Charateodory function and $|F(t,x)|\leq a(|x|)b(t)$, where $a\in C(\mathbb{R}_+,\mathbb{R}_+)$
and $b\in L^1([0,T])$. We suppose that $\Phi\in\Delta_2$.  Then the functional \eqref{integral_accion} is weakly lower semicontinuous (w.l.s.c.). 
\end{lem}

\begin{proof}
 
\end{proof}







\bibliographystyle{plain}
\bibliography{biblio}
\end{document}
