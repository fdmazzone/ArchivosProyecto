\documentclass[twoside]{article}
%%Paquetes


\usepackage{amssymb,amsthm}
\usepackage{amsmath}
\usepackage{color}
\usepackage{ esint }
%\usepackage{graphicx}
%\usepackage{wrapfig}
%\usepackage{subfigure}
\usepackage{fancyhdr}
\usepackage{times}
%\usepackage{theorem}
\usepackage[latin1]{inputenc}
%\usepackage{showkeys}
\usepackage{comment}
\usepackage{url}
\usepackage{xcolor}
\usepackage{adjustbox}
%Teorema y similes
\usepackage[maxnames=6,backend=bibtex]{biblatex}
\bibliography{biblio.bib}

\definecolor{rosa}{rgb}{1,0.3,0.9}
\definecolor{violeta1}{rgb}{0.5,0.3,0.5}
\definecolor{violeta}{rgb}{0.5,0.1,0.5}
\definecolor{negro}{rgb}{0.5,0.2,0.4}
\definecolor{celeste}{rgb}{0.1,0.4,1}
\definecolor{naranja}{rgb}{1,0.5,0}
\definecolor{color_nota_fer}{HTML}{DEBFDB}


\newenvironment{colbox}[2]{%
    \begin{adjustbox}{minipage={\linewidth},margin=1ex,bgcolor=#1,env=center}
        #2}{%
    \end{adjustbox}%
}
\newcounter{nota_fer_cont}
\newenvironment{nota_fer}[1]{\refstepcounter{nota_fer_cont}\begin{colbox}{color_nota_fer}{\textbf{Comentario Leo-Graciela-Fernando \arabic{nota_fer_cont}.} #1}}{\end{colbox}}


\newtheorem{thm}{Theorem}[section]
\newtheorem{cor}[thm]{Corollary}
\newtheorem{lem}[thm]{Lemma}
\newtheorem{rem}[thm]{Remark}
\newtheorem{defi}[thm]{Definition}
\newtheorem{prop}[thm]{Proposition}
\theoremstyle{remark}
\newtheorem{comentario}{Remark}


\title{Euler-Lagragian equations in an Orlicz-Sobolev space setting}
\author{Sonia Acinas \thanks{SECyT-UNRC}\\
Dpto. de Matem\'atica, Facultad de Ciencias Exactas y Naturales\\
Universidad Nacional de La Pampa\\
(6300) Santa Rosa, La Pampa, Argentina\\
\url{sonia.acinas@gmail.com}\\[3mm]
Leopoldo Buri \thanks{SECyT-UNRC}\\
Dpto. de Matem\'atica, Facultad de Ciencias Exactas, F\'{\i}sico-Qu\'{\i}micas y Naturales\\
Universidad Nacional de R\'{i}o Cuarto\\
(5800) R\'{\i}o Cuarto, C\'ordoba, Argentina,\\
\url{lburi@exa.unrc.edu.ar}\\[3mm]
Graciela Giubergia \thanks{SECyT-UNRC and CONICET}\\
Dpto. de Matem\'atica, Facultad de Ciencias Exactas, F\'{\i}sico-Qu\'{\i}micas y Naturales\\
Universidad Nacional de R\'{i}o Cuarto\\
(5800) R\'{\i}o Cuarto, C\'ordoba, Argentina,\\
\url{ggiubergia@exa.unrc.edu.ar}\\[3mm]
Fernando D. Mazzone \thanks{SECyT-UNRC and CONICET}\\
Dpto. de Matem\'atica, Facultad de Ciencias Exactas, F\'{\i}sico-Qu\'{\i}micas y Naturales\\
Universidad Nacional de R\'{i}o Cuarto\\
(5800) R\'{\i}o Cuarto, C\'ordoba, Argentina,\\
\url{fmazzone@exa.unrc.edu.ar}\\[3mm]
Erica L. Schwindt\thanks{ANR. AVENTURES - ANR-12-BLAN-BS01-0001-01}\\
Universit\'{e} d'{O}rl\'{e}ans, Laboratoire MAPMO, CNRS, UMR 7349, \\
F\'ed\'eration Denis Poisson, FR 2964,\\
B\^{a}timent de Math\'{e}matiques, BP 6759, 45067 Orl\'{e}ans Cedex 2, France,\\
\url{leris98@gmail.com}}

\date{}

\newcommand{\orlnor}{\|_{L^{\Phi}}}
\newcommand{\lurnor}{\|^{*}_{L^{\Phi}}}
\newcommand{\linf}{\|_{L^{\infty}}}
\newcommand{\lphi}{L^{\Phi}}
\newcommand{\lpsi}{L^{\Psi}}
\newcommand{\ephi}{E^{\Phi}}
\newcommand{\claseor}{C^{\Phi}}
\newcommand{\wphi}{W^{1}\lphi}
\newcommand{\sobnor}{\|_{W^{1}\lphi}}
\newcommand{\domi}{\mathcal{E}^{\Phi}_n(\lambda)}
\renewcommand{\b}[1]{\boldsymbol{#1}}
\newcommand{\rr}{\mathbb{R}}
\newcommand{\nn}{\mathbb{N}}
\newcommand{\ccdot}{\b{\cdot}}
\renewcommand{\leq}{\leqslant} 
\newcommand{\epsi}{E^{\Psi}}

\begin{document}



\maketitle
%
\begingroup%Locallizing the change to `thefootnote'.
    \renewcommand{\thefootnote}{}%Removing the footnote symbol.
    %
    \footnotetext{%
    %   2010 Mathematics Subject Classification
    %   http://www.ams.org/msc/
    \textbf{2010  AMS Subject Classification.} Primary: .
    Secondary: .
    }%
        \footnotetext{%
    \textbf{Keywords and phrases.}  .
    }%
    \endgroup
%
%
%
%

\begin{abstract}
...
\end{abstract}




\pagestyle{fancy} \headheight 35pt \fancyhead{} \fancyfoot{}

\fancyfoot[C]{\thepage} \fancyhead[CE]{\nouppercase{S. Acinas, L. Buri, G. Giubergia, F. Mazzone and E. Schwindt}} \fancyhead[CO]{\nouppercase{\section}}

\fancyhead[CO]{\nouppercase{\leftmark}}


%\tableofcontents

\section{Introduction}

\section{Preliminaries}

%\subsection{$N$-functions}
For reader convenience, we give a short introduction to Orlicz and Orlicz Sobolev spaces of vector valued functions and a  list  of results that we will use throughout the article. We refer to \cite{adams_sobolev,KR,wroblewska2012application} for additional details and proofs. In the first two references scalar valued function are considered, however the generalization of the results enumerated below  to vector valued functions is direct. Last one reference consider vector valued functions.

Hereafter we denote  by $\mathbb{R}^+$ to the set of all non negative real numbers. A function $\Phi:\mathbb{R}^+\to \mathbb{R}^+ $ is called an \emph{$N$-function} if it has the form
\[
\Phi(t)=\int_{0}^t \varphi(\tau)\ d\tau,\quad\hbox{for } u\geq 0,
\]
where $\varphi:\mathbb{R}^+\rightarrow :\mathbb{R}^+$ is a right continuous nondecreasing function  satisfying   $\varphi(0)=0$, $\varphi(t)>0$ for $t>0$ and
$\lim_{t\rightarrow \infty}\varphi(t)=+\infty$.

Given a function $\varphi$ as above, we also consider the so-called right inverse function $\psi$ of $\varphi$ which is defined $\psi(s)=\sup_{\varphi(t)\leq s}t$.
The function $\psi$ satisfies the same properties that function $\varphi$, therefore we have an $N$-function $\Psi$ such that $\Psi'=\psi$ . The function $\Psi$ is called the
\emph{complementary function} of $\Phi$.

We say that $\Phi$ is a \emph{function of the $\Delta_2$ class} when there exists a constant $K>0$ and a $t_0\geq 0$ such that $\Phi(2t)\leq K\Phi(t)$, for every $t\geq t_0$. If $t_0=0$ we said that $\Phi$ is $\Delta_2$ \emph{global}.  

 In this paper we adopt the convention of to use bold symbols for denote points in $\mathbb{R}^n$ and plain symbols for scalar ones.

For  $n$  positive integer we denote by $M_n:=M_n([0,T])$ the set of all measurable functions defined in $[0,T]$ with values in $\mathbb{R}^n$.  Given  a $N$-function $\Phi$ we define the \emph{modular function} $\rho_{\Phi}:M_n\to \mathbb{R}^+\cup\{+\infty\}$ by
\[\rho_{\Phi}(\b{u}):= \int_0^T \Phi(|\b{u}|)\ dt.\]
Here $|\cdot|$ is the euclidean norm of $\mathbb{R}^n$.
The \emph{Orlicz class} $C_n^{\Phi}=C_n^{\Phi}([0,T])$  is defined by
\begin{equation}\label{claseOrlicz}
  C^{\Phi}_n:=\left\{\b{u}\in M_n | \rho_{\Phi}(\b{u})< \infty \right\}.
\end{equation}
The \emph{Orlicz space} $\lphi_n=L^{\Phi}_n([0,T])$ is the linear hull of $\claseor_n$.
Equivalently
\begin{equation}\label{espacioOrlicz}
\lphi_n:=\left\{ \b{u}\in M_n | \exists \lambda>0: \rho_{\Phi}(\lambda \b{u}) < \infty   \right\}.
\end{equation}
  The Orlicz space $\lphi_n$ equipped with the \emph{Orlicz norm}
\[
\|  \b{u}  \orlnor:=\sup \left\{  \left.\int_0^T \b{u}\b{\cdot} \b{v} dt \right| \rho_{\Psi}(\b{v})\leq 1\right\},
\]
is a Banach space. By $\b{u}\b{\cdot} \b{v}$ we denote the usual dot product in $\mathbb{R}^{n}$ between $\b{u}$ and $\b{v}$.  Sometimes the following alternative expression for the norm, known as \emph{Amemiya norm},     will  be useful (see \cite[Th. 10.5]{KR} and \cite{hudzik2000amemiya}). For every $\b{u}\in\lphi$,

\begin{equation}\label{amemiya}
\|\b{u}\orlnor=\inf\limits_{k>0}\frac{1}{k}\left\{1+\rho_{\Phi}(k\b{u})\right\}.
\end{equation}



The subspace $\ephi_n=\ephi_n([0,T])$ is defined as the closure in $\lphi_n$ of the subspace $L^{\infty}_n$ of all the $\mathbb{R}^n$-valued essentially bounded functions. It is showed that  $\ephi_n$ is the only one maximal subspace contained in the Orlicz class $\claseor$, that is $\b{u}\in\ephi_n$ if and only if for any $\lambda>0$ we have $\rho_{\Phi}(\lambda \b{u})<\infty$.  

A generalizated version of \emph{H\"older inequality} holds in the setting of Orlicz spaces (ver \cite[Th 9.3]{KR} ). Namely, if $\b{u}\in\lphi_n$ and $\b{v}\in\lpsi_n$ then $\b{u}\ccdot\b{v}\in L_1^1$ and
\begin{equation}\label{holder}
\int_0^T\b{v}\ccdot\b{u}dt\leq \|\b{u}\orlnor\|\b{v}\|_{L^{\Psi}}.
\end{equation}




If $X$ and $Y$ are  Banach spaces, with $Y\subset X^*$ we denote by $\langle\cdot,\cdot\rangle:Y\times X\to\mathbb{R}$ to the bilinear pairing  map given by $\langle x^*,x\rangle=x^*(x)$. H\"older inequality shows that $\lpsi_n\subset \left[\lphi_n\right]^*$, where the pairing $\langle \b{u},\b{v}\rangle$, $\b{u}\in\lphi_n$ and $\b{v}\in\lpsi_n$, is defined by 
\begin{equation}\label{pairing}
  \langle \b{v},\b{u}\rangle=\int_0^T\b{u}\ccdot\b{v}dt.
\end{equation}
 Unless $\Phi$ be a $\Delta_2$ function, the relation $\lpsi_n= \left[\lphi_n\right]^*$ does not holds. It is true in general that  $\left[\ephi_n\right]^*=\lpsi_n$.


Likes in \cite{KR}, we will consider the subset $\Pi(\ephi_n,r)$ of $\lphi_n$ defined by
\[\Pi(\ephi_n,r):=\{\b{u}\in\lphi_n| d(\b{u},\ephi_n)<r\}.\]
This set is related to the Orlicz class $\claseor_n$ by means of inclusions
\begin{equation}\label{inclusiones}\Pi(\ephi_n,1)\subset \claseor_n \subset\overline{\Pi(\ephi_n,1)}.\end{equation}
The proof of this fact, and similar ones, is given by real valued function in \cite{KR},
the extension to $\mathbb{R}^n$-valued functions does not involve any difficulty. When the function $\Phi$ is of the $\Delta_2$ class then the four sets $\lphi_n$, $\ephi_n$, $\Pi(\ephi_n,1)$ and $\claseor_n$ are equal.

We will use the following elementary fact frequently 
\begin{equation}\label{inclusion2}
\b{u}\in\Pi(\ephi_n,\lambda)\implies \frac{\b{u}}{\lambda}\in\Pi(\ephi_n,1)\subset\claseor_n.
\end{equation}

We define the \emph{Sobolev-Orlicz space} $\wphi_n$ (see \cite{adams_sobolev}) by
\[\wphi_n:=\{\b{u}| \b{u} \hbox{ is absolutely continuous and } \b{u},\b{\dot{u}}\in \lphi_n\}.\]
This space is a Banach space  equipped with the norm
\[
\|  \b{u}  \|_{\wphi}= \|  \b{u}  \|_{\lphi} + \|\b{\dot{u}}\orlnor.
\]



For a  function $\b{u}\in L^1_n([0,T])$, we write $\b{u}=\overline{\b{u}}+\widetilde{\b{u}}$, where $\overline{\b{u}} =\frac1T\int_0^T \b{u}(t)\ dt$ and $\widetilde{\b{u}}=\b{u}-\overline{\b{u}}$.

 An important aspect of the theory of Sobolev spaces is related to embedding theorems. There is an extensive literature on this question in the setting of Orlicz-Sobolev spaces, see for example
 \cite{cianchi1999some,cianchi2000fully,claverooptimal,edmunds2000optimal,kerman2006optimal}.
For this reason the following simple  Lemma, which we will use systematically, it is well known. We include a brief proof for sake of completeness.

 % As is usual, if $X$ and $Y$ are normed spaces, with $X\subset Y$,  we write $X\hookrightarrow Y$ when the identity map is an bounded operator between $X$ and $Y$.

\begin{lem}\label{inclusion orlicz} Let $\b{u}\in\wphi_n$. Then $\b{u}\in L^{\infty}_n([0,T])$ and
\begin{align}
  \|\widetilde{\b{u}}\|_{L^{\infty}} &\leq T\Psi^{-1}\left(\frac{1}{T}\right)\|\b{\dot u}\orlnor&\text{  (Wirtinger's inequality)}\label{wirtinger}\\
\|\b{u}\|_{L^{\infty}} &\leq\Psi^{-1}\left(\frac{1}{T}\right)\max\{1,T\}\|\b{u}\sobnor&\text{  (Sobolev's inequality)}\label{sobolev}
\end{align}

\end{lem}
\begin{proof}
Since $\b{u}$ is contiuous, from the mean value theorem there exists $\tau$ such that
$\b{u}(\tau)=\overline{\b{u}}$, thus
\begin{equation}\label{desigualdad1}\begin{split}
|\b{u}(t)-\overline{\b{u}}|\leq \int\limits_{\tau}^{t}|\b{\dot{u}}(s)|ds
\leq \|\b{\dot{u}}\|_{L^{\Phi}}\|1\|_{L^{\Psi}}\leq T\Psi^{-1}\left(\frac{1}{T}\right)\|\b{\dot u}\orlnor.
\end{split}
\end{equation}
Here we have used H\"older inequality and the formula for the norm of a characteristic function (ver  \cite[Eq. 9.11]{KR}). Inequality \eqref{desigualdad1} proves Wirtinger's inequality \eqref{wirtinger}. 

On the other hand, again by H\"older inequality and \cite[Eq. 9.11]{KR}, we obtain
\begin{equation}\label{desigualdad2}\begin{split}
|\overline{\b{u}}|\leq \frac{1}{T}\int\limits_{0}^{T}|\b{u}(s)|ds\leq \Psi^{-1}\left(\frac{1}{T}\right)\|\b{u}\orlnor.
\end{split}
\end{equation}
From \eqref{desigualdad1},\eqref{desigualdad2} and since $\b{u}=\overline{\b{u}}+\widetilde{\b{u}}$  we obtain \eqref{sobolev}.
\end{proof}

If $(X,\|\cdot\|_X)$ is a Banach space and $(Y,\|\cdot \|_Y)$ is a subespace of $X$, as is usual we write $Y\hookrightarrow X$ and we say that $Y$ is \emph{embeeded} in $X$  when the restricted identity map $i_Y:Y\to X$ is bounded. That means that there exists $C>0$ such that  for any $y\in Y$ we have $\|y\|_X\leq C\|y\|_Y$.  With this notation, the Lemma \ref{inclusion orlicz} states $\wphi_n \hookrightarrow L_n^{\infty}$ and H\"older inequality states that  $\lpsi_n\hookrightarrow  \left[\lphi_n\right]^*$.


 Given a contiuous function $a\in C(\mathbb{R}^+,\mathbb{R}^+)$, we define the composition operator $\b{a}:M_n\to M_n$ by $\b{a}(\b{u})(t)= a(|\b{u}(t)|)$.
We will use repeatedly the following elementary consequence of the previous lemma. 
\begin{cor}\label{a_bound} If $a\in C(\mathbb{R}^+,\mathbb{R}^+)$ then $\b{a}:\wphi_n\to L^{\infty}_1([0,T])$ is bounded. More concretely  there exists a non decreasing function $c:\mathbb{R}^+\to\mathbb{R}^+$ such that
 $\|\b{a}(\b{u})\|_{L^{\infty}([0,T])}\leq c(\|\b{u}\|_{\wphi})$.
\end{cor}
\begin{proof}  Let $\alpha\in C(\mathbb{R}^+,\mathbb{R}^+)$ be a  non-decreasing mayorant of $a$, for example $\alpha(s):=\sup_{0\leq t\leq s}a(t)$.  If $\b{u}\in \wphi_n$ then by Lemma \ref{inclusion orlicz} 
\[a(|\b{u}(t)|)\leq \alpha(\|\b{u}\|_{L^{\infty}})\leq a\left(\Psi^{-1}\left(\frac{1}{T}\right)\max\{1,T\} \|\b{u}\|_{\wphi}\right)=: c(\|\b{u}\|_{\wphi}).\]
\end{proof}


The following lemma is an inmediate consequence of principles  related to  operators of Nemitskii type, see \cite[�17]{KR}.

\begin{lem}\label{phi_comp}   The  composition operator  $\boldsymbol{\varphi}$  acts from $\Pi(\ephi_n,1)$ into $C_1^{\Psi}$.
\end{lem}
\begin{proof}
  As consequence of \cite[Lemma 9.1]{KR} we have that  $\boldsymbol{\varphi}\left(B_{\lphi}(0,1)\right)\subset C_1^{\Psi}$, where
$B_{X}(\b{u}_0,r)$ is the open ball with center $\b{u}_0$ and radius $r>0$ in the space $X$. Therefore, applying \cite[Lemma 17.1]{KR} we deduce that $\boldsymbol{\varphi}$ acts from $\Pi(\ephi_n,1)$ into $C_1^{\Psi}$.
\end{proof}

We need also the following technical lemma.
\begin{lem}\label{segundo lema}
Let $\lambda>0$ and  $\{\b{u}_n\}_{n\in \mathbb{N}}$ be a sequence of  functions in $\Pi(\ephi_n,\lambda)$ converging to  $\b{u}\in \Pi(\ephi_n,\lambda)$  in the $\lphi$-norm. Then there exist a subsequence
$\b{u}_{n_k}$ and a real valued function $h\in\Pi\left(\ephi_1\left([0,T]\right),\lambda\right)$ such that $\b{u}_{n_k}\rightarrow \b{u} \quad\text{a.e.}$ and $|\b{u}_{n_k}|\leq h\quad\text{a.e.}$.
\end{lem}



\begin{proof}
Let $r:=d(\b{u},\ephi_n)$, $r<\lambda$. Because $\b{u}_n$ converges to $\b{u}$, there exists a subsequence $(n_k)$ such that
\[\|\b{u}_{n_k}-\b{u}\orlnor<\frac{\lambda-r}{2}\quad \text{ and }\quad \|\b{u}_{n_k}-\b{u}_{n_{k+1}}\orlnor<2^{-(k+1)}(\lambda-r)\]
Let $h:[0,T]\rightarrow\mathbb{R}$ defined by
\begin{equation}\label{serie} h(x)=|\b{u}_{n_1}(x)|+\sum_{k=2}^{\infty}|\b{u}_{n_k}(x)-\b{u}_{n_{k-1}}(x)|.
\end{equation}
As a consequence  of \cite[Lemma 10.1]{KR} we have that, for any $\b{v}\in\lphi_n$, $d(\b{v},\ephi_n)=d(|\b{v}|,\ephi_1)$. Therefore
\[d(|\b{u}_{n_1}|,\ephi_1)= d(\b{u}_{n_1},\ephi_n)\leq d(\b{u}_{n_1},\b{u})+d(\b{u},\ephi_n)<\frac{\lambda+r}{2}.\]
Then
\[d(h,\ephi_1)\leq d(h,|\b{u}_{n_1}|)+d(|\b{u}_{n_1}|,\ephi_1)< \lambda.\]
Therefore, $h\in\Pi(\ephi_1,\lambda)$.  In particular,  $|h|<\infty$ a.e. We conclude that the series $\b{u}_{n_1}(x)+\sum_{k=2}^{\infty}(\b{u}_{n_k}(x)-\b{u}_{n_{k-1}}(x))$
is absolutely convergent a.e.  This imply that $\b{u}_{n_k}\rightarrow \b{u} \quad\text{a.e.}$. The inequality $|\b{u}_{n_k}|\leq h$ is clear from the definition of $h$.
\end{proof}

A common obstacle with Orlicz spaces, that distinguishes it from $L^p$ spaces, is that a  sequence $\b{u}_n\in\lphi_n$ which is  uniformly bounded by $ h\in\lphi_1$ and a.e. convergent to $\b{u}$ is not necessarily norm convergent.
Fortunately the subspace $\ephi_n$ has that property. 

\begin{lem}\label{lema_conv_may}
Suppose that $\b{u_n} \in\lphi_n$ is a sequence such that $\b{u_n}\to \b{u}$ a.e. and suppose that there exist $h\in\ephi_1$ with $|\b{u_n}|\leq h$ a.e. 
then $\|\b{u_n}-\b{u}\orlnor\to 0$.
\end{lem}
\begin{proof}\cite[p.84]{rao1991theory} and \cite[Th. 10.3]{KR})
\end{proof}


  We recall the definition of Gate\^{a}ux derivative, see \cite{ambrosetti} for details. Given a function $I:U\to\mathbb{R}$ where $U$ is an open set of a Banach space $X$,
we say that $I$ has a G\^ateaux derivative en $\b{u} \in U$ if there exists $\b{u}^*\in X^*$ such that for every $\b{v} \in X$
\[
\lim_{s \rightarrow 0}\frac{I(\b{u}+s\b{v})-I(\b{u}) }{s}=\langle \b{u}^* , \b{v}\rangle.
\]

We recall the following definition. 
\begin{defi}[see \cite{kato1964demicontinuity}] Let $X$ a Banach space and $D\subset X$. A non linear operator $T:D\to X^*$ is called \emph{demicontinuous} if it is continuous when $X$ is equipped with the strong topology and $X^*$ with tha weak${}^*$ topology. 
\end{defi} 

\section{Differetiability of action integrals on Orlicz spaces}

\begin{defi} We said that a function $\mathcal{L}:[0,T]\times \mathbb{R}^n \times \mathbb{R}^n \rightarrow \mathbb{R}$ is a Caratheodory function if for fixed $(\b{x},\b{y})$
the map $t \mapsto \mathcal{L}(t, \b{x},\b{y})$ is measurable  and for fixed $t$ the map  $(\b{x},\b{y}) \mapsto \mathcal{L}(t, \b{x}, \b{y})$ is continuously differentiable for almost everywhere $t\in [0,T]$.

\end{defi}


In this paper we will consider Lagrangian functions satisfying the following structure conditions. We assume  that there
exists $\lambda>0$ and non negative functions  $a \in C(\mathbb{R}^+, \mathbb{R}^+)$, $b \in L^1_1([0,T])$, $c \in \lpsi_1([0,T])$ and $d\in\ephi_1$ such that

\begin{eqnarray}
|\mathcal{L}(t,\b{x},\b{y})| &\leq a(|\b{x}|)\left(b(t)+ \Phi\left(\frac{|\b{y}|}{\lambda}+d(t) \right)\right),\label{cotaL}\\
|D_{\b{x}}\mathcal{L}(t,\b{x},\b{y})| &\leq a(|\b{x}|)\left(b(t)+ \Phi\left(\frac{|\b{y}|}{\lambda}+d(t) \right)\right),\label{cotaDxL}\\
|D_{\b{y}}\mathcal{L}(t,\b{x},\b{y})| &\leq a(|\b{x}|)\left(c(t)+ \varphi\left(\frac{|\b{y}|}{\lambda}+d(t)\right)  \right).\label{cotaDyL}
\end{eqnarray}

\begin{comentario}These conditions are a direct  generalization of the conditions
\cite[Eq (a), p. 10]{mawhin2010critical}. 
\end{comentario}

In the following comments we discuss the relevance of the function $d$ in the inequalities \eqref{cotaL},  \eqref{cotaDxL} and \eqref{cotaDyL}. In particular, we are interested in to see when it is possible to find, for every  $d\in \ephi_1$, a function  $b\in L^1_1$and $C>0$ such that for every $s>0$
\begin{equation}\label{cotadb}
\Phi(s+d(t))\leq C\Phi(s)+b(t).
\end{equation} 
In that case we can suppose $d=0$ in \eqref{cotaL} and \eqref{cotaDxL}. The same considerations should be done with $\varphi\left(s+d(t)\right)$.

\begin{comentario}  As a direct consequence of convexity, we can  bound the term $\Phi(s+d(t))$, $d\in \ephi_1$, by the expression  $\frac12\Phi(2s)+b(t)$ where $b(t):=\tfrac12\Phi(2d(t))\in L^1_1$. Therefore, we can always assume $d = 0$ in \eqref{cotaL} and \eqref{cotaDxL} at the price of making smaller the value of $\lambda$. 
\end{comentario}

  


\begin{comentario} If $\Phi\in\Delta_2$ then we can asssume $d=0$ keeping the same value of $\lambda$. This is consequence of that a non decreasing $\Delta_2$ function $G:\mathbb{R}_+\to\mathbb{R}_+$ is quasi-subadditive. In fact, we suppose $s_1\leq s_2$, then  
\[G(s_1+s_2)\leq G(2s_2)\leq KG(s_2)\leq K\left(G(s_1)+G(s_2)\right).\]
Moreover, if $\Phi$ is $\Delta_2$  then $\varphi$ is also $\Delta_2$, as the following simple argument shows
\[2s\varphi(2s)\leq \alpha \Phi(2s)\leq K\Phi(s)\leq Ks\varphi(s) \]
Here we have used \cite[Th. 4.1]{KR}, the $\Delta_2$ condition for $\Phi$ and the inequality $\Phi(s)\leq s\varphi(s)$ valid for any $N$-function. Therefore if $\Phi$ is $\Delta_2$ we have that

\[\Phi\left(s+d(t) \right)\leq K\Phi(s)+
K\Phi\left(d(t) \right)=K\Phi(s)+b_1(t),\]
where $b_1(t)=b(t)+ K\Phi\left(d(t) \right)\in L^1_1([0,T])$. A similar fact holds with $\varphi$ instead $\Phi$ namely
\[ \varphi\left(s+d(t)\right) \leq c_1(t)+ \varphi\left(s\right),\]
where, as consequence of Lemma \ref{phi_comp} and the $\Delta_2$ condition for $\Phi$, we have   $c_1(t):=K\varphi\left(d(t)\right)\in \lpsi_1$. 


\end{comentario}

\begin{comentario} If $\Phi\notin\Delta_2$ then \eqref{cotadb} may be true or not.  For example, we consider the $N$-function $\Phi(s)=e^s-s-1$ which is not a $\Delta_2$ function. We have that $\ephi_1=L_1^{\infty}$. In fact, if $d\in \ephi_1$ then from the inequality $1/2e^s\leq \Phi(s)+1$ and since $pd\in\claseor_1$, for every $p>0$, we have that $\int_0^Te^{pd(t)}dt<\infty$, for every $p>0$. This imply $d\in L_1^{\infty}$.  Therefore,
\[\Phi(s+d(t))\leq e^{s+d(t)}\leq 2 \|e^d\|_{L^{\infty}}\Phi(s)+2 \|e^d\|_{L^{\infty}}.\] 

On the other hand, we consider  the $N$-function $\Phi(s)=e^{s^2}-1$.  Suppose that $0\leq d\in\ephi_1$ and $b\in L_1^1$ satisfy  \eqref{cotadb}. Then
\[e^{s^2}e^{2sd(t)}\leq \Phi(s+d(t))+1\leq Ce^{s^2}+b(t).\]
Dividing by $e^{s^2}$ and taking the limit $s\to\infty$ we obtain that  $d= 0$ a.e.. In other words, if $d\neq 0$ on a set with positive measure, the bounds in \eqref{cotaL} and \eqref{cotaDxL} are essentially bigger than a bound of the type  $a(|\b{x}|)\left(b(t)+ \Phi\left(|\b{y}|/\lambda \right)\right)$.


\end{comentario}








\begin{thm}\label{teorema_acotacion}
Let $\mathcal{L}$ be a Caratheodory function satisfying \eqref{cotaL},\eqref{cotaDxL}, \eqref{cotaDyL}. Then the following statements hold
\begin{enumerate}
\item \label{T1item1} \label{A1} The \emph{action integral}  
\begin{equation}\label{integral_accion}
I(\b{u}):=\int_{0}^T \mathcal{L}(t,\b{u}(t),\b{\dot{u}}(t))\ dt
\end{equation}
is finitely defined in $ \mathcal{E}^{\Phi}_n(\lambda):=W^{1}\lphi\cap\{\b{u}|\b{\dot{u}}\in\Pi(\ephi_n,\lambda)\}$.

\item\label{T1item3} The function  $I$ is G\^ateaux differentiable on $\domi$ and  its derivative $I'$ is demicontinuous from $\domi$  into $\left[\wphi \right]^*$. Moreover $I'$ is given by the following expression
\begin{equation}\label{DerAccion}
\langle  I'(\b{u}),\b{v}\rangle= \int_0^T \left\{D_{\b{x}}\mathcal{L}\big(t,\b{u},\b{\dot{u}}\big)\ccdot \b{v}+ D_{\b{y}}\mathcal{L}\big(t,\b{u},\b{\dot{u}}\big)\ccdot\b{\dot{v}}\right\} \ dt.
\end{equation}

\item\label{T1item4}  If  $\Psi$ is $\Delta_2$ then 
  $I'$ is continuous from $\domi$ into $\left[\wphi\right]^*$ when both spaces are equipped with the strong topology.


\end{enumerate}
\end{thm}
\begin{proof} From \eqref{inclusion2} we have   $\b{\dot{u}}/\lambda\in\Pi(\ephi_n,1)$. Thus, as $d\in\ephi_1$ and attending to \eqref{inclusiones}, we get 

\begin{equation}\label{inclusion3}
|\b{\dot{u}}|/\lambda+d\in\Pi(\ephi_1,1)\subset \claseor_1.
\end{equation}
From Corollary \ref{a_bound} we get a constant $c=c(\|\b{u}\sobnor )$ such that  $a(|\b{u}(t)|)\leq c$, $t\in [0,T]$.
 Thus,
 \[|\mathcal{L}(t,\b{u},\b{\dot{u}})| \leq c\left(b(t)+ \Phi\left (\frac{|\b{\dot{u}}(t)|}{\lambda}+d(t)\right)  \right)\in
 L^1_1.\]
This fact proves item \ref{T1item1}.

 We split the proof of  \ref{T1item3} in three steps.

\noindent\emph{Step 1. The non linear operator  $\b{u} \mapsto D_{\b{x}}\mathcal{L}(t,\b{u},\b{\dot{u}})$ is continuous from $\domi$ into $L^{1}_n([0,T])$ whith the strong topology on both sets.} 

We take   $\{\b{u}_n\}_{n\in \mathbb{N}}$ a sequence of  functions in $\domi$, and $\b{u}\in \domi$ such that $\b{u}_n\rightarrow \b{u}$ in $\wphi_n$.
Then $\b{u}_n\rightarrow \b{u}$ in $\lphi_n$ and $\b{\dot{u}}_n\rightarrow \b{\dot{u}}$ in $\lphi_n$. By Lemma \ref{segundo lema} there exist a subsequence $\b{u}_{n_k}$ and $h\in \Pi(\ephi_1,\lambda))$
such that $\b{u}_{n_k}\rightarrow \b{u} \quad\text{a.e.}$, $\b{\dot{u}}_{n_k}\rightarrow \b{\dot{u}} \quad\text{a.e.}$ and $|\b{\dot{u}}_{n_k}|\leq h\quad\text{a.e.}$.  Since $\b{u}_{n_k}$, $k=1,2,\ldots$ is a strong convergent sequence in $\wphi_n$, it is a bounded sequence in $\wphi_n$. According to Lemmas \ref{inclusion orlicz} and Corollary \ref{a_bound} there exists $M>0$ such that $\|\b{a}(\b{u}_{n_k})\|_{L^{\infty}} \leq M$, $k=1,2,\ldots$.  From the previous facts, \eqref{cotaDxL} and \eqref{inclusion3} we get
\begin{equation}\label{DxL1}
|D_{\b{x}}\mathcal{L}(t,\b{u}_{n_k}(t),\b{\dot{u}}_{n_k}(t))|\leq M\left(b(t)+\Phi\left(\frac{|h|}{\lambda}+d(t)\right)\right) \in L^1_1.
\end{equation}
By the Caratheodory condition
\[D_{\b{x}}\mathcal{L}(t,\b{u}_{n_k}(t),\b{\dot{u}}_{n_k}(t))\to D_x\mathcal{L}(t,\b{u}(t),\b{\dot{u}}(t))\quad\hbox{ for a.e }t\in[0,T].\]
Applying the Dominated Convergence Theorem we conclude the proof of step 1.

\noindent\emph{Step 2. The non linear operator   $\b{u}
 \mapsto  D_{y}\mathcal{L}(t,\b{u},\b{\dot{u}})$ is continuous from $\domi$ with the strong topology  into $\left[\lphi\right]^*$  with the weak$^*$ topology.}

 Let $\b{u}\in \domi$.  It follows from  \eqref{inclusion3}, Lemma \ref{phi_comp} and Corollary \ref{a_bound} that 
\begin{equation}\label{AcotOperphi}
\varphi\left(\frac{|\b{\dot{u}}|}{\lambda}+d\right)\in C^{\Psi}_1
\end{equation}
and $\b{a}(|\b{u}|)\in L^{\infty}_1$. Therefore, in virtue of  \eqref{cotaDyL} we get
\begin{equation}\label{DyLpsi}
   \left|D_{\b{y}}\mathcal{L}(t,\b{u}(t),\b{\dot{u}}(t))\right|\leq  c(\|\b{u}\|_{\wphi} )  \left(c(t)+\varphi\left( \frac{|\b{\dot{u}}(t)|}{\lambda}+d(t)\right  ) \right)\in\lpsi_1.
\end{equation}
 We note that \eqref{DxL1},  \eqref{DyLpsi} , the imbedding $\wphi_n \hookrightarrow L_n^{\infty}$ and  $\lpsi_n\hookrightarrow  \left[\lphi_n\right]^*$ imply that the second member 
\eqref{DerAccion} defines an element in $\left[\wphi_n\right]^*$.

Now, let us to prove the continuity of the map   $\b{u}\mapsto D_y\mathcal{L}(\cdot,\b{u},\b{\dot{u}})$. We take $\b{u}_n,\b{u}\in \domi$ with $\b{u}_n\to \b{u}$ in the norm of $\wphi_n$. We must prove that  $D_y\mathcal{L}(\cdot,\b{u}_n,\dot{\b{u}_n})\overset{w^*}{\rightharpoonup} D_y\mathcal{L}(\cdot,\b{u},\b{\dot{u}})$. Suppose, on the contrary, that there exists $\b{v}\in\lphi_n$, $\epsilon>0$ and a subsequence of $\{\b{u}_n\}$ (again denoted for simplicity $\{\b{u}_n\}$)  such that
\begin{equation}\label{cota_eps}
 \left| \langle D_{\b{y}}\mathcal{L}(\cdot,\b{u}_n,\b{\dot{u}_n}),\b{v} \rangle - \langle  D_{\b{y}}\mathcal{L}(\cdot,\b{u},\b{\dot{u}}),\b{v} \rangle\right|\geq \epsilon.
\end{equation}
We have $\b{u}_n\rightarrow \b{u}$ in $\lphi_n$ and
$\b{\dot{u}}_n\rightarrow \b{\dot{u}}$ in $\lphi_n$. By Lemma \ref{segundo lema}, there exist a subsequence $\b{u}_{n_k}$ and $h\in \Pi(\ephi,\lambda)$ such that $\b{u}_{n_k}\rightarrow \b{u} \quad\text{a.e.}$, $\b{\dot{u}}_{n_k}\rightarrow \b{\dot{u}} \quad\text{a.e.}$ and $|\b{\dot{u}}_{n_k}|\leq h\quad\text{a.e.}$. As in the previous step, since $\b{u}_n$ is a convergent sequence, the Corrollary \ref{a_bound} implies that $a(|\b{u}_n(t)|)$ is uniformly bounded by certain constant $C$. Therefore, from \eqref{cotaDyL},  \eqref{AcotOperphi}, the fact that $c\in\lpsi_1$, H\"older inequality we obtain
\[
  \left | D_y\mathcal{L}(\cdot,\b{u}_n,\b{\dot{u}}_n) \ccdot \b{ v} \right| \leq C\left(c+\varphi\left(\frac{h}{\lambda}+d\right)\right)|\b{v}|\in L_1^1.
\]
 From the Lebesgue dominated convergence theorem we deduce
\begin{equation}\label{conv_debil}\int_0^T  D_{\b{y}}\mathcal{L}(t,\b{u}_{n_k},\b{\dot{u}}_{n_k})\ccdot\b{ v} dt \to \int_0^T D_{\b{y}}\mathcal{L}(t,\b{u},\b{\dot{u}})\ccdot\b{ v} dt \end{equation}
which contradict the inequality \eqref{cota_eps}. This completes the proof of step 2.

\emph{Step 3.} Finally we prove \ref{T1item3}. The proof follows similar lines that \cite[Theorem 1.4]{mawhin2010critical}. For $\b{u}\in \domi$ and $0\neq\b{v}\in\wphi_n$ we define the function
\[f(s,t):=\mathcal{L}(t,\b{u}(t)+s\b{v}(t),\b{\dot{u}}(t)+s\b{\dot{v}}(t)).\]
From \cite[Th. 10.1]{KR} we obtain that if $|\b{u}|\leq |\b{v}|$ then    $d(\b{u},\ephi_n)\leq d(\b{v},\ephi_n)$. Therefore, for  $|s|\leq s_0:=\left(\lambda-d(\b{\dot{u}},\ephi_n)\right)/\|\b{v}\sobnor$ we have
\[
d \left(\b{\dot{u}}+s\b{\dot{v}}, \ephi_n \right)\leq
d \left(|\b{\dot{u}}|+s|\b{\dot{v}}|, \ephi_1 \right)
\leq d \left(|\b{\dot{u}}|,\ephi_1 \right)+ s \|\b{\dot{v}}\orlnor < \lambda.
\]
As a consequence $\b{\dot{u}}+s\b{\dot{v}} \in \Pi(\ephi_n,\lambda)$ and  $|\b{\dot{u}}|+s|\b{\dot{v}}| \in \Pi(\ephi_1,\lambda)$. These facts imply, in virtue of Theorem \ref{teorema_acotacion}(\ref{T1item1}) that $I(\b{u}+s\b{v})$ is well defined and it is finite for $|s|\leq s_0$. 
Using  Corollary \ref{a_bound} we see that
\[ \|a(|\b{u}+s\b{v}|)\|_{L^{\infty}}\leq  c(\|\b{u}+s\b{v}\sobnor)\leq
 c(\|\b{u}\sobnor+s_0\|\b{v}\sobnor).
\]
Consequently, applying chain rule,  inequalities \eqref{cotaDxL}-\eqref{cotaDyL}, the previous inequality and using that $\varphi$ and $\Phi$ are non decreasing, we obtain
\begin{equation}\label{ctg}
\begin{split}
|D_s f(s,t)|&=\left| D_{\b{x}}\mathcal{L}(t,\b{u}+s\b{v},\b{\dot{u}}+s\b{\dot{v}})\ccdot \b{v} +  D_{\b{y}}\mathcal{L}(t,\b{u}+s\b{v},\b{\dot{u}}+s\b{\dot{v}})\ccdot\b{\dot{v}}\right| \\
 & \leq c \left[\left( b(t)+ \Phi\left(\frac{|\b{\dot{u}}|+s_0|\b{\dot{v}}|}{\lambda}+d\right)\right)|\b{v}|\right.\\
&\left. \quad+ \left(c(t)+ \varphi\left (\frac{|\b{\dot{u}}|+s_0|\b{\dot{v}}|}{\lambda}+d\right)\right)|\b{\dot{v}}| \right]
\end{split}
\end{equation}
 Invoking \eqref{DxL1}, \eqref{DyLpsi} with $ |\b{\dot{u}}|+s_0|\b{\dot{v}}|$ instead $\b{\dot{u}}$ and taking account of $\b{v}\in L^{\infty}$ and $\b{\dot{v}}\in\lphi$ we show that there exists a function $g \in L^1_1([0,T], \mathbb{R}^{+})$
such that $|D_s f(s,t)| \leq g(t)$. Consequently, $I$ has a directional derivative and
\[
\langle I'(\b{u}),\b{v} \rangle=\frac{d}{ds}I(\b{u}+s\b{v})\big|_{s=0}=\int_0^T  D_{\b{x}}\mathcal{L}(t,\b{u},\b{\dot{u}})\ccdot \b{v}+ D_{\b{y}}\mathcal{L}(t,\b{u},\b{\dot{u}})\ccdot\b{\dot{v}}  dt.
\]
Moreover, from \eqref{DxL1}, \eqref{DyLpsi}, Lemma \ref{inclusion orlicz} and previous formula
\[
|\langle I'(\b{u}),\b{v} \rangle| \leq c \|v\linf + c \|\b{\dot{v}}\orlnor \leq c \|\b{v}\sobnor.
\]
This complete the proof of the G\^ateaux differentiability of $I$. Finally, the demicontinuity of $I':\domi  \to \left[\wphi
\right]^* $ is a consequence of the continuity of the mappings $\b{u} \mapsto D_{\b{x}}\mathcal{L}(t,\b{u},\b{\dot{u}})$ and $\b{u} \mapsto
D_{\b{y}}\mathcal{L}(t,\b{u},\b{\dot{u}})$. Indeed, we set $\b{u}_n,\b{u}\in \domi$ with $\b{u}_n\to \b{u}$ in the norm of $\wphi$ and $\b{v} \in
\wphi$ , then
\[
\begin{split}
\left\langle  I'(\b{u}_{n}),\b{v} \right\rangle &= \int_0^T \left\{  D_{\b{x}}\mathcal{L}\left(t,\b{u}_n,\b{\dot{u}}_n\right)\ccdot
\b{v} +
 D_{\b{y}}\mathcal{L}\left(t,\b{u}_n,\b{\dot{u}}_n\right)\ccdot\b{\dot{v}}\right\} \ dt\\
&\rightarrow \int_0^T \left\{ D_{\b{x}}\mathcal{L}\left(t,\b{u},\b{\dot{u}}\right)\ccdot \b{v}+ 
D_{\b{y}}\mathcal{L}\left(t,\b{u},\b{\dot{u}}\right)\ccdot\b{\dot{v}}\right\} \ dt\\
&=\left\langle  I'(\b{u}),\b{v} \right\rangle.
\end{split}
\]


In order to prove  \ref{T1item4}, let us see that the maps $\b{u}\mapsto D_{\b{x}}\mathcal{L}(\cdot,\b{u}(\cdot),\b{\dot{u}}(\cdot))$  and $\b{u}\mapsto D_{\b{y}}\mathcal{L}(\cdot,\b{u}(\cdot),\b{\dot{u}}(\cdot))$  are norm continuous
from $\domi $ into $L^1$ and
 $\lpsi$ respectively.  The continuity of the first map has already been proved in step 1. We will prove the continuity of the second map. We repeat an argument similar to the one given in step 2.  We consider $\b{u}_n$ and $\b{u}$ in $\domi$ with $\|\b{u}_n- \b{u}\sobnor\to 0$.
By Lemma \ref{segundo lema}, there exist a subsequence $\b{u}_{n_k}\in \domi$ and $h\in\Pi(\ephi_1,\lambda)$ such that $\b{u}_{n_k}\rightarrow \b{u} \quad\text{a.e.}$, $\b{\dot{u}}_{n_k}\rightarrow \b{\dot{u}} \quad\text{a.e.}$ and $|\b{\dot{u}}_{n_k}|\leq h\quad\text{a.e.}$. Then  since $\mathcal{L}$ is a Caratheodory function
 we have $ D_{\b{y}}\mathcal{L}(t,\b{u}_{n_k}(t),\b{\dot{u}}_{n_k}(t))\to D_{\b{y}}\mathcal{L}(t,\b{u}(t),\b{\dot{u}}(t))$ a.e. $t\in [0,T]$. Using \eqref{cotaDyL} and that $\Psi$ is of the $\Delta_2$ class, we obtain
 \[\begin{split}
    |D_{\b{y}}\mathcal{L}(t,\b{u}_{n_k}(t),\b{\dot{u}}_{n_k}(t))| &\leq a(|\b{u}_{n_k}(t)|)\left( c(t) + \varphi \left(\frac{|\b{\dot{u}}_{n_k}(t)|}{\lambda}+d(t)\right)\right)\\
    &\leq C\left( c(t) + \varphi \left(\frac{|h(t)|}{\lambda}+d(t)\right)\right)\in \lpsi=\epsi
   \end{split}
\]
Therefore, invoking  Lemma \ref{lema_conv_may}, we have proved that
  from any sequence $\b{u}_n$ which converge to $\b{u}$ in $\wphi$ we can
extract a subsequence with $D_{\b{y}}\mathcal{L}(t,\b{u}_{n_k},\b{\dot{u}}_{n_k})\to D_{\b{y}}\mathcal{L}(t,\b{u},\b{\dot{u}})$ in the strong topology. The desired result follows from a standard argument.

The continuity of $I'$  follows  of the  previously established continuity for $D_{\b{x}}\mathcal{L}$ and $D_{\b{y}}\mathcal{L}$ by using the representation \eqref{DerAccion}.
\end{proof}



\section{Critical points and Euler-Lagrange equations}


In this section we derive the Euler-Lagrange equations associated to critical points of action integrals.  We denote by $\wphi_T$ the subspace of $\wphi_n$ of all  $T$-periodic functions. Similarly we consider the subspaces $\ephi_T$, $\lphi_T$. As is usual, when $Y$ is a subspace of
the Banach space $X$, we denote by $Y^{\perp}$ the \emph{annihilator subspace} of $X^*$, tghat means the subspace
consistent of all the bounded linear functions which are identically zero on $Y$.

We recall that  a function $f: \mathbb{R}^n \to \mathbb{R}$ is called \emph{strictly convex} if $f\left(\tfrac{\b{x}+\b{y}}{2}\right)< \tfrac{1}{2} \left(f\left(
\b{x}\right)+f\left( \b{y}\right)\right)$ for  $\b{x}\neq\b{y}$.  It is a well known that if $f$ is a strictly convex and differentiable functions then
$D_{\b{x}}f:\mathbb{R}^n\to\mathbb{R}^n$ is a one-to-one map  (see, for instance \cite[Theorem 12.17]{rockafellar2009variational}).


\begin{thm} Let $\b{u}\in\domi$. The following statements are equivalent
\begin{enumerate}
 \item $I'(\b{u})\in\left( \wphi_T\right)^{\perp}$
 \item  $D_{\b{y}}\mathcal{L}(t,\b{u}(t),\b{\dot{u}}(t))$ is an absolutely continuous function and $\b{u}$ solve the following boundary value problem
 \begin{equation}\label{ecualagran2}
    \left\{%
\begin{array}{ll}
   \frac{d}{dt} D_{y}\mathcal{L}(t,\b{u}(t),\b{\dot{u}}(t))= D_{\b{x}}\mathcal{L}(t,\b{u}(t),\b{\dot{u}}(t)) \quad \hbox{a.e.}\ t \in (0,T)\\
    \b{u}(0)-\b{u}(T)=D_{\b{y}}\mathcal{L}(0,\b{u}(0),\b{\dot{u}}(0))-D_{\b{y}}\mathcal{L}(T,\b{u}(T),\b{\dot{u}}(T))=0.
\end{array}%
\right.
\end{equation}
\end{enumerate}
Moreover if $D_{\b{y}}\mathcal{L}(t,x,y)$ is $T$-periodic with respect to the variable $t$ and strictly convex with respect to $\b{y}$, then
$D_{\b{y}}\mathcal{L}(0,\b{u}(0),\b{\b{\dot{\b{u}}}}(0))-D_{\b{y}}\mathcal{L}(T,\b{u}(T),\b{\dot{u}}(T))=0$ is equivalent to $\b{\dot{u}}(0)=\b{\dot{u}}(T)$.
\end{thm}

\begin{proof} The condition  $I'(\b{u})\in\left( \wphi_T\right)^{\perp}$ and \eqref{DerAccion} imply 
\[\int_0^T  D_{\b{y}} \mathcal{L}(t,\b{u}(t),\b{\dot{u}}(t))\ccdot \b{\dot{v}}(t) dt=-\int_0^T  D_{\b{x}}\mathcal{L}(t,\b{u}(t),\b{\dot{u}}(t)) \ccdot\b{ v}(t) dt \]
Using \cite[pag. 6]{mawhin2010critical} we obtain that  $D_{\b{y}}\mathcal{L}(t,\b{u}(t),\b{\dot{u}}(t))$ is absolutely continuous and $T$-periodic, therefore it is differentiable a.e.on $[0,T]$ and the first equality of \eqref{ecualagran2} holds true.
This complete the proof  1. implies 2. The proof of 2.implies 1.  is still easier and so  we will omit it.

The last part of the Corollary is a consequence of that $D_{\b{y}}\mathcal{L}(T,\b{u}(T),\b{\dot{u}}(T))=D_{\b{y}}\mathcal{L}(0,\b{u}(0),\b{\dot{u}}(0))=D_{\b{y}}\mathcal{L}(T,u(T),\b{\dot{u}}(0))$ and the injectivity of $D_{\b{y}}\mathcal{L}(T,u(T),\cdot)$.
\end{proof}


\section{Coercivity discussion}

We recall the following usual definition in the context  of calculus of variations. 

\begin{defi} Let $X$ be a Banach space and let $D$ be an unbounded subset of $X$. Suppose $J:D\subset X\to\rr$. We said that $J$ is \emph{coercitive} if $J(u)\to +\infty$ when $\|\b{u}\|\to +\infty$. 
\end{defi}

It is well known that coercitivity is an ingredient useful in order to establish existence of minima. Therefore we are interestent in finding conditions which insure the coercitivity of the action integral $I$ acting on $\domi$. For this purpose we need to introduce the following  extra condition on Lagrange function $\mathcal{L}$  
\begin{equation}\label{cota_inf}
\mathcal{L}(t,\b{x},\b{y})\geq \alpha_0\Phi\left(\frac{|\b{y}|}{\Lambda}\right)+ F(t,\b{x}),
\end{equation}
where $\alpha_0,\Lambda>0$ and  the function $F:\rr\times\rr^n\to\rr$ is measurable respect to $t$ for every fixed  $\b{x}\in\rr^n$ and continuously differentiable in $\b{x}$ for a.e. $t\in [0,T]$. We note that in virtue of \eqref{cota_inf} and \eqref{cotaL} we have that $F(t,\b{x})\leq a(|\b{x}|)b_0(t)$, with $b_0(t):=b(t)+\Phi(d(t))\in L^1_1([0,T])$. In order to ensure that integral $\int_0^TF(t,\b{u})dt$ be finite for $\b{u}\in\wphi$,  we need to assume 

\begin{equation}\label{condA1} |F(t,\b{x})|\leq a(|\b{x}|)b_0(t),\quad\text{for a.e. }t\in [0,T] \text{ and all }\b{x}\in\rr^n.
\end{equation}

As we shall see in Theorem \ref{coercitividad1}, when $\mathcal{L}$ satisfies \eqref{cotaL}, \eqref{cotaDxL}, \eqref{cotaDyL}, \eqref{cota_inf} and \eqref{condA1},  the coercitivity of the action integral $I$ is related to the coercitivity of the functional
\begin{equation}\label{func_phi}
  J_{C,\nu}(\b{u}):=\int_0^T\Phi\left(\frac{|\b{u}|}{\Lambda}\right)dt-C\|\b{u}\orlnor^{\nu},
\end{equation}
for $C,\nu>0$. If $\Phi(x)=|x|^p/p$ then $J_{C,\nu}$ is clearly coercitive for $\nu<p$. For more general $\Phi$ the situation is more interesting   as will be shown in the following lemma.

\begin{lem}\label{lem_coer} Let $\Phi$ and $\Psi$ complementary $N$-functions. Then
\begin{enumerate}
  \item If $C\Lambda<1$ then $J_{C,1}$ is coercive. 
  
  \item If $\Psi$ is $\Delta_2$ global, then there exist a constant $\alpha_{\Phi}>1$ such that for any $0<\mu<\alpha_{\Phi}$,
\begin{equation}\label{coer_modular} \lim\limits_{\|\b{u}\orlnor \to \infty} \frac{\rho_{\Phi}\left(\frac{\b{u}}{\Lambda}\right)}{\|\b{u}\orlnor^{\mu}}=+\infty.
\end{equation}
In particular, the functional $J_{C,\mu}$ is coercive for every $C>0$ and  $0<\mu<a_{\Phi}$. The constant $\alpha_{\Phi}$ is one of the so called \emph{ Matuszewska-Orlicz indices} (see \cite[Ch. 11]{M}).
\item If $J_{C,1}$ is coercitive with $C\Lambda>1$, then $\Psi$ is $\Delta_2$ at infinity.  
\end{enumerate}
\end{lem}

\begin{proof} Using \eqref{amemiya} we obtain
\[(1-C\Lambda)\|\b{u}\orlnor+C\Lambda\|\b{u}\orlnor=\|\b{u}\orlnor\leq \Lambda +\Lambda \rho_{\Phi}\left(\frac{\b{u}}{\Lambda}\right).\]
Then
\[\frac{(1-C\Lambda)}{\Lambda}\|\b{u}\orlnor-1\leq \rho_{\Phi}\left(\frac{\b{u}}{\Lambda}\right)- C\|\b{u}\orlnor=J_{C,1}(\b{u}).\]
This show that $J_{C,1}$ is coercitive, and therefore it proves item 1.  

In virtue of \cite[Eq. (2.8)]{AGMS}, the $\Delta_2$ condition for $\Psi$, \cite[Th 11.7]{M} and \cite[Cor. 11.6]{M} we obtain a constant $K>0$ and $\alpha_{\Phi}>1$ such that for any $0<\nu<\alpha_{\Phi}$, $s\geq 0$ and $r>1$ 
\begin{equation}\label{delta2-consecuencia}
\Phi(r s)\geq Kr^{\nu}\Phi(s).
\end{equation}
Let $1<\mu<\alpha_{\Phi}$ and we consider a constant $r>\Lambda$ that later will be specify.  Then, from \eqref{delta2-consecuencia} and \eqref{amemiya}, we get
\[
\begin{split}
\frac{\int_0^T \Phi\left(\frac{|\b{u}|}{\Lambda}\right)dt}{\|\b{u}\orlnor^{\mu}}
&\geq
K \left(\frac{r}{\Lambda}\right)^{\nu}\frac{\int_0^T \Phi(r^{-1}|\b{u}|)dt}{\|\b{u}\orlnor^{\mu}}\\
&\geq
K \left(\frac{r}{\Lambda}\right)^{\nu}\frac{r^{-1}\|\b{u}\orlnor-1}{\|\b{u}\orlnor^{\mu}}\\
\end{split}
\]
We choose $r=\|\b{u}\orlnor/2$. Since $\|\b{u}\orlnor\to+\infty$   we can assume $\|\b{u}\orlnor>2\Lambda$.  Thus $r>\Lambda$ and 

\[
\frac{\int_0^T \Phi\left(\frac{|\b{u}|}{\Lambda}\right) dt}{\|\b{u}\orlnor^{\mu}}\geq
\frac{K}{2^{\nu}\Lambda^{\nu}} \|\b{u}\orlnor^{\nu-\mu}\to +\infty\quad\text{for }\|\b{u}\orlnor\to+\infty,
\]
because $\nu>\mu$.

In order to prove the last item, we assume that $\Psi \notin \Delta_2$. By \cite[Th. 4.1]{KR},  there exists a sequence of real  numbers  $r_n$ such that
$r_n \to \infty$ and 
\begin{equation}\label{eq: un-_tiende_inf}
\lim\limits_{n \to \infty} \frac{r_n \psi(r_n)}{\Psi(r_n)}=+\infty.
\end{equation}
Now, we choose $\b{u}_n$, such that
$|\b{u}_n|=\Lambda\psi(r_n)\chi_{[0,\frac{1}{\Psi(r_n)}]}$, then 
by \cite[Eq. (9.11)]{KR}, we get 
\[
\|\b{u}_n\orlnor =\Lambda\frac{\psi(r_n)}{\Psi(r_n)}\Psi^{-1}(\Psi(r_n))=
\Lambda\frac{r_n\psi(r_n)}{\Psi(r_n)}\to \infty,\quad\text{as}\quad n \to \infty.
\]
Next, using Young's equality (see \cite[Eq. (2.7)]{KR}), we have
\[
\begin{split}
J_{C,1}(\b{u}_n)&=\int_0^T \Phi\left(\frac{|\b{u}_n|}{\Lambda}\right)-C\|\b{u}_n\orlnor\\
&=
\frac{1}{\Psi(r_n)}\left[\Phi(\psi(r_n))  -C\Lambda r_n\psi(r_n)\right]\\
&=
\frac{1}{\Psi(r_n)} \left[ r_n\psi(r_n)-\Psi(r_n)- C\Lambda r_n\psi(r_n) \right]\\
&=\frac{(1- C\Lambda) r_n\psi(r_n)}{\Psi(r_n)}-1.
\end{split}
\]
From \eqref{eq: un-_tiende_inf} and the condition $C\Lambda>1$, we obtain  $J_{C,1}(\b{u}_n)\to-\infty$, which is a contradiction.
\end{proof}

We present two theorems establishing coercitivity of action integrals. 



\begin{thm}\label{coercitividad1}
Let  $\mathcal{L}$ be a Lagrangian function satisfying \eqref{cotaL}, \eqref{cotaDxL}, \eqref{cotaDyL}, \eqref{cota_inf} and \eqref{condA1}. We assume  following conditions 
\begin{enumerate}
\item There exists a non negative function  $b_1 \in L^1_1$ and a constant $\mu>0$  such that for any $\b{x_1},\b{x_2}\in\rr^n$ and a.e. $t\in [0,T]$
\begin{equation}\label{holder_cont}
  \left| F(t,\b{x_2})- F(t,\b{x_1}) \right|\leq b_1(t)(1+|\b{x_2}-\b{x_1}|^{\mu}).
\end{equation}
We suppose that the constant $\mu$ satisfies that $\mu< \alpha_{\Phi}$,  with $\alpha_{\Phi}$ as in Lemma \ref{lem_coer}, when $\Psi\in\Delta_2$ and that $\mu=1$  if $\Psi$ is an  arbitrary $N$-function. 
\item
\begin{equation}\label{propiedad1coercividad}
\int_{0}^{T}F(t,\b{x})\ dt \rightarrow \infty \quad \hbox{as} \quad |\b{x}|\rightarrow \infty,
\end{equation}
\item\label{hipot_coer}  $\Psi\in\Delta_2$ or, alternatively, the following inequality $\alpha_0^{-1}T\Psi^{-1}\left(1/T\right)\|b_1\|_{L^1}\Lambda<1$.
\end{enumerate}
Then  the action integral $I$ is coercive.
\end{thm}

\begin{proof} In the following estimates we will use \eqref{cota_inf}, the descomposition $\b{u}=\b{\overline{u}}+\b{\tilde{u}}$, H\"older and the Wirtinger inequality \eqref{wirtinger}. 
\begin{equation}\label{cota_inf_I}
\begin{split}
I(\b{u})&\geq\alpha_0\rho_{\Phi}\left( \frac{|\b{\dot{u}}|}{\Lambda}\right)+\int_0^TF(t,\b{u})dt\\ 
&=\alpha_0\rho_{\Phi}\left( \frac{|\b{\dot{u}}|}{\Lambda}\right)+ \int_0^TF(t,\b{u})-F(t,\b{\overline{u}})dt +  \int_0^TF(t,\b{\overline{u}})dt\\
&\geq\alpha_0\rho_{\Phi}\left( \frac{|\b{\dot{u}}|}{\Lambda}\right)- \int_0^Tb_1(t)(1+|\b{\tilde{u}}(t)|^{\mu})dt +  \int_0^TF(t,\b{\overline{u}})dt\\
&\geq \alpha_0\rho_{\Phi}\left( \frac{|\b{\dot{u}}|}{\Lambda}\right)- \|b_1\|_{L^1}(1+\|\b{\tilde{u}}\|_{L^{\infty}}^{\mu}) +  \int_0^TF(t,\b{\overline{u}})dt\\
&\geq\alpha_0\rho_{\Phi}\left( \frac{|\b{\dot{u}}|}{\Lambda}\right)- \|b_1\|_{L^1}\left(1+\left[T\Psi^{-1}\left(\frac{1}{T}\right)\right]^{\mu}\|\b{\dot u}\orlnor^{\mu}\right) \\
&\quad+  \int_0^TF(t,\b{\overline{u}})dt\\
&=\alpha_0J_{C,\mu}(\b{\dot{u}})- \|b_1\|_{L^1}+ \int_0^TF(t,\b{\overline{u}})dt,
\end{split}
\end{equation}
where $C=\alpha_0^{-1}\left[T\Psi^{-1}\left(1/T\right)\right]^{\mu}\|b_1\|_{L^1}$.
Suppose  $\b{u}_n$ a sequence in $\domi$ such that i) the  sequence  $\b{\overline{u}}_n$ is bounded in $\rr^n$ and ii) $\|\b{u}_n\sobnor\to\infty$. Then  the Wirtinger inequality \eqref{wirtinger} implies that $\|\b{\dot{u}}_n\orlnor\to\infty$. Therefore one of the following affirmation holds true $\|\b{\dot{u}}_n\orlnor\to\infty$ or $|\b{\overline{u}}_n|\to \infty$. On the other hand,  \eqref{holder_cont} and \eqref{propiedad1coercividad}
imply that the integral $\int_0^TF(t,\b{\overline{u}}_n)dt$ is bounded from below.  These observations, the lower bound \eqref{cota_inf_I} of $I$, 
assumption \ref{hipot_coer} in Theorem \ref{coercitividad1} and Lemma \ref{lem_coer} imply the desidered result.
\end{proof}

Following \cite{mawhin2010critical} we said that $F$ satisfies the condition (A) if it satisfies \eqref{condA1} and 
\begin{equation}\label{condA2} |D_{\b{x}}F(t,\b{x})|\leq a(|\b{x}|)b_0(t),\quad\text{for a.e. }t\in [0,T] \text{ and all }\b{x}\in\rr^n,
\end{equation}
The following result was proved in \cite[p. 18]{mawhin2010critical}. 
\begin{lem}\label{lema_pto_cri} Suppose that $F$ satisfies condition (A),\eqref{propiedad1coercividad}, that $F(t,\cdot)$ is  differentiable and convex  a.e. $t\in [0,T]$. Then there exists $\b{x}_0\in\rr^n$ such that
\begin{equation}\label{der_cero}
 \int_0^T D_{\b{x}} F(t,\b{x}_0) dt=0.
\end{equation}
\end{lem}


\begin{thm}
Let $\mathcal{L}$ be as in Theorem \ref{coercitividad1} and  $F$ as in Lemma \ref{lema_pto_cri}. In addition assume $\Psi\in\Delta_2$ or, alternatively, the following inequality $\alpha_0^{-1}T\Psi^{-1}\left(1/T\right)a(|\b{x}_0|)\|b_0\|_{L^1}<1$, with $a$ and $b_0$ as in \eqref{condA1} and $\b{x}_0\in\rr^n$  any point satisfying  \eqref{der_cero}. Then $I$ is coercive. 

\end{thm}


\begin{proof}  Using \eqref{cota_inf}, \cite[Eq. 18, p. 17]{mawhin2010critical},  descomposition $\b{u}=\b{\overline{u}}+\b{\tilde{u}}$,  \eqref{der_cero}, \eqref{holder} and  \eqref{wirtinger} we deduce

\begin{equation}\label{cota_con _upunto}
\begin{split}
I(\b{u})&\geq\alpha_0\rho_{\Phi}\left( \frac{|\b{\dot{u}}|}{\Lambda}\right)+\int_0^T F(t,\b{x}_0) dt + \int_0^T D_{\b{x}} F (t,\b{x}_0) \ccdot (\b{u}-\b{x}_0) dt\\
&\geq\alpha_0\rho_{\Phi}\left( \frac{|\b{\dot{u}}|}{\Lambda}\right) +\int_0^T F(t,\b{x}_0) dt + \int_0^TD_{\b{x}}F (t,\b{x}_0) \ccdot \b{\widetilde{u}} dt\\
&\quad + \int_0^T D_{\b{x}} F (t,x_0) \ccdot (\b{\overline{u}}  -\b{x}_0) dt\\
&\geq\alpha_0\rho_{\Phi}\left( \frac{|\b{\dot{u}}|}{\Lambda}\right)+\int_0^T F(t,\b{x}_0) dt + \int_0^T D_{\b{x}} F (t,\b{x}_0) \ccdot \b{\widetilde{u}} dt\\
&\geq\alpha_0\rho_{\Phi}\left( \frac{|\b{\dot{u}}|}{\Lambda}\right)-a(|\b{x}_0|)\|b_0\|_{L^1} -a(|\b{x}_0|)\|b_0\|_{L^1}T\Psi^{-1}\left(\frac{1}{T}\right) \|\b{\dot{u}}  \|_{\Phi}\\
&= \alpha_0^{-1}J_{C,1}(\b{\dot{u}})-a(|\b{x}_0|)\|b_0\|_{L^1} .
\end{split}
\end{equation}
with $C:=\alpha_0^{-1}a(|\b{x}_0|)\|b_0\|_{L^1}T\Psi^{-1}(1/T)$. 

Let $\alpha$ be as in Corollary \ref{a_bound}, it is  a non decresing mayorant of $a$. Using that  $F(t, \b{\overline{u}} /2) \leq (1/2)F(t,\b{u}) + (1/2) F(t, -\b{\widetilde{u}})$ and taking account of the non negativity of $\Phi$, inequality \eqref{condA1}, H\"older inequality,  Corollary \ref{a_bound} and Wirtinger inequality \eqref{wirtinger}, we obtain
\begin{equation}\label{cota_con _ubarra}
\begin{split}
I(\b{u}) &\geq\alpha_0\rho_{\Phi}\left( \frac{|\b{\dot{u}}|}{\Lambda}\right)  +2 \int_0^T F(t,\b{\overline{u}} /2) dt - \int_0^T F(t, -\b{\widetilde{u}}) dt\\
&\geq 2 \int_0^T F(t,\b{\overline{u}} /2) dt -\|b_0\|_{L^1} \|\b{a}(\b{\tilde{u}})\|_{L^{\infty}}\\
&\geq 2 \int_0^T F(t,\b{\overline{u}} /2) dt -\|b_0\|_{L^1} \alpha(\|\b{\tilde{u}}\|_{L^{\infty}})\\
&\geq 2 \int_0^T F(t,\b{\overline{u}} /2) dt -C'\|\b{\dot{u}}\orlnor.
\end{split}
\end{equation}

Let $\b{u}_n$ be a sequence in $\wphi$ with $\|\b{u}_n\sobnor\to\infty$. We need consider two situations: i) $\|\b{u}_n\orlnor\to\infty$, in this case \eqref{cota_con _upunto} and Lemma \ref{lem_coer} imply $I(\b{u}_n)\to\infty$, ii) $\|\b{\dot{u}}_n\orlnor$ bounded and $\|\b{u}_n\orlnor\to\infty$, in this case we have, by a similar reasoning  that in proof of Theorem \ref{coercitividad1},  $\b{\overline{u}}_n \to\infty$. This fact, together \eqref{cota_con _ubarra}, finished the proof. \end{proof}

\section{Duality and weak lower semicontinuity of actions integrals}

The objective of this section is to obtain condictions 
which ensure lower semicontinuity of the action integral $I$ on its domain $\domi$. The strategy we will use is to split the action integral into two functional integrals, one convex and the other continuous. 
It will be shown that the convex part is lower  semicontinuous by a duality argument involving the Fenchel transform of $I$. 

We star for considering Lagrangians $\mathcal{L}:\rr\times\rr^n\to\rr^n$  satisfying 
\begin{equation}\label{reduc_lagran}
|\mathcal{L}(t,\b{y})|\leq b(t)+a_0\Phi\left(\frac{|\b{y}|}{\lambda}\right),
\end{equation}
with $a_0>0$ and $b\in L^1_1([0,T])$. The operators induced by $\mathcal{L}$ were studied in the Lieterature \cite[]{KR}The following lemma is an inmediate consequence of principles  related to  operators of Nemitskii type, see \cite[�17]{KR}.


 \begin{lem}\label{unif_conv}
If the sequence $\{\b{u}_{k}\}_{k \geq 1}$ converges weakly to $\b{u}$ in $\wphi$, then $\{\b{u}_{k}\}_{k\geq 1}$ converges uniformly to $\b{u}$ on $[0,T]$.
\end{lem}
\begin{proof}
By Lemma \ref{inclusion orlicz}, the injection of $\wphi$ in $L^{\infty}$ is continuous. Since $\b{u}_{k}\rightharpoonup \b{u}$ in $\wphi$ it follows that
$\b{u}_{k}\rightharpoonup \b{u}$ in $C(0,T;\mathbb{R}^n)$. Since $\b{u}_{k}\rightharpoonup \b{u}$ in $\wphi$, we know that $\{\b{u}_{k}\}_{k \geq 1}$ is bounded in
$\wphi$ and, hence by \eqref{estimacion} in $C(0,T;\mathbb{R}^n)$. Moreover, the sequence $\{\b{u}_{k}\}_{k \geq 1}$ is equi-uniformly continuous since, for $0 \leq
s\leq t \leq T $, we have
\[
\begin{split}
\left|\b{u}_{k}(t)-\b{u}_{k}(s) \right|&\leq \int_{s}^t \left| \dot{\b{u}}_{k}(\tau)\right|\ \ d\tau \leq \| t-s\|_{\lpsi}\|\dot{\b{u}}_{k}\|_{\lphi}\\
&\leq \| t-s\|_{\lpsi}\|\b{u}_{k}\|_{\wphi} \leq C \| t-s\|_{\lpsi}.
\end{split}
\]
By Arzela-Ascoli theorem, $\{\b{u}_{k}\}_{k \geq 1}$ is relatively compact in $C(0,T;\mathbb{R}^n)$. By the uniqueness of the weak limit in $C(0,T;\mathbb{R}^n)$,
every uniformly convergent subsequence of $\{\b{u}_{k}\}_{k \geq 1}$ converges to $\b{u}$. Thus, $\{\b{u}_{k}\}_{k \geq 1}$ converges uniformly on $[0,T]$.

\end{proof}

\begin{thm}
We suppose that $\mathcal{L}(t,\b{x},\b{y})$ is a Charateodory functions satisfying \eqref{cotaL}-\eqref{cotaDyL}.
Moreover we assume $\mathcal{L}(t,\b{x},\cdot)$ is convex for each $t,\b{x}$. We suppose that $\Phi,\Psi$ are $\Delta_2$ functions. Then the functional \eqref{integral_accion} is weakly lower semicontinuous (w.l.s.c.).
\end{thm}



\printbibliography

\end{document}
