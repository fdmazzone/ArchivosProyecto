%%% Preambulo%%%%%%%%%%%%%%%%%%%%%%%%%

\documentclass[twoside]{article}
%%Paquetes


\usepackage{amssymb,amsthm}
\usepackage{amsmath}
\usepackage{color}
\usepackage{ esint }
%\usepackage{graphicx}
%\usepackage{wrapfig}
%\usepackage{subfigure}
\usepackage{fancyhdr}
\usepackage{times}
%\usepackage{theorem}
\usepackage[latin1]{inputenc}
%\usepackage{showkeys}
\usepackage{comment}
\usepackage{url}
\usepackage{xcolor}
\usepackage{adjustbox}
%Teorema y similes

\definecolor{rosa}{rgb}{1,0.3,0.9}
\definecolor{violeta1}{rgb}{0.5,0.3,0.5}
\definecolor{violeta}{rgb}{0.5,0.1,0.5}
\definecolor{negro}{rgb}{0.5,0.2,0.4}
\definecolor{celeste}{rgb}{0.1,0.4,1}
\definecolor{naranja}{rgb}{1,0.5,0}
\definecolor{color_nota_fer}{HTML}{DEBFDB}


\newenvironment{colbox}[2]{%
    \begin{adjustbox}{minipage={\linewidth},margin=1ex,bgcolor=#1,env=center}
        #2}{%
    \end{adjustbox}%
}
\newcounter{nota_fer_cont}
\newenvironment{nota_fer}[1]{\refstepcounter{nota_fer_cont}\begin{colbox}{color_nota_fer}{\textbf{Comentario Leo-Graciela-Fernando \arabic{nota_fer_cont}.} #1}}{\end{colbox}}


\newtheorem{thm}{Theorem}[section]
\newtheorem{cor}[thm]{Corollary}
\newtheorem{lem}[thm]{Lemma}
\newtheorem{rem}[thm]{Remark}
\newtheorem{defi}[thm]{Definition}
\newtheorem{prop}[thm]{Proposition}
\theoremstyle{remark}
\newtheorem{comentario}{Remark}


\title{Euler-Lagragian equations in an Orlicz-Sobolev space setting}
\author{Sonia Acinas \thanks{SECyT-UNRC}\\
Dpto. de Matem\'atica, Facultad de Ciencias Exactas y Naturales\\
Universidad Nacional de La Pampa\\
(6300) Santa Rosa, La Pampa, Argentina\\
\url{sonia.acinas@gmail.com}\\[3mm]
Leopoldo Buri \thanks{SECyT-UNRC}\\
Dpto. de Matem\'atica, Facultad de Ciencias Exactas, F\'{\i}sico-Qu\'{\i}micas y Naturales\\
Universidad Nacional de R\'{i}o Cuarto\\
(5800) R\'{\i}o Cuarto, C\'ordoba, Argentina,\\
\url{lburi@exa.unrc.edu.ar}\\[3mm]
Graciela Giubergia \thanks{SECyT-UNRC and CONICET}\\
Dpto. de Matem\'atica, Facultad de Ciencias Exactas, F\'{\i}sico-Qu\'{\i}micas y Naturales\\
Universidad Nacional de R\'{i}o Cuarto\\
(5800) R\'{\i}o Cuarto, C\'ordoba, Argentina,\\
\url{ggiubergia@exa.unrc.edu.ar}\\[3mm]
Fernando D. Mazzone \thanks{SECyT-UNRC and CONICET}\\
Dpto. de Matem\'atica, Facultad de Ciencias Exactas, F\'{\i}sico-Qu\'{\i}micas y Naturales\\
Universidad Nacional de R\'{i}o Cuarto\\
(5800) R\'{\i}o Cuarto, C\'ordoba, Argentina,\\
\url{fmazzone@exa.unrc.edu.ar}\\[3mm]
Erica L. Schwindt\thanks{ANR. AVENTURES - ANR-12-BLAN-BS01-0001-01}\\
Universit\'{e} d'{O}rl\'{e}ans, Laboratoire MAPMO, CNRS, UMR 7349, \\
F\'ed\'eration Denis Poisson, FR 2964,\\
B\^{a}timent de Math\'{e}matiques, BP 6759, 45067 Orl\'{e}ans Cedex 2, France,\\
\url{leris98@gmail.com}}

\date{}

\newcommand{\orlnor}{\|_{L^{\Phi}}}
\newcommand{\lurnor}{\|^{*}_{L^{\Phi}}}
\newcommand{\linf}{\|_{L^{\infty}}}
\newcommand{\lphi}{L^{\Phi}}
\newcommand{\lpsi}{L^{\Psi}}
\newcommand{\ephi}{E^{\Phi}}
\newcommand{\claseor}{C^{\Phi}}
\newcommand{\wphi}{W^{1}\lphi}
\newcommand{\sobnor}{\|_{W^{1}\lphi}}
\newcommand{\domi}{\mathcal{E}^{\Phi}_n(\lambda)}
\renewcommand{\b}[1]{\boldsymbol{#1}}
\newcommand{\rr}{\mathbb{R}}
\newcommand{\nn}{\mathbb{N}}
\newcommand{\ccdot}{\b{\cdot}}
\renewcommand{\leq}{\leqslant} 


\begin{document}



\maketitle
%
\begingroup%Locallizing the change to `thefootnote'.
    \renewcommand{\thefootnote}{}%Removing the footnote symbol.
    %
    \footnotetext{%
    %   2010 Mathematics Subject Classification
    %   http://www.ams.org/msc/
    \textbf{2010  AMS Subject Classification.} Primary: .
    Secondary: .
    }%
        \footnotetext{%
    \textbf{Keywords and phrases.}  .
    }%
    \endgroup
%
%
%
%

\begin{abstract}
...
\end{abstract}




\pagestyle{fancy} \headheight 35pt \fancyhead{} \fancyfoot{}

\fancyfoot[C]{\thepage} \fancyhead[CE]{\nouppercase{S. Acinas, L. Buri, G. Giubergia, F. Mazzone and E. Schwindt}} \fancyhead[CO]{\nouppercase{\section}}

\fancyhead[CO]{\nouppercase{\leftmark}}


%\tableofcontents

\section{Introduction}

\section{Preliminaries}

%\subsection{$N$-functions}
For reader convenience, we give a short introduction to Orlicz and Orlicz Sobolev spaces of vector valued functions and a  list  of results that we will use throughout the article. We refer to \cite{adams_sobolev,KR,wroblewska2012application} for additional details and proofs. In the first two references scalar valued function are considered, however the generalization of the results enumerated below  to vector valued functions is direct. Last one reference consider vector valued functions.

Hereafter we denote  by $\mathbb{R}^+$ to the set of all non negative real numbers. A function $\Phi:\mathbb{R}^+\to \mathbb{R}^+ $ is called an \emph{$N$-function} if it has the form
\[
\Phi(t)=\int_{0}^t \varphi(\tau)\ d\tau,\quad\hbox{for } u\geq 0,
\]
where $\varphi:\mathbb{R}^+\rightarrow :\mathbb{R}^+$ is a right continuous nondecreasing function  satisfying   $\varphi(0)=0$, $\varphi(t)>0$ for $t>0$ and
$\lim_{t\rightarrow \infty}\varphi(t)=+\infty$.

Given a function $\varphi$ as above, we also consider the so-called right inverse function $\psi$ of $\varphi$ which is defined $\psi(s)=\sup_{\varphi(t)\leq s}t$.
The function $\psi$ satisfies the same properties that function $\varphi$, therefore we have an $N$-function $\Psi$ such that $\Psi'=\psi$ . The function $\Psi$ is called the
\emph{complementary function} of $\Phi$.

We say that $\Phi$ is a \emph{function of the $\Delta_2$ class} when there exists a constant $k>0$ and a $t_0\geq 0$ such that $\Phi(2t)\leq K\Phi(t)$, for every $t\geq t_0$.

 In this paper we adopt the convention of to use bold symbols for denote points in $\mathbb{R}^n$ and plain symbols for scalar ones.

For  $n$  positive integer we denote by $M_n:=M_n([0,T])$ the set of all measurable functions defined in $[0,T]$ with values in $\mathbb{R}^n$.  Given  a $N$-function $\Phi$ we define the \emph{modular function} $\rho_{\Phi}:M_n\to :\mathbb{R}^+\cup\{+\infty\}$ by
\[\rho_{\Phi}(\b{u}):= \int_{[0,T]} \Phi(|\b{u}|)\ dt.\]
Here $|\cdot|$ is the euclidean norm of $\mathbb{R}^n$.
The \emph{Orlicz class} $C_n^{\Phi}=C_n^{\Phi}([0,T])$  is defined by
\begin{equation}\label{claseOrlicz}
  C^{\Phi}_n:=\left\{\b{u}\in M_n | \rho_{\Phi}(\b{u})< \infty \right\}.
\end{equation}
The \emph{Orlicz space} $\lphi_n=L^{\Phi}_n([0,T])$ is the linear hull of $\claseor_n$.
Equivalently
\begin{equation}\label{espacioOrlicz}
\lphi_n:=\left\{ \b{u}\in M_n | \exists \lambda>0: \rho_{\Phi}(\lambda \b{u}) < \infty   \right\}.
\end{equation}
  The Orlicz space $\lphi_n$ equipped with the \emph{Orlicz norm}
\[
\|  \b{u}  \orlnor:=\sup \left\{  \left.\int_0^T \b{u}\b{\cdot} \b{v} dt \right| \rho_{\Psi}(\b{v})\leq 1\right\},
\]
is a Banach space. By $\b{u}\b{\cdot} \b{v}$ we denote the usual dot product in $\mathbb{R}^{n}$ between $\b{u}$ and $\b{v}$.




The subspace $\ephi_n=\ephi_n([0,T])$ is defined as the closure in $\lphi_n$ of the subspace $L^{\infty}_n$ of all the $\mathbb{R}^n$-valued essentially bounded functions. It is showed that  $\ephi_n$ is the only one maximal subspace contained in the Orlicz class $\claseor$, that is $\b{u}\in\ephi_n$ if and only if for any $\lambda>0$ we have $\rho_{\Phi}(\lambda \b{u})<\infty$.  

A generalizated version of \emph{H\"older inequality} holds in the setting of Orlicz spaces (ver \cite[Th 9.3]{KR} ). Namely, if $\b{u}\in\lphi_n$ and $\b{v}\in\lpsi_n$ then $\b{u}\ccdot\b{v}\in L_1^1$ and
\begin{equation}\label{holder}
\int_0^T\b{u}\ccdot\b{v}dt\leq \|\b{u}\orlnor\|\b{v}\|_{L^{\Psi}}.
\end{equation}




If $X$ and $Y$ are  Banach spaces, with $Y\subset X^*$ we denote by $\langle\cdot,\cdot\rangle:X\times Y\to\mathbb{R}$ to the bilinear pairing  map given by $\langle x,x^*\rangle=x^*(x)$. H\"older inequality shows that $\lpsi_n\subset \left[\lphi_n\right]^*$, where the pairing $\langle \b{u},\b{v}\rangle$, $\b{u}\in\lphi_n$ and $\b{v}\in\lpsi_n$, is defined by 
\begin{equation}\label{pairing}
  \langle \b{u},\b{v}\rangle=\int_0^T\b{u}\ccdot\b{v}dt.
\end{equation}
 Unless $\Phi$ be a $\Delta_2$ function, the relation $\lpsi_n= \left[\lphi_n\right]^*$ does not holds. It is true in general that  $\left[\ephi_n\right]^*=\lpsi_n$.


Likes in \cite{KR}, we will consider the subset $\Pi(\ephi_n,r)$ of $\lphi_n$ defined by
\[\Pi(\ephi_n,r):=\{\b{u}\in\lphi_n| d(\b{u},\ephi_n)<r\}.\]
This set is related to the Orlicz class $\claseor_n$ by means of inclusions
\begin{equation}\label{inclusiones}\Pi(\ephi_n,1)\subset \claseor_n \subset\overline{\Pi(\ephi_n,1)}.\end{equation}
The proof of this fact, and similar ones, is given by real valued function in \cite{KR},
the extension to $\mathbb{R}^n$-valued functions does not involve any difficulty. When the function $\Phi$ is of the $\Delta_2$ class then the four sets $\lphi_n$, $\ephi_n$ $\Pi(\ephi_n,1)$ and $\claseor_n$ are equal.

We will use the following elementary fact frequently 
\begin{equation}\label{inclusion2}
\b{u}\in\Pi(\ephi_n,\lambda)\implies \frac{\b{u}}{\lambda}\in\Pi(\ephi_n,1)\subset\claseor_n.
\end{equation}

We define the \emph{Sobolev-Orlicz space} $\wphi_n$ (see \cite{adams_sobolev}) by
\[\wphi_n:=\{\b{u}| \b{u} \hbox{ is absolutely continuous and } \b{u},\b{\dot{u}}\in \lphi_n\}.\]
This space is a Banach space  equipped with the norm
\[
\|  \b{u}  \|_{\wphi}= \|  \b{u}  \|_{\lphi} + \|\b{\dot{u}}\orlnor.
\]



For a  function $\b{u}\in L^1_n([0,T])$, we write $\b{u}=\overline{\b{u}}+\widetilde{\b{u}}$, where $\overline{\b{u}} =\frac1T\int_0^T \b{u}(t)\ dt$ and $\widetilde{\b{u}}=\b{u}-\overline{\b{u}}$.

 An important aspect of the theory of Sobolev spaces is related to embedding theorems. There is an extensive literature on this question in the setting of Orlicz-Sobolev spaces, see for example
 \cite{cianchi1999some,cianchi2000fully,claverooptimal,edmunds2000optimal,kerman2006optimal}.
For this reason the following simple  Lemma, which we will use systematically, it is well known. We include a brief proof for sake of completeness.

 % As is usual, if $X$ and $Y$ are normed spaces, with $X\subset Y$,  we write $X\hookrightarrow Y$ when the identity map is an bounded operator between $X$ and $Y$.

\begin{lem}\label{inclusion orlicz} Let $\b{u}\in\wphi_n$. Then $\b{u}\in L^{\infty}_n([0,T])$ and
\begin{align}
  \|\widetilde{\b{u}}\|_{L^{\infty}} &\leq T\Psi^{-1}\left(\frac{1}{T}\right)\|\b{\dot u}\orlnor&\text{  (Wirtinger's inequality)}\label{wirtinger}\\
\|\b{u}\|_{L^{\infty}} &\leq\Psi^{-1}\left(\frac{1}{T}\right)\max\{1,T\}\|\b{u}\sobnor&\text{  (Sobolev's inequality)}\label{sobolev}
\end{align}

\end{lem}
\begin{proof}
Since $\b{u}$ is contiuous, from the mean value theorem there exists $\tau$ such that
$\b{u}(\tau)=\overline{\b{u}}$, thus
\begin{equation}\label{desigualdad1}\begin{split}
|\b{u}(t)-\overline{\b{u}}|\leq \int\limits_{\tau}^{t}|\b{\dot{u}}(s)|ds
\leq \|\b{\dot{u}}\|_{L^{\Phi}}\|1\|_{L^{\Psi}}\leq T\Psi^{-1}\left(\frac{1}{T}\right)\|\b{\dot u}\orlnor.
\end{split}
\end{equation}
Here we have used H\"older inequality and the formula for the norm of a characteristic function (ver  \cite[Eq. 9.11]{KR}). Inequality \eqref{desigualdad1} proves Wirtinger's inequality \eqref{wirtinger}. 

On the other hand, again by H\"older inequality and \cite[Eq. 9.11]{KR}, we obtain
\begin{equation}\label{desigualdad2}\begin{split}
|\overline{\b{u}}|\leq \frac{1}{T}\int\limits_{0}^{T}|\b{u}(s)|ds\leq \Psi^{-1}\left(\frac{1}{T}\right)\|\b{u}\orlnor.
\end{split}
\end{equation}
From \eqref{desigualdad1},\eqref{desigualdad2} and since $\b{u}=\overline{\b{u}}+\widetilde{\b{u}}$  we obtain \eqref{sobolev}.
\end{proof}

If $(X,\|\cdot\|_X)$ is a Banach space and $(Y,\|\cdot \|_Y)$ is a subespace of $X$, as is usual we write $Y\hookrightarrow X$ and we say that $Y$ is \emph{embeeded} in $X$  when the restricted identity map $i_Y:Y\to X$ is bounded. That means that there exists $C>0$ such that  for any $y\in Y$ we have $\|y\|_X\leq C\|y\|_Y$.  With this notation, the Lemma \ref{inclusion orlicz} states $\wphi_n \hookrightarrow L_n^{\infty}$ and H\"older inequality states that  $\lpsi_n\hookrightarrow  \left[\lphi_n\right]^*$.


 Given a contiuous function $a\in C(\mathbb{R}^+,\mathbb{R}^+)$, we define the composition operator $\b{a}:M_n\to M_n$ by $\b{a}(\b{u})(t)= a(|\b{u}(t)|)$.
We will use repeatedly the following elementary consequence of the previous lemma. 
\begin{cor}\label{a_bound} If $a\in C(\mathbb{R}^+,\mathbb{R}^+)$ then $\b{a}:\wphi_n\to L^{\infty}_1([0,T])$ is bounded. More concretely  there exists a non decreasing function $c:\mathbb{R}^+\to\mathbb{R}^+$ such that
 $\|\b{a}(\b{u})\|_{L^{\infty}([0,T])}\leq c(\|\b{u}\|_{\wphi})$.
\end{cor}
\begin{proof}  Let $\alpha$ be a  non-decreasing mayorant of $a$, for example $\alpha(s):=\sup_{0\leq t\leq s}a(t)$.  If $\b{u}\in \wphi_n$ then by Lemma \ref{inclusion orlicz} 
\[a(|\b{u}(t)|)\leq \alpha(\|\b{u}\|_{L^{\infty}})\leq a\left(\Psi^{-1}\left(\frac{1}{T}\right)\max\{1,T\} \|\b{u}\|_{\wphi}\right)=: c(\|\b{u}\|_{\wphi}).\]
\end{proof}


The following lemma is an inmediate consequence of principles  related to  operators of Nemitskii type, see \cite[�17]{KR}.

\begin{lem}\label{phi_comp}   The  composition operator  $\boldsymbol{\varphi}$  acts from $\Pi(\ephi_n,1)$ into $C_1^{\Psi}$.
\end{lem}
\begin{proof}
  As consequence of \cite[Lemma 9.1]{KR} we have that  $\boldsymbol{\varphi}\left(B_{\lphi}(0,1)\right)\subset C_1^{\Psi}$, where
$B_{X}(\b{u}_0,r)$ is the open ball with center $\b{u}_0$ and radius $r>0$ in the space $X$. Therefore, applying \cite[Lemma 17.1]{KR} we deduce that $\boldsymbol{\varphi}$ acts from $\Pi(\ephi_n,1)$ into $C_1^{\Psi}$.
\end{proof}

We need also the following technical lemma.
\begin{lem}\label{segundo lema}
Let $\lambda>0$ and  $\{\b{u}_n\}_{n\in \mathbb{N}}$ be a sequence of  functions in $\Pi(\ephi_n,\lambda)$ converging to  $\b{u}\in \Pi(\ephi_n,\lambda)$  in the $\lphi$-norm. Then there exist a subsequence
$\b{u}_{n_k}$ and a real valued function $h\in\Pi\left(\ephi_1\left([0,T]\right),\lambda\right)$ such that $\b{u}_{n_k}\rightarrow \b{u} \quad\text{a.e.}$ and $|\b{u}_{n_k}|\leq h\quad\text{a.e.}$.
\end{lem}



\begin{proof}
Let $r:=d(\b{u},\ephi_n)$, $r<\lambda$. Because $\b{u}_n$ converges to $\b{u}$, there exists a subsequence $(n_k)$ such that
\[\|\b{u}_{n_k}-\b{u}\orlnor<\frac{\lambda-r}{2}\quad \text{ and }\quad \|\b{u}_{n_k}-\b{u}_{n_{k+1}}\orlnor<2^{-(k+1)}(\lambda-r)\]
Let $h:[0,T]\rightarrow\mathbb{R}$ defined by
\begin{equation}\label{serie} h(x)=|\b{u}_{n_1}(x)|+\sum_{k=2}^{\infty}|\b{u}_{n_k}(x)-\b{u}_{n_{k-1}}(x)|.
\end{equation}
As a consequence  of \cite[Lemma 10.1]{KR} we have that, for any $\b{v}\in\lphi_n$, $d(\b{v},\ephi_n)=d(|\b{v}|,\ephi_1)$. Therefore
\[d(|\b{u}_{n_1}|,\ephi_1)= d(\b{u}_{n_1},\ephi_n)\leq d(\b{u}_{n_1},\b{u})+d(\b{u},\ephi_n)<\frac{\lambda+r}{2}.\]
Then
\[d(h,\ephi_1)\leq d(h,|\b{u}_{n_1}|)+d(|\b{u}_{n_1}|,\ephi_1)< \lambda.\]
Therefore, $h\in\Pi(\ephi_1,\lambda)$.  In particular,  $|h|<\infty$ a.e. We conclude that the series $\b{u}_{n_1}(x)+\sum_{k=2}^{\infty}(\b{u}_{n_k}(x)-\b{u}_{n_{k-1}}(x))$
is absolutely convergent a.e.  This imply that $\b{u}_{n_k}\rightarrow \b{u} \quad\text{a.e.}$. The inequality $|\b{u}_{n_k}|\leq h$ is clear from the definition of $h$.
\end{proof}

A common obstacle with Orlicz spaces, that distinguishes it from $L^p$ spaces, is that a  sequence $\b{u}_n\in\lphi_n$ which is  uniformly bounded by $ h\in\lphi_1$ and a.e. convergent to $\b{u}$ is not necessarily norm convergent.
Fortunately the subspace $\ephi_n$ has that property. 

\begin{lem}\label{lema_conv_may}
Suppose that $\b{u_n} \in\lphi_n$ is a sequence such that $\b{u_n}\to \b{u}$ a.e. and suppose that there exist $h\in\ephi_1$ with $|\b{u_n}|\leq h$ a.e. 
then $\|\b{u_n}-\b{u}\orlnor\to 0$.
\end{lem}
\begin{proof}\cite[p.84]{rao1991theory} and \cite[Th. 10.3]{KR})
\end{proof}


  We recall the definition of Gate\^{a}ux derivative, see \cite{ambrosetti} for details. Given a function $I:U\to\mathbb{R}$ where $U$ is an open set of a Banach space $X$,
we say that $I$ has a G\^ateaux derivative en $\b{u} \in U$ if there exists $\b{u}^*\in X^*$ such that for every $\b{v} \in X$
\[
\lim_{s \rightarrow 0}\frac{I(\b{u}+s\b{v})-I(\b{u}) }{s}=\langle \b{u},\b{u}^*\rangle.
\]



\section{Differetiability of action integrals on Orlicz spaces}

\begin{defi} We said that a function $\mathcal{L}:[0,T]\times \mathbb{R}^n \times \mathbb{R}^n \rightarrow \mathbb{R}$ is a Caratheodory function if for fixed $(\b{x},\b{y})$
the map $t \mapsto \mathcal{L}(t, \b{x},\b{y})$ is measurable  and for fixed $t$ the map  $(\b{x},\b{y}) \mapsto \mathcal{L}(t, \b{x}, \b{y})$ is continuously differentiable for almost everywhere $t\in [0,T]$.

\end{defi}


In this paper we will consider Lagrangian functions satisfying the following structure conditions. We assume  that there
exists $\lambda>0$ and non negative functions  $a \in C(\mathbb{R}^+, \mathbb{R}^+)$, $b \in L^1_1([0,T])$, $c \in \lpsi_1([0,T])$ and $d\in\ephi_1$ such that

\begin{eqnarray}
|\mathcal{L}(t,\b{x},\b{y})| &\leq a(|\b{x}|)\left(b(t)+ \Phi\left(\frac{|\b{y}|}{\lambda}+d(t) \right)\right),\label{cotaL}\\
|D_{\b{x}}\mathcal{L}(t,\b{x},\b{y})| &\leq a(|\b{x}|)\left(b(t)+ \Phi\left(\frac{|\b{y}|}{\lambda}+d(t) \right)\right),\label{cotaDxL}\\
|D_{\b{y}}\mathcal{L}(t,\b{x},\b{y})| &\leq a(|\b{x}|)\left(c(t)+ \varphi\left(\frac{|\b{y}|}{\lambda}+d(t)\right)  \right).\label{cotaDyL}
\end{eqnarray}

\begin{comentario}These conditions are a generalization of the frequently considered condition (A) (see \cite{tang1995periodic,
xu2007some,tang2010periodic,zhao2004periodic} ). In fact, conditions  \eqref{cotaL},\eqref{cotaDxL}, \eqref{cotaDyL} are equivalent to condition (A)  when  $\mathcal{L}(t,\b{x},\b{y})=|\b{y}|^p/p+F(t,\b{x})$, $\Phi_p(s)=s^p/p$,  and $d=0$. 
\end{comentario}

\begin{comentario} Let us note that if $\Phi\in\Delta_2$ then we can asssume $d=0$. This is consequence of that a non decreasing $\Delta_2$ function $G:\mathbb{R}_+\to\mathbb{R}_+$ is quasi-subadditive. In fact, we suppose $y\leq x$, then  
\[G(x+y)\leq G(2x)\leq KG(x)\leq K\left(G(x)+G(y)\right).\]
Moreover, if $\Phi$ is $\Delta_2$  then $\varphi$ is also $\Delta_2$, as the following simple argument shows
\[2x\varphi(2x)\leq \alpha \Phi(2x)\leq K\Phi(x)\leq Kx\varphi(x) \]
Here we have used \cite[Th. 4.1]{KR}, the $\Delta_2$ condition for $\Phi$ and the inequality $\Phi(x)\leq x\varphi(x)$ valid for any $N$-function. 

Therefore if $\Phi$ is $\Delta_2$ we have that

\[b(t)+ \Phi\left(\frac{|\b{y}|}{\lambda}+d(t) \right)\leq 
b(t)+ K\Phi\left(d(t) \right) +\Phi\left(\frac{|\b{y}|}{\lambda}\right)=b_1(t)+K\Phi\left(\frac{|\b{y}|}{\lambda}\right),\]
where $b_1(t)=b(t)+ K\Phi\left(d(t) \right)\in L^1_1([0,T])$. A similar fact holds with $\varphi$ instead $\Phi$ namely
\[ c(t)+ \varphi\left(\frac{|\b{y}|}{\lambda}+d(t)\right) \leq c_1(t)+ \varphi\left(\frac{|\b{y}|}{\lambda}\right),\]
where, as consequence of Lemma \ref{phi_comp} and the $\Delta_2$ condition for $\Phi$, we have   $c_1(t):=c(t)+K\varphi\left(d(t)\right)\in \lpsi$.    

  
\end{comentario}

\begin{thm}\label{teorema_acotacion}
Let $\mathcal{L}$ be a Caratheodory function satisfying \eqref{cotaL},\eqref{cotaDxL}, \eqref{cotaDyL}. Then the following statements hold
\begin{enumerate}
\item \label{T1item1} \label{A1} The \emph{action integral}  
\begin{equation}\label{integral_accion}
I(\b{u}):=\int_{0}^T \mathcal{L}(t,\b{u}(t),\b{\dot{u}}(t))\ dt
\end{equation}
is finitely defined in $ \mathcal{E}^{\Phi}_n(\lambda):=W^{1}\lphi\cap\{\b{u}|\b{\dot{u}}\in\Pi(\ephi_n,\lambda)\}$.

\item\label{T1item3} The function  $I$ is G\^ateaux differentiable on $\domi$ and  its derivative $I'$ is continuous from $\domi$ with the strong topology into $\left[\wphi \right]^*$ equipped with the $w^*$-topology. Moreover $I'$ is given by the expression
\begin{equation}\label{DerAccion}
\langle \b{v}, I'(\b{u})\rangle= \int_0^T \left\{D_{\b{x}}\mathcal{L}\big(t,\b{u},\b{\dot{u}}\big)\ccdot \b{v}+ D_{\b{y}}\mathcal{L}\big(t,\b{u},\b{\dot{u}}\big)\ccdot\b{\dot{v}}\right\} \ dt.
\end{equation}

\item\label{T1item4}  If  $\Psi$ is $\Delta_2$ then 
  $I'$ is continuous from $\domi$ into $\left[\wphi\right]^*$ when both spaces are equipped with the strong topology.


\end{enumerate}
\end{thm}
\begin{proof} From \eqref{inclusion2} we have   $\b{\dot{u}}/\lambda\in\Pi(\ephi_n,1)$. Thus, as $d\in\ephi_1$ and attending to \eqref{inclusiones}, we get 

\begin{equation}\label{inclusion3}
|\b{\dot{u}}|/\lambda+d\in\Pi(\ephi_1,1)\subset \claseor_1.
\end{equation}
From Corollary \ref{a_bound} we get a constant $c=c(\|\b{u}\sobnor )$ such that  $a(|\b{u}(t)|)\leq c$, $t\in [0,T]$.
 Thus,
 \[|\mathcal{L}(t,\b{u},\b{\dot{u}})| \leq c\left(b(t)+ \Phi\left (\frac{|\b{\dot{u}}(t)|}{\lambda}+d(t)\right)  \right)\in
 L^1.\]
This fact proves item \ref{T1item1}.

 We split the proof of  \ref{T1item3} in three steps.

\noindent\textbf{Step 1.} We prove that $\b{u} \mapsto D_{\b{x}}\mathcal{L}(t,\b{u},\b{\dot{u}})$ is continuous from $\domi$ into $L^{1}_n([0,T])$ whith the strong topology on both sets. We take   $\{\b{u}_n\}_{n\in \mathbb{N}}$ a sequence of  functions in $\domi$, and $\b{u}\in \domi$ such that $\b{u}_n\rightarrow \b{u}$ in $\wphi_n$.
Then $\b{u}_n\rightarrow \b{u}$ in $\lphi_n$ and $\b{\dot{u}}_n\rightarrow \b{\dot{u}}$ in $\lphi_n$. By Lemma \ref{segundo lema} there exist a subsequence $\b{u}_{n_k}$ and $h\in \Pi(\ephi_1,\lambda))$
such that $\b{u}_{n_k}\rightarrow \b{u} \quad\text{a.e.}$, $\b{\dot{u}}_{n_k}\rightarrow \b{\dot{u}} \quad\text{a.e.}$ and $|\b{\dot{u}}_{n_k}|\leq h\quad\text{a.e.}$.  Since $\b{u}_{n_k}$, $k=1,2,\ldots$ is a strong convergent sequence in $\wphi_n$, it is a bounded sequence in $\wphi_n$. According to Lemmas \ref{inclusion orlicz} and Corollary \ref{a_bound} there exists $M>0$ such that $\|\b{a}(\b{u}_{n_k})\|_{L^{\infty}} \leq M$, $k=1,2,\ldots$.  From the previous facts, \eqref{cotaDxL} and \eqref{inclusion3} we get
\begin{equation}\label{DxL1}
|D_{\b{x}}\mathcal{L}(t,\b{u}_{n_k}(t),\b{\dot{u}}_{n_k}(t))|\leq M\left(b(t)+\Phi\left(\frac{|h|}{\lambda}+d(t)\right)\right) \in L^1.
\end{equation}
By the Caratheodory condition
\[D_{\b{x}}\mathcal{L}(t,\b{u}_{n_k}(t),\b{\dot{u}}_{n_k}(t))\to D_x\mathcal{L}(t,\b{u}(t),\b{\dot{u}}(t))\quad\hbox{ for a.e }t\in[0,T].\]
Applying the Dominated Convergence Theorem we conclude the proof of step 1.

\noindent\textbf{Step 2.} We will prove that the mapping  $\b{u}
 \mapsto  D_{y}\mathcal{L}(t,\b{u},\b{\dot{u}})$ is continuous from $\domi$ with the strong topology  into $\left[\lphi\right]^*$  with the weak$^*$ topology. Let $\b{u}\in \domi$.  It follows from  \eqref{inclusion3}, Lemma \ref{phi_comp} and Corollary \ref{a_bound} that 
\begin{equation}\label{AcotOperphi}
\varphi\left(\frac{|\b{u}|}{\lambda}+d\right)\in C^{\Psi}_1
\end{equation}
and $\b{a}(|\b{u}|)\in L^{\infty}_1$. Therefore, in virtue of  \eqref{cotaDyL} we get
\begin{equation}\label{DyLpsi}
   \left|D_{\b{y}}\mathcal{L}(t,\b{u}(t),\b{\dot{u}}(t))\right|\leq  c(\|\b{u}\|_{\wphi} )  \left(c+\boldsymbol{\varphi}\left( \frac{|\b{u}|}{\lambda}+d\right  ) \right)\in\lpsi.
\end{equation}
 We note that \eqref{DxL1},  \eqref{DyLpsi} , the imbedding $\wphi_n \hookrightarrow L_n^{\infty}$ and  $\lpsi_n\hookrightarrow  \left[\lphi_n\right]^*$ imply that the second member 
\eqref{DerAccion} defines an element in $\left[\wphi_n\right]^*$.

Now, let us to prove the continuity of the map   $\b{u}\mapsto D_y\mathcal{L}(\cdot,\b{u},\b{\dot{u}})$. We take $\b{u}_n,\b{u}\in \domi$ with $\b{u}_n\to \b{u}$ in the norm of $\wphi_n$. We must prove that  $D_y\mathcal{L}(\cdot,\b{u}_n,\dot{\b{u}_n})\overset{w^*}{\rightharpoonup} D_y\mathcal{L}(\cdot,\b{u},\b{\dot{u}})$. Suppose, on the contrary, that there exists $\b{v}\in\lphi_n$, $\epsilon>0$ and a subsequence of $\{\b{u}_n\}$ (again denoted for simplicity $\{\b{u}_n\}$)  such that
\begin{equation}\label{cota_eps}
 \left| \langle \b{v}, D_y\mathcal{L}(\cdot,\b{u}_n,\b{\dot{u}_n}) \rangle - \langle \b{v}, D_y\mathcal{L}(\cdot,\b{u},\b{\dot{u}}) \rangle\right|\geq \epsilon.
\end{equation}
We have $\b{u}_n\rightarrow \b{u}$ in $\lphi_n$ and
$\b{\dot{u}}_n\rightarrow \b{\dot{u}}$ in $\lphi_n$. By Lemma \ref{segundo lema}, there exist a subsequence $\b{u}_{n_k}$ and $h\in \Pi(\ephi,\lambda)$ such that $\b{u}_{n_k}\rightarrow \b{u} \quad\text{a.e.}$, $\b{\dot{u}}_{n_k}\rightarrow \b{\dot{u}} \quad\text{a.e.}$ and $|\b{\dot{u}}_{n_k}|\leq h\quad\text{a.e.}$. As in the previous step, since $\b{u}_n$ is a convergent sequence, the Corrollary \ref{a_bound} implies that $a(|\b{u}_n(t)|)$ is uniformly bounded by certain constant $C$. Therefore, from \eqref{cotaDyL},  \eqref{AcotOperphi}, the fact that $c\in\lphi_1$, H\"older inequality we obtain
\[
  \left | D_y\mathcal{L}(\cdot,\b{u}_n,\b{\dot{u}}_n) \ccdot \b{ v} \right| \leq C\left(c|\b{v}|+\varphi\left(\frac{h}{\lambda}+d\right)\right)|\b{v}|\in L_1^1.
\]
 From the Lebesgue dominated convergence theorem we deduce
\begin{equation}\label{conv_debil}\int_0^T \langle D_y\mathcal{L}(t,\b{u}_{n_k}(t),\b{\dot{u}}_{n_k}(t)),\b{ v}(t) \rangle dt \to \int_0^T\langle D_y\mathcal{L}(t,\b{u}(t),\b{\dot{u}}(t)),\b{ v}(t) \rangle dt \end{equation}
which contradict the inequality \eqref{cota_eps}. This completes the proof of step 2.

\textbf{Step 3.} Finally we prove \ref{T1item3}. The proof follows similar lines that \cite[Theorem 1.4]{mawhin2010critical}. For $\b{u}\in \domi$ and $\b{v}\in\wphi_n$ we define the function
\[f(s,t):=\mathcal{L}(t,\b{u}(t)+s\b{v}(t),\b{\dot{u}}(t)+s\b{\dot{v}}(t)).\]
From \cite[Th. 10.1]{KR} we obtain that if $|\b{u}|\leq |\b{v}|$ then    $d(\b{u},\ephi_n)\leq d(\b{v},\ephi_n)$. Therefor, for  $|s|\leq s_0:=\left(\lambda-d(\b{\dot{u}},\ephi_n)\right)/\|\b{v}\sobnor$ we have
\[
d \left(\b{\dot{u}}+s_0\b{\dot{v}}, \ephi_n \right)\leq
d \left(|\b{\dot{u}}|+s_0|\b{\dot{v}}|, \ephi_1 \right)
\leq d \left(|\b{\dot{u}}|,\ephi_1 \right)+ s_0 \|\b{\dot{v}}\orlnor < \lambda.
\]
As a consequence $\b{\dot{u}}+s_0\b{\dot{v}} \in \Pi(\ephi_n,\lambda)$ and  $|\b{\dot{u}}|+s_0|\b{\dot{v}}| \in \Pi(\ephi_1,\lambda)$. These facts imply, in virtue of Theorem \ref{teorema_acotacion}(\ref{T1item1}) that $I(\b{u}+s\b{v})$ well defined and finite for $|s|\leq s_0$. 
Using  Corollary \ref{a_bound} we see that
\[ \|a(|\b{u}+s\b{v}|)\|_{L^{\infty}}\leq  c(\|\b{u}+s\b{v}\sobnor)\leq
 c(\|\b{u}\sobnor+s_0\|\b{v}\sobnor).
\]
Consequently, applying chain rule,  inequalities \eqref{cotaDxL}-\eqref{cotaDyL}, the previous inequality and using that $\varphi$ and $\Phi$ are non decreasing, we obtain
\begin{equation}\label{ctg}
\begin{split}
|D_s f(s,t)|&=\left| D_{\b{x}}\mathcal{L}(t,\b{u}+s\b{v},\b{\dot{u}}+s\b{\dot{v}})\ccdot \b{v} +  D_{\b{y}}\mathcal{L}(t,\b{u}+s\b{v},\b{\dot{u}}+s\b{\dot{v}})\ccdot\b{\dot{v}}\right| \\
 & \leq c \left[\left( b(t)+ \Phi\left(\frac{|\b{\dot{u}}|+s_0|\b{\dot{v}}|}{\lambda}\right)\right)|\b{v}|\right.\\
&\left. \quad+ \left(c(t)+ \varphi\left (\frac{|\b{\dot{u}}|+s_0|\b{\dot{v}}|}{\lambda}\right)\right)|\b{\dot{v}}| \right]
\end{split}
\end{equation}
 Then $b+\Phi(|\b{\dot{u}}|+s_0|\b{\dot{v}}|) \in L^1$ and since $\b{v} \in L^{\infty}$ we have that
$(b+\Phi(|\b{\dot{u}}|+s_0|\b{\dot{v}}|))|\b{v}| \in L^1$. On the other hand, from Lemma \ref{phi_comp} and \eqref{AcotOperphi}  we obtain $c(t)+ \varphi(|\b{\dot{u}}|+s_0|\b{\dot{v}}|) \in L^{\Psi}$ and since $\b{\dot{v}} \in L^{\Phi}$, applying the H\"older inequality
$(c(t)+ \varphi(|\b{\dot{u}}|+s_0|\b{\dot{v}}|))|\b{\dot{v}}| \in L^1$. Thus, from \eqref{ctg} and the above discussion there exists a function $g \in L^1([0,T], \mathbb{R}^{+})$
such that $|D_s f(s,t)| \leq g(t)$. Consequently, $I$ has a directional derivative and
\[
\langle v, I'(\b{u}) \rangle=\frac{d}{ds}I(\b{u}+sv)\big|_{s=0}=\int_0^T \left(\langle D_{x}\mathcal{L}(t,\b{u},\b{\dot{u}}), \b{v}\rangle+ \langle D_{y}\mathcal{L}(t,\b{u},\b{\dot{u}}),\b{\dot{v}}\rangle\right) \ dt.
\]
Moreover, from \eqref{DxL1}, \eqref{DyLpsi}, Lemma \ref{inclusion orlicz} and previous formula
\[
|\langle I'(\b{u}), \b{v} \rangle| \leq c \|v\linf + c \|\b{\dot{v}}\orlnor \leq c \|\b{v}\sobnor.
\]
This complete the proof of the G\^ateaux differentiability of $I$. Finally, the continuity of $I': \left(\domi, \|\cdot \sobnor\right) \to \left(\left[\wphi
\right]^*, w^* \right)$ is a consequence of the continuity of the mappings $\b{u} \mapsto D_{\b{x}}\mathcal{L}(t,\b{u},\b{\dot{u}})$ and $\b{u} \mapsto
D_{\b{y}}\mathcal{L}(t,\b{u},\b{\dot{u}})$. Indeed, we set $\b{u}_n,\b{u}\in W^{1}(\lphi,\Pi(\ephi,1))$ with $\b{u}_n\to \b{u}$ in the norm of $\wphi$ and $\b{v} \in
\wphi$ , then
\[
\begin{split}
\left\langle \b{v}, I'(\b{u}_{n}) \right\rangle &= \int_0^T \left\{\left\langle D_{\b{x}}\mathcal{L}\left(t,\b{u}_n(t),\b{\dot{u}}_n(t)\right),
\b{v}(t)\right\rangle+
\left\langle D_{y}\mathcal{L}\left(t,\b{u}_n(t),\b{\dot{u}}_n(t)\right),\dot{\b{v}}(t)\right\rangle\right\} \ dt\\
&\rightarrow \int_0^T \left\{\left\langle D_{\b{x}}\mathcal{L}\left(t,\b{u}(t),\b{\dot{u}}(t)\right), \b{v}(t)\right\rangle+ \left\langle
D_{y}\mathcal{L}\left(t,\b{u}(t),\b{\dot{u}}(t)\right),\dot{\b{v}}(t)\right\rangle\right\} \ dt\\
&=\left\langle \b{v}, I'(\b{u}) \right\rangle.
\end{split}
\]


In order to prove  \ref{T1item4}, let us see that the maps $\b{u}\mapsto D_{\b{x}}\mathcal{L}(\cdot,\b{u}(\cdot),\b{\dot{u}}(\cdot))$  and $\b{u}\mapsto D_{\b{y}}\mathcal{L}(\cdot,\b{u}(\cdot),\b{\dot{u}}(\cdot))$  are continuous
from $\left(\wphi, \|\cdot \sobnor\right) $ into $\left( L^1, \|\cdot \|_{L^1}\right)$ and
 $\left(\lpsi,\|\cdot\|_{L^{\Psi}}\right)$ respectively.  The continuity of the first map is an immediate consequence of the step 1 in the proof of item \ref{T1item3} and the fact that $\Pi(\ephi,1) =\lphi$  when $\Phi$ is of the $\Delta_2$ class. We will prove the continuity of the second map.  We consider $\b{u}_n$ and $\b{u}$ with $\|\b{u}_n- \b{u}\sobnor\to 0$.
By Lemma \ref{segundo lema}, there exist a subsequence $\b{u}_{n_k}$ and $h\in \lphi$ such that $\b{u}_{n_k}\rightarrow \b{u} \quad\text{a.e.}$, $\b{\dot{u}}_{n_k}\rightarrow \b{\dot{u}} \quad\text{a.e.}$ and $|\b{\dot{u}}_{n_k}|\leq h\quad\text{a.e.}$.
 Then  since $\mathcal{L}$ is a Caratheodory function
 we have $ D_{\b{y}}\mathcal{L}(t,\b{u}_{n_k}(t),\b{\dot{u}}_{n_k}(t))\to D_{\b{y}}\mathcal{L}(t,\b{u}(t),\b{\dot{u}}(t))$ a.e. $t\in [0,T]$.  From Corollary \ref{a_bound} and the fact that $\|\b{u}_{n_k}\sobnor$ are uniformly bounded, we get a constant $C>0$ such that  $\|\b{a}(\b{u}_{n_k})\|_{L^{\infty}}\leq C$.
 By using \eqref{cotaDyL} and as $\Psi$ is of the $\Delta_2$ class, we obtain
 \[\begin{split}
    |D_{\b{y}}\mathcal{L}(t,\b{u}_{n_k}(t),\b{\dot{u}}_{n_k}(t))| &\leq a(|\b{u}_{n_k}|)\left( c(t) + \varphi (|\b{\dot{u}}_{n_k}(t)|)\right)\\
    &\leq C\left( c(t) + \varphi (|h|)\right)\in \lpsi
   \end{split}
\]
Therefore, invoking  Lemma \ref{lema_conv_may}, we have proved that
  of all sequence $\b{u}_n$ which converge to $\b{u}$ in $\wphi$ we can
extract a subsequence with $D_{\b{y}}\mathcal{L}(t,\b{u}_{n_k},\b{\dot{u}}_{n_k})\to D_{\b{y}}\mathcal{L}(t,\b{u},\b{\dot{u}})$ in the strong topology. The desired result follows from a standard argument.

Now we are  in conditions of  to prove the continuity of $I'$. Let $\b{u}, \b{u_0},\b{v}\in\lphi$. Then
\[
  \begin{split}
   \langle I'(\b{u})-I'(\b{u}_0),\b{v} \rangle
      &= \int_0^T \left\{ \left(D_{\b{x}}\mathcal{L}(t,\b{u}(t),\b{\dot{u}}(t))-D_{\b{x}}\mathcal{L}(t,\b{u}_0(t),\dot{\b{u}_0}(t))\right)\cdot v(t)\right.\\
      &\quad\left.+
      \left(D_{\b{y}}\mathcal{L}(t,\b{u}(t),\b{\dot{u}}(t))-D_{\b{y}}\mathcal{L}(t,\b{u}_0(t),\dot{\b{u}_0}(t))\right)\cdot \b{\dot{v}}(t)\right\}dt \\
      &\leq \left\{\| D_{\b{x}}\mathcal{L}(\cdot,\b{u}(\cdot),\b{\dot{u}}(\cdot))-D_{\b{x}}\mathcal{L}(\cdot,\b{u}_0(\cdot),\dot{\b{u}_0}(\cdot))\|_{L^1}\|\b{v}\|_{L^{\infty}}\right.\\
      &\quad+\left.
       \| D_{\b{y}}\mathcal{L}(\cdot,\b{u}(\cdot),\b{\dot{u}}(\cdot))-D_{\b{y}}\mathcal{L}(\cdot,\b{u}_0(\cdot),\dot{\b{u}_0}(\cdot))\|_{\lpsi}\|\b{\dot{v}}\|_{\lphi}\right\}\\
  \end{split}
\]
Taking supremum on $\b{v}$ with $\|\b{v}\sobnor\leq 1$ in the previous inequality and using Lemma \ref{inclusion orlicz} we obtain
% \[\begin{split}
%    \|I'(\b{u})-I'(\b{u}_0)\|_{\wphi} &\leq  C\bigg(\| D_{\b{x}}\mathcal{L}(\cdot,\b{u}(\cdot),\b{\dot{u}}(\cdot))-D_{\b{x}}\mathcal{L}(\cdot,\b{u}_0(\cdot),\dot{\b{u}_0}(\cdot))\|_{L^1}\right.\\
%      & + \| D_{\b{y}}\mathcal{L}(\cdot,\b{u}(\cdot),\b{\dot{u}}(\cdot))-D_{\b{y}}\mathcal{L}(\cdot,\b{u}_0(\cdot),\dot{\b{u}_0}(\cdot))\|_{\lpsi}\bigg)
%   \end{split}
%   \]
Therefore the results follows  of the  previously established continuity for $D_{\b{x}}\mathcal{L}$ and $D_{\b{y}}\mathcal{L}$.
\end{proof}



\section{Critical points and Euler-Lagrange equations}


In this section we derive the Euler-Lagrange equations associated to critical points of action integrals.


% \begin{equation}\label{ecualagran}
%     \left\{%
% \begin{array}{ll}
%    \frac{d}{dt} D_{y}\mathcal{L}(t,\b{u}(t),\b{\dot{u}}(t))= D_{\b{x}}\mathcal{L}(t,\b{u}(t),\b{\dot{u}}(t)) \quad \hbox{a.e.}\ t \in (0,T)\\
%     \b{u}(0)-\b{u}(T)=\b{\dot{u}}(0)-\b{\dot{u}}(T)=0.
% \end{array}%
% \right.
% \end{equation}

We denote by $\wphi_T$ the subspace of $\wphi$ of all functions $T$-periodic. Similarly we consider the subspaces $\ephi_T$, $\lphi_T$. As is usual, when $Y$ is a subspace of
the Banach space $X$, we denote by $Y^{\perp}$ the subspace of $X^*$ of all the bounded linear functions which are identically zero on $Y$.

We recall that  a function $f: \mathbb{R}^n \to \mathbb{R}$ is called \emph{strictly convex} if $f\left(\tfrac{\b{x}+\b{y}}{2}\right)< \tfrac{1}{2} \left(f\left(
\b{x}\right)+f\left( \b{y}\right)\right)$.  It is a well known that if $f$ is a strictly convex and differentiable functions then
$D_{\b{x}}f:\mathbb{R}^n\to\mathbb{R}^n$ is a one-to-one map  (see, for instance \cite[Theorem 12.17]{rockafellar2009variational}).


\begin{thm} The following statements are equivalent
\begin{enumerate}
 \item $I'(\b{u})\in\left( \wphi_T\right)^{\perp}$
 \item  $D_{\b{y}}\mathcal{L}(t,\b{u}(t),\b{\dot{u}}(t))$ is an absolutely continuous function and $\b{u}$ solve the following boundary value problem
 \begin{equation}\label{ecualagran2}
    \left\{%
\begin{array}{ll}
   \frac{d}{dt} D_{y}\mathcal{L}(t,\b{u}(t),\b{\dot{u}}(t))= D_{\b{x}}\mathcal{L}(t,\b{u}(t),\b{\dot{u}}(t)) \quad \hbox{a.e.}\ t \in (0,T)\\
    \b{u}(0)-\b{u}(T)=D_{\b{y}}\mathcal{L}(0,\b{u}(0),\b{\dot{u}}(0))-D_{\b{y}}\mathcal{L}(T,\b{u}(T),\b{\dot{u}}(T))=0.
\end{array}%
\right.
\end{equation}
\end{enumerate}
Moreover if $D_{\b{y}}\mathcal{L}(t,x,y)$ is $T$-periodic with respect to the variable $t$ and strictly convex with respect to $\b{y}$, then
$D_{\b{y}}\mathcal{L}(0,\b{u}(0),\b{\b{\dot{\b{u}}}}(0))-D_{\b{y}}\mathcal{L}(T,\b{u}(T),\b{\dot{u}}(T))=0$ is equivalent to $\b{\dot{u}}(0)=\b{\dot{u}}(T)$.
\end{thm}

\begin{proof} The condition $I'(\b{u})\in\left( \wphi_T\right)^{\perp}$ means that for every $\b{v}\in \wphi_T$ we have $\langle \b{v}, I'(\b{u})\rangle=0$. According to Theorem
\ref{teorema_acotacion} we have

\[\int_0^T\langle D_{\b{y}} \mathcal{L}(t,\b{u}(t),\b{\dot{u}}(t)), \b{\dot{v}}(t)\rangle dt=-\int_0^T \langle D_{\b{x}}\mathcal{L}(t,\b{u}(t),\b{\dot{u}}(t)),\b{ v}(t)\rangle dt \]
Using \cite[pag. 6]{mawhin2010critical} we obtain that  $D_{\b{y}}\mathcal{L}(t,\b{u}(t),\b{\dot{u}}(t))$ is absolutely continuous and $T$-periodic, therefore it is differentiable a.e.on $[0,T]$ and the first equality of \eqref{ecualagran2} holds true.
This complete the proof  1. implies 2. The proof of 2.implies 1.  is still easier and so  we will omit it.

The last part of the Corollary is a consequence of that $D_{\b{y}}\mathcal{L}(T,\b{u}(T),\b{\dot{u}}(T))=D_{\b{y}}\mathcal{L}(0,\b{u}(0),\b{\dot{u}}(0))=D_{\b{y}}\mathcal{L}(T,u(T),\b{\dot{u}}(0))$ and the injectivity of $D_{\b{y}}\mathcal{L}(T,u(T),\cdot)$.
\end{proof}

\section{Weak lower semicontinuity of actions integrals}

 \begin{lem}\label{unif_conv}
If the sequence $\{\b{u}_{k}\}_{k \geq 1}$ converges weakly to $\b{u}$ in $\wphi$, then $\{\b{u}_{k}\}_{k\geq 1}$ converges uniformly to $\b{u}$ on $[0,T]$.
\end{lem}
\begin{proof}
By Lemma \ref{inclusion orlicz}, the injection of $\wphi$ in $L^{\infty}$ is continuous. Since $\b{u}_{k}\rightharpoonup \b{u}$ in $\wphi$ it follows that
$\b{u}_{k}\rightharpoonup \b{u}$ in $C(0,T;\mathbb{R}^n)$. Since $\b{u}_{k}\rightharpoonup \b{u}$ in $\wphi$, we know that $\{\b{u}_{k}\}_{k \geq 1}$ is bounded in
$\wphi$ and, hence by \eqref{estimacion} in $C(0,T;\mathbb{R}^n)$. Moreover, the sequence $\{\b{u}_{k}\}_{k \geq 1}$ is equi-uniformly continuous since, for $0 \leq
s\leq t \leq T $, we have
\[
\begin{split}
\left|\b{u}_{k}(t)-\b{u}_{k}(s) \right|&\leq \int_{s}^t \left| \dot{\b{u}}_{k}(\tau)\right|\ \ d\tau \leq \| t-s\|_{\lpsi}\|\dot{\b{u}}_{k}\|_{\lphi}\\
&\leq \| t-s\|_{\lpsi}\|\b{u}_{k}\|_{\wphi} \leq C \| t-s\|_{\lpsi}.
\end{split}
\]
By Arzela-Ascoli theorem, $\{\b{u}_{k}\}_{k \geq 1}$ is relatively compact in $C(0,T;\mathbb{R}^n)$. By the uniqueness of the weak limit in $C(0,T;\mathbb{R}^n)$,
every uniformly convergent subsequence of $\{\b{u}_{k}\}_{k \geq 1}$ converges to $\b{u}$. Thus, $\{\b{u}_{k}\}_{k \geq 1}$ converges uniformly on $[0,T]$.

\end{proof}

\begin{thm}
We suppose that $\mathcal{L}(t,\b{x},\b{y})$ is a Charateodory functions satisfying \eqref{cotaL}-\eqref{cotaDyL}.
Moreover we assume $\mathcal{L}(t,\b{x},\cdot)$ is convex for each $t,\b{x}$. We suppose that $\Phi,\Psi$ are $\Delta_2$ functions. Then the functional \eqref{integral_accion} is weakly lower semicontinuous (w.l.s.c.).
\end{thm}


\begin{proof} We fix any $\b{u}\in\wphi$. What we must prove that for any sequence $\{\b{u}_n\}$ with $\b{u}_n\rightharpoonup \b{u}$ in $\wphi$ we have that $I(\b{u})\leq \liminf_n I(\b{u}_n)$. We write
\[
\begin{split}I(\b{v})&=\int_0^T\mathcal{L}(t,\b{v}(t),\b{\dot{v}}(t))dt\\
 &=\int_0^T\mathcal{L}(t,\b{v}(t),\b{\dot{v}}(t))-\mathcal{L}(t,\b{u}(t),\b{\dot{v}}(t))dt +\int_0^T\mathcal{L}(t,\b{u}(t),\b{\dot{v}}(t))dt\\
 &=:J(\b{v})+H(\b{v}).
\end{split}
\]
 As $\{\b{u}_n\}$ is a weakly convergent sequence,  by the Lemma \ref{unif_conv}  we have that $\b{u}_n\to \b{u}$ in $L^{\infty}$. By the mean value theorem for derivatives, we obtain
 a function $\b{\xi_n}(t)$, with $\b{\xi_n}(t)$ belonging to line segment joining $\b{u}_n(t)$ and $\b{u}(t)$, such that
 \begin{equation}\label{opJ}
 \begin{split}
   &\left|  \mathcal{L}(t,\b{u}_n(t),\b{\dot{u}}_n(t))-\mathcal{L}(t,\b{u}(t),\b{\dot{u}}_n(t))\right|\\
  &\leq|D_{\b{x}}\mathcal{L}(t,\b{\xi_n}(t),\b{\dot{u}}_n(t))||\b{u_n}(t)-\b{u}(t)|.
  \end{split}
 \end{equation}
The functions $\b{u_n}$, and therefore the functions $\b{\xi_n}$, are uniformly bounded in $L^{\infty}$. Thus, there exists $C>0$ such that $a(|\b{\xi_n}(t)|)\leq C $. Then,
using \eqref{cotaDxL} we get
\begin{equation}\label{acot_Dx}
 \left|D_{\b{x}}\mathcal{L}(t,\b{\xi_n}(t),\b{\dot{u}}_n(t))\right|\leq C \left(b(t)+\Phi(|\b{\dot{u}}_n(t))|)\right)
\end{equation}
Since $\Phi$ is a function of the $\Delta_2$ class, we have that the operator $\b{v}\mapsto\Phi(|\b{v}|)$  acts from $\lphi$ in $L^1$. Therefore, by
\cite[Lemma 17.4]{KR} we have that $\{\Phi(|\b{v}|): \|\b{v}\|_{\lphi}\leq r\}$ is bounded in $ L^1$ for any $r>0$. Hence
there exists a constant $C>0$ such that $\|\Phi(|\b{\dot{u}}_n(t))|)\|_{L^1}\leq C$. Then, from
\eqref{opJ}, \eqref{acot_Dx}, H\"older inequality and since $\|\b{u}_n-\b{u}\|_{L^{\infty}}\to 0$ and $b\in L^1$ we get $J(\b{u}_n)\to 0    $.

Now we will prove that $H(\b{v})$ is w.l.s.c. Since $H(\b{v})$ is convex it is sufficient to prove that $H$ is l.s.c (see \cite[Proposition 4.26]{fonseca2007modern}).  We suppose that  $\|\b{v_n}- \b{v}\sobnor\to 0$.

There exists $s=s_{n,t}\in[0,1]$
such that
\[|\mathcal{L}(t,\b{u}(t),\b{\dot{v}}_n(t))-\mathcal{L}(t,\b{u}(t),\b{\dot{v}}(t))|\leq|D_{\b{y}}\mathcal{L}(t,\b{u}(t),(1-s)\b{\dot{v}}_n(t)+s \b{\dot{v}}(t))||\b{\dot{v}}_n-\b{\dot{v}}|.\]
Let $\mathfrak{G}_n$ be the set $\{|\b{\dot{v}}_n(t)|\geq |\b{\dot{v}}(t)|\}$. Then
\[|(1-s)\b{\dot{v}}_n(t)+ s\b{\dot{v}}(t)|\leq \max\{|\b{\dot{v}}_n(t)|,|\b{\dot{v}}(t)|\}=\chi_{\mathfrak{G}_n}(t)|\b{\dot{v}}_n(t)|+
 \chi_{\mathfrak{G}_n^c}(t)|\b{\dot{v}}(t)|
\]
Therefore, using \eqref{cotaDyL} and taking account that $a(|\b{u}(t)|)\in L^{\infty}$ we get
\[\begin{split}|\mathcal{L}(t,\b{u}(t),\b{\dot{v}}_n(t))-\mathcal{L}(t,\b{u}(t),\b{\dot{v}}(t))|&\leq C\left(c(t)+\varphi(\chi_{\mathfrak{G}}|\b{\dot{v}}_n(t)|+
 \chi_{\mathfrak{G}^c}|\b{\dot{v}}(t)|)\right)|\b{\dot{v}}_n-\b{\dot{v}}|\\
 &=C\left(c(t)+\varphi(\chi_{\mathfrak{G}}|\b{\dot{v}}_n(t)|)+\varphi(
 \chi_{\mathfrak{G}^c}|\b{\dot{v}}(t)|)\right)|\b{\dot{v}}_n-\b{\dot{v}}|\\
 &\leq C\left(c(t)+\varphi(|\b{\dot{v}}_n(t)|)+\varphi(
 |\b{\dot{v}}(t)|)\right)|\b{\dot{v}}_n-\b{\dot{v}}|.
 \end{split}
 \]

 Now, in virtue of \cite[Lemma 9.1]{KR}, \cite[Lemma 17.1]{KR}, \cite[Theorem 17.4]{KR} and the uniform boundedness of $\b{\dot{u}}_n$ in
 $\lphi$ we have
 \[|H(\b{v}_n)-H(\b{v})|\leq C\|\b{\dot{v}}_n-\b{\dot{v}}\orlnor\to 0.\]
 Which completes the proof.
\end{proof}

\section{Coercivity discussion}




%Sea $\mathcal{L}(t, \b{x}, \b{y})=F(t,\b{x})+\Phi(|\b{y}|)$.

We consider the problem   (introduced in \eqref{ecualagran2}):
\begin{equation}\label{ecualagran3}
\frac{d}{dt} D_{y}\mathcal{L}(t,\b{u}(t),\b{\dot{u}}(t))= D_{\b{x}}\mathcal{L}(t,\b{u}(t),\b{\dot{u}}(t)) \quad \hbox{a.e.}\ t \in (0,T).
\end{equation}

\begin{nota_fer}
Me parece que vamos a tener que presentar el resultado siguiente de otra forma. Como est� escrito, resultar� un caso particular de otros teoremas. Tener en cuenta que, al fin y al cabo, se termina pidiendo que $\Phi,\Psi\in\Delta_2$, con lo cual se aplicar�an los resultados que escribe Sonia m�s abajo y la funcional ser�a coercitiva sin necesidad de la hip�tesis \eqref{cota}.  Para poder mantener el resultado de abajo, habr�a que contextualizarlo en un teorema donde no se pida  $\Phi,\Psi\in\Delta_2$. Se me ocurre que un buen lugar ser�a despu�s del comentario que sigue a la estimaci�n \eqref{eq:no_coerciva}. Pues all� se habla del papel que juega la constante $C$ y me parece bueno relacionarlo con eso.

Todo el razonamiento hasta llegar a la estimaci�n \eqref{eq:est_abajo_I} se puede tomar para usarlo con lo que escribi� Sonia. Justamente lo que demostr� Sonia es que la cota inferior en  \eqref{eq:est_abajo_I} tiende a infinito cuando $\|\b{\dot{u}}\orlnor\to\infty$.\textbf{ Eso si, habr�a que intetar generalizar este razonamiento, sin pedir la condici�n $\mathcal{L}(t,\b{x},\b{y})=\Phi(|\b{y}|)+F(t,\b{x})$.} Algo habr� que pedirle a $\mathcal{L}$, intentar que sea alguna desigualdad, en lugar de adjudicarle una forma m�s expl�cita. 

\end{nota_fer}

\begin{thm}[Theorem 1.5 M-W]
Sea $\mathcal{L}(t,\b{x},\b{y})=\Phi(|\b{y}|)+F(t,\b{x})$. We suppose that $\Phi,\Psi$ are $\Delta_2$ functions and that there exists $f \in L^1$  such that
\[
\left|\nabla F(t,\b{x}) \right|\leq f(t)
\]
for a.e. $t \in [0,T]$ and all $\b{x}\in \mathbb{R}^n $ with $f$ satisfying
\begin{equation}\label{cota}
\|f\|_{L^1([0,T])}\leq\frac{1}{2\|1\|_{\lpsi([0,T])}}.
\end{equation}
If
\begin{equation}\label{propiedad1coercividad}
\int_{0}^{T}F(t,\b{x})\ dt \rightarrow \infty \quad \hbox{as} \quad |\b{x}|\rightarrow \infty,
\end{equation}
then problem \eqref{ecualagran3} has at least one solution which minimizes the functional $I$ given by \eqref{integral_accion} on $\wphi$.
\end{thm}

\begin{proof}
For $\b{w} \in \wphi$, we write $\b{w}=\overline{\b{w}}+\widetilde{\b{w}}$ where $\overline{\b{w}} =\frac1T\int_0^T \b{w}(t)\ dt$.


\begin{comment}
\textbf{Case 1}:
\[
\begin{split}
I(\b{u})=&\int_{0}^T \mathcal{L}(t,\b{u}(t),\dot{\b{u}}(t))\ dt\\
=& \int_{0}^T (\mathcal{L}(t,\b{u}(t),\dot{\b{u}}(t))- \mathcal{L}(t,\overline{\b{u}},\dot{\b{u}}(t)))\ dt + \int_{0}^T
(\mathcal{L}(t,\overline{\b{u}},\dot{\b{u}}(t))-\mathcal{L}(t,\overline{\b{u}},\overline{\dot{\b{u}}}))\ dt\\
&+ \int_0^T \mathcal{L}(t,\overline{\b{u}},\overline{\dot{\b{u}}})\ dt\\
=&\int_{0}^T \int_{0}^{1} \langle D_{\b{x}}\mathcal{L}(t,\overline{\b{u}}+s\widetilde{\b{u}}(t),\dot{\b{u}}(t)),\widetilde{\b{u}}(t)\rangle\ ds \ dt +
\int_{0}^T \int_{0}^{1} \langle D_{\b{y}}\mathcal{L}(t,\overline{\b{u}},\dot{\b{u}}(t)+s\widetilde{\dot{\b{u}}}(t)),\widetilde{\dot{\b{u}}}(t)\rangle\ ds \ dt\\
&+  \int_{0}^T \mathcal{L}(t,\overline{\b{u}},\overline{\dot{\b{u}}})\ dt  \\
\geq &-C_1\|f\|_{L^1}\|\widetilde{\b{u}}\|_{L^{\infty}}-C_2\|g\|_{\lpsi}\|\widetilde{\dot{\b{u}}}\|_{\lphi} +  \int_{0}^T \mathcal{L}(t,\overline{\b{u}},\overline{\dot{\b{u}}})\ dt  \\
\geq &-C_1\|f\|_{L^1}\|\widetilde{\b{u}}\sobnor-C_2\|g\|_{\lpsi}\|\widetilde{\dot{\b{u}}}\|_{\lphi} +  \int_{0}^T \mathcal{L}(t,\overline{\b{u}},\overline{\dot{\b{u}}})\ dt  \\
\geq & -C_1\|f\|_{L^1}\|\dot{\widetilde{\b{u}}}\|_{\lphi} -C_2\|g\|_{\lpsi}\|\widetilde{\dot{\b{u}}}\|_{\lphi} +  \int_{0}^T
\mathcal{L}(t,\overline{\b{u}},\overline{\dot{\b{u}}})\ dt\\
\geq & -C_1\|f\|_{L^1}\|\dot{\b{u}}\|_{\lphi} -C_2\|g\|_{\lpsi}\|\widetilde{\dot{\b{u}}}\|_{\lphi} + {\color{celeste}\int_{0}^T
\mathcal{L}(t,\overline{\b{u}},\overline{\dot{\b{u}}})\ dt}.
\end{split}
\]
\end{comment}




\[
\begin{split}
I(\b{u})=&\int_{0}^T \mathcal{L}(t,\b{u}(t),\dot{\b{u}}(t))\ dt\\
=& \int_{0}^T (\mathcal{L}(t,\b{u}(t),\dot{\b{u}}(t))- \mathcal{L}(t,\overline{\b{u}},\dot{\b{u}}(t)))\ dt + \int_{0}^T
\mathcal{L}(t,\overline{\b{u}},\dot{\b{u}}(t))\ dt\\
=&\int_{0}^T \int_{0}^{1} \langle D_{\b{x}}\mathcal{L}(t,\overline{\b{u}}+s\widetilde{\b{u}}(t),\dot{\b{u}}(t)),\widetilde{\b{u}}(t)\rangle\ ds \ dt + + \int_{0}^T
\mathcal{L}(t,\overline{\b{u}},\dot{\b{u}}(t))\ dt\\
\geq &-\|f\|_{L^1}\|\widetilde{\b{u}}\|_{L^{\infty}} + \int_{0}^T
\mathcal{L}(t,\overline{\b{u}},\dot{\b{u}}(t))\ dt\\
\geq & -C_1\|f\|_{L^1}\|\dot{\widetilde{\b{u}}}\|_{\lphi} + \int_{0}^T
\mathcal{L}(t,\overline{\b{u}},\dot{\b{u}}(t))\ dt\\
\geq & -C_1\|f\|_{L^1}\|\dot{\b{u}}\|_{\lphi} {\color{celeste}+ \int_{0}^T \mathcal{L}(t,\overline{\b{u}},\dot{\b{u}}(t))\ dt}.
\end{split}
\]

Cuando tomamos $\mathcal{L}$
\[
\mathcal{L}(t,\b{x},\b{y})=\Phi(|\b{y}|)+F(t,\b{x}).
\]

\begin{comment}
Para caso 1:
\[
I(\b{u})\geq-C_1\|f\|_{L^1}\|\dot{\b{u}}\|_{\lphi} -C_2\|g\|_{\lpsi}\|\widetilde{\dot{\b{u}}}\|_{\lphi}+\int_{0}^T
\Phi(|\overline{\dot{\b{u}}}|)+F(t,\overline{\b{u}}).
\]
\end{comment}

Tenemos
\begin{equation}\label{eq:est_abajo_I}
I(\b{u})\geq-C_1\|f\|_{L^1}\|\dot{\b{u}}\|_{\lphi}+\int_{0}^T \Phi(|\dot{\b{u}}|)+F(t,\overline{\b{u}}).
\end{equation}
Para $\b{w}$ denotemos por $\|| \b{w} \||_{\lphi}$ la norma de Luxembourg de $\b{w}$.

Es sabido que que si $\|| \b{w} \||_{\lphi} \geq 1$ entonces $\int_{0}^T \Phi(|\b{w}|) \geq \|| \b{w} \||_{\lphi}$ y que $\|| \b{w}\||_{\lphi} \leq
\|\b{w}\|_{\lphi}\leq 2\|| \b{w}\||_{\lphi}$. Entonces se tiene para $\||\b{u}\|| \geq 1$

\[
\begin{split}
I(\b{u})&\geq-C_1\|f\|_{L^1}\|\dot{\b{u}}\|_{\lphi}+\int_{0}^T \Phi(|\dot{\b{u}}|)+F(t,\overline{\b{u}})\\
&\geq -C_1\|f\|_{L^1}\|\dot{\b{u}}\|_{\lphi}+\|| \dot{\b{u}} \||_{\lphi}+\int_{0}^TF(t,\overline{\b{u}})\\
&\geq-C_1\|f\|_{L^1}\|\dot{\b{u}}\|_{\lphi}+\frac{1}{2}\| \dot{\b{u}} \|_{\lphi}+\int_0^TF(t,\overline{\b{u}})\\
&= \left(-C_1\|f\|_{L^1}+\frac{1}{2}\right)\|\dot{\b{u}}\|_{\lphi}+\int_{0}^{T}F(t,\overline{\b{u}}).
\end{split}
\]
Therefore, if $\|\b{u}\sobnor \rightarrow \infty$ then $(|\overline{\b{u}}|^2+ \|\dot{\b{u}}\|_{\lphi})\rightarrow \infty$. If $|\overline{\b{u}}|^2 \rightarrow
\infty$ by \eqref{propiedad1coercividad} we have $I(\b{u}) \rightarrow \infty$. If $\|\dot{\b{u}}\|_{\lphi}\rightarrow \infty$ usamos \eqref{cota}. En efecto,

Si $-C_1\|f\|_{L^1}+\frac{1}{2} \geq 0$, entonces $I(\b{u}) \rightarrow \infty$ cuando $\|\dot{\b{u}}\|_{\lphi}\rightarrow \infty$. Pero $-C_1\|f\|_{L^1}+\frac{1}{2}
\geq 0$ si y solamente si $\|f\|_{L^1([0,T])}\leq\frac{1}{2 C_1}=\frac{1}{2\|1\|_{\lpsi([0,T])}}$, lo cual es justamente \eqref{cota}.
 ($C_1=\|1\|_{\lpsi([0,T])}$ por \eqref{estimacion2}).
\end{proof}





In the sequel, we will discuss  the  conditions that guarantee the coercivity of  the functional $u \to \int_0^T \Phi(|u|)\,dx$ in $L^{\Phi}$.



\subsection{Condition that guarantees coercivity (a la antigua...sin $\lambda$!!!) }
{\color{red}
Habr\'ia que poner s\'olo el resultado que sigue, en lugar de TODOS los INTENTOS de la versi\'on de junio?????

En la �ltima secci\'on dej\'e los intentos previos en caso de que alguno deba ser rescatado del ostracismo.}

\begin{thm}
$\Psi \in \Delta_2^0$ if and only if 
$\lim\limits_{\|u\orlnor \to \infty} \frac{\int_0^T \Phi(|u|)\,dx}{\|u\orlnor}=\infty$
\end{thm}

\begin{proof}
$\Rightarrow)$
$\Psi \in \Delta_2^0$ if and only if $\Phi \in \nabla_2$ see \cite[Th.3 , pp. 22-23 ]{rao1991theory}. 
Then, there exists $k>2$ such that 
%$\Phi(2x)\geq k \Phi(x)$ for every $x>0$ and consequently
$\Phi(2^n x)\geq k^n \Phi(x)$ for every $x>0$.
\\
If $\lambda>1$, there exists  $n\in \nn$ such that $2^n\leq \lambda <2^{n+1}$
and consequently
%$n\leq \log_2\lambda <n+1$ and 
$\Phi(\lambda x)\geq 
%\Phi(2^n x)\geq 
%k^n \Phi(x)\geq k^{\log_2\lambda-1}\Phi(x)=
k^{-1}\lambda^{\nu}\Phi(x)$
where  $k=2^{\nu}$ with $\nu>1$. 
%because $k>2$. 
In that way, we have 
\begin{equation}\label{delta2-consecuencia}
\Phi(\lambda x)\geq C_2\lambda^{\nu}\Phi(x) \;\;\mbox{with}\;\;\lambda>1.
\end{equation}
We also know, by \cite[Th. 10.5, pp. 92 ]{KR}, that
\begin{equation}\label{norma-orlicz-cota}
\|u\orlnor\leq \frac{1}{\lambda}\left(1+\int_0^T \Phi(\lambda|u|)\,dx\right)
\;\;\mbox{for every}\;\;\lambda>0.
\end{equation}
Let $\lambda<1$. Then, from \eqref{delta2-consecuencia} and \eqref{norma-orlicz-cota}, we get
\[
\frac{\int_0^T \Phi(|u|)\,dx}{\|u\orlnor}
%=
%\frac{\int_0^T \Phi(\frac{1}{\lambda}\lambda |u|)\,dx}{\|u\orlnor}
\geq
\frac{C_2}{\lambda^{\nu}}\frac{\int_0^T \Phi(\lambda|u|)\,dx}{\|u\orlnor}
\geq 
%\frac{C_2}{\lambda^{\nu}}\frac{\lambda \|u\orlnor-1}{\|u\orlnor}=
\frac{C_2}{\lambda^{\nu-1}}\left(1-\frac{1}{\lambda \|u\orlnor}\right).
\]
As $\|u\orlnor \to \infty$, we choose $\lambda=\frac{2}{\|u\orlnor}>1$ and  thus 
\[
\frac{\int_0^T \Phi(|u|)\,dx}{\|u\orlnor}\geq
C (\|u\orlnor)^{\nu-1}\to \infty
\]
because $\nu>1$.

$\Leftarrow)$ Assume that $\Psi \notin \Delta_2$.
\\
Now, by \cite[Th. 4.2, pp. 24]{KR},  there exists a sequence of real  numbers  $r_n$ such that
$r_n \to \infty$ and 
\[
\lim\limits_{n \to \infty} \frac{r_n \psi(r_n)}{\Psi(r_n)}=+\infty.
\]
Now, we choose
$u_n=\psi(r_n)\chi_{[0,\frac{1}{\Psi(r_n)}]}$, then 
by \cite[pp. 72]{KR}, we get 
\[
\|u_n\orlnor =\frac{\psi(r_n)}{\Psi(r_n)}\Psi^{-1}(\Psi(r_n))=
\frac{r_n\psi(r_n)}{\Psi(r_n)}\to \infty\;\;\mbox{as}\;\;n \to \infty.
\]
Next, using Young's equality, we have
\[
\frac{\int_0^T \Phi(u_n)}{\|u_n\orlnor}=
\frac{\Phi(\psi(r_n))[\Psi(r_n)]^{-1}}{\frac{r_n\psi(r_n)}{\Psi(r_n)}}=
\frac{r_n\psi(r_n)-\Psi(r_n)}{r_n\psi(r_n)}=1-\frac{\Psi(r_n)}{r_n\psi(r_n)}.
\]
As $\frac{\Psi(r_n)}{r_n\psi(r_n)}\to 0$ when $r_n \to \infty$, then 
$\frac{\int_0^T \Phi(u_n)}{\|u_n\orlnor}$ is bounded and therefore
$\lim\limits_{\|u\orlnor \to \infty} 
\frac{\int_0^T \Phi(|u|)\,dx}{\|u\orlnor}\neq \infty$.
\end{proof}

\section{Algo para rescatar????}
\begin{nota_fer}
\textbf{Tenemos la conjetura que todo sale si pedimos $\Phi$ que satisface $\nabla_2$. Problema: probarla o refutarla} Un problema m�s amplio ser�a tratar de sacar hip�tesis ya sea la condici�n $\Delta_2$ de alguna de las funciones ($\Phi$ o $\Psi$) o usar estas condiciones para el infinito o a�n para cero(????).
\end{nota_fer}


We know that  $\Phi\in \Delta_2$ if and only if 
for every $\epsilon>0$ there exists $C_\epsilon>0$ such that
\begin{equation}\label{delta2-control-potencia}
C^{-1}_\epsilon \min\{\lambda^{\alpha-\epsilon},\lambda^{\beta+\epsilon}\} \Phi(x)
\leq \Phi(\lambda x)\leq
C_\epsilon \max\{\lambda^{\alpha-\epsilon},\lambda^{\beta+\epsilon}\} \Phi(x)
\end{equation}
for every $x,\lambda>0$.

Recall that by Theorem 11.11 in \cite{M} we have $p\leq \alpha\leq \beta\leq q$ where $\alpha,\beta$ are lower and upper Orlicz indices, and $p,q$ are lower and upper Simonenko indices.
We also have that $\Phi,\Psi\in \Delta_2$ implies that $\Phi(x)\sim x\varphi(x)\sim \Psi(\varphi(x))$.

By Theorem 4.3 in \cite{KR} or Corollary 4.4 in \cite{rao1991theory} we know that $p>1$ if and only if $\Phi \in \nabla_2$ and by  Theorem 11.7 in \cite{M} we have that $\beta<\infty$ if and only if $\Phi \in \Delta_2$.

Theorem 10.4 in \cite{KR} says that if there exists $k^*$ such that $\int \Psi[\varphi(k^*|u|)]\,dx=1$, then $\|u{\orlnor}=\int \varphi(k^*|u|)|u|\,dx$.

\begin{comentario}
If $\varphi$ is a continuous function, then there exists $k^*$ in Theorem 4.3 of \cite{KR}. See \cite[pages 89, 90]{KR}.
\end{comentario}












\begin{lem}\label{kn_0} Let $\Phi\in \nabla_2\cap\Delta_2$ and  $\{k_n\}$ such that $\int \Psi[\varphi(k_n|u_n|)]\,dx=1$. If
$\|u_n{\orlnor}\to\infty $ as $n \to \infty$, then $k_n \to 0$ as $n \to \infty$.
\end{lem}


\begin{proof}
Since $\Phi\in \nabla_2\cap\Delta_2$, then $\Psi \circ \varphi\in \Delta_2$ and \eqref{delta2-control-potencia} holds.

Now, for such a sequence $k_n$ we have
\begin{equation}
1=\int \Psi[\varphi(k_n|u_n|)]\,dx\geq 
C_{\epsilon}^{-1}\min\{k_n^{\alpha-\epsilon},k_n^{\beta+\epsilon}\}\int \Psi(\varphi(|u_n|))\,dx
\end{equation}
for every $\epsilon>0$. Due to $\Phi \in \Delta_2$, there exists $C_{\Lambda}>0$ such that 
\begin{equation}
C_{\epsilon}^{-1}\min\{k_n^{\alpha-\epsilon},k_n^{\beta+\epsilon}\}\int \Psi(\varphi(|u_n|))\,dx
\geq 
C_{\Lambda}C^{-1}_{\epsilon}\min\{k_n^{\alpha-\epsilon},k_n^{\beta+\epsilon}\}
\int \Phi(|u_n|)\,dx
\end{equation}
for every $\epsilon>0$. 
\\
As $\|u_n{\orlnor}\to\infty $ as $n \to \infty$, there exists $N\in \nn$ such that $\|u_n{\orlnor}\geq 1$ for every $n> N$
and consequently 
$\int \Phi(|u_n|)\,dx\geq \||u_n\||_{\lphi}$ for every $n> N$. 
\\
In that way, we have 
\begin{equation}
1 \geq C_{\Lambda}C_{\epsilon}^{-1}\min\{k_n^{\alpha-\epsilon},k_n^{\beta+\epsilon}\}
\||u_n\||_{\lphi}
\end{equation}
for every $\epsilon>0$ and for every $n>N$. 
\\
{\color{violeta} In addition, $\alpha>1$ because $\Phi \in \nabla_2$}, then $\alpha-	\epsilon>0$ for every $0<\epsilon<1$.
\\
Now, we have
\begin{equation}
\frac{1}{C_{\Lambda}C_{\epsilon}^{-1} \|| u_n\||_{ \lphi}}\geq \min\{k_n^{\alpha-\epsilon},k_n^{\beta+\epsilon}\}
\end{equation}
for every $n>N$ and provided that $\alpha-\epsilon>0$.
\\
Because of the 
equivalence between Orlicz and Luxemburg norms, we have 
$\||u_n\||_{\lphi}\to \infty$ as $n \to \infty$, 
then
\begin{equation}
\min\{k_n^{\alpha-\epsilon},k_n^{\beta+\epsilon}\}\to 0\;\;\mbox{as}\;\;n\to \infty
\end{equation}
where $\alpha-\epsilon>0$ and $\beta+\epsilon>0$; therefore,  $k_n \to  0$ as $n \to \infty$.
\end{proof}




{\bf El resultado que sigue es v\'alido para cualquier funci\'on u?????}
\begin{nota_fer}  Si me parece que habr�a que decir que entendemos por coercitivo en el teorema de abajo. Yo creo que la demostraci�n \textbf{Problema: verificarlo y escribarlo} implica el sentido m�s fuerte de coercitividad
\[\lim_{\|u\orlnor\to\infty}\frac{\int_0^T\Phi(|u_n|)dx}{\|u\orlnor}=\infty\]
\end{nota_fer}

\begin{thm}
If $\Phi\in \Delta_2\cap \nabla_2$, then 
the functional $u \to \int_0^t \Phi(|u|)\,dx$ is  coercive in $L^\Phi$.
\end{thm}

\begin{proof}
Assume $\|u_n{\orlnor}\to\infty $ when $n\to \infty$ and 
{\color{violeta} suppose that there exists $\{k_n\}$} such that 
%\linebreak 
$\int \Psi[\varphi(k_n|u_n|)]\,dx=1$.
\\
In that way, by Theorem 10.4 in \cite{KR}, we have
\begin{equation}
\int \Phi(|u_n|)\,dx-C\|u_n{\orlnor}=
\int \Phi(|u_n|)\,dx-C\int \varphi(k_n|u_n|)|u_n|\,dx
\end{equation}
As $\Phi \in \Delta_2$, there exists $\Lambda_{\Phi}>0$ such that
\begin{equation}
\int \Phi(|u_n|)\,dx-\frac{C}{k_n}\int \varphi(k_n|u_n|)|u_n|k_n\,dx\geq 
\int \Phi(|u_n|)\,dx-\frac{C\Lambda_{\Phi}}{k_n}\int \Phi(k_n|u_n|)\,dx\
\end{equation}
%%
{\color{violeta} Since $\Phi\in \nabla_2$, then $\alpha>1$ \cite{KR,rao1991theory};
now, we choose $\epsilon>0$ such that $\alpha-\epsilon>1$.}
\\
In addition,  by Lemma \ref{kn_0},   
there exists $N \in \nn$ such that $k_n<1$ for every $n>N$ and therefore
\begin{equation}
\Phi(k_n |u_n|) \leq C_\epsilon k_n^{\alpha-\epsilon}\Phi(|u_n|)
\end{equation}
for every $n>N$ and where $\alpha-\epsilon>1$.
\\
Now, 
\begin{equation}
\int \Phi(|u_n|)\,dx-\frac{C\Lambda_{\Phi}}{k_n}
\int \Phi(k_n|u_n|)\,dx
\geq
\left(1-C\Lambda_{\Phi}C_\epsilon k_n^{\alpha-\epsilon-1}\right)\int \Phi(|u_n|)\,dx
\end{equation}
with $\alpha-\epsilon-1>0$.
\\
As $\|u_n{\orlnor}\to\infty $, there exists $N_2\in \nn$ such  $\|u_n\orlnor>1$ for every $n>N_2$, then
\begin{equation}
\left(1-C\Lambda_{\Phi}C_\epsilon k_n^{\alpha-\epsilon-1}\right)\int \Phi(|u_n|)\,dx
\geq 
\left(1-C\Lambda_{\Phi}C_\epsilon k_n^{\alpha-\epsilon-1}\right)\|| u_n|\|_{\lphi}
\end{equation}
with $\alpha-\epsilon-1>0$.
\\
Finally, due to $\|u_n{\orlnor}\to\infty $, 
we have $k_n^{\alpha-\epsilon-1}\to 0$, $\|| u_n|\|_{\lphi}\to \infty$ and consequently
$\int \Phi(|u_n|)\,dx-C\|u_n{\orlnor}\to \infty$ for every $C>0$; that is, 
the functional $u \to \int_0^t \Phi(|u|)\,dx$ is coercive in $L^\Phi$.
\end{proof}

\begin{nota_fer}
Lamentablemente el libro de Rao pag. 43 dice que $\nabla'$ implica $\nabla_2$. Luego la primera parte de este terorema es consecuencia de la proposici�n anterior.
 El otro inciso si me parece que tiene sentido pues para ver que no es coercitiva basta mostrar contraejemplos. Este comentario no aporta problemas abiertos, solo es una observaci�n de la redacci�n
\end{nota_fer} 


{\color{red}
Sonia: Totalmente de acuerdo, por eso escrib\'i la observaci\'on al final de la prueba. Si el resultado usando $\nabla_2$ es correcto, el primer inciso sobra.}

\begin{prop}
{\color{violeta} Let $u$ be a characteristic function.
\\
If $\Phi \in \Delta_2\cap\nabla'$, then the functional $u \to \int_0^t \Phi(|u|)\,dx$ is coercive in $L^\Phi$.}
\\
If $\Phi \in \nabla_3$, then the functional (la suelen llamar {\color{violeta} modular}) $u \to \int_0^t \Phi(|u|)\,dx$ is not coercive in $L^\Phi$.
\end{prop}

\begin{proof}
Let  $u=\alpha \chi_A$ with $\alpha \in \rr$  and where $A$ is a subset of $[0,T]$.
\\
Assume that 
$\|u\orlnor \to \infty$ then $\|u\orlnor=\alpha\|\chi_A\orlnor=\alpha m(A) \Psi^{-1}\left(\frac{1}{m(A)}\right) \to \infty$ and consequently $m(A) \Psi^{-1}\left(\frac{1}{m(A)}\right)\to  \infty$ as {\color{violeta} $m(A)\to 0$}.
\\

On the other hand, we have
\begin{equation}
\frac{\int \Phi(|u|)\,dx}{\|u\orlnor}=\frac{\Phi(\alpha)m(A)}{\alpha m(A) \Phi^{-1}(\frac{1}{m(A)})}=\frac{\Phi(\alpha)}{\alpha \Psi^{-1}(\frac{1}{m(A)})}.
\end{equation}
Let $r=\frac{1}{m(A)}\geq \frac{1}{T}$, then $\alpha \frac{\Psi^{-1}(r)}{r}\to \infty$ as $\alpha, r\to \infty$ and
\begin{equation}
\frac{\int \Phi(|u|)\,dx}{\|u\orlnor}=\frac{\Phi(\alpha)}{\alpha \Psi^{-1}(r)}
\end{equation}
As $\Phi \in \nabla '$, there exists  $C_1>0$ such that $\Phi(xy)\geq C_1\Phi(x)\Phi(y)$ for every $x,y>0$.
Taking $y=\frac{\alpha}{x}$, we have 
$
\Phi(\alpha)\geq C_1\Phi(x)\Phi(\frac{\alpha}{x})$ and choosing
$x=\Phi^{-1}(r)$ we get 
\begin{equation}\label{Phi-r-1}
\frac{\Phi(\alpha)}{r}\geq C_1 \Phi\left(\frac{\alpha}{\Phi^{-1}(r)}\right).
\end{equation}
\\
As $\Phi \in \Delta_2$, there exists $C_2>0$ such that 
\begin{equation}\label{Phi-r-2}
\Phi\left(\frac{\alpha}{\Phi^{-1}(r)}\right)\geq C_2 \frac{\alpha}{\Phi^{-1}(r)}\varphi\left(\frac{\alpha}{\Phi^{-1}(r)}\right).
\end{equation}
We also have that $r \leq \Phi^{-1}(r)\Psi^{-1}(r)\leq 2r$ for every $r>0$, then 
\begin{equation}\label{Phi-r-3}
C_2 \frac{\alpha}{\Phi^{-1}(r)}\varphi\left(\frac{\alpha}{\Phi^{-1}(r)}\right)\geq 
C_2\frac{\alpha \Psi^{-1}(r)}{2r}\varphi\left(\frac{\alpha \Psi^{-1}(r)}{2r}\right).
\end{equation}
Thus, from \eqref{Phi-r-1}-\eqref{Phi-r-3},
\begin{equation}
\frac{\Phi(\alpha)}{r}\geq C_1 C_2\frac{\alpha \Psi^{-1}(r)}{2r}\varphi\left(\frac{\alpha \Psi^{-1}(r)}{2r}\right);
\end{equation}
and then ({\bf Ac\'a ten\'ia que despejar y no irme por la ramas!!!)}


\begin{equation}
\frac{\Phi(\alpha)}{\alpha\Psi^{-1}(r)}\geq \frac{C_1 C_2}{2}\varphi\left(\frac{\alpha \Psi^{-1}(r)}{2r}\right).
\end{equation}
{\color{violeta} Due to $\alpha\frac{\Psi^{-1}(r)}{r}\to \infty$ as $\alpha,r\to \infty$} and 
the fact that $\varphi(x) \to \infty$ as $x \to \infty$, we obtain
\begin{equation}
\frac{C_1 C_2}{2}\varphi\left(\frac{\alpha \Psi^{-1}(r)}{2r}\right)\to \infty
\end{equation}
and consequently
\begin{equation}
\frac{\Phi(\alpha)}{\alpha\Psi^{-1}(r)}\to  \infty\;\;\mbox{as}\;\;\alpha,r\to \infty;
\end{equation}
that is, 
\begin{equation}
\frac{\int \Phi(|u|)\,dx}{\|u\orlnor}\to\infty\;\;\mbox{as}\;\;\|u\orlnor\to \infty.
\end{equation}
Now, we will see that  $\nabla_3$ functions do not imply the wished coercivity
in case of $u$ being a characteristic function.
\begin{nota_fer} Me parece que es posible ver que no es coercitiva en el sentido m�s debil que
\[\int \Phi(|u_n|)\,dx-C\|u_n{\orlnor}\]
no tiende a infinito.\textbf{Demostrarlo}
Por otra parte, notar que al final de la demostraci�n se toma $\alpha=r$, como lo que estamos mostrando es un contraejemplo, no es necesario mantener la distinci�n entre $\alpha$ y $r$. Podr�amos haber tomado de movida 
\[u=u_r=r\chi_{[0,1/r]},\]
\end{nota_fer}
To do so, suppose there exists a function  $\Phi \in \nabla_3$ such that 
\begin{equation}
\frac{\int \Phi(|u|)\,dx}{\|u\orlnor}=\frac{\Phi(\alpha)}{\alpha \Psi^{-1}(r)}\to \infty\;\;\mbox{as}\;\;\|u\orlnor\to \infty,
\end{equation}
then 
\begin{equation}\label{Phi-r-nabla3-1}
\frac{\int \Phi(|u|)\,dx}{\|u\orlnor}=\frac{\Phi(\alpha)}{\alpha \Psi^{-1}(r)} 
\leq 
\frac{\Phi(\alpha)\Phi^{-1}(r)}{\alpha r}\to \infty \;\mbox{as}\;\;\|u\orlnor\to \infty
\end{equation} 
for any $\alpha>0$.
\\
However,  $\Phi \in \nabla_3$ if and only if there exists $C_3>0$ such that
 \begin{equation}
 \frac {\Phi(r)\Phi^{-1}(r)} {r^2} \leq C_3\;\;\mbox{for every}\;\;r>0
 \end{equation}
 which contradicts \eqref{Phi-r-nabla3-1} choosing $\alpha=r$.
 \\
 Therefore,  if $\Phi\in \nabla_3$ then the functional $u \to \int_0^t \Phi(|u|)\,dx$ 
is not coercive in $L^{\Phi}$.
\end{proof}





\begin{rem}
$\nabla'$ functions are a subset of $\nabla_2$ functions and 
$\nabla_3$ functions belong to  the set of  $\Delta_2$ functions \cite[Chapter 2]{ rao1991theory}.
\end{rem}

We have just proved that 
coercivity  in $L^\Phi$ for characteristic functions holds when
$\Phi$ belongs  to a subclass of $\Delta_2$ functions.
\\
Nevertheless,  $\Delta_2$ condition is not sufficient  to get the wished coercivity in $L^\Phi$;
in fact, $\nabla_3$ functions, which belong to the set of $\Delta_2$ functions, do not imply coercivity in $L^\Phi$.

We can also see that $\Delta_2$ condition is not sufficient  to have coercivity in $L^\Phi$  providing a counterexample.
\begin{nota_fer} Deje de entender. Al fin y al cabo, el de abajo ser�a un contraejemplo para mostrar la misma cosa que ya vimos en otro contraejemplo? 


Pero se nos plantea este problema, la $\Phi$ del contraejemlo de abajo no ser� $\nabla_3$? Si fuese as� lo que decimos abajo ya lo sabemos \textbf{Problema abierto: probar o refutar que la $\Phi$ es $\nabla_3$}
\end{nota_fer}

{\color{red}
Sonia: En efecto, quise decir que como $\nabla_3$ implica $\Delta_2$, el ejemplo que viene a continuaci\'on es redundante para probar que $\Delta_2$ no implica coercividad.
\\
Por otra parte, en la p\'agina 39 del Rao dice que la funci\'on $\Psi$ que  usamos es $ \Delta_3$, luego $\Phi\in\nabla_3$ (ver Teorema 3 p\'agina 38 de Rao). Conclusi\'on: O ponemos el enunciado general para $\nabla_3$ o el contraejemplo que viene a continuaci\''on, porque dicen lo mismo.}

{\color{green}  Hay una diferencia entre el enunciado general y el ejemplo de abajo. El ejemplo de abajo es un contraejemplo a la afirmaci�n
\begin{equation}\label{eq:paraborrar}
\lim_{\|u\orlnor\to\infty}\int \Phi(|u_n|)\,dx-C\|u_n{\orlnor}=\infty
\end{equation}
Mientras que el resultado general lo es de
\begin{equation}\label{eq:paraborrar2}\lim_{\|u\orlnor\to\infty}\frac{\int_0^T\Phi(|u_n|)dx}{\|u\orlnor}=\infty
\end{equation}
Como \eqref{eq:paraborrar2} implica \eqref{eq:paraborrar},  el contraejemplo de abajo no es consecuencia del resultado general. Adem�s nosotros lo que necesitamos es \eqref{eq:paraborrar2}. Todo esto nos lleva a:

\textbf{Problema abierto (jaja): ver si el enunciado general  sigue sirviendo con \eqref{eq:paraborrar}
  }}



 In fact, suppose that $\Phi(r)=(r+1)\log(r+1)-r$  then $\Phi\in\Delta_2$; and, we also have $\Psi(r)=e^r-r-1$
which is not a $\Delta_2$ function \cite[Chapter 1, pp 22]{rao1991theory}.
\\
Let $u_n=\frac{1}{m(A_n)}\chi_{A_n}$ where $A_n \subset[0,T]$ such that $m(A_n)\to 0$.
\\
If $r_n=\frac{1}{m(A_n)}$, we have
\begin{equation}
\int \Phi(|u_n|)\,dx-C\|u_n{\orlnor}=\frac{\Phi(r_n)}{r_n}-C\Psi^{-1}(r_n)
\end{equation}
then 
\begin{equation}
\begin{split}
  \frac{\Phi(r_n)}{r_n}-C\Psi^{-1}(r_n) \leq \left(1+\frac{1}{r_n}\right)\log(r_n+1)-C\log(r_n+1)=
	\\
  \left(1+\frac{1}{r_n}-C\right)\log (r_n+1) 
	\end{split}
  \end{equation}
Now, as $r_n$ goes to $\infty$, we get 
 \begin{equation}\label{eq:no_coerciva}
  \lim\limits_{r_n \to \infty}
	\left(\frac{\Phi(r_n)}{r_n}-C\Psi^{-1}(r_n)\right)  \leq  \left(1-C\right) \infty.
  \end{equation}
In that way, 
$\int \Phi(|u_n|)\,dx-C\|u_n {\orlnor} $ is an  upper bounded function provided that $C>1$, which means that the functional
 $u \to \int_0^t \Phi(|u|)\,dx$ is not coercive in $L^\Phi$ for such a particular $\Delta_2$ function
$\Phi$.

{\bf Porque para tener coercividad, la expresi\'on no deber\'ia estar acotada para $C$ grande, es as\'i????}

\bibliographystyle{plain}
\bibliography{biblio}

%\printbibliography

\end{document}
