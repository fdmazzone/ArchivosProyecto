\documentclass[twoside]{article}
%%Paquetes


\usepackage{amssymb,amsthm}
\usepackage{amsmath}
\usepackage{color}
\usepackage{ esint }
%\usepackage{graphicx}
%\usepackage{wrapfig}
%\usepackage{subfigure}
\usepackage{fancyhdr}
\usepackage{times}
%\usepackage{theorem}
\usepackage[latin1]{inputenc}
%\usepackage{showkeys}
\usepackage{comment}
\usepackage{url}
\usepackage{xcolor}
\usepackage{adjustbox}
%Teorema y similes
\usepackage[maxnames=6,backend=bibtex]{biblatex}
\bibliography{biblio.bib}

\definecolor{rosa}{rgb}{1,0.3,0.9}
\definecolor{violeta1}{rgb}{0.5,0.3,0.5}
\definecolor{violeta}{rgb}{0.5,0.1,0.5}
\definecolor{negro}{rgb}{0.5,0.2,0.4}
\definecolor{celeste}{rgb}{0.1,0.4,1}
\definecolor{naranja}{rgb}{1,0.5,0}
\definecolor{color_nota_fer}{HTML}{DEBFDB}


\newenvironment{colbox}[2]{%
    \begin{adjustbox}{minipage={\linewidth},margin=1ex,bgcolor=#1,env=center}
        #2}{%
    \end{adjustbox}%
}
\newcounter{nota_fer_cont}
\newenvironment{nota_fer}[1]{\refstepcounter{nota_fer_cont}\begin{colbox}{color_nota_fer}{\textbf{Comentario Leo-Graciela-Fernando \arabic{nota_fer_cont}.} #1}}{\end{colbox}}


\newtheorem{thm}{Theorem}[section]
\newtheorem{cor}[thm]{Corollary}
\newtheorem{lem}[thm]{Lemma}
\newtheorem{rem}[thm]{Remark}
\newtheorem{defi}[thm]{Definition}
\newtheorem{prop}[thm]{Proposition}
\theoremstyle{remark}
\newtheorem{comentario}{Remark}


\title{Some existence results on periodic solutions of 
Euler-Lagrange equations in an Orlicz-Sobolev space setting}
\author{Sonia Acinas \thanks{SECyT-UNRC, UNSL and CONICET}\\
Instituto de Matem\'atica Aplicada San Luis (CONICET-UNSL)\\
(5700) San Luis, Argentina\\
Universidad Nacional de La Pampa\\
(6300) Santa Rosa, La Pampa, Argentina\\
\url{sonia.acinas@gmail.com}\\[3mm]
Leopoldo Buri \thanks{SECyT-UNRC}\\
Dpto. de Matem\'atica, Facultad de Ciencias Exactas, F\'{\i}sico-Qu\'{\i}micas y Naturales\\
Universidad Nacional de R\'{i}o Cuarto\\
(5800) R\'{\i}o Cuarto, C\'ordoba, Argentina,\\
\url{lburi@exa.unrc.edu.ar}\\[3mm]
Graciela Giubergia \thanks{SECyT-UNRC and CONICET}\\
Dpto. de Matem\'atica, Facultad de Ciencias Exactas, F\'{\i}sico-Qu\'{\i}micas y Naturales\\
Universidad Nacional de R\'{i}o Cuarto\\
(5800) R\'{\i}o Cuarto, C\'ordoba, Argentina,\\
\url{ggiubergia@exa.unrc.edu.ar}\\[3mm]
Fernando D. Mazzone \thanks{SECyT-UNRC and CONICET}\\
Dpto. de Matem\'atica, Facultad de Ciencias Exactas, F\'{\i}sico-Qu\'{\i}micas y Naturales\\
Universidad Nacional de R\'{i}o Cuarto\\
(5800) R\'{\i}o Cuarto, C\'ordoba, Argentina,\\
\url{fmazzone@exa.unrc.edu.ar}\\[3mm]
Erica L. Schwindt\thanks{ANR. AVENTURES - ANR-12-BLAN-BS01-0001-01}\\
Universit\'{e} d'{O}rl\'{e}ans, Laboratoire MAPMO, CNRS, UMR 7349, \\
F\'ed\'eration Denis Poisson, FR 2964,\\
B\^{a}timent de Math\'{e}matiques, BP 6759, 45067 Orl\'{e}ans Cedex 2, France,\\
\url{leris98@gmail.com}}

\date{}

\newcommand{\orlnor}{\|_{L^{\Phi}}}
\newcommand{\lurnor}{\|^{*}_{L^{\Phi}}}
\newcommand{\linf}{\|_{L^{\infty}}}
\newcommand{\lphi}{L^{\Phi}}
\newcommand{\lpsi}{L^{\Psi}}
\newcommand{\ephi}{E^{\Phi}}
\newcommand{\claseor}{C^{\Phi}}
\newcommand{\wphi}{W^{1}\lphi}
\newcommand{\sobnor}{\|_{W^{1}\lphi}}
\newcommand{\domi}{\mathcal{E}^{\Phi}_d(\lambda)}
\renewcommand{\b}[1]{\boldsymbol{#1}}
\newcommand{\rr}{\mathbb{R}}
\newcommand{\nn}{\mathbb{N}}
\newcommand{\ccdot}{\b{\cdot}}
\renewcommand{\leq}{\leqslant} 
\newcommand{\epsi}{E^{\Psi}}

\begin{document}



\maketitle
%
\begingroup%Locallizing the change to `thefootnote'.
    \renewcommand{\thefootnote}{}%Removing the footnote symbol.
    %
    \footnotetext{%
    %   2010 Mathematics Subject Classification
    %   http://www.ams.org/msc/
    \textbf{2010  AMS Subject Classification.} Primary: .
    Secondary: .
    }%
        \footnotetext{%
    \textbf{Keywords and phrases.}  .
    }%
    \endgroup
%
%
%
%

\begin{abstract}
In this paper we consider the problem of finding periodic solutions of certain Euler-Lagrange equations. We employ the direct method of the calculus of variations, this is we obtain solutions minimizing certain functional $I$. We give conditions which ensure that $I$ is defined and it is  differentiable on certain subsets of  Orlicz-Sobolev spaces $W^1L^{\Phi}$ associated to a $N$-function $\Phi$. We discuss various conditions for the  coercitivity of $I$. We show that, in some sense, it is necessary for the coercitivity that  the complementary function of $\Phi$ be a $\Delta_2$ function.  We conclude by discussing conditions for existence of minima for $I$. 
\end{abstract}




\pagestyle{fancy} \headheight 35pt \fancyhead{} \fancyfoot{}

\fancyfoot[C]{\thepage} \fancyhead[CE]{\nouppercase{S. Acinas, L. Buri, G. Giubergia, F. Mazzone and E. Schwindt}} \fancyhead[CO]{\nouppercase{\section}}

\fancyhead[CO]{\nouppercase{\leftmark}}


%\tableofcontents

\section{Introduction} This paper is concerned with the existence of periodic solutions of the following problem
\begin{equation}\label{ProbPrin}
    \left\{%
\begin{array}{ll}
   \frac{d}{dt} D_{y}\mathcal{L}(t,\b{u}(t),\b{\dot{u}}(t))= D_{\b{x}}\mathcal{L}(t,\b{u}(t),\b{\dot{u}}(t)) \quad \hbox{a.e.}\ t \in (0,T)\\
    \b{u}(0)-\b{u}(T)=\b{\dot{u}}(0)-\b{\dot{u}}(T)=0.
\end{array}%
\right.
\end{equation}
where $T>0$, $\b{u}:[0,T]\to\rr^d$ is absolutely continuous and $\mathcal{L}:[0,T]\times\rr^n\times\rr^n\to\rr$ is a Caratheodory function satisfying the conditions 
\begin{eqnarray}
|\mathcal{L}(t,\b{x},\b{y})| &\leq a(|\b{x}|)\left(b(t)+ \Phi\left(\frac{|\b{y}|}{\lambda}+f(t) \right)\right),\label{cotaL}\\
|D_{\b{x}}\mathcal{L}(t,\b{x},\b{y})| &\leq a(|\b{x}|)\left(b(t)+ \Phi\left(\frac{|\b{y}|}{\lambda}+f(t) \right)\right),\label{cotaDxL}\\
|D_{\b{y}}\mathcal{L}(t,\b{x},\b{y})| &\leq a(|\b{x}|)\left(c(t)+ \varphi\left(\frac{|\b{y}|}{\lambda}+f(t)\right)  \right).\label{cotaDyL}
\end{eqnarray}
In this inequalities we assume that $\lambda>0$,  $a\in C(\mathbb{R}^+,\mathbb{R}^+)$, $\Phi$ is a $N$-function (see preliminaries section for definitions), $\varphi$ is the right continuous derivative of $\Phi$ and the non negative functions  $b,$ and $f$ belong to certain Banach spaces that  will be introduced later. 


\section{Preliminaries}

%\subsection{$N$-functions}
For reader convenience, we give a short introduction to Orlicz and Orlicz Sobolev spaces of vector valued functions and a  list  of results that we will use throughout the article. We refer to \cite{adams_sobolev,KR,wroblewska2012application} for additional details and proofs. In the first two references scalar valued functions are considered, however the generalization of the results  to vector valued functions, which is enumerated below, is direct. Last one reference considers vector valued functions.

Hereafter we denote  by $\mathbb{R}^+$  the set of all non negative real numbers. A function $\Phi:\mathbb{R}^+\to \mathbb{R}^+ $ is called an \emph{$N$-function} if $\Phi$ is given by 
\[
\Phi(t)=\int_{0}^t \varphi(\tau)\ d\tau,\quad\hbox{for } t\geq 0,
\]
where $\varphi:\mathbb{R}^+\rightarrow \mathbb{R}^+$ is a right continuous nondecreasing function  satisfying   $\varphi(0)=0$, $\varphi(t)>0$ for $t>0$ and
$\lim_{t\rightarrow \infty}\varphi(t)=+\infty$.
{\color{red} Throughout this paper, we assume that $\varphi$ is a continuous function.} 

Given a function $\varphi$ as above, we  consider the so-called right inverse function $\psi$ of $\varphi$ which is 
defined by $\psi(s)=\sup_{\varphi(t)\leq s}t$.
The function $\psi$ satisfies the same properties as the function $\varphi$, therefore we have an $N$-function $\Psi$ such that $\Psi'=\psi$ .
 The function $\Psi$ is called the \emph{complementary function} of $\Phi$.


We say that $\Phi$ satisfies the  \emph{$\Delta_2$-condition}, denoted by $\Phi \in \Delta_2$, 
if there exist  constants $K>0$ and  $t_0\geq 0$ such that $\Phi(2t)\leq K\Phi(t)$
for every $t\geq t_0$. 
If $t_0=0$,  we say that $\Phi$ satisfies the \emph{$\Delta_2$-condition globally} ($\Phi \in \Delta_2$ globally).  

 In this paper we adopt the convention that bold symbols denote points in $\mathbb{R}^d$ and plain symbols indicate scalars.

Let $d$ be a positive integer. We denote by $\mathcal{M}_d:=\mathcal{M}_d([0,T])$ the set of all measurable functions defined in $[0,T]$ with values in $\mathbb{R}^d$ and  we write $\b{u}=(u_1,\dots,u_d)$ for  $\b{u}\in \mathcal{M}_d$.


Given  an $N$-function $\Phi$ we define the \emph{modular function} 
$\rho_{\Phi}:\mathcal{M}_d\to \mathbb{R}^+\cup\{+\infty\}$ by
\[\rho_{\Phi}(\b{u}):= \int_0^T \Phi(|\b{u}|)\ dt.\]
Here $|\cdot|$ is the euclidean norm of $\mathbb{R}^d$.
The \emph{Orlicz class} $C_d^{\Phi}=C_d^{\Phi}([0,T])$  is defined by
\begin{equation}\label{claseOrlicz}
  C^{\Phi}_d:=\left\{\b{u}\in \mathcal{M}_d | \rho_{\Phi}(\b{u})< \infty \right\}.
\end{equation}
The \emph{Orlicz space} $\lphi_d=L^{\Phi}_d([0,T])$ is the linear hull of $\claseor_d$;
equivalently,
\begin{equation}\label{espacioOrlicz}
\lphi_d:=\left\{ \b{u}\in \mathcal{M}_d | \exists \lambda>0: \rho_{\Phi}(\lambda \b{u}) < \infty   \right\}.
\end{equation}
  The Orlicz space $\lphi_d$ equipped with the \emph{Orlicz norm}
\[
\|  \b{u}  \orlnor:=\sup \left\{  \int_0^T \b{u}\b{\cdot} \b{v}\ dt \big| \rho_{\Psi}(\b{v})\leq 1\right\},
\]
is a Banach space. By $\b{u}\b{\cdot} \b{v}$ we denote the usual dot product in $\mathbb{R}^{d}$ between $\b{u}$ and $\b{v}$.  
The following alternative expression for the norm, known as \emph{Amemiya norm},     will  be useful (see \cite[Thm. 10.5]{KR} and \cite{hudzik2000amemiya}). For every $\b{u}\in\lphi$,

\begin{equation}\label{amemiya}
\|\b{u}\orlnor=\inf\limits_{k>0}\frac{1}{k}\left\{1+\rho_{\Phi}(k\b{u})\right\}.
\end{equation}



The subspace $\ephi_d=\ephi_d([0,T])$ is defined as the closure in $\lphi_d$ of the subspace $L^{\infty}_d$ of all $\mathbb{R}^d$-valued essentially bounded functions. It is shown that  $\ephi_d$ is the only one maximal subspace contained in the Orlicz class $\claseor_d$, i.e. 
$\b{u}\in\ephi_d$ if and only if $\rho_{\Phi}(\lambda \b{u})<\infty$ for any $\lambda>0$.  

A generalized version of \emph{H\"older's inequality} holds in Orlicz spaces (see \cite[Thm. 9.3]{KR} ). Namely, if $\b{u}\in\lphi_d$ and $\b{v}\in\lpsi_d$ then $\b{u}\ccdot\b{v}\in L_1^1$ and
\begin{equation}\label{holder}
\int_0^T\b{v}\ccdot\b{u}\ dt\leq \|\b{u}\orlnor\|\b{v}\|_{L^{\Psi}}.
\end{equation}




If $X$ and $Y$ are  Banach spaces, with $Y\subset X^*$ we denote by $\langle\cdot,\cdot\rangle:Y\times X\to\mathbb{R}$ the bilinear pairing  map given by $\langle x^*,x\rangle=x^*(x)$. H\"older's inequality shows that $\lpsi_d\subset \left[\lphi_d\right]^*$, where the pairing  
$\langle \b{v}, \b{u}\rangle$
is defined by 
\begin{equation}\label{pairing}
  \langle \b{v},\b{u}\rangle=\int_0^T\b{v}\ccdot\b{u}\ dt
\end{equation}
with  $\b{u}\in\lphi_d$ and $\b{v}\in\lpsi_d$.
 Unless $\Phi \in \Delta_2$, the relation $\lpsi_d= \left[\lphi_d\right]^*$ will not hold. In general, it is true  that  $\left[\ephi_d\right]^*=\lpsi_d$.


Like in \cite{KR}, we will consider the subset $\Pi(\ephi_d,r)$ of $\lphi_d$ defined by
\[\Pi(\ephi_d,r):=\{\b{u}\in\lphi_d| d(\b{u},\ephi_d)<r\}.\]
This set is related to the Orlicz class $\claseor_d$ by means of inclusions, that is,
\begin{equation}\label{inclusiones}\Pi(\ephi_d, r )\subset r \claseor_d\subset\overline{\Pi(\ephi_d,r)}
\end{equation}
for any positive $r$.
The proof of this fact, and similar ones, is given by real valued function in \cite{KR},
the extension to $\mathbb{R}^d$-valued functions does not involve any difficulty. 
If $\Phi \in \Delta_2$,  then the sets $\lphi_d$, $\ephi_d$, $\Pi(\ephi_d,r)$ and $\claseor_d$ are equal.

%Frequently, we will use the following elementary fact 
%\begin{equation}\label{inclusion2}
%\b{u}\in\Pi(\ephi_d,\lambda)\implies \frac{\b{u}}{\lambda}\in\Pi(\ephi_d,1)\subset\claseor_d.
%\end{equation}

We define the \emph{Sobolev-Orlicz space} $\wphi_d$ (see \cite{adams_sobolev}) by
\[\wphi_d:=\{\b{u}| \b{u} \hbox{ is absolutely continuous and } \b{u},\b{\dot{u}}\in \lphi_d\}.\]
$\wphi_d$ is a Banach space when equipped with the norm
\[
\|  \b{u}  \|_{\wphi}= \|  \b{u}  \|_{\lphi} + \|\b{\dot{u}}\orlnor.
\]



For a  function $\b{u}\in L^1_d([0,T])$, we write $\b{u}=\overline{\b{u}}+\widetilde{\b{u}}$ where $\overline{\b{u}} =\frac1T\int_0^T \b{u}(t)\ dt$ and $\widetilde{\b{u}}=\b{u}-\overline{\b{u}}$.

As usual, if $(X,\|\cdot\|_X)$ is a Banach space and $(Y,\|\cdot \|_Y)$ is a subespace of $X$,  we write $Y\hookrightarrow X$ and we say that $Y$ is \emph{embedded} in $X$  when the restricted identity map $i_Y:Y\to X$ is bounded. That is, there exists $C>0$ such that  for any $y\in Y$ we have $\|y\|_X\leq C\|y\|_Y$.  With this notation, H\"older's inequality states that  $\lpsi_d\hookrightarrow  \left[\lphi_d\right]^*$.




 An important aspect of the theory of Sobolev spaces is related to embedding theorems. There is an extensive literature on this question in the  Orlicz-Sobolev space setting, see for example
 \cite{cianchi2000fully,cianchi1999some,claverooptimal,edmunds2000optimal,kerman2006optimal}.
The following simple lemma is well known and we will use it systematically. For the sake of completeness, we include a brief proof of it.



\begin{lem}\label{inclusion orlicz} Let $\b{u}\in\wphi_d$. Then $\b{u}\in L^{\infty}_d([0,T])$ and
\begin{align}
& \left|\b{u}(t)-\b{u}(s) \right| \leq  \|\b{\dot{u}}\orlnor | t-s|\Phi^{-1}\left(\frac{1}{|t-s|}\right)&\label{in-sob-cont}
\\
& \|\widetilde{\b{u}}\|_{L^{\infty}} \leq T\Phi^{-1}\left(\frac{1}{T}\right)\|\b{\dot u}\orlnor&\text{  (Wirtinger's inequality)}\label{wirtinger}
\\
& \|\b{u}\|_{L^{\infty}} \leq\Phi^{-1}\left(\frac{1}{T}\right)\max\{1,T\}\|\b{u}\sobnor&\text{  (Sobolev's inequality)}\label{sobolev}
\end{align}
\end{lem}
\begin{comentario}
 As usual, a function   $w:[0,+\infty)\to[0,+\infty)$ is called  a \emph{modulus of continuity} if $w$ is a continuous increasing function which satisfies $w(0)=0$. We say that $\b{u}:[0,T]\to\rr^d$  has modulus of continuity $w$  when there exists a constant $C>0$ such that 
\begin{equation}\label{w-holder}|\b{u}(t)-\b{u}(s)|\leq Cw(|t-s|).
\end{equation}
The inequality \eqref{in-sob-cont}
establishes that $w(s):=s\Phi^{-1}(1/s)$ is a modulus of continuity for all functions $\b{u}\in\wphi$. 
Since $\Phi$ is an $N$-function then $w(0)=0$; 
and, from the concavity of $\Phi^{-1}$, we have that $w$ is increasing.
\end{comentario}

\begin{proof}
For $0 \leq
s\leq t \leq T $, we get
\begin{equation}\label{equicont}
\begin{split}
\left|\b{u}(t)-\b{u}(s) \right| &\leq \int_{s}^t \left| \b{\dot{u}}(\tau)\right|\ \ d\tau\\
&\leq \| \chi_{[s,t]}\|_{\lpsi}\|\b{\dot{u}}\|_{\lphi}\\
&= \|\b{\dot{u}}\|_{\lphi} ( t-s)\Phi^{-1}\left(\frac{1}{t-s}\right),
\end{split}
\end{equation}
using \cite[Eq. (9.11)]{KR} in the last equality. This completes the proof of \eqref{in-sob-cont}.

Since $u_i$ is continuous, from Mean Value Theorem for integrals, 
there exists  $s_i\in [0,T]$ such that $u_i(s_i)=\overline{u}_i$.
Using this $s_i$ value in \eqref{in-sob-cont} with $u_i$ instead of $\b{u}$ and taking into account that $s\Phi^{-1}(1/s)$ is increasing, 
we obtain  Wirtinger's inequality for each $u_i$. The inequality \eqref{wirtinger} 
follows easily from the corresponding result for each component of $\b{u}$.

On the other hand, again by H\"older's inequality and \cite[Eq. (9.11)]{KR}, we have
\begin{equation}\label{desigualdad2}\begin{split}
|\overline{\b{u}}|= \frac{1}{T}\int\limits_{0}^{T}|\b{u}(s)|ds\leq \Phi^{-1}\left(\frac{1}{T}\right)\|\b{u}\orlnor.
\end{split}
\end{equation}
From \eqref{wirtinger}, \eqref{desigualdad2} and the fact that $\b{u}=\overline{\b{u}}+\widetilde{\b{u}}$,  we obtain \eqref{sobolev}.
\end{proof}

\begin{comentario}Inequalies \eqref{in-sob-cont} and \eqref{wirtinger}  proves  the embedding  $\wphi_d\hookrightarrow C^w([0,T],\rr^d)$, where $C^w([0,T],\rr^d)$ denotes the space of generalized $w$-H\"older continuous functions. This is the space of all functions satisfying \eqref{w-holder} for some $C>0$ and $w(s)=s\Phi^{-1}(1/s)$. This space is a Banach space with norm
\[\|\b{u}\|_{  C^w([0,T],\rr^d) }  :=\|\b{u}\|_{L^{\infty}}+\sup\limits_{t\neq s}\frac{|\b{u}(t)-\b{u}(s)|}{w(|t-s|)}.\]
As it is well known, Arzela-Ascoli Theorem implies that  the embedding $C^w([0,T],\rr^d)\hookrightarrow C([0,T],\rr^d)$ is compact (see \cite[Ch. 5]{driver} for the case $w(s)=|s|^{\alpha}$, $0< \alpha\leq 1$, the case  $w$ arbitrary follows with obvious modifications). Therefore we have the following result.
\end{comentario}

 \begin{cor}\label{unif_conv} Every bounded sequence $\{\b{u}_n\}$ in  $\wphi_d$  has an uniformly convergent subsequence. 
\end{cor}



 Given a continuous function $a\in C(\mathbb{R}^+,\mathbb{R}^+)$, we define the composition operator $\b{a}:\mathcal{M}_d\to \mathcal{M}_d$ by $\b{a}(\b{u})(t)= a(|\b{u}(t)|)$.
We will often use the following elementary consequence of the previous lemma. 
\begin{cor}\label{a_bound} If $a\in C(\mathbb{R}^+,\mathbb{R}^+)$ then $\b{a}:\wphi_d\to L^{\infty}_1([0,T])$ is bounded. 
More concretely,  there exists a non decreasing function $A:\mathbb{R}^+\to\mathbb{R}^+$ such that
 $\|\b{a}(\b{u})\|_{L^{\infty}([0,T])}\leq A(\|\b{u}\|_{\wphi})$.
\end{cor}

\begin{proof}  Let $\alpha\in C(\mathbb{R}^+,\mathbb{R}^+)$ be a  non-decreasing majorant of $a$, for example 
$\alpha(s):=\sup_{0\leq t\leq s}a(t)$.  If $\b{u}\in \wphi_d$ then, by Lemma \ref{inclusion orlicz}, 
\[a(|\b{u}(t)|)\leq \alpha(\|\b{u}\|_{L^{\infty}})\leq 
\alpha\left(\Phi^{-1}\left(\frac{1}{T}\right)\max\{1,T\} \|\b{u}\|_{\wphi}\right)=: 
A(\|\b{u}\|_{\wphi}).\]
\end{proof}


The following lemma is an inmediate consequence of principles  related to  operators of Nemitskii type, see \cite[�17]{KR}.

\begin{lem}\label{phi_comp}   
The  composition operator  $\boldsymbol{\varphi}$  acts from $\Pi(\ephi_d,1)$ into $C_1^{\Psi}$.
\end{lem}
\begin{proof}
  As a consequence of \cite[Lemma 9.1]{KR} we have that  $\boldsymbol{\varphi}\left(B_{\lphi}(0,1)\right)\subset C_1^{\Psi}$, where
$B_{X}(\b{u}_0,r)$ is the open ball with center $\b{u}_0$ and radius $r>0$ in the space $X$. Therefore, applying \cite[Lemma 17.1]{KR}, we deduce that $\boldsymbol{\varphi}$ acts from $\Pi(\ephi_d,1)$ into $C_1^{\Psi}$.
\end{proof}

We also need the following technical lemma.
\begin{lem}\label{segundo lema}
Let $\lambda>0$ and let $\{\b{u}_n\}_{n\in \mathbb{N}}$ be a sequence of  functions in $\Pi(\ephi_d,\lambda)$ converging to  $\b{u}\in \Pi(\ephi_d,\lambda)$  in the $\lphi$-norm. Then, there exist a subsequence
$\b{u}_{n_k}$ and a real valued function $h\in\Pi\left(\ephi_1\left([0,T]\right),\lambda\right)$ such that $\b{u}_{n_k}\rightarrow \b{u} \quad\text{a.e.}$ and $|\b{u}_{n_k}|\leq h\quad\text{a.e.}$
\end{lem}



\begin{proof}
Let $r:=d(\b{u},\ephi_d)$, $r<\lambda$. As $\b{u}_n$ converges to $\b{u}$, there exists a subsequence $(n_k)$ such that
\[\|\b{u}_{n_k}-\b{u}\orlnor<\frac{\lambda-r}{2}\quad \text{ and }\quad \|\b{u}_{n_k}-\b{u}_{n_{k+1}}\orlnor<2^{-(k+1)}(\lambda-r).\]
Let $h:[0,T]\rightarrow\mathbb{R}$ defined by
\begin{equation}\label{serie} h(x)=|\b{u}_{n_1}(x)|+\sum_{k=2}^{\infty}|\b{u}_{n_k}(x)-\b{u}_{n_{k-1}}(x)|.
\end{equation}
As a consequence  of \cite[Lemma 10.1]{KR} we have that $d(\b{v},\ephi_d)=d(|\b{v}|,\ephi_1)$ for any $\b{v}\in\lphi_d$. 
Now
\[d(|\b{u}_{n_1}|,\ephi_1)= d(\b{u}_{n_1},\ephi_d)\leq d(\b{u}_{n_1},\b{u})+d(\b{u},\ephi_d)<\frac{\lambda+r}{2}.\]
Then
\[d(h,\ephi_1)\leq d(h,|\b{u}_{n_1}|)+d(|\b{u}_{n_1}|,\ephi_1)< \lambda.\]
Therefore, $h\in\Pi(\ephi_1,\lambda)$ and  $|h|<\infty$ a.e. 
We conclude that the series \linebreak $\b{u}_{n_1}(x)+\sum_{k=2}^{\infty}(\b{u}_{n_k}(x)-\b{u}_{n_{k-1}}(x))$
is absolutely convergent a.e. and this fact implies that $\b{u}_{n_k}\rightarrow \b{u} \quad\text{a.e.}$ 
The inequality $|\b{u}_{n_k}|\leq h$ follows straightforwardly from the definition of $h$.
\end{proof}

A common obstacle with Orlicz spaces, that distinguishes it from $L^p$ spaces, is that a  sequence $\b{u}_n\in\lphi_d$ which is  uniformly bounded by $ h\in\lphi_1$ and a.e. convergent to $\b{u}$ is not necessarily norm convergent.
Fortunately, the subspace $\ephi_d$ has that property. 

\begin{lem}\label{lema_conv_may}
Suppose that $\b{u}_n \in\lphi_d$ is a sequence such that $\b{u}_n\to \b{u}$ a.e. and assume that there exist $h\in\ephi_1$ with $|\b{u}_n|\leq h$ a.e. 
then $\|\b{u}_n-\b{u}\orlnor\to 0$.
\end{lem}
\begin{proof}\cite[p. 84]{rao1991theory} and \cite[Thm. 10.3]{KR}.
\end{proof}


  We recall some useful concepts.

	\begin{defi} 
	Given a function $I:U\to\mathbb{R}$ where $U$ is an open set of a Banach space $X$,
we say that $I$ has a G\^ateaux derivative at $\b{u} \in U$ if there exists $\b{u}^*\in X^*$ such that for every $\b{v} \in X$
\[
\lim_{s \rightarrow 0}\frac{I(\b{u}+s\b{v})-I(\b{u}) }{s}=\langle \b{u}^* , \b{v}\rangle.
\]
See \cite{ambrosetti} for details. 
\end{defi}

%We recall the following definition. 
\begin{defi} Let $X$ be a Banach space and let $D\subset X$. A non linear operator $T:D\to X^*$ is called \emph{demicontinuous} if it is continuous when $X$ is equipped with the strong topology and $X^*$ with the weak${}^*$ topology 
(see \cite{kato1964demicontinuity}).
\end{defi} 

\section{Differentiability of action integrals in Orlicz spaces}

\begin{defi} We say that a function $\mathcal{L}:[0,T]\times \mathbb{R}^d \times \mathbb{R}^d \rightarrow \mathbb{R}$ is a Carath\'eodory function if for fixed $(\b{x},\b{y})$
the map $t \mapsto \mathcal{L}(t, \b{x},\b{y})$ is measurable  and for fixed $t$ the map  $(\b{x},\b{y}) \mapsto \mathcal{L}(t, \b{x}, \b{y})$ is continuously differentiable for almost everywhere $t\in [0,T]$.

\end{defi}





\begin{comentario}These conditions are a direct  generalization of the conditions
\cite[Eq (a), p. 10]{mawhin2010critical}. 
\end{comentario}

In the following comments we discuss the relevance of the function $f$ in the inequalities \eqref{cotaL},  \eqref{cotaDxL} and \eqref{cotaDyL}. In particular, we are interested in seeing when for every  $f\in \ephi_1$ there exist
$b\in L^1_1$ and $C>0$ such that 
\begin{equation}\label{cotadb}
\Phi(s+f(t))\leq C\Phi(s)+b(t)\;\;\mbox{for every}\;s>0.
\end{equation} 
In this case we can suppose $f=0$ in \eqref{cotaL} and \eqref{cotaDxL}. The same considerations should be done with $\varphi\left(s+f(t)\right)$.

\begin{comentario}  As a direct consequence of convexity, we can  bound the term $\Phi(s+f(t))$ by the expression  $\frac12\Phi(2s)+b(t)$ where $b(t):=\tfrac12\Phi(2f(t))\in L^1_1$ and $f\in \ephi_1$. Therefore, we can always assume $f = 0$ in \eqref{cotaL} and \eqref{cotaDxL} at the price of making smaller the value of $\lambda$. 
\end{comentario}

  


\begin{comentario} If $\Phi\in\Delta_2$ then we can asssume $f=0$ keeping the same value of $\lambda$. This is a consequence of that a non decreasing $\Delta_2$ function $G:\mathbb{R}^+\to\mathbb{R}^+$ is quasi-subadditive (see \cite[Prop. 4.2]{AF12}) i.e. there exists $K>0$ such that 
\[G(s_1+s_2)\leq K\left(G(s_1)+G(s_2)\right).\]
%%
Moreover, if $\Phi \in \Delta_2$  then $\varphi \in \Delta_2$ (see \cite[Eq. (2.15)]{FZ01}).
Therefore if $\Phi \in \Delta_2$ we have that
\[\Phi\left(s+f(t) \right)\leq K\Phi(s)+
K\Phi\left(f(t) \right)=K\Phi(s)+b_1(t),\]
where $b_1(t)=b(t)+ K\Phi\left(f(t) \right)\in L^1_1([0,T])$. A similar fact holds with $\varphi$ instead of $\Phi$, namely
\[ \varphi\left(s+f(t)\right) \leq c_1(t)+ \varphi\left(s\right),\]
where $c_1(t):=K\varphi\left(f(t)\right)\in \lpsi_1$, as  a consequence of Lemma \ref{phi_comp} and the \linebreak $\Delta_2$-condition on $\Phi$. 
\end{comentario}

\begin{comentario} If $\Phi\notin\Delta_2$, then \eqref{cotadb} may or may not be true.  For example, if we consider the $N$-function $\Phi(s)=e^s-s-1$ which does not satisfy the $\Delta_2$-condition, we have that $\ephi_1=L_1^{\infty}$. In fact, if $f\in \ephi_1$ then,
since $1/2e^s\leq \Phi(s)+1$ and  $pf\in\claseor_1$ for every $p>0$, we get $\int_0^Te^{pf(t)}dt<\infty$ for every $p>0$. This  implies that $f\in L_1^{\infty}$.  Therefore,
\[\Phi(s+f(t))\leq e^{s+f(t)}\leq 2 \|e^f\|_{L^{\infty}}\Phi(s)+2 \|e^f\|_{L^{\infty}}.\] 

On the other hand, we consider  the $N$-function $\Phi(s)=e^{s^2}-1$.  Suppose that $0\leq f\in\ephi_1$ and $b\in L_1^1$ satisfy  \eqref{cotadb}, then
\[e^{s^2}e^{2sf(t)}\leq \Phi(s+f(t))+1\leq Ce^{s^2}+b(t).\]
Dividing by $e^{s^2}$ and taking the limit as $s\to\infty$, we obtain that  $f=0$ a.e. In other words, if $f\neq 0$ on a set with positive measure, the estimations in \eqref{cotaL} and \eqref{cotaDxL} are essentially bigger than a bound of the type  $a(|\b{x}|)\left(b(t)+ \Phi\left(|\b{y}|/\lambda \right)\right)$.
\end{comentario}


%
%
%
%
%
%
%}

\begin{thm}\label{teorema_acotacion}
Let $\mathcal{L}$ be a Carath\'eodory function satisfying \eqref{cotaL}, \eqref{cotaDxL} and \eqref{cotaDyL}. 
Then the following statements hold:
\begin{enumerate}
\item \label{T1item1} \label{A1} The \emph{action integral}  
\begin{equation}\label{integral_accion}
I(\b{u}):=\int_{0}^T \mathcal{L}(t,\b{u}(t),\b{\dot{u}}(t))\ dt
\end{equation}
is finitely defined in $\domi:=W^{1}\lphi_d\cap\{\b{u}|\b{\dot{u}}\in\Pi(\ephi_d,\lambda)\}$.

\item\label{T1item3} The function  $I$ is G\^ateaux differentiable on $\domi$ and  its derivative $I'$ is demicontinuous from $\domi$  into $\left[\wphi_d \right]^*$. Moreover, $I'$ is given by the following expression
\begin{equation}\label{DerAccion}
\langle  I'(\b{u}),\b{v}\rangle= \int_0^T \left\{D_{\b{x}}\mathcal{L}\big(t,\b{u},\b{\dot{u}}\big)\ccdot \b{v}+ D_{\b{y}}\mathcal{L}\big(t,\b{u},\b{\dot{u}}\big)\ccdot\b{\dot{v}}\right\} \ dt.
\end{equation}

\item\label{T1item4}  If  $\Psi \in \Delta_2$ then 
  $I'$ is continuous from $\domi$ into $\left[\wphi_d\right]^*$ when both spaces are equipped with the strong topology.


\end{enumerate}
\end{thm}
\begin{proof}
 Since  $\lambda\Pi(\ephi_d,r)=\Pi(\ephi_d,\lambda r)$, we have   $\b{\dot{u}}/\lambda\in\Pi(\ephi_d,1)$. 
Thus, as $f\in\ephi_1$ and attending to \eqref{inclusiones}, we get 

\begin{equation}\label{inclusion3}
|\b{\dot{u}}|/\lambda+f\in\Pi(\ephi_1,1)\subset \claseor_1.
\end{equation}
By Corollary \ref{a_bound}, we get 
 \[|\mathcal{L}(\cdot,\b{u},\b{\dot{u}})| \leq A(\|\b{u}\sobnor ) \left(b+ \Phi\left (\frac{|\b{\dot{u}}|}{\lambda}+f\right)  \right)\in
 L^1_1.\]
This fact proves item \ref{T1item1}.

 We split the proof of  \ref{T1item3} in three steps.

\noindent\emph{Step 1. The non linear operator  $\b{u} \mapsto D_{\b{x}}\mathcal{L}(t,\b{u},\b{\dot{u}})$ is continuous from $\domi$ into $L^{1}_d([0,T])$ with the strong topology on both sets.} 


If $\b{u}\in \domi$, from \eqref{cotaDxL} and \eqref{inclusion3}, we obtain 
\begin{equation}\label{DxL1}
|D_{\b{x}}\mathcal{L}(\cdot,\b{u},\b{\dot{u}})|\leq A(\|u\sobnor) \left(b+\Phi\left(\frac{|\b{\dot{u}}|}{\lambda}+f\right)\right) \in L^1_1.
\end{equation}


Let   $\{\b{u}_n\}_{n\in \mathbb{N}}$ be a sequence of  functions in $\domi$, and $\b{u}\in \domi$ such that $\b{u}_n\rightarrow \b{u}$ in $\wphi_d$.
Then $\b{u}_n\rightarrow \b{u}$ in $\lphi_d$ and $\b{\dot{u}}_n\rightarrow \b{\dot{u}}$ in $\lphi_d$. By Lemma \ref{segundo lema} there exist a subsequence $\b{u}_{n_k}$ and a function  $h\in \Pi(\ephi_1,\lambda))$
such that $\b{u}_{n_k}\rightarrow \b{u} \quad\text{a.e.}$, $\b{\dot{u}}_{n_k}\rightarrow \b{\dot{u}} \quad\text{a.e.}$ and $|\b{\dot{u}}_{n_k}|\leq h\quad\text{a.e}$.  Since $\b{u}_{n_k}$, $k=1,2,\ldots,$ is a strong convergent sequence in $\wphi_d$, it is a bounded sequence in $\wphi_d$. According to Lemma \ref{inclusion orlicz} and Corollary \ref{a_bound}, there exists $M>0$ such that $\|\b{a}(\b{u}_{n_k})\|_{L^{\infty}} \leq M$, $k=1,2,\ldots$.  From the previous facts and \eqref{DxL1}, we get
\begin{equation*}\label{DxL1-bis}
|D_{\b{x}}\mathcal{L}(\cdot,\b{u}_{n_k},\b{\dot{u}}_{n_k})|\leq M\left(b+\Phi\left(\frac{|h|}{\lambda}+f\right)\right) \in L^1_1.
\end{equation*}
On the other hand, by the Carath\'eodory condition, we have
\[D_{\b{x}}\mathcal{L}(t,\b{u}_{n_k}(t),\b{\dot{u}}_{n_k}(t))\to D_{\b{x}}\mathcal{L}(t,\b{u}(t),\b{\dot{u}}(t))\quad\hbox{ for a.e. } t\in[0,T].\]
Applying the Dominated Convergence Theorem we conclude the proof of step 1.

\noindent\emph{Step 2. The non linear operator   $\b{u}
 \mapsto  D_{y}\mathcal{L}(t,\b{u},\b{\dot{u}})$ is continuous from $\domi$ with the strong topology  into $\left[\lphi_d\right]^*$  with the weak$^*$ topology.}

 Let $\b{u}\in \domi$.  From  \eqref{inclusion3}, Lemma \ref{phi_comp} and Corollary \ref{a_bound}, it follows that 
\begin{equation}\label{AcotOperphi}
\varphi\left(\frac{|\b{\dot{u}}|}{\lambda}+f\right)\in C^{\Psi}_1
\end{equation}
and $\b{a}(|\b{u}|)\in L^{\infty}_1$. 
Therefore, in virtue of  \eqref{cotaDyL} we get
\begin{equation}\label{DyLpsi}
   \left|D_{\b{y}}\mathcal{L}(\cdot,\b{u},\b{\dot{u}})\right|\leq  A(\|\b{u}\|_{\wphi} )  \left(c+\varphi\left( \frac{|\b{\dot{u}}|}{\lambda}+f\right  ) \right)\in\lpsi_1.
\end{equation}
 We note that \eqref{DxL1},  \eqref{DyLpsi}, the imbeddings $\wphi_d \hookrightarrow L_d^{\infty}$ and  $\lpsi_d\hookrightarrow  \left[\lphi_d\right]^*$ imply that the second member of
\eqref{DerAccion} defines an element in $\left[\wphi_d\right]^*$.

Let $\b{u}_n,\b{u}\in \domi$ such that $\b{u}_n\to \b{u}$ in the norm of $\wphi_d$. 
We must prove that  $D_{\b{y}}\mathcal{L}(\cdot,\b{u}_n,\dot{\b{u}_n})\overset{w^*}{\rightharpoonup} D_{\b{y}}\mathcal{L}(\cdot,\b{u},\b{\dot{u}})$. On the contrary, if there exist $\b{v}\in\lphi_d$, $\epsilon>0$ and a subsequence of $\{\b{u}_n\}$ (denoted  $\{\b{u}_n\}$ for simplicity)  such that
\begin{equation}\label{cota_eps}
 \left| \langle D_{\b{y}}\mathcal{L}(\cdot,\b{u}_n,\b{\dot{u}_n}),\b{v} \rangle - \langle  D_{\b{y}}\mathcal{L}(\cdot,\b{u},\b{\dot{u}}),\b{v} \rangle\right|\geq \epsilon.
\end{equation}
We have $\b{u}_n\rightarrow \b{u}$ in $\lphi_d$ and
$\b{\dot{u}}_n\rightarrow \b{\dot{u}}$ in $\lphi_d$. By Lemma \ref{segundo lema}, there exist a subsequence $\b{u}_{n_k}$ and a function $h\in \Pi(\ephi_1,\lambda)$ such that $\b{u}_{n_k}\rightarrow \b{u} \quad\text{a.e.}$, $\b{\dot{u}}_{n_k}\rightarrow \b{\dot{u}} \quad\text{a.e.}$ and $|\b{\dot{u}}_{n_k}|\leq h\quad\text{a.e.}$ 
As in the previous step, since $\b{u}_n$ is a convergent sequence, the Corollary \ref{a_bound} implies that $a(|\b{u}_n(t)|)$ is uniformly bounded by a certain constant $M>0$. 
Therefore,  with $\b{u}_n$ instead of $\b{u}$, inequality  \eqref{DyLpsi} becomes  
%\eqref{cotaDyL},  \eqref{AcotOperphi}, the fact that $c\in\lpsi_1$ and H\"older's inequality, we obtain
\begin{equation}\label{Dy-suc}
  \left | D_{\b{y}}\mathcal{L}(\cdot,\b{u}_n,\b{\dot{u}}_n)  \right| 
	\leq M\left(c+\varphi\left(\frac{h}{\lambda}+f\right)\right)\in \lpsi_1.
\end{equation}

{\color{red} La siguiente frase hab\'ia quedado incompleta}

Consequently, as $v \in \lphi_d$ and employing H\"older's inequality, we obtain that 
 $\sup_n|D_{\b{y}}\mathcal{L}(\cdot,\b{u}_n,\b{\dot{u}}_n)\ccdot v| \in L^1_1$.  Finally, from the Lebesgue Dominated Convergence Theorem, we deduce
\begin{equation}\label{conv_debil}\int_0^T  D_{\b{y}}\mathcal{L}(t,\b{u}_{n_k},\b{\dot{u}}_{n_k})\ccdot\b{ v} \ dt \to \int_0^T D_{\b{y}}\mathcal{L}(t,\b{u},\b{\dot{u}})\ccdot\b{ v}\ dt \end{equation}
which contradicts the inequality \eqref{cota_eps}. This completes the proof of step 2.

\emph{Step 3.} Finally we prove \ref{T1item3}. The proof follows similar lines that \cite[Thm. 1.4]{mawhin2010critical}. For $\b{u}\in \domi$ and $0\neq\b{v}\in\wphi_d$, we define the function
\[H(s,t):=\mathcal{L}(t,\b{u}(t)+s\b{v}(t),\b{\dot{u}}(t)+s\b{\dot{v}}(t)).\]

From \cite[Thm. 10.1]{KR} we obtain that if $|\b{u}|\leq |\b{v}|$ then    $d(\b{u},\ephi_d)\leq d(\b{v},\ephi_d)$. 
Therefore, for  $|s|\leq s_0:=\left(\lambda-d(\b{\dot{u}},\ephi_d)\right)/\|\b{v}\sobnor$ we have
\[
d \left(\b{\dot{u}}+s\b{\dot{v}}, \ephi_d \right)\leq
d \left(|\b{\dot{u}}|+s|\b{\dot{v}}|, \ephi_1 \right)
\leq d \left(|\b{\dot{u}}|,\ephi_1 \right)+ s \|\b{\dot{v}}\orlnor < \lambda.
\]
As a consequence $\b{\dot{u}}+s\b{\dot{v}} \in \Pi(\ephi_d,\lambda)$ and  $|\b{\dot{u}}|+s|\b{\dot{v}}| \in \Pi(\ephi_1,\lambda)$. These facts imply, in virtue of Theorem \ref{teorema_acotacion} item \ref{T1item1}, that $I(\b{u}+s\b{v})$ is well defined and finite for $|s|\leq s_0$. 
Using  Corollary \ref{a_bound} we see that
\[ \|a(|\b{u}+s\b{v}|)\|_{L^{\infty}}\leq  A(\|\b{u}+s\b{v}\sobnor)\leq
 A(\|\b{u}\sobnor+s_0\|\b{v}\sobnor)=:M
\]
Now, applying Chain Rule, \eqref{DxL1}, \eqref{DyLpsi} the monotonicity of $\varphi$ and $\Phi$, 
the fact that $\b{v}\in L^{\infty}_d$ and $\b{\dot{v}}\in\lphi_d$ and H\"older's inequality, we have
\begin{equation}\label{ctg}
\begin{split}
|D_s H(s,t)|&=\left| D_{\b{x}}\mathcal{L}(t,\b{u}+s\b{v},\b{\dot{u}}+s\b{\dot{v}})\ccdot \b{v} +  D_{\b{y}}\mathcal{L}(t,\b{u}+s\b{v},\b{\dot{u}}+s\b{\dot{v}})\ccdot\b{\dot{v}}\right| \\
 & \leq M \left[\left( b(t)+ \Phi\left(\frac{|\b{\dot{u}}|+s_0|\b{\dot{v}}|}{\lambda}+f(t)\right)\right)|\b{v}|\right.\\
&\left. \quad+ \left(c(t)+ \varphi\left (\frac{|\b{\dot{u}}|+s_0|\b{\dot{v}}|}{\lambda}+f(t)\right)\right)|\b{\dot{v}}| \right]\in L^1_1.
\end{split}
\end{equation}
Consequently, $I$ has a directional derivative and
\[
\langle I'(\b{u}),\b{v} \rangle=\frac{d}{ds}I(\b{u}+s\b{v})\big|_{s=0}=\int_0^T  
\left\{D_{\b{x}}\mathcal{L}(t,\b{u},\b{\dot{u}})\ccdot \b{v}+ D_{\b{y}}\mathcal{L}(t,\b{u},\b{\dot{u}})\ccdot\b{\dot{v}}\right\} \ dt.
\]
Moreover, from \eqref{DxL1}, \eqref{DyLpsi}, Lemma \ref{inclusion orlicz} and previous formula, we have
\[
|\langle I'(\b{u}),\b{v} \rangle| \leq \|D_{\b{x}}\mathcal{L}\|_{L^1} \| \b{v}\linf + 
\|D_{\b{y}}\mathcal{L}\|_{\lpsi} \|\b{\dot{v}}\orlnor \leq C \|\b{v}\sobnor
\]
with a appropriate constant $C$.
This completes the proof of the G\^ateaux differentiability of $I$. 

Finally, the demicontinuity of $I':\domi  \to \left[\wphi_d
\right]^* $ is a consequence of the continuity of the mappings $\b{u} \mapsto D_{\b{x}}\mathcal{L}(t,\b{u},\b{\dot{u}})$ and $\b{u} \mapsto
D_{\b{y}}\mathcal{L}(t,\b{u},\b{\dot{u}})$. Indeed, if $\b{u}_n,\b{u}\in \domi$ with $\b{u}_n\to \b{u}$ in the norm of $\wphi_d$ and $\b{v} \in
\wphi_d$, then
\[
\begin{split}
\left\langle  I'(\b{u}_{n}),\b{v} \right\rangle &= \int_0^T \left\{  D_{\b{x}}\mathcal{L}\left(t,\b{u}_n,\b{\dot{u}}_n\right)\ccdot
\b{v} +
 D_{\b{y}}\mathcal{L}\left(t,\b{u}_n,\b{\dot{u}}_n\right)\ccdot\b{\dot{v}}\right\} \ dt\\
&\rightarrow \int_0^T \left\{ D_{\b{x}}\mathcal{L}\left(t,\b{u},\b{\dot{u}}\right)\ccdot \b{v}+ 
D_{\b{y}}\mathcal{L}\left(t,\b{u},\b{\dot{u}}\right)\ccdot\b{\dot{v}}\right\} \ dt\\
&=\left\langle  I'(\b{u}),\b{v} \right\rangle.
\end{split}
\]


In order to prove  \ref{T1item4}, it is necessary to see that the maps $\b{u}\mapsto D_{\b{x}}\mathcal{L}(\cdot,\b{u}(\cdot),\b{\dot{u}}(\cdot))$  and $\b{u}\mapsto D_{\b{y}}\mathcal{L}(\cdot,\b{u}(\cdot),\b{\dot{u}}(\cdot))$  are norm continuous
from $\domi $ into $L^1_d$ and
 $\lpsi_d$ respectively.  The continuity of the first map has already been proved in step 1. 
We will prove the continuity of the second map developing a similar argument to the one given in step 2 of item 2.  
We consider $\b{u}_n$ and $\b{u}$ in $\domi$ with $\|\b{u}_n- \b{u}\sobnor\to 0$.
By Lemma \ref{segundo lema}, there exist a subsequence $\b{u}_{n_k}\in \domi$ and a function $h\in\Pi(\ephi_1,\lambda)$ such that $\b{u}_{n_k}\rightarrow \b{u} \quad\text{a.e.}$, $\b{\dot{u}}_{n_k}\rightarrow \b{\dot{u}} \quad\text{a.e.}$ and $|\b{\dot{u}}_{n_k}|\leq h\quad\text{a.e.}$ Then,  since $\mathcal{L}$ is a Carath\'eodory function
 we have $ D_{\b{y}}\mathcal{L}(t,\b{u}_{n_k}(t),\b{\dot{u}}_{n_k}(t))\to D_{\b{y}}\mathcal{L}(t,\b{u}(t),\b{\dot{u}}(t))$ a.e. $t\in [0,T]$. By \eqref{cotaDyL} and the fact that $\Psi\in \Delta_2$, we obtain

{\color{red} Llegamos hasta ac\'a!!! Tenemos que corregir/resumir  citando \eqref{Dy-suc}}

 \[\begin{split}
    |D_{\b{y}}\mathcal{L}(t,\b{u}_{n_k}(t),\b{\dot{u}}_{n_k}(t))| &\leq a(|\b{u}_{n_k}(t)|)\left( c(t) + \varphi \left(\frac{|\b{\dot{u}}_{n_k}(t)|}{\lambda}+f(t)\right)\right)\\
    &\leq C\left( c(t) + \varphi \left(\frac{|h(t)|}{\lambda}+f(t)\right)\right)\in \lpsi_1=\epsi_1.
   \end{split}
\]
Therefore, invoking  Lemma \ref{lema_conv_may}, we have proved that
  from any sequence $\b{u}_n$ which converges to $\b{u}$ in $\wphi_d$ we can
extract a subsequence such that  \linebreak $D_{\b{y}}\mathcal{L}(t,\b{u}_{n_k},\b{\dot{u}}_{n_k})\to D_{\b{y}}\mathcal{L}(t,\b{u},\b{\dot{u}})$ in the strong topology. The desired result is obtained by a standard argument.

The continuity of $I'$  follows  from the continuity 
of $D_{\b{x}}\mathcal{L}$ and $D_{\b{y}}\mathcal{L}$ by using the formula \eqref{DerAccion}.
\end{proof}



\section{Critical points and Euler-Lagrange equations}


In this section we derive the Euler-Lagrange equations associated to critical points of action integrals.  We denote by $\wphi_T$ the subspace of $\wphi_d$ containing all \linebreak $T$-periodic functions. Similarly we consider the subspaces $\ephi_T$, $\lphi_T$. As usual, when $Y$ is a subspace of
the Banach space $X$, we denote by $Y^{\perp}$ the \emph{annihilator subspace} of $X^*$, i.e. the subspace
that consists of all  bounded linear functions which are identically zero on $Y$.

We recall that  a function $f: \mathbb{R}^d \to \mathbb{R}$ is called \emph{strictly convex} if 
\linebreak
$f\left(\tfrac{\b{x}+\b{y}}{2}\right)< \tfrac{1}{2} \left(f\left(
\b{x}\right)+f\left( \b{y}\right)\right)$ for  $\b{x}\neq\b{y}$.  It is  well known that if $f$ is a strictly convex and differentiable function, then
$D_{\b{x}}f:\mathbb{R}^d\to\mathbb{R}^d$ is a one-to-one map  (see, for instance \cite[Thm. 12.17]{rockafellar2009variational}).


\begin{thm} Let $\b{u}\in\domi$. The following statements are equivalent:
\begin{enumerate}
 \item $I'(\b{u})\in\left( \wphi_T\right)^{\perp}$.
 \item  $D_{\b{y}}\mathcal{L}(t,\b{u}(t),\b{\dot{u}}(t))$ is an absolutely continuous function and $\b{u}$ solves the following boundary value problem
 \begin{equation}\label{ecualagran2}
    \left\{%
\begin{array}{ll}
   \frac{d}{dt} D_{y}\mathcal{L}(t,\b{u}(t),\b{\dot{u}}(t))= D_{\b{x}}\mathcal{L}(t,\b{u}(t),\b{\dot{u}}(t)) \quad \hbox{a.e.}\ t \in (0,T)\\
    \b{u}(0)-\b{u}(T)=D_{\b{y}}\mathcal{L}(0,\b{u}(0),\b{\dot{u}}(0))-D_{\b{y}}\mathcal{L}(T,\b{u}(T),\b{\dot{u}}(T))=0.
\end{array}%
\right.
\end{equation}
\end{enumerate}
Moreover if $D_{\b{y}}\mathcal{L}(t,x,y)$ is $T$-periodic with respect to the variable $t$ and strictly convex with respect to $\b{y}$, then
$D_{\b{y}}\mathcal{L}(0,\b{u}(0),\b{\b{\dot{\b{u}}}}(0))-D_{\b{y}}\mathcal{L}(T,\b{u}(T),\b{\dot{u}}(T))=0$ is equivalent to $\b{\dot{u}}(0)=\b{\dot{u}}(T)$.
\end{thm}

\begin{proof} The condition  $I'(\b{u})\in\left( \wphi_T\right)^{\perp}$ and \eqref{DerAccion} imply 
\[\int_0^T  D_{\b{y}} \mathcal{L}(t,\b{u}(t),\b{\dot{u}}(t))\ccdot \b{\dot{v}}(t)\ dt
=-\int_0^T  D_{\b{x}}\mathcal{L}(t,\b{u}(t),\b{\dot{u}}(t)) \ccdot\b{ v}(t)\ dt. \]
Using \cite[pp. 6-7]{mawhin2010critical} we obtain that  $D_{\b{y}}\mathcal{L}(t,\b{u}(t),\b{\dot{u}}(t))$ is absolutely continuous and 
\linebreak
$T$-periodic, therefore it is differentiable a.e. on $[0,T]$ and the first equality of \eqref{ecualagran2} holds true.
This completes the proof of  1 implies 2. The proof of 2 implies  1  follows easily 
from \eqref{DerAccion}  and \eqref{ecualagran2}.

The last part of the theorem is a consequence of that \linebreak
$D_{\b{y}}\mathcal{L}(T,\b{u}(T),\b{\dot{u}}(T))=D_{\b{y}}\mathcal{L}(0,\b{u}(0),\b{\dot{u}}(0))=D_{\b{y}}\mathcal{L}(T,u(T),\b{\dot{u}}(0))$ and the injectivity of $D_{\b{y}}\mathcal{L}(T,u(T),\cdot)$.
\end{proof}


\section{Coercivity discussion}

We recall the following usual definition in the context  of calculus of variations. 

\begin{defi} Let $X$ be a Banach space and let $D$ be an unbounded subset of $X$. Suppose $J:D\subset X\to\rr$. We say that $J$ is \emph{coercive} if $J(u)\to +\infty$ when \linebreak $\|\b{u}\|_X\to +\infty$. 
\end{defi}

It is well known that coercivity is a useful ingredient in order to establish existence of minima. Therefore, we are interested in finding conditions which ensure the coercivity of the action integral $I$ acting on $\domi$. For this purpose, we need to introduce the following  extra condition on lagrangian function $\mathcal{L}$  
\begin{equation}\label{cota_inf}
\mathcal{L}(t,\b{x},\b{y})\geq \alpha_0\Phi\left(\frac{|\b{y}|}{\Lambda}\right)+ F(t,\b{x}),
\end{equation}
where $\alpha_0,\Lambda>0$ and  $F:\rr\times\rr^d\to\rr$ is a Carath\'eodory function, i.e. $F(t,\b{x})$ is  measurable with respect to $t$ for every fixed  $\b{x}\in\rr^d$ and it is continuous at $\b{x}$ for a.e. $t\in [0,T]$. We note that, in virtue of \eqref{cota_inf} and \eqref{cotaL}, $F(t,\b{x})\leq a(|\b{x}|)b_0(t)$  with $b_0(t):=b(t)+\Phi(f(t))\in L^1_1([0,T])$. In order to ensure that integral $\int_0^TF(t,\b{u})dt$ is finite for $\b{u}\in\wphi$,  we need to assume 
\begin{equation}\label{condA1} |F(t,\b{x})|\leq a(|\b{x}|)b_0(t),\quad\text{for \,a.e. }t\in [0,T] \quad\text{and for every } \b{x}\in\rr^d.
\end{equation}
As we shall see in Theorem \ref{coercitividad1}, when $\mathcal{L}$ satisfies \eqref{cotaL}, \eqref{cotaDxL}, \eqref{cotaDyL}, \eqref{cota_inf} and \eqref{condA1},  the coercivity of the action integral $I$ is related to the coercivity of the functional
\begin{equation}\label{func_phi}
  J_{C,\nu}(\b{u}):= \rho_{\Phi}\left(\frac{\b{u}}{\Lambda}\right)-C\|\b{u}\orlnor^{\nu},
\end{equation}
for $C,\nu>0$. If $\Phi(x)=|x|^p/p$ then $J_{C,\nu}$ is clearly coercive for $\nu<p$. For more general $\Phi$ the situation is more interesting   as it will be shown in the following lemma.

\begin{lem}\label{lem_coer} Let $\Phi$ and $\Psi$ be complementary $N$-functions. Then:
\begin{enumerate}
  \item If $C\Lambda<1$ then $J_{C,1}$ is coercive. 
  
  \item If $\Psi \in \Delta_2$ globally, then there exists a constant $\alpha_{\Phi}>1$ such that, for any $0<\mu<\alpha_{\Phi}$,
\begin{equation}\label{coer_modular} \lim\limits_{\|\b{u}\orlnor \to \infty} \frac{\rho_{\Phi}\left(\frac{\b{u}}{\Lambda}\right)}{\|\b{u}\orlnor^{\mu}}=+\infty.
\end{equation}
In particular, the functional $J_{C,\mu}$ is coercive for every $C>0$ and  $0<\mu<a_{\Phi}$. The constant $\alpha_{\Phi}$ is one of the so-called \emph{ Matuszewska-Orlicz indices} (see \cite[Ch. 11]{M}).
\item If $J_{C,1}$ is coercive with $C\Lambda>1$, then $\Psi \in \Delta_2$.  
\end{enumerate}
\end{lem}

\begin{proof} By \eqref{amemiya} we have
\[(1-C\Lambda)\|\b{u}\orlnor+C\Lambda\|\b{u}\orlnor=\|\b{u}\orlnor\leq \Lambda +\Lambda \rho_{\Phi}\left(\frac{\b{u}}{\Lambda}\right),\]
then
\[\frac{(1-C\Lambda)}{\Lambda}\|\b{u}\orlnor-1\leq \rho_{\Phi}\left(\frac{\b{u}}{\Lambda}\right)- C\|\b{u}\orlnor=J_{C,1}(\b{u}).\]
This shows that $J_{C,1}$ is coercive and therefore item 1 is proved.  

In virtue of \cite[Eq. (2.8)]{AGMS}, the $\Delta_2$-condition on $\Psi$, \cite[Thm. 11.7]{M} and \linebreak \cite[Cor. 11.6]{M}, we obtain constants $K>0$ and $\alpha_{\Phi}>1$ such that 
\begin{equation}\label{delta2-consecuencia}
\Phi(r s)\geq Kr^{\nu}\Phi(s)
\end{equation}
for any $0<\nu<\alpha_{\Phi}$,  $s\geq 0$ and $r>1$.

Let $1<\mu<\alpha_{\Phi}$ and let $r>\Lambda$ be a constant that will be specified later.  
Then, from \eqref{delta2-consecuencia} and \eqref{amemiya}, we get
\[
\begin{split}
\frac{\int_0^T \Phi\left(\frac{|\b{u}|}{\Lambda}\right)\ dt}{\|\b{u}\orlnor^{\mu}}
&\geq
K \left(\frac{r}{\Lambda}\right)^{\nu}\frac{\int_0^T \Phi(r^{-1}|\b{u}|)\ dt}{\|\b{u}\orlnor^{\mu}}\\
&\geq
K \left(\frac{r}{\Lambda}\right)^{\nu}\frac{r^{-1}\|\b{u}\orlnor-1}{\|\b{u}\orlnor^{\mu}}.\\
\end{split}
\]
We choose $r=\|\b{u}\orlnor/2$. Since $\|\b{u}\orlnor\to+\infty$   we can assume $\|\b{u}\orlnor>2\Lambda$.  Thus $r>\Lambda$ and 

\[
\frac{\int_0^T \Phi\left(\frac{|\b{u}|}{\Lambda}\right) dt}{\|\b{u}\orlnor^{\mu}}\geq
\frac{K}{2^{\nu}\Lambda^{\nu}} \|\b{u}\orlnor^{\nu-\mu}\to +\infty\quad\text{as }\|\b{u}\orlnor\to+\infty,
\]
because $\nu>\mu$.

In order to prove the last item, we assume that $\Psi \notin \Delta_2$. 
By \cite[Thm. 4.1]{KR},  there exists a sequence of real  numbers  $r_n$ such that
$r_n \to \infty$ and 
\begin{equation}\label{eq: un-_tiende_inf}
\lim\limits_{n \to \infty} \frac{r_n \psi(r_n)}{\Psi(r_n)}=+\infty.
\end{equation}
Now, we choose $\b{u}_n$ such that
$|\b{u}_n|=\Lambda\psi(r_n)\chi_{[0,\frac{1}{\Psi(r_n)}]}$. Then, 
by \cite[Eq. (9.11)]{KR}, we get 
\[
\|\b{u}_n\orlnor =\Lambda\frac{\psi(r_n)}{\Psi(r_n)}\Psi^{-1}(\Psi(r_n))=
\Lambda\frac{r_n\psi(r_n)}{\Psi(r_n)}\to \infty,\quad\text{as}\quad n \to \infty.
\]
Next, using Young's equality (see \cite[Eq. (2.7)]{KR}), we have
\[
\begin{split}
J_{C,1}(\b{u}_n)&=\int_0^T \Phi\left(\frac{|\b{u}_n|}{\Lambda}\right)\,dt-C\|\b{u}_n\orlnor\\
&=
\frac{1}{\Psi(r_n)}\left[\Phi(\psi(r_n))  -C\Lambda r_n\psi(r_n)\right]\\
&=
\frac{1}{\Psi(r_n)} \left[ r_n\psi(r_n)-\Psi(r_n)- C\Lambda r_n\psi(r_n) \right]\\
&=\frac{(1- C\Lambda) r_n\psi(r_n)}{\Psi(r_n)}-1.
\end{split}
\]
From \eqref{eq: un-_tiende_inf} and the condition $C\Lambda>1$, we obtain  $J_{C,1}(\b{u}_n)\to-\infty$, which contradicts the coercivity  of $J_{C,1}$.
\end{proof}



Next, we present two theorems that establish coercivity of action integrals. 



\begin{thm}\label{coercitividad1}
Let  $\mathcal{L}$ be a Lagrangian function satisfying \eqref{cotaL}, \eqref{cotaDxL}, \eqref{cotaDyL}, \eqref{cota_inf} and \eqref{condA1}. We assume the following conditions:
\begin{enumerate}
\item There exist a non negative function  $b_1 \in L^1_1$ and a constant $\mu>0$  such that for any $\b{x_1},\b{x_2}\in\rr^d$ and a.e. $t\in [0,T]$
\begin{equation}\label{holder_cont}
  \left| F(t,\b{x_2})- F(t,\b{x_1}) \right|\leq b_1(t)(1+|\b{x_2}-\b{x_1}|^{\mu}).
\end{equation}
We suppose that $\mu< \alpha_{\Phi}$,  with $\alpha_{\Phi}$ as in Lemma \ref{lem_coer}, in the case that $\Psi\in\Delta_2$; and $\mu=1$  if $\Psi$ is an  arbitrary $N$-function. 
\item
\begin{equation}\label{propiedad1coercividad}
\int_{0}^{T}F(t,\b{x})\ dt \rightarrow \infty \quad \hbox{as} \quad |\b{x}|\rightarrow \infty.
\end{equation}
\item\label{hipot_coer}  $\Psi\in\Delta_2$ or, alternatively, 
%the following inequality 
$\alpha_0^{-1}T\Phi^{-1}\left(1/T\right)\|b_1\|_{L^1}\Lambda<1$.
\end{enumerate}
Then  the action integral $I$ is coercive.
\end{thm}


{\color{red}
\begin{comentario}
\eqref{holder_cont} es m\'as d\'ebil que las hip\'otesis anteriores...
\end{comentario}}


\begin{proof} In the following estimates, we will use \eqref{cota_inf}, the decomposition $\b{u}=\b{\overline{u}}+\b{\tilde{u}}$, H\"older's inequality and Wirtinger's inequality. Namely,
\begin{equation}\label{cota_inf_I}
\begin{split}
I(\b{u})&\geq\alpha_0\rho_{\Phi}\left( \frac{\b{\dot{u}}}{\Lambda}\right)+\int_0^TF(t,\b{u})\ dt\\ 
&=\alpha_0\rho_{\Phi}\left( \frac{\b{\dot{u}}}{\Lambda}\right)+ \int_0^T \left[F(t,\b{u})-F(t,\b{\overline{u}})\right]\ dt +  \int_0^TF(t,\b{\overline{u}})\ dt\\
&\geq\alpha_0\rho_{\Phi}\left( \frac{\b{\dot{u}}}{\Lambda}\right)- \int_0^Tb_1(t)(1+|\b{\tilde{u}}(t)|^{\mu})\ dt +  \int_0^TF(t,\b{\overline{u}})\ dt\\
&\geq \alpha_0\rho_{\Phi}\left( \frac{\b{\dot{u}}}{\Lambda}\right)- \|b_1\|_{L^1}(1+\|\b{\tilde{u}}\|_{L^{\infty}}^{\mu}) +  \int_0^TF(t,\b{\overline{u}})\ dt\\
&\geq\alpha_0\rho_{\Phi}\left( \frac{\b{\dot{u}}}{\Lambda}\right)- \|b_1\|_{L^1}\left(1+\left[T\Phi^{-1}\left(\frac{1}{T}\right)\right]^{\mu}\|\b{\dot u}\orlnor^{\mu}\right) \\
&\quad+  \int_0^TF(t,\b{\overline{u}})\ dt\\
&=\alpha_0J_{C,\mu}(\b{\dot{u}})- \|b_1\|_{L^1}+ \int_0^TF(t,\b{\overline{u}})\ dt,
\end{split}
\end{equation}
where $C=\alpha_0^{-1}\left[T\Phi^{-1}\left(1/T\right)\right]^{\mu}\|b_1\|_{L^1}$.
Suppose that $\b{u}_n$ is a sequence in $\domi$ such that 
the  sequence  $\b{\overline{u}}_n$ is bounded in $\rr^d$ and 
$\|\b{u}_n\sobnor\to\infty$. Then  the Wirtinger's inequality implies that $\|\b{\dot{u}}_n\orlnor\to\infty$. 
Therefore, one of the following statements holds true $\|\b{\dot{u}}_n\orlnor\to\infty$ or $|\b{\overline{u}}_n|\to \infty$. 
On the other hand,  \eqref{condA1} and \eqref{propiedad1coercividad}
imply that the integral $\int_0^TF(t,\b{\overline{u}}_n)\ dt$ is lower bounded. 
These observations, the lower bound of $I$ given by \eqref{cota_inf_I}, 
assumption \ref{hipot_coer} in Theorem \ref{coercitividad1} and Lemma \ref{lem_coer} imply the desired result.
\end{proof}


Following \cite{mawhin2010critical} we say that $F$ satisfies the condition (A) if  $F(t,\b{x})$ is a Carath\'eodory function, $F$ verifies \eqref{condA1} and $F$ is continuously differentiable with respect to $\b{x}$. Moreover, the next inequality holds 

\begin{equation}\label{condA2} |D_{\b{x}}F(t,\b{x})|\leq a(|\b{x}|)b_0(t),\quad\text{for a.e. }t\in [0,T] \text{ and for every }\b{x}\in\rr^d.
\end{equation}
The following result was proved in \cite[p. 18]{mawhin2010critical}. 
\begin{lem}\label{lema_pto_cri} Suppose that $F$ satisfies condition (A) and \eqref{propiedad1coercividad}, $F(t,\cdot)$ is  differentiable and convex  a.e. $t\in [0,T]$. Then there exists $\b{x}_0\in\rr^d$ such that
\begin{equation}\label{der_cero}
 \int_0^T D_{\b{x}} F(t,\b{x}_0)\ dt=0.
\end{equation}
\end{lem}


\begin{thm}
Let $\mathcal{L}$ be as in Theorem \ref{coercitividad1} and let $F$ be as in Lemma \ref{lema_pto_cri}. In addition, assume that $\Psi\in\Delta_2$ or, alternatively  $\alpha_0^{-1}T\Phi^{-1}\left(1/T\right)a(|\b{x}_0|)\|b_0\|_{L^1} \Lambda<1$, with $a$ and $b_0$ as in \eqref{condA1} and $\b{x}_0\in\rr^d$  any point satisfying  \eqref{der_cero}. Then $I$ is coercive. 

\end{thm}


\begin{proof}  Using \eqref{cota_inf}, \cite[Eq. (18), p.17]{mawhin2010critical},  the decomposition $\b{u}=\b{\overline{u}}+\b{\tilde{u}}$,  \eqref{der_cero}, \eqref{holder} and Wirtinger's inequality, we get
%
\begin{equation}\label{cota_con _upunto}
\begin{split}
I(\b{u})&\geq\alpha_0\rho_{\Phi}\left( \frac{\b{\dot{u}}}{\Lambda}\right)+\int_0^T F(t,\b{x}_0)\ dt 
+ \int_0^T D_{\b{x}} F (t,\b{x}_0) \ccdot (\b{u}-\b{x}_0)\ dt\\
&=\alpha_0\rho_{\Phi}\left( \frac{\b{\dot{u}}}{\Lambda}\right) +\int_0^T F(t,\b{x}_0)\ dt 
+ \int_0^TD_{\b{x}}F (t,\b{x}_0) \ccdot \b{\widetilde{u}} \ dt\\
&\quad + \int_0^T D_{\b{x}} F (t,x_0) \ccdot (\b{\overline{u}}  -\b{x}_0) \ dt\\
&=\alpha_0\rho_{\Phi}\left( \frac{\b{\dot{u}}}{\Lambda}\right)+\int_0^T F(t,\b{x}_0)\ dt + 
\int_0^T D_{\b{x}} F (t,\b{x}_0) \ccdot \b{\widetilde{u}}\ dt\\
&\geq\alpha_0 \rho_{\Phi}\left( \frac{\b{\dot{u}}}{\Lambda}\right)-a(|\b{x}_0|)\|b_0\|_{L^1} 
-a(|\b{x}_0|)\|b_0\|_{L^1}T\Phi^{-1}\left(\frac{1}{T}\right) \|\b{\dot{u}}  \|_{\lphi}\\
&= \alpha_0 J_{C,1}(\b{\dot{u}})-a(|\b{x}_0|)\|b_0\|_{L^1} 
\end{split}
\end{equation}
with $C:=\alpha_0^{-1}a(|\b{x}_0|)\|b_0\|_{L^1}T\Phi^{-1}(1/T)$. 

Let $\alpha$ be as in Corollary \ref{a_bound}, then it is  a non decreasing majorant of $a$. Using that  $F(t, \b{\overline{u}} /2) \leq (1/2)F(t,\b{u}) + (1/2) F(t, -\b{\widetilde{u}})$ and taking into account that $\Phi$ is a non negative function, inequality \eqref{condA1}, H\"older's inequality,  Corollary \ref{a_bound} and Wirtinger's inequality, we obtain
\begin{equation}\label{cota_con _ubarra}
\begin{split}
I(\b{u}) &\geq\alpha_0\rho_{\Phi}\left( \frac{\b{\dot{u}}}{\Lambda}\right)  +2 \int_0^T F(t,\b{\overline{u}} /2)\ dt - \int_0^T F(t, -\b{\widetilde{u}})\ dt\\
&\geq 2 \int_0^T F(t,\b{\overline{u}} /2)\ dt -\|b_0\|_{L^1} \|\b{a}(\b{\tilde{u}})\|_{L^{\infty}}\\
&\geq 2 \int_0^T F(t,\b{\overline{u}} /2)\ dt -\|b_0\|_{L^1} \alpha(\|\b{\tilde{u}}\|_{L^{\infty}})\\
&\geq 2 \int_0^T F(t,\b{\overline{u}} /2)\  dt - C \alpha(C\|\b{\dot{u}}\orlnor)
\end{split}
\end{equation}
with a certain constant $C>0$.

Let $\b{u}_n$ be a sequence in $\wphi_d$ such that $\|\b{u}_n\sobnor\to\infty$. We need to consider two situations:
\\
i) If $\|\b{\dot {u}}_n\orlnor\to\infty$ then, 
from  \eqref{cota_con _upunto} and Lemma \ref{lem_coer}, we have $I(\b{u}_n)\to\infty$. 
\\
ii) If $\|\b{\dot{u}}_n\orlnor$ is bounded and $\|\b{u}_n\orlnor\to\infty$,  then we obtain $\b{\overline{u}}_n \to\infty$,
reasoning in a similar way to that developed in the proof of Theorem \ref{coercitividad1}. 
This fact together with  \eqref{cota_con _ubarra} finishes the proof. 

\end{proof}



\section{Main results}




For the lower semicontinuity of  $I$ we only need to adapt some results of \cite{ekeland1999convex} to our problem. 


\begin{thm}
We suppose that $\mathcal{L}(t,\b{x},\b{y})$, $F(t,\b{x})$ are Charath\'eodory functions satisfying
\begin{equation}\label{cota_inf_2}
\mathcal{L}(t,\b{x},\b{y})\geq \Phi\left(|\b{y}|\right)+ F(t,\b{x}),
\end{equation}
where $\Phi$ is an $N$-function. 
We also assume that the function $F$ satisfies inequality \eqref{condA1} and $\mathcal{L}(t,\b{x},\cdot)$ is convex in $\rr^d$ for each $(t,\b{x})\in [0,T]\times\rr^d$.  Let $\{\b{u}_n\}\subset\wphi$ be a sequence such that $\b{u}_n$ converges  uniformly  to a function $\b{u}\in\wphi$ and $\b{\dot{u}}_n$ converges in the weak topology of $L^1_d$ to $\b{\dot{u}}$.   Then
\begin{equation}\label{liminf0}I(\b{u})\leq \liminf_{n\to\infty}I(\b{u}_n).
\end{equation}

\end{thm}

\begin{proof} First we note that \eqref{cota_inf_2} and \eqref{condA1} imply that $I$ is defined in $\wphi$ taking values in the interval $(-\infty,+\infty]$. 

Let $\{\b{u}_n\}$ be a sequence  satisfying the assumptions of the Theorem.   We define the Caratheodory function $\mathcal{L}'=\mathcal{L}-F$. Let $I'$ be the associated to $\mathcal{L}'$ action integral. Using  \cite[Th. 2.1, p. 243]{ekeland1999convex} with  $\mathcal{L}'$ instaead $f$, we have that
\begin{equation}\label{liminf1}\int_0^T\mathcal{L}'(t,\b{u},\b{\dot{u}})dt\leq \liminf_{n\to\infty}\int_0^T\mathcal{L}'(t,\b{u}_n,\b{\dot{u}}_n)dt.
\end{equation}

From the uniform convergence of $\b{u}_n$ and the Caratheodory conditions for $F$ we obtain that $F(t,\b{u}_n(t))\to F(t,\b{u}(t))$ a.e. $t\in[0,T]$.  Since $\b{u}_n$ are uniformly bouded, the inequaluity  \eqref{condA1} imply that there exists $g\in L_1^1([0,T])$ such that $|F(t,\b{u}_n(t))|\leq g(t)$. From the Dominated Convergence Theorem we have that 
\begin{equation}\label{liminf2}\lim_{n\to\infty}\int_0^TF(t,\b{u}_n(t))dt=\int_0^TF(t,\b{u}(t))dt.
\end{equation}
Taking account of \eqref{liminf1} and  \eqref{liminf2} we obtain \eqref{liminf0}.

\end{proof}


\begin{thm} We suppose $\Phi$ and $\Psi$ complementary $N$-functions with $\Psi\in\Delta_2$. We assume that the Caratheodory function $\mathcal{L}(t,\b{x},\b{y})$ is strictly convex in $\b{y}$, $D_{\b{y}}\mathcal{L}$ is $T$-periodic with respect to $T$  and that it satisfies \eqref{cotaL}, \eqref{cotaDxL}, \eqref{cotaDyL}, \eqref{cota_inf}, \eqref{condA1}, \eqref{holder_cont} and \eqref{propiedad1coercividad}.  Then problem \eqref{ProbPrin} has a solution.
\end{thm}

\begin{proof} Let $\{\b{u}_n\}\subset \wphi_T$ be a  minimizing sequence of the problem
\[\min\limits_{\b{u}\in\wphi_T}\int_0^T\mathcal{L}(t,\b{u},\b{\dot{u}})dt.\]
 Then $I(\b{u}_n)$ is bounded. Theorem \ref{coercitividad1} imply that $\{\b{u}_n\}$ is norm bounded in $\wphi_d$. Therefore $\b{\dot{u}}_n$ is bounded in $\lphi_d$. The space  $\lphi_d$ is a predual space, concretely $\lphi_d=\left[\epsi_d\right]^*=\left[\lpsi_d\right]^*$ and according to \cite[Prop. 4, p. 147]{rao1991theory} (note that $\Phi\in\nabla_2\Leftrightarrow \Psi\in\Delta_2$) there exists a subsequence 

\end{proof}

\printbibliography

\end{document}
