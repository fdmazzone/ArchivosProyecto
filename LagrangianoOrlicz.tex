%%% Preambulo%%%%%%%%%%%%%%%%%%%%%%%%%

\documentclass[twoside]{article}
%%Paquetes


\usepackage{amsmath,amssymb,amsthm}
\usepackage{color}
\usepackage{ esint }
\usepackage{graphicx}
\usepackage{wrapfig}
\usepackage{subfigure}
\usepackage{fancyhdr}
\usepackage{times}
%\usepackage{theorem}
\usepackage[latin1]{inputenc}
%\usepackage{showkeys}
\usepackage{comment}


%Teorema y similes



\newtheorem{thm}{Theorem}[section]
\newtheorem{cor}[thm]{Corollary}
\newtheorem{lem}[thm]{Lemma}
\newtheorem{rem}[thm]{Remark}
\newtheorem{defi}{Definition}
\newtheorem{prop}[thm]{Proposition}



\title{Propuesta de trabajo 07-08 de noviembre 2013}
\author{Sonia Acinas \thanks{SECyT-UNRC}\\
Dpto. de Matem\'atica, Facultad de Ciencias Exactas y Naturales\\
Universidad Nacional de La Pampa\\
(6300) Santa Rosa, La Pampa, Argentina\\
sonia.acinas@gmail.com\\[3mm]
Leopoldo Buri \thanks{SECyT-UNRC}\\
Dpto. de Matem\'atica, Facultad de Ciencias Exactas, F\'{\i}sico-Qu\'{\i}micas y Naturales\\
Universidad Nacional de R\'{i}o Cuarto\\
(5800) R\'{\i}o Cuarto, C\'ordoba, Argentina,\\
lburi@exa.unrc.edu.ar\\[3mm]
Graciela Giubergia \thanks{SECyT-UNRC and CONICET}\\
Dpto. de Matem\'atica, Facultad de Ciencias Exactas, F\'{\i}sico-Qu\'{\i}micas y Naturales\\
Universidad Nacional de R\'{i}o Cuarto\\
(5800) R\'{\i}o Cuarto, C\'ordoba, Argentina,\\
ggiubergia@exa.unrc.edu.ar\\[3mm]
Fernando D. Mazzone \thanks{SECyT-UNRC and CONICET}\\
Dpto. de Matem\'atica, Facultad de Ciencias Exactas, F\'{\i}sico-Qu\'{\i}micas y Naturales\\
Universidad Nacional de R\'{i}o Cuarto\\
(5800) R\'{\i}o Cuarto, C\'ordoba, Argentina,\\
fmazzone@exa.unrc.edu.ar\\[3mm]
Erica L. Schwindt\thanks{ANR. AVENTURES - ANR-12-BLAN-BS01-0001-01}\\
Universit\'{e} d'{O}rl\'{e}ans, Laboratoire MAPMO, CNRS, UMR 7349, \\
F\'ed\'eration Denis Poisson, FR 2964,\\
B\^{a}timent de Math\'{e}matiques, BP 6759, 45067 Orl\'{e}ans Cedex 2, France,\\
leris98@gmail.com}

\date{}

\newcommand{\orlnor}{\|_{L^{\Phi}}}
\newcommand{\lurnor}{\|^{*}_{L^{\Phi}}}
\newcommand{\linf}{\|_{L^{\infty}}}
\newcommand{\lphi}{L^{\Phi}}
\newcommand{\lpsi}{L^{\Psi}}
\newcommand{\ephi}{E^{\Phi}}
\newcommand{\claseor}{\widetilde{L}^{\Phi}}
\newcommand{\wphi}{W^{1}\lphi}
\newcommand{\sobnor}{\|_{W^{1}\lphi}}
\newcommand{\domi}{W^{1}\left(\lphi,\Pi\left(\ephi,1\right)\right)}
\renewcommand{\b}[1]{\boldsymbol{#1}}

\begin{document}



\maketitle
%
\begingroup%Locallizing the change to `thefootnote'.
    \renewcommand{\thefootnote}{}%Removing the footnote symbol.
    %
    \footnotetext{%
    %   2010 Mathematics Subject Classification
    %   http://www.ams.org/msc/
    \textbf{2010  AMS Subject Classification.} Primary: .
    Secondary: .
    }%
        \footnotetext{%
    \textbf{Keywords and phrases.}  .
    }%
    \endgroup
%
%
%
%

\begin{abstract}
...
\end{abstract}




\pagestyle{fancy} \headheight 35pt \fancyhead{} \fancyfoot{}

\fancyfoot[C]{\thepage} \fancyhead[CE]{\nouppercase{S. Acinas, L. Buri, G. Giubergia, F. Mazzone and E. Schwindt}} \fancyhead[CO]{\nouppercase{\section}}

\fancyhead[CO]{\nouppercase{\leftmark}}


%\tableofcontents

\section{Preliminaries}

%\subsection{$N$-functions}
For reader convenience, a short introduction to Orlicz and Orlicz Sobolev spaces is given, we refer to \cite{adams_sobolev,KR} for additional details and proofs.

A function $\Phi:[0,+\infty)\to [0,+\infty)$ is called an \emph{$N$-function} if it has the form
\[
\Phi(t)=\int_{0}^t \varphi(\tau)\ d\tau,\quad\hbox{for } u\geq 0,
\]
where $\varphi: [0, \infty)\rightarrow [0, \infty)$ is a right continuous nondecreasing function  satisfying   $\varphi(0)=0$, $\varphi(t)>0$ for $t>0$ and
$\lim_{t\rightarrow \infty}\varphi(t)=+\infty$.

Given a function $\varphi$ as above, we also consider the so-called right inverse function $\psi$ of $\varphi$ which is defined $\psi(s)=\sup_{\varphi(t)\leq s}t$.
The function $\psi$ satisfies the same properties that function $\varphi$, therefore we have an $N$-function $\Psi$ associated to $\psi$. We say that $\Psi$ is the
\emph{complementary function} of $\Phi$.

We say that $\Phi$ is a \emph{function of the $\Delta_2$ class} when there exists a constant $k>0$ and a $t_0\geq 0$ such that $\Phi(2t)\leq K\Phi(t)$, for every $t\geq t_0$.

Throughout this article we will try with spaces of $\mathbb{R}^n$ valued functions defined on an interval $[0,T]\subset\mathbb{R}$. Given an $N$-function $\Phi$ we
define the \emph{Orlicz class} $\widetilde{L}^{\Phi}([0,T],\mathbb{R}^n)$ by
\begin{equation}\label{claseOrlicz}
  \widetilde{L}^{\Phi}([0,T],\mathbb{R}^n):=\left\{\b{u}: [0,T] \rightarrow\mathbb{R}^n,\b{u}\ \hbox{mesurable},\ \int_{[0,T]} \Phi(|\b{u}|)\ dx < \infty \right\}.
\end{equation}
here $|\cdot|$ is the euclidean norm of $\mathbb{R}^n$. When clear from the context, we will omit the domain and codomain in the notation of function spaces and
classes ($\claseor=\widetilde{L}^{\Phi}([0,T],\mathbb{R}^n)$). The \emph{Orlicz space} $\lphi=L^{\Phi}([0,T],\mathbb{R}^n)$ is defined as the linear hull of $\claseor$.
Equivalently
\begin{equation}\label{espacioOrlicz}
\lphi:=\left\{ \b{u}: [0,T] \rightarrow \mathbb{R}^n \bigg| \b{u}\ \hbox{is mesurable and},\ \int_{[0,T]} \Phi(\alpha|\b{u}|)\ dx < \infty \ \hbox{for some}\ \alpha >0   \right\}.
\end{equation}
 We adopt the convention of to use bold symbols for denote $\mathbb{R}^n$-valued fuction and plain symbols for scalar ones. The Orlicz space $\lphi$ equipped with the Orlicz norm
\[
\|  \b{u}  \orlnor:=\sup \left\{  \int_0^T\langle \b{u}, \b{v}\rangle dt: \int_0^T\Psi(|\b{v}|)\leq 1\right\},
\]
is a Banach space. Here by $\langle \b{u}, \b{v}\rangle$ we denote the usual dot product in $\mathbb{R}^{n}$ between $\b{u}$ and $\b{v}$.

%Alternatively  the Luxemburg norm
%\[
%\|  u  \lurnor =\inf \left\{ k>0:  \int_{0}^T \Phi \left(\frac{|u|}{k}\right)\ dx \leq 1 \right\},
%\]
%defines an equivalent norm.



The space $\ephi=\ephi([0,T],\mathbb{R}^n)$ is defined as the closure in $\lphi$ of the subspace $L^{\infty}$. The space $\ephi$ is the maximal subspace of the
Orlicz class $\claseor$ and, it is known that, $\left[\ephi\right]^*=\lpsi$.


Likes in \cite{KR}, we will consider the subset $\Pi(\ephi,r)=\Pi\left(\ephi\left([0,T],\mathbb{R}^n\right),r\right)$ of $\lphi([0,T],\mathbb{R}^n)$ defined by
\[\Pi(\ephi,r):=\{\b{u}\in\lphi: d(\b{u},\ephi)<r\}.\]
This set is related to the Orlicz class $\claseor$ by means of inclusions
\begin{equation}\label{inclusiones}\Pi(\ephi,1)\subset \claseor \subset\overline{\Pi(\ephi,1)}.\end{equation}
The proof of this fact, and similar ones, is given by real valued function in \cite{KR}, 
the extension to $\mathbb{R}^n$-valued functions does not involve any difficulty.

Let $X$ and $Y$ be subsets of certain vector spaces of $\mathbb{R}^n$-valued measurable functions defined in $[0,T]$.   We denote by  $W^1(X,Y)$ to the set defined by
\[W^1(X,Y):=\{\b{u}| \b{u} \hbox{ is absolutely continuous and } \b{u}\in X, \b{\dot{u}}\in Y\}.\]
If $X=Y$ we simply write $W^1(X,X)=W^1X$. In this paper $X$ and $Y$ will be some subset of an Orlicz space.  When $X=Y=\lphi$ we have the usual Sobolev-Orlicz space
 $\wphi$ (see \cite{adams_sobolev}) , which is a Banach space  equipped with the norm
\[
\|  \b{u}  \|_{\wphi}= \|  \b{u}  \|_{\lphi} + \|\b{\dot{u}}\orlnor.
\]

 An important aspect of the theory of Sobolev spaces is related to embedding theorems. There is an extensive literature on this question in the setting of Orlicz-Sobolev spaces, see for example
 \cite{cianchi1999some,cianchi2000fully,claverooptimal,edmunds2000optimal,kerman2006optimal}.
For this reason the following simple  Lemma, which we will use systematically, it is well known. We include a brief proof for sake of completeness. As is usual, if $X$ and $Y$ are normed spaces, with $X\subset Y$,  we write $X\hookrightarrow Y$ when the identity map is an bounded operator between $X$ and $Y$. 


\begin{lem}\label{inclusion orlicz}$\wphi\hookrightarrow L^{\infty}$.
\end{lem}
\begin{proof}
Let $\b{u}\in \wphi$. From the mean value theorem there exists $\tau$ such that
$\b{u}(\tau)=\frac{1}{T}\int\limits_{0}^{T}\b{u}(s)ds$, thus
\begin{equation}\label{desigualdad1}\begin{split}
|\b{u}(t)|\leqslant |\b{u}(\tau)|+\int\limits_{\tau}^{t}|\b{\dot{u}}(s)|ds
%&\leq |\b{u}(\tau)|+\int\limits_{0}^{T}|\b{\dot{u}}(s)|ds\\
\leq |\b{u}(\tau)|+\|\b{\dot{u}}\|_{L^{\Phi}}\|1\|_{L^{\Psi}}.
\end{split}
\end{equation}
Moreover
\begin{equation}\label{desigualdad2}\begin{split}
|\b{u}(\tau)|\leq \frac{1}{T}\int\limits_{0}^{T}|\b{u}(s)|ds\leq \frac{1}{T}\|\b{u}\|_{L^{\Phi}}\|1\|_{L^{\Psi}}.
\end{split}
\end{equation}
From \eqref{desigualdad1} and \eqref{desigualdad2} we obtain
\[
\|\b{u}\|_{L^{\infty}}\leq C(T)\|\b{u}\|_{\wphi}.
\]
\end{proof}
We will use repeatedly the following elementary consequence of the previous theorem. Hereafter we denote  by $\mathbb{R}^+$ to the set of all non negative real numbers. 
\begin{cor}\label{a_bound} Let $a\in C(\mathbb{R}^+,\mathbb{R}^+)$ and we define the composition operator $\b{a}$ by $\b{a}(\b{u})(t)= a(|\b{u}(t)|)$. Then $a:\wphi\to L^{\infty}([0,T],\mathbb{R})$ is bounded, which is there exists a non decreasing function $c:\mathbb{R}^+\to\mathbb{R}^+$ such that 
 $\|\b{a}(\b{u})\|_{L^{\infty}}\leq c(\|\b{u}\|_{\wphi})$.
\end{cor}
\begin{proof}  If $\b{u}\in \wphi$ then, by Lemma \ref{inclusion orlicz} ,
 $\b{u}\in L^{\infty}$ and $\|\b{u}\|_{L^{\infty}}\leq C(T)\|\b{u}\|_{\wphi}=:A$.   By hypothesis
 $a:[0,A]\rightarrow\mathbb{R}$ is bounded, hence the supremum $c(A):=\sup_{[0,A]}a$ is well defined. Clearly $c$ satisfies the statement of Corollary.
\end{proof}


The following lemma is an easy consequence of more general principles  related to  operators of Nemitskii type, see \cite[�17]{KR}. 

\begin{lem}\label{phi_comp}   Let $\boldsymbol{\varphi}$ be the  composition operator
 defined by $\boldsymbol{\varphi}(\b{u})(t)= \varphi(|\b{u}(t)|)$. Then  $\boldsymbol{\varphi}$ acts from $\Pi(\ephi,1)$ into $\tilde{L}^{\Psi}$.
\end{lem}
\begin{proof}
  As consequence of \cite[Lemma 9.1]{KR} we have that  $\boldsymbol{\varphi}\left(B_{\lphi}(0,1)\right)\subset \tilde{L}^{\Psi}$, where
$B_{\lphi}(\b{u}_0,r)$ is the open ball with center $\b{u}_0$ and radius $r>0$. Therefore, applying \cite[Lemma 17.1]{KR} we deduce that $\boldsymbol{\varphi}$ acts from $\Pi(\ephi,1)$ into $\tilde{L}^{\Psi}$. 
\end{proof}

We need also the following technical lemma.
\begin{lem}\label{segundo lema}
Let $\{\b{u}_n\}_{n\in \mathbb{N}}$ a sequence of  functions in $\Pi(\ephi,1)$, and $\b{u}\in \lphi$ such that $\b{u}_n\rightarrow \b{u}$ in $\lphi$. Then there exist a subsequence
$\b{u}_{n_k}$ and a real valued function $h\in\Pi\left(\ephi\left([0,T],\mathbb{R}\right),1\right)$ such that $\b{u}_{n_k}\rightarrow \b{u} \quad\text{a.e.}$ and $|\b{u}_{n_k}|\leq h\quad\text{a.e.}$.
\end{lem}



\begin{proof}
Let $r:=d(\b{u},\ephi)$, $r<1$. Because $\b{u}_n$ converges to $\b{u}$, there exists a subsequence $(n_k)$ such that
\[\|\b{u}_{n_k}-\b{u}\|_{\Phi}<\frac{1-r}{2}\quad \text{ and }\quad \|\b{u}_{n_k}-\b{u}_{n_{k+1}}\|_{\Phi}<2^{-k}(1-r)\]
Let $h:[0,T]\rightarrow\mathbb{R}$ defined by
\begin{equation}\label{serie} h(x)=|\b{u}_{n_1}(x)|+\sum_{k=2}^{\infty}|\b{u}_{n_k}(x)-\b{u}_{n_{k-1}}(x)|.
\end{equation}
Since 
\[d(|\b{u}_{n_1}|,\ephi)\leq d(\b{u}_{n_1},\ephi)\leq d(\b{u}_{n_1},\b{u})+d(\b{u},\ephi)<\frac{1+r}{2},\] 
we have
\[d(h,\ephi)\leq d(h,|\b{u}_{n_1}|)+d(|\b{u}_{n_1}|,\ephi)< 1.\]
Therefore, $h\in\Pi(\ephi,1)$.  In particular,  $|h|<\infty$ a.e. We conclude that la serie $\b{u}_1(x)+\sum_{k=2}^{\infty}(\b{u}_{n_k}(x)-\b{u}_{n_{k-1}}(x))$ 
is absolutely convergent a.e.  This imply that $\b{u}_{n_k}\rightarrow \b{u} \quad\text{a.e.}$. The inequality $|\b{u}_{n_k}|\leq h$ is clear from the definition of $h$.
\end{proof}

The following simple result will be useful.
\begin{lem}\label{lema_conv_may}
Suppose that $\Phi$ is a $\Delta_2$ function.  If $u_n \in\lphi$ is a sequence such that $u_n\to 0$ a.e. and suppose that there exist $M\in\lphi$ with $|u_n|\leq M$
then $\|u_n\orlnor\to 0$.
\end{lem}
\begin{proof}
 According to \cite[Theorem 9.4]{KR} it is sufficient to prove that
 \[\int_0^T\Phi(|\b{u}_n|)dt\to 0,\quad\text{for }n\to\infty.\]
 This is an immediate consequence of $\Phi(|\b{u}_n|)\leq \Phi(M)\in L^1$ and the Dominated Convergence Theorem.
\end{proof}


If $X$ is a Banach spaces, we denote by $\langle\cdot,\cdot\rangle:X\times X^*\to\mathbb{R}$ the bilinear map given by the pairing between $X$ and its dual space $X^*$.  We recall the definition of Gate\^{a}ux derivative, see \cite{ambrosetti} for details. Given a function $I:U\to\mathbb{R}$ where $U$ is an open set of a Banach space $X$,
we say that $I$ has a G\^ateaux derivative en $\b{u} \in U$ if there exists $\b{u}^*\in X^*$ such that for every $\b{v} \in X$
\[
\lim_{s \rightarrow 0}\frac{I(\b{u}+s\b{v})-I(\b{u}) }{s}=\langle \b{u},\b{u}^*\rangle.
\]



\section{Action integrals on Orlicz spaces}

\begin{defi} We said that a function $\mathcal{L}:[0,T]\times \mathbb{R}^n \times \mathbb{R}^n \rightarrow \mathbb{R}$ is a Caratheodory function if for fixed $(\b{x},\b{y})$
the map $t \mapsto \mathcal{L}(t, \b{x},\b{y})$ is measurable  and for fixed $t$ the map  $(\b{x},\b{y}) \mapsto \mathcal{L}(t, \b{x}, \b{y})$ is continuously differentiable for almost everywhere $t\in [0,T]$.

\end{defi}



\begin{thm}\label{teorema_acotacion}
Let $\mathcal{L}:[0,T]\times \mathbb{R}^n \times \mathbb{R}^n \rightarrow \mathbb{R}$ be a Caratheodory function and $\Phi,\Psi$ be complementary  $N$-functions. Suppose that there
exists $a \in C(\mathbb{R}^+, \mathbb{R}^+)$, $b \in L^1$, $c \in \lpsi$ such that

\begin{eqnarray}
|\mathcal{L}(t,\b{x},\b{y})| &\leq a(|\b{x}|)\left(b(t)+ \Phi(|\b{y}|)  \right),\label{cotaL}\\
|D_{\b{x}}\mathcal{L}(t,\b{x},\b{y})| &\leq a(|\b{x}|)\left(b(t)+ \Phi(|\b{y}|)  \right),\label{cotaDxL}\\
|D_{\b{y}}\mathcal{L}(t,\b{x},\b{y})| &\leq a(|\b{x}|)\left(c(t)+ \varphi(|\b{y}|)  \right).\label{cotaDyL}
\end{eqnarray}




Then the following statements hold
\begin{enumerate}
\item \label{T1item1} \label{A1} The \emph{action integral}  $I: W^{1}(\lphi,\claseor) \rightarrow \mathbb{R}$ defined by
\begin{equation}\label{integral_accion}
I(\b{u})=\int_{0}^T \mathcal{L}(t,\b{u}(t),\b{\dot{u}}(t))\ dt
\end{equation}
is finite for every $\b{u}\in W^{1}(\lphi,\claseor)$.

\item\label{T1item3} The function  $I$ is G\^ateaux differentiable on $W^{1}\left(\lphi,\Pi\left(\ephi,1\right)\right)$ and  its derivative $I'$ is continuous from $\domi$ with the strong topology into $\left[\wphi \right]^*$ equipped with the $w^*$-topology. Moreover $I'$ is given by the expression
\[
\langle v, I'(\b{u})\rangle= \int_0^T \bigg\{\bigg\langle D_{\b{x}}\mathcal{L}\big(t,\b{u}(t),\b{\dot{u}}(t)\big), \b{v}(t)\bigg\rangle+ \bigg\langle D_{y}\mathcal{L}\big(t,\b{u}(t),\b{\dot{u}}(t)\big),\dot{\b{v}}(t)\bigg\rangle\bigg\} \ dt.
\]

\item\label{T1item4}  If $\Phi$ and $\Psi$ are $\Delta_2$ functions,
  $I'$ is continuous from $\wphi$ into $\left[\wphi\right]^*$ when both spaces are equipped with the strong topology.
 

\end{enumerate}
\end{thm}
\begin{proof} From Corollary \ref{a_bound} we obtain a constant $c=c(\|\b{u}\sobnor )$ such that  $a(|\b{u}|)\leq c$.
 Thus,
 \[|\mathcal{L}(t,\b{u},\b{\dot{u}})| \leq M\left(b(t)+ \Phi(|\b{\dot{u}}|)  \right)\in
 L^1.\]

 We split the proof of  \ref{T1item3} in three steps. 

\noindent\textbf{Step 1.} We prove that $\b{u} \mapsto D_{\b{x}}\mathcal{L}(t,\b{u},\b{\dot{u}})$ is continuous from $\domi$ into $\mathcal{L}^{1}([0,T])$ whith the strong topology on both sets. We take   $\{\b{u}_n\}_{n\in \mathbb{N}}$ a sequence of  functions in $W^{1}(\lphi,\Pi(\ephi,1))$, and $\b{u}\in W^{1}(\lphi,\Pi(\ephi,1))$ such that $\b{u}_n\rightarrow \b{u}$ in $\wphi$.
Then $\b{u}_n\rightarrow \b{u}$ in $\lphi$ and $\b{\dot{u}}_n\rightarrow \b{\dot{u}}$ in $\lphi$. By Lemma \ref{segundo lema} there exist a subsequence $\b{u}_{n_k}$ and $h\in \Pi(\ephi,1))$
such that $\b{u}_{n_k}\rightarrow \b{u} \quad\text{a.e.}$, $\b{\dot{u}}_{n_k}\rightarrow \b{\dot{u}} \quad\text{a.e.}$ and $|\b{\dot{u}}_{n_k}|\leq h\quad\text{a.e.}$.  Since $\b{u}_{n_k}$, $k=1,2,\ldots$ is a strong convergent sequence in $\wphi$, it is a bounded sequence in $\wphi$. According to Lemmas \ref{inclusion orlicz} and Corollary \ref{a_bound} there exists $M>0$ such that $\|\b{a}(\b{u}_{n_k})\|_{L^{\infty}} \leq M$, $k=1,2,\ldots$.  From the previous facts and \eqref{cotaDxL} we get
\begin{equation}\label{DxL1}
|D_x\mathcal{L}(t,\b{u}_{n_k}(t),\b{\dot{u}}_{n_k}(t))|\leq M\left(b(t)+\Phi(|h|)\right) \in L^1.
\end{equation}
By the Caratheodory condition
\[D_x\mathcal{L}(t,\b{u}_{n_k}(t),\b{\dot{u}}_{n_k}(t))\to D_x\mathcal{L}(t,\b{u}(t),\b{\dot{u}}(t))\quad\hbox{ for a.e }t\in[0,T].\]
Applying the Dominated Convergence Theorem we conclude the proof of step 1.

\noindent\textbf{Step 2.} We will prove that the mapping  $\b{u}
 \mapsto  D_{y}\mathcal{L}(t,\b{u},\b{\dot{u}})$ is continuous from $\domi$ with the strong topology  into $\left[\lphi\right]^*$  with the weak$^*$ topology. Let $\b{u}\in W^{1}(\lphi,\Pi(\ephi,1))$.  It follows from Lemma \ref{phi_comp} and Corollary \ref{a_bound} that $\varphi(\b{u}(t))\in\tilde{L}^{\Psi}$ and $a(|\b{u}(t)|)\in L^{\infty}$ respectively. Therefore, in virtue of  \eqref{cotaDyL} we get 
\begin{equation}\label{DyLpsi}
   \left\|D_y\mathcal{L}(t,\b{u}(t),\b{\dot{u}}(t))\right\|_{L^{\Psi}}\leq  c(\|\b{u}\|_{\wphi} )\left(\|c\|_{L^{\Psi}}+\|\boldsymbol{\varphi}(\b{u})\|_{L^{\Psi}} \right)
\end{equation}
Hence $D_y\mathcal{L}(t,\b{u}(t),\b{\dot{u}}(t))\in L^{\Psi}$.

Now, let us to prove the continuity of the map   $\b{u}\mapsto D_y\mathcal{L}(t,\b{u}(t),\b{\dot{u}}(t))$. We take $\b{u}_n,\b{u}\in W^{1}(\lphi,\Pi(\ephi,1))$ with $\b{u}_n\to \b{u}$ in the norm of $\wphi$. We must prove that  $D_y\mathcal{L}(t,\b{u}_n(t),\dot{\b{u}_n}(t))\rightharpoonup D_y\mathcal{L}(t,\b{u}(t),\b{\dot{u}}(t))$. Suppose, on the contrary, that there exists $\b{v}\in\lphi$, $\epsilon>0$ and a subsequence of $\{\b{u}_n\}$ (again denoted for simplicity $\{\b{u}_n\}$)  such that
\begin{equation}\label{cota_eps}
 \left| \langle D_y\mathcal{L}(t,\b{u}_n(t),\b{\dot{u}_n}(t)), \b{v} \rangle - \langle D_y\mathcal{L}(t,\b{u}(t),\b{\dot{u}}(t)), \b{v} \rangle\right|\geq \epsilon.
\end{equation}
We have $\b{u}_n\rightarrow \b{u}$ in $\lphi$ and
$\b{\dot{u}}_n\rightarrow \b{\dot{u}}$ in $\lphi$. By Lemma \ref{segundo lema}, there exist a subsequence $\b{u}_{n_k}$ and $h\in \Pi(\ephi,1)$ such that $\b{u}_{n_k}\rightarrow \b{u} \quad\text{a.e.}$, $\b{\dot{u}}_{n_k}\rightarrow \b{\dot{u}} \quad\text{a.e.}$ and $|\b{\dot{u}}_{n_k}|\leq h\quad\text{a.e.}$. As in the previous step, since $\b{u}_n$ is a convergent sequence, the Corrollary \ref{a_bound} implies that $a(|\b{u}_n(t)|)$ is uniformly bounded by certain constant $C$. Therefore, from \eqref{cotaDyL} we get
\[
  \left |\langle D_y\mathcal{L}(t,\b{u}_n(t),\b{\dot{u}}_n(t)) ,\b{ v}\rangle \right| \leq C\left(c(t)|\b{v}(t)|+\varphi(|h(t)|)|\b{v}(t)|\right).
\]
We have that $c(t),\varphi(h(t))\in\lpsi$ and $|\b{v}|\in\lphi$. Then,  H\"older inequality for Orlicz spaces implies that  $\langle D_y\mathcal{L}(t,\b{u}_{n_k}(t),\b{\dot{u}}_{n_k}(t)), \b{v}\rangle$ is dominated by a function in $L^1$. Thus, from the Lebesgue dominated convergence Theorem we deduce
\begin{equation}\label{conv_debil} \langle D_y\mathcal{L}(t,\b{u}_{n_k}(t),\b{\dot{u}}_{n_k}(t)),\b{ v} \rangle \to \langle D_y\mathcal{L}(t,\b{u}(t),\b{\dot{u}}(t)),\b{ v} \rangle. \end{equation}
which contradict the inequality \eqref{cota_eps}. This completes the proof of step 2.

\textbf{Step 3.} Finally we prove \ref{T1item3}. The proof follows similar lines that \cite[Theorem 1.4]{mawhin2010critical}. For $\b{u}\in \domi$ and $\b{v}\in\wphi$ we define the function
\[f(s,t):=\mathcal{L}(t,\b{u}(t)+s\b{v}(t),\b{\dot{u}}(t)+s\b{\dot{v}}(t)).\]
We remark that by \ref{T1item1}
\[I(\b{u}+s\b{v})=\int_0^Tf(s,t)\ dt\]
is well defined and it is finite valued for $t\in [0,T]$ and  $|s|\leq s_0:=\left(1-d(\b{\dot{u}},\ephi)\right)/\|\b{v}\sobnor$ ($\b{v}\neq 0$). Using  Corollary \ref{a_bound} we have
where
\[ \|a(|\b{u}+s\b{v}|)\|_{L^{\infty}}\leq  c(\|\b{u}+s\b{v}\sobnor)\leq
 c(\|\b{u}\sobnor+s_0\|\b{v}\sobnor).
\]
Consequently, applying chain rule,  inequalities \eqref{cotaDxL}-\eqref{cotaDyL}, the previous inequality and using that $\varphi$ and $\Phi$ are non decreasing, we obtain
\begin{equation}\label{ctg}
\begin{split}
|D_s f(s,t)|&=\big|\langle D_x\mathcal{L}(t,\b{u}+s\b{v},\b{\dot{u}}+s\b{\dot{v}}), \b{v}\rangle + \langle D_y\mathcal{L}(t,\b{u}+s\b{v},\b{\dot{u}}+s\b{\dot{v}}), \b{\dot{v}}\rangle\big| \\
 & \leq c \big[(b(t)+\Phi(|\b{\dot{u}}|+s_0|\b{\dot{v}}|))|\b{v}|+ (c(t)+ \varphi(|\b{\dot{u}}|+s_0|\b{\dot{v}}|))|\b{\dot{v}}| \big]
\end{split}
\end{equation}
It is easy to show that $d(\b{w},\ephi)\leq d(|\b{w}|,\ephi)$ for every $\b{w} \in \lphi$. Then
\[
d \left(|\b{\dot{u}}|+s_0|\b{\dot{v}}|, \ephi \right) %d(|\b{\dot{u}}|,\ephi)+ d(|\b{\dot{u}}|+s_0|\b{\dot{v}}|, |\b{\dot{u}}|)
\leq d \left(|\b{\dot{u}}|,\ephi \right)+ s_0 \|\b{\dot{v}}\orlnor < 1.
\]
As a consequence $|\b{\dot{u}}|+s_0|\b{\dot{v}}| \in \Pi(\ephi,1) \subset \claseor$. Then $b+\Phi(|\b{\dot{u}}|+s_0|\b{\dot{v}}|) \in L^1$ and since $\b{v} \in L^{\infty}$ we have that
$(b+\Phi(|\b{\dot{u}}|+s_0|\b{\dot{v}}|))|\b{v}| \in L^1$. On the other hand, from Lemma \ref{phi_comp} we obtain $c(t)+ \varphi(|\b{\dot{u}}|+s_0|\b{\dot{v}}|) \in L^{\Psi}$ and since $\b{\dot{v}} \in L^{\Phi}$, applying the H\"older inequality
$(c(t)+ \varphi(|\b{\dot{u}}|+s_0|\b{\dot{v}}|))|\b{\dot{v}}| \in L^1$. Thus, from \eqref{ctg} and the above discussion there exists a function $g \in L^1([0,T], \mathbb{R}^{+})$
such that $|D_s f(s,t)| \leq g(t)$. Consequently, $I$ has a directional derivative and
\[
\langle v, I'(\b{u}) \rangle=\frac{d}{ds}I(\b{u}+sv)\big|_{s=0}=\int_0^T \left(\langle D_{x}\mathcal{L}(t,\b{u},\b{\dot{u}}), \b{v}\rangle+ \langle D_{y}\mathcal{L}(t,\b{u},\b{\dot{u}}),\b{\dot{v}}\rangle\right) \ dt.
\]
Moreover, from \eqref{DxL1}, \eqref{DyLpsi}, Lemma \ref{inclusion orlicz} and previous formula
\[
|\langle I'(\b{u}), \b{v} \rangle| \leq c \|v\linf + c \|\b{\dot{v}}\orlnor \leq c \|\b{v}\sobnor.
\]
This complete the proof of the G\^ateaux differentiability of $I$. Finally, the continuity of $I': \left(\domi, \|\cdot \sobnor\right) \to \left(\left[\wphi
\right]^*, w^* \right)$ is a consequence of item \ref{T1item2}. \textcolor{red}{Erica desarrollara este punto}

In order to prove  \ref{T1item4}, let us see that the maps $\b{u}\mapsto D_{\b{x}}\mathcal{L}(\cdot,\b{u}(\cdot),\b{\dot{u}}(\cdot))$  and $\b{u}\mapsto D_{\b{y}}\mathcal{L}(\cdot,\b{u}(\cdot),\b{\dot{u}}(\cdot))$  are continuous
from $\left(\wphi, \|\cdot \sobnor\right) $ into $\left( L^1, \|\cdot \|_{L^1}\right)$ and
 $\left(\lpsi,\|\cdot\|_{L^{\Psi}}\right)$ respectively.  The continuity of the first map is an immediate consequence of the step 1 in the proof of item \ref{T1item3} and the fact that $\Pi(\ephi,1) =\lphi$  when $\Phi$ is of the $\Delta_2$ class. We will prove the continuity of the second map.  We consider $\b{u}_n$ and $\b{u}$ with $\|\b{u}_n- \b{u}\sobnor\to 0$.
By Lemma \ref{segundo lema}, there exist a subsequence $\b{u}_{n_k}$ and $h\in \lphi$ such that $\b{u}_{n_k}\rightarrow \b{u} \quad\text{a.e.}$, $\b{\dot{u}}_{n_k}\rightarrow \b{\dot{u}} \quad\text{a.e.}$ and $|\b{\dot{u}}_{n_k}|\leq h\quad\text{a.e.}$.
 Then  since $\mathcal{L}$ is a Caratheodory function
 we have $ D_{\b{y}}\mathcal{L}(t,\b{u}_{n_k}(t),\b{\dot{u}}_{n_k}(t))\to D_{\b{y}}\mathcal{L}(t,\b{u}(t),\b{\dot{u}}(t))$ a.e. $t\in [0,T]$.  From Lemma \ref{inclusion orlicz} Corollary \ref{a_bound} and the fact that $\|\b{u}_{n_k}\sobnor$ are uniformly bounded, we get a constant $C>0$ such that  $\|\b{a}(\b{u}_{n_k})\|_{L^{\infty}}\leq C$. 
 By using \eqref{cotaDyL} and as $\Psi$ is of the $\Delta_2$ class, we obtain 
 \[\begin{split}
    |D_{\b{y}}\mathcal{L}(t,\b{u}_{n_k}(t),\b{\dot{u}}_{n_k}(t))| &\leq a(|\b{u}_{n_k}|)\left( c(t) + \varphi (|\b{\dot{u}}_{n_k}(t)|)\right)\\
    &\leq C\left( c(t) + \varphi (|h|)\right)\in \lpsi
   \end{split}
\]
Therefore, invoking  Lemma \ref{lema_conv_may}, we have proved that
  of all sequence $\b{u}_n$ which converge to $\b{u}$ in $\wphi$ we can
extract a subsequence with $\|D_{\b{y}}\mathcal{L}(t,\b{u}_{n_k},\b{\dot{u}}_{n_k})-D_{\b{y}}\mathcal{L}(t,\b{u},\b{\dot{u}})\|_{\lpsi}\to 0$. The desired result follows from a standard argument.

\[
  \begin{split}
    \|I'(\b{u})&-I'(\b{u}_0)\|_{\left[\wphi\right]^*}=\sup_{v\in \wphi,\|v\sobnor\leq 1} \langle I'(\b{u})-I'(\b{u}_0),v \rangle\\
      &=\sup_{v\in \wphi, \|v\sobnor\leq 1} \int_0^T \left\{ \left(D_{\b{x}}\mathcal{L}(t,\b{u}(t),\b{\dot{u}}(t))-D_{\b{x}}\mathcal{L}(t,\b{u}_0(t),\dot{\b{u}_0}(t))\right)\cdot v(t)\right.\\
      &\quad\left.+
      \left(D_{\b{y}}\mathcal{L}(t,\b{u}(t),\b{\dot{u}}(t))-D_{\b{y}}\mathcal{L}(t,\b{u}_0(t),\dot{\b{u}_0}(t))\right)\cdot \b{\dot{v}}(t)\right\}dt \\
      &\leq \sup_{v\in \wphi, \|v\sobnor\leq 1} \left\{\| D_{\b{x}}\mathcal{L}(\cdot,\b{u}(\cdot),\b{\dot{u}}(\cdot))-D_{\b{x}}\mathcal{L}(\cdot,\b{u}_0(\cdot),\dot{\b{u}_0}(\cdot))\|_{L^1}\|v\|_{L^{\infty}}\right.\\
      &\quad+\left.
       \| D_{\b{y}}\mathcal{L}(\cdot,\b{u}(\cdot),\b{\dot{u}}(\cdot))-D_{\b{y}}\mathcal{L}(\cdot,\b{u}_0(\cdot),\dot{\b{u}_0}(\cdot))\|_{\lpsi}\|\b{\dot{v}}\|_{\lphi}\right\}\\
       &\leq C\left(\| D_{\b{x}}\mathcal{L}(\cdot,\b{u}(\cdot),\b{\dot{u}}(\cdot))-D_{\b{x}}\mathcal{L}(\cdot,\b{u}_0(\cdot),\dot{\b{u}_0}(\cdot))\|_{L^1}\right.\\
       &\left. \quad +
       \| D_{\b{y}}\mathcal{L}(\cdot,\b{u}(\cdot),\b{\dot{u}}(\cdot))-D_{\b{y}}\mathcal{L}(\cdot,\b{u}_0(\cdot),\dot{\b{u}_0}(\cdot))\|_{\lpsi}\right)
  \end{split}
\]
Therefore the results follows  of the  previously established continuity for $D_{\b{x}}\mathcal{L}$ and $D_{\b{y}}\mathcal{L}$.

\end{proof}






We recall that the definition of a strictly convex function, that is a function $f: \mathbb{R}^n \to \mathbb{R}$ such that
\[
f\left(\frac{x+y}{2}\right)< \frac{1}{2} \left(f\left( x\right)+f\left( y\right)\right).
\]


In this paper, we will consider the following problem


\begin{equation}\label{ecualagran}
    \left\{%
\begin{array}{ll}
   \frac{d}{dt} D_{y}\mathcal{L}(t,\b{u}(t),\b{\dot{u}}(t))= D_{\b{x}}\mathcal{L}(t,\b{u}(t),\b{\dot{u}}(t)) \quad \hbox{a.e.}\ t \in (0,T)\\
    \b{u}(0)-\b{u}(T)=\b{\dot{u}}(0)-\b{\dot{u}}(T)=0.
\end{array}%
\right.
\end{equation}

We denote by $\wphi_T$ the subspace of $\wphi$ of all functions $T$-periodic. Similarly we consider the subspaces $\ephi_T$, $\lphi_T$. As is usual, when $Y$ is a subspace of
the Banach space $X$, we denote by $Y^{\perp}$ the subspace of $X^*$ of all the functions which are identically zero on $Y$.

\begin{cor} The following statements are equivalent
\begin{enumerate}
 \item $I'(\b{u})\in\left( \wphi_T\right)^{\perp}$
 \item  $D_{\b{y}}\mathcal{L}(t,\b{u}(t),\b{\dot{u}}(t))dt$ is an absolutely continuous function and $\b{u}$ solve the following boundary value problem
 \begin{equation}\label{ecualagran2}
    \left\{%
\begin{array}{ll}
   \frac{d}{dt} D_{y}\mathcal{L}(t,\b{u}(t),\b{\dot{u}}(t))= D_{\b{x}}\mathcal{L}(t,\b{u}(t),\b{\dot{u}}(t)) \quad \hbox{a.e.}\ t \in (0,T)\\
    \b{u}(0)-\b{u}(T)=D_{\b{y}}\mathcal{L}(0,\b{u}(0),\b{\dot{u}}(0))-D_{\b{y}}\mathcal{L}(T,\b{u}(T),\b{\dot{u}}(T))=0.
\end{array}%
\right.
\end{equation}

\end{enumerate}
Moreover if $D_{\b{y}}\mathcal{L}(t,x,y)$ is $T$-periodic with respect to the variable $t$ and strictly convex with respect to $y$, then
$D_{\b{y}}\mathcal{L}(0,\b{u}(0),\b{\b{\dot{\b{u}}}}(0))-D_{\b{y}}\mathcal{L}(T,\b{u}(T),\b{\dot{u}}(T))=0$ is equivalent to $\b{\dot{u}}(0)=\b{\dot{u}}(T)$.

\end{cor}

\begin{proof} The condition $I'(\b{u})\in\left( \wphi_T\right)^{\perp}$ means that for every $v\in \wphi_T$ we have $\langle I'(\b{u}),v\rangle=0$. According to Theorem
\ref{teorema_acotacion} we have

\[\int_0^TD_{\b{y}}\mathcal{L}(t,\b{u}(t),\b{\dot{u}}(t))\cdot \b{\dot{v}}(t)dt=-\int_0^TD_{\b{x}}\mathcal{L}(t,\b{u}(t),\b{\dot{u}}(t))\cdot v(t)dt \]
Using \cite[pag. 6]{mawhin2010critical} we obtain that $D_{\b{y}}\mathcal{L}(t,\b{u}(t),\b{\dot{u}}(t))$ is absolutely continuous, therefore it is differentiable a.e.on $[0,T]$ and that the first equality
of \ref{ecualagran} holds true. Moreover the same reference says that $D_{\b{y}}\mathcal{L}(t,\b{u}(t),\b{\dot{u}}(t))$ is $T$-periodic.
This complete the proof of 1. implies 2. The converse is still easier than the previous one, we omit the proof of it.

The last part of the Corollary is a consequence of that $D_{\b{y}}\mathcal{L}(T,\b{u}(T),\b{\dot{u}}(T))=D_{\b{y}}\mathcal{L}(0,\b{u}(0),\b{\dot{u}}(0))=D_{\b{y}}\mathcal{L}(T,u(T),\b{\dot{u}}(0))$ and the well known fact strictly
convexity implies injectivity (see, for instance \cite[Theorem 12.17]{rockafellar2009variational}).
\end{proof}


 \begin{lem}\label{unif_conv}
{\color{red} Proposicion 1.2 de Mawhin y Willem, Ejercicio Erica} If the sequence $\{\b{u}_{n}\}_{n \geq 1}$ converges weakly to $\b{u}$ in $\wphi$, then $\{\b{u}_{n}\}_{n\geq 1}$ converges
uniformly to $\b{u}$ on $[0,T]$.
\end{lem}
\begin{proof}
By Lemma \ref{inclusion orlicz}, the injection of $\wphi$ in $L^{\infty}$ is continuous. Since $\b{u}_{n}\rightharpoonup \b{u}$ in $\wphi$ it follows that
$\b{u}_{k}\rightharpoonup \b{u}$ in $L^{\infty}$ ($\langle \b{u}_{n}-\b{u}, f\rangle_{L^{\infty},[L^{\infty}]^*} \leq \|\b{u}_{n}-\b{u}\|_{L^{\infty}} \|f\|_{[L^{\infty}]^*} \leq C
\|\b{u}_{n}-\b{u}\|_{\wphi}\|f\|_{[L^{\infty}]^*} \rightarrow 0$)
\end{proof}

\begin{lem}
We suppose that $\mathcal{L}(t,x,y)$ is a Charateodory functions satisfying \eqref{cotaL}-\eqref{cotaDyL}. Moreover we assume $\mathcal{L}(t,x,\cdot)$ is convex for each $t,x$. We suppose that $\Phi,\Psi\in\Delta_2$.  
Then the functional \eqref{integral_accion} is weakly lower semicontinuous (w.l.s.c.). 
\end{lem}


\begin{proof} 

\textcolor{red}{Necesito la proposicion 1.2 de Mawhin y Willem pero para espacios de Orlicz. Quien se le anima?}

We fix any $\b{u}\in\wphi$. What we must prove that for any sequence $\{\b{u}_n\}$ with $\b{u}_n\rightharpoonup \b{u}$ in $\wphi$ we have that $I(\b{u})\leq \liminf_n I(\b{u}_n)$. We write
\[
\begin{split}I(v)&=\int_0^T\mathcal{L}(t,v(t),\b{\dot{v}}(t))dt\\
 &=\int_0^T\mathcal{L}(t,v(t),\b{\dot{v}}(t))-\mathcal{L}(t,\b{u}(t),\b{\dot{v}}(t))dt +\int_0^T\mathcal{L}(t,\b{u}(t),\b{\dot{v}}(t))dt\\
 &=:J(v)+H(v).
\end{split}
\]
 As $\{\b{u}_n\}$ is a weakly convergent sequence,  by the Lemma XXX  we have that $\b{u}_n\to \b{u}$ in $L^{\infty}$. By the mean value theorem for derivatives, we obtain 
 a function $\xi_n(t)$, with $\xi_n(t)$ belonging to line segment joyning $\b{u}_n(t)$ and $\b{u}(t)$, such that
 \begin{equation}\label{opJ}
 \begin{split}
   &\left|  \mathcal{L}(t,\b{u}_n(t),\b{\dot{u}}_n(t))-\mathcal{L}(t,\b{u}(t),\b{\dot{u}}_n(t))\right|\\
  &\leq|D_{\b{x}}\mathcal{L}(t,\xi_n(t),\b{\dot{u}}_n(t))||u_n(t)-u(t)|.
  \end{split}
 \end{equation}

The functions $u_n$, and therefore the functions $\xi_n$, are uniformly bounded in $L^{\infty}$. Thus, there exists $C>0$ such that $a(|\xi_n(t)|)\leq C $. Then, 
using \eqref{cotaDxL} we get

\begin{equation}\label{acot_Dx}
 \left|D_{\b{x}}\mathcal{L}(t,\xi_n(t),\b{\dot{u}}_n(t))\right|\leq C \left(b(t)+\Phi(|\b{\dot{u}}_n(t))|)\right)
\end{equation}
Since $\Phi\in\Delta_2$ we have that the operator $v\mapsto\Phi(v)$  acts from $\lphi$ in $L^1$. Therefore, by 
\cite[Lemma 17.4]{KR} we have that $\{\Phi(v): \|v\|_{\lphi}\leq r\}$ is bounded in $ L^1$ for any $r>0$. This fact implies that
there exists a constant $C>0$ such that $\|\Phi(|\b{\dot{u}}_n(t))|)\|_{L^1}\leq C$. Then, from 
\eqref{opJ}, \eqref{acot_Dx}, H\"older inequality and since $\|u_n-u\|_{L^{\infty}}\to 0$ and $b\in L^1$ we get $J(u_n)\to 0 	$

Now we will prove that $H(v)$ is w.l.s.c. Since $H(v)$ is convex it is sufficient to prove that $H$ is l.s.c (see
\cite[Theorem 1.2]{mawhin2010critical}). We suppose that  $\|u_n- u\sobnor\to 0$. \textcolor{red}{Deber Fernando: $v_n$ tiende a v}.

There exists $s=s_{n,t}\in[0,1]$
such that
\[|\mathcal{L}(t,u(t),\b{\dot{u}}_n(t))-\mathcal{L}(t,u(t),\b{\dot{u}}(t))|\leq|D_{\b{y}}\mathcal{L}(t,u(t),(1-s)\b{\dot{u}}_n(t)+s \b{\dot{u}}(t))||\b{\dot{u}}_n-\b{\dot{u}}|.\]
Let $\mathfrak{G}$ be the set $\{|\b{\dot{u}}_n(t)|\geq |\b{\dot{u}}(t)|\}$. Then
\[|(1-s)\b{\dot{u}}_n(t)+ s\b{\dot{u}}(t)|\leq \max\{|\b{\dot{u}}_n(t)|,|\b{\dot{u}}(t)|\}=\chi_{\mathfrak{G}}(t)|\b{\dot{u}}_n(t)|+
 \chi_{\mathfrak{G}^c}(t)|\b{\dot{u}}(t)|
\]
Therefore, using \eqref{cotaDyL} and by the uniform boundedness of $a(|u_n(t)|)$ we get
\[\begin{split}|\mathcal{L}(t,u(t),\b{\dot{u}}_n(t))-\mathcal{L}(t,u(t),\b{\dot{u}}(t))|&\leq C\left(c(t)+\varphi(\chi_{\mathfrak{G}}|\b{\dot{u}}_n(t)|+
 \chi_{\mathfrak{G}^c}|\b{\dot{u}}(t)|)\right)|\b{\dot{u}}_n-\b{\dot{u}}|\\
 &=C\left(c(t)+\varphi(\chi_{\mathfrak{G}}|\b{\dot{u}}_n(t)|)+\varphi(
 \chi_{\mathfrak{G}^c}|\b{\dot{u}}(t)|)\right)|\b{\dot{u}}_n-\b{\dot{u}}|\\
 &\leq C\left(c(t)+\varphi(|\b{\dot{u}}_n(t)|)+\varphi(
 |\b{\dot{u}}(t)|)\right)|\b{\dot{u}}_n-\b{\dot{u}}|.
 \end{split}
 \]
 
 Now, in virtue of \cite[Lemma 9.1]{KR}, \cite[Lemma 17.1]{KR}, \cite[Theorem 17.4]{KR} and the uniform boundedness of $\b{\dot{u}}_n$ in 
 $\lphi$ we have
 \[|H(u_n)-H(u)|\leq C\|\b{\dot{u}}_n-\b{\dot{u}}\orlnor\to 0.\]
 Which completes the proof (o eso me gustara creer a mi)
 

\end{proof}





\bibliographystyle{plain}
\bibliography{biblio}
\end{document}
