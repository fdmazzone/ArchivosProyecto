\documentclass[twoside]{article}


%\usepackage{hyperref}
\usepackage{amssymb,amsthm}
\usepackage{amsmath}
\usepackage{color}
\usepackage{ esint }
%\usepackage{graphicx}
%\usepackage{wrapfig}
%\usepackage{subfigure}
\usepackage{fancyhdr}
\usepackage{times}
%\usepackage{theorem}
\usepackage[latin1]{inputenc}
%\usepackage{showkeys}
\usepackage{comment}
\usepackage{url}
\usepackage{xcolor}
\usepackage{adjustbox}
%Teorema y similes
\usepackage[maxnames=6,backend=bibtex]{biblatex}
\bibliography{biblio.bib}

\definecolor{rosa}{rgb}{1,0.3,0.9}
\definecolor{violeta1}{rgb}{0.5,0.3,0.5}
\definecolor{violeta}{rgb}{0.5,0.1,0.5}
\definecolor{negro}{rgb}{0.5,0.2,0.4}
\definecolor{celeste}{rgb}{0.1,0.4,1}
\definecolor{naranja}{rgb}{1,0.5,0}
\definecolor{color_nota_fer}{HTML}{DEBFDB}


\newenvironment{colbox}[2]{%
    \begin{adjustbox}{minipage={\linewidth},margin=1ex,bgcolor=#1,env=center}
        #2}{%
    \end{adjustbox}%
}
\newcounter{nota_fer_cont}
\newenvironment{nota_fer}[1]{\refstepcounter{nota_fer_cont}\begin{colbox}{color_nota_fer}{\textbf{Comentario Leo-Graciela-Fernando \arabic{nota_fer_cont}.} #1}}{\end{colbox}}


\newtheorem{thm}{Theorem}[section]
\newtheorem{cor}[thm]{Corollary}
\newtheorem{lem}[thm]{Lemma}
\newtheorem{rem}[thm]{Remark}
\newtheorem{defi}[thm]{Definition}
\newtheorem{prop}[thm]{Proposition}
\theoremstyle{remark}
\newtheorem{comentario}{Remark}


\title{Some existence results on periodic solutions of 
Euler-Lagrange equations in an Orlicz-Sobolev space setting}
\author{Sonia Acinas \thanks{SECyT-UNRC, UNSL and CONICET}\\
Instituto de Matem\'atica Aplicada San Luis (CONICET-UNSL)\\
(5700) San Luis, Argentina\\
Universidad Nacional de La Pampa\\
(6300) Santa Rosa, La Pampa, Argentina\\
\url{sonia.acinas@gmail.com}\\[3mm]
Leopoldo Buri \thanks{SECyT-UNRC}\\
Dpto. de Matem\'atica, Facultad de Ciencias Exactas, F\'{\i}sico-Qu\'{\i}micas y Naturales\\
Universidad Nacional de R\'{i}o Cuarto\\
(5800) R\'{\i}o Cuarto, C\'ordoba, Argentina,\\
\url{lburi@exa.unrc.edu.ar}\\[3mm]
Graciela Giubergia \thanks{SECyT-UNRC and CONICET}\\
Dpto. de Matem\'atica, Facultad de Ciencias Exactas, F\'{\i}sico-Qu\'{\i}micas y Naturales\\
Universidad Nacional de R\'{i}o Cuarto\\
(5800) R\'{\i}o Cuarto, C\'ordoba, Argentina,\\
\url{ggiubergia@exa.unrc.edu.ar}\\[3mm]
Fernando D. Mazzone \thanks{SECyT-UNRC and CONICET}\\
Dpto. de Matem\'atica, Facultad de Ciencias Exactas, F\'{\i}sico-Qu\'{\i}micas y Naturales\\
Universidad Nacional de R\'{i}o Cuarto\\
(5800) R\'{\i}o Cuarto, C\'ordoba, Argentina,\\
\url{fmazzone@exa.unrc.edu.ar}\\[3mm]
Erica L. Schwindt\thanks{ANR. AVENTURES - ANR-12-BLAN-BS01-0001-01}\\
Universit\'{e} d'{O}rl\'{e}ans, Laboratoire MAPMO, CNRS, UMR 7349, \\
F\'ed\'eration Denis Poisson, FR 2964,\\
B\^{a}timent de Math\'{e}matiques, BP 6759, 45067 Orl\'{e}ans Cedex 2, France,\\
\url{leris98@gmail.com}}

\date{}

\newcommand{\orlnor}{\|_{L^{\Phi}}}
\newcommand{\lurnor}{\|^{*}_{L^{\Phi}}}
\newcommand{\linf}{\|_{L^{\infty}}}
\newcommand{\lphi}{L^{\Phi}}
\newcommand{\lpsi}{L^{\Psi}}
\newcommand{\ephi}{E^{\Phi}}
\newcommand{\claseor}{C^{\Phi}}
\newcommand{\wphi}{W^{1}\lphi}
\newcommand{\sobnor}{\|_{W^{1}\lphi}}
\newcommand{\domi}{\mathcal{E}^{\Phi}_d(\lambda)}
\renewcommand{\b}[1]{\boldsymbol{#1}}
\newcommand{\rr}{\mathbb{R}}
\newcommand{\nn}{\mathbb{N}}
\newcommand{\ccdot}{\b{\cdot}}
\renewcommand{\leq}{\leqslant} 
\newcommand{\epsi}{E^{\Psi}}

\begin{document}



\maketitle
%
\begingroup%Locallizing the change to `thefootnote'.
    \renewcommand{\thefootnote}{}%Removing the footnote symbol.
    %
    \footnotetext{%
    %   2010 Mathematics Subject Classification
    %   http://www.ams.org/msc/
    \textbf{2010  AMS Subject Classification.} Primary: .
    Secondary: .
    }%
        \footnotetext{%
    \textbf{Keywords and phrases.}  .
    }%
    \endgroup
%
%
%
%

\begin{abstract}

In this paper we consider the problem of finding periodic solutions of certain Euler-Lagrange equations. We employ the direct method of the calculus of variations, that is we obtain solutions minimizing certain functional $I$. We give conditions which ensure that $I$ is finitely defined and   differentiable on certain subsets of  Orlicz-Sobolev spaces $W^1L^{\Phi}$ associated to an $N$-function $\Phi$. We show that, in some sense, it is necessary for the coercitivity of the functional $I$ that  the complementary function of $\Phi$ satisfy the $\Delta_2$ condition.  We conclude by discussing conditions for existence of minima for $I$. 


\end{abstract}




\pagestyle{fancy} \headheight 35pt \fancyhead{} \fancyfoot{}

\fancyfoot[C]{\thepage} \fancyhead[CE]{\nouppercase{S. Acinas, L. Buri, G. Giubergia, F. Mazzone and E. Schwindt}} \fancyhead[CO]{\nouppercase{\section}}

\fancyhead[CO]{\nouppercase{\leftmark}}


%\tableofcontents

\section{Introduction}
This paper is concerned with the existence of periodic solutions of the problem
\begin{equation}\label{ProbPrin}
    \left\{%
\begin{array}{ll}
   \frac{d}{dt} D_{y}\mathcal{L}(t,\b{u}(t),\b{\dot{u}}(t))= D_{\b{x}}\mathcal{L}(t,\b{u}(t),\b{\dot{u}}(t)) \quad \hbox{a.e.}\ t \in (0,T)\\
    \b{u}(0)-\b{u}(T)=\b{\dot{u}}(0)-\b{\dot{u}}(T)=0
\end{array}%
\right.
\end{equation}
where $T>0$, $\b{u}:[0,T]\to\rr^d$ is absolutely continuous and the \emph{Lagrangian} $\mathcal{L}:[0,T]\times\rr^d\times\rr^d\to\rr$ is a Carath\'eodory function satisfying the conditions
\begin{eqnarray}
|\mathcal{L}(t,\b{x},\b{y})| &\leq a(|\b{x}|)\left(b(t)+ \Phi\left(\frac{|\b{y}|}{\lambda}+f(t) \right)\right),\label{cotaL}\\
|D_{\b{x}}\mathcal{L}(t,\b{x},\b{y})| &\leq a(|\b{x}|)\left(b(t)+ \Phi\left(\frac{|\b{y}|}{\lambda}+f(t) \right)\right),\label{cotaDxL}\\
|D_{\b{y}}\mathcal{L}(t,\b{x},\b{y})| &\leq a(|\b{x}|)\left(c(t)+ \varphi\left(\frac{|\b{y}|}{\lambda}+f(t)\right)  \right).\label{cotaDyL}
\end{eqnarray}
In these inequalities we assume that  $a\in C(\mathbb{R}^+,\mathbb{R}^+)$, $\lambda>0$, $\Phi$ is an $N$-function (see section  Preliminaries  for definitions), $\varphi$ is the right continuous derivative of $\Phi$ and the non negative functions  $b,$ $c$ and $f$ belong to certain Banach spaces that  will be introduced later.



It is well known that problem \eqref{ProbPrin} comes from a variational one, that is,  a solution of \eqref{ProbPrin}  
is a critical point of the \emph{action integral}
\begin{equation}\label{integral_accion}
I(\b{u})=\int_{0}^T \mathcal{L}(t,\b{u}(t),\b{\dot{u}}(t))\ dt.
\end{equation}



Variational problems and hamiltonian systems  have been studied extensively. Classic references of these subjects are
\cite{mawhin2010critical,struwe2008variational,ekeland1999convex}. Problems like \eqref{ProbPrin} have maintained the interest of researchers as the recent literature on the topic testifies. For lagrangian functions of the type $\mathcal{L}(t,\b{x},\b{y})=\frac{|\b{y}|^2}{2}+F(t,\b{x})$  many solvability conditions have been given expanding the results in \cite{mawhin2010critical}.  
In  \cite{tang1995periodic} the function $F$
was split up into two potentials, one of them with a property of subadditivity and the other with a bounded
gradient.
%%
In \cite{tang1998periodic} it was required a certain sublinearity condition on the gradient of the potential $F$; and, 
in \cite{wu1999periodic} it was considered a potential $F$ given by a sum of a subconvex function and a  subquadratic one. 
%%
In  \cite{tang2001periodic} the uniform coercivity of $\int_0^T F(t,\b{x})\,dt$
was replaced by local coercivity of $F$ in some positive measure subset of $[0,T]$.
In \cite{zhao2004periodic}, the authors took a similar potential to that in \cite{wu1999periodic} getting new solvability conditions 
and they also studied the case in which the two potentials do not have any convexity.
%

The Lagrangian  $\mathcal{L}(t,\b{x},\b{y})=\frac{|\b{y}|^p}{p}+F(t,\b{x})$
for $p>1$ was treated in more recent papers.
By using the dual least action principle, in \cite{Tian2007192}
it was performed the extension of some results given in \cite{mawhin2010critical};
and, in \cite{tang2010periodic} the authors improved the work done in \cite{wu1999periodic}.
On the other hand, by the minimax methods in critical point theory 
some existence theorems were obtained.
In \cite{xu2007some} it was employed a subquadratic potential $F$ 
%which is subquadratic 
in Rabinowitz{}'s sense 
and in \cite{ye2008periodic} $F$ was taken as in \cite{tang1998periodic}.
%a sum of two potentials where one of them had a sublinear gradient.

Another source of problems, close to our proposal, is the one in which a \emph {p-laplacian-like} operator is involved.
Assuming that the function $\varphi$ is a homeomorphism  from $\rr^d$ into itself, it is considered the differential operator
$\b u \mapsto (\varphi(\b u'))'$. 
In  \cite{C-2007,C-2009, 
Cristia-2008, manasevich2000boundary, manasevich1998periodic}, using 
the Leray-Schauder degree theory,
some existence results of solutions of equations 
like 
$(\varphi({\b u}'))'={\b f}(t,{\b u}(t),{\b u}'(t))$ were obtained 
under different boundary conditions (periodic, Dirichlet, von Neumann) 
and where $\b f$ is not necessarily a gradient. 
We point out that our approach 
differs from that of previous articles because
we tackle the direct method of the calculus of variations.

  In the Orlicz-Sobolev space setting, 
	in \cite{m-1999} a constrained minimization problem associated to the existence of eigenvalues for certain differential operators involving $N$-functions was studied.  
Slightly away from the problems to be treated in this paper, 
we can mention \cite{cianchi2000local, cianchi1999gradient} where A. Cianchi dealt with 
the regularity of minimizers of action integrals defined on several variable functions.

In this article  we consider lagrangian functions  defined on Orlicz-Sobolev spaces $W^{1}L^{\Phi}$    
(see \cite{adams_sobolev,KR, rao1991theory,2002applications}) 
and we use  the direct method of calculus of variations. 
The exposition is organized as follows. 
In Section \ref{preliminares} we enumerate results related to Orlicz spaces, Orlicz-Sobolev spaces and composition operators. 
Almost all results in this section  are essentially known. Conditions \eqref{cotaL}, \eqref{cotaDxL} and \eqref{cotaDyL} are the means to ensure that $I$ is finitely defined on 
a non trivial subset of $W^{1}\lphi_d$ and $I$ is G\^ateaux differentiable in this subspace. 
We develop 
these issues in Theorem \ref{teorema_acotacion}  of Section \ref{sec:dif}. In Section \ref{sec:equa-min} we prove that critical points of \eqref{integral_accion}
 are solution of \eqref{ProbPrin}. Conditions for the coercitivity of action integrals are discussed in Section \ref{sec:coer}. Finally, our main theorem about existence of solutions of  \eqref{ProbPrin} is introduced and proved in Section \ref{sec:main}. 

We put an emphasis on that we take care of using $\Delta_2$-condition. 
In some results where we have used it,
we show that $\Delta_2$-condition is necessary in a certain sense (see, for example, Lemma \ref{lem_coer}).
   




\section{Preliminaries}\label{preliminares}

%\subsection{$N$-functions}
For reader convenience, we give a short introduction to Orlicz and Orlicz-Sobolev spaces of vector valued functions and a  list  of results that we will use throughout the article. 
Classic references for Orlicz spaces of real valued functions are \cite{adams_sobolev,KR,rao1991theory}.
For  Orlicz spaces of vector valued functions, see \cite{Orliczvectorial2005} and the references therein.

Hereafter we denote  by $\mathbb{R}^+$  the set of all non negative real numbers. A function $\Phi:\mathbb{R}^+\to \mathbb{R}^+ $ is called an \emph{$N$-function} if $\Phi$ is given by 
\[
\Phi(t)=\int_{0}^t \varphi(\tau)\ d\tau,\quad\hbox{for } t\geq 0,
\]
where $\varphi:\mathbb{R}^+\rightarrow \mathbb{R}^+$ is a right continuous nondecreasing function  satisfying   $\varphi(0)=0$, $\varphi(t)>0$ for $t>0$ and
$\lim_{t\rightarrow \infty}\varphi(t)=+\infty$.

Given a function $\varphi$ as above, we  consider the so-called right inverse function $\psi$ of $\varphi$ which is 
defined by $\psi(s)=\sup_{\varphi(t)\leq s}t$.
The function $\psi$ satisfies the same properties as the function $\varphi$, therefore we have an $N$-function $\Psi$ such that $\Psi'=\psi$ .
 The function $\Psi$ is called the \emph{complementary function} of $\Phi$.


We say that $\Phi$ satisfies the  \emph{$\Delta_2$-condition}, denoted by $\Phi \in \Delta_2$, 
if there exist  constants $K>0$ and  $t_0\geq 0$ such that 
\begin{equation}\label{delta2defi}\Phi(2t)\leq K\Phi(t)
\end{equation}
for every $t\geq t_0$. 
If $t_0=0$,  we say that $\Phi$ satisfies the \emph{$\Delta_2$-condition globally} ($\Phi \in \Delta_2$ globally).  

% and plain symbols indicate scalars.

Let $d$ be a positive integer. We denote by $\mathcal{M}_d:=\mathcal{M}_d([0,T])$ the set of all measurable functions defined on $[0,T]$ with values on $\mathbb{R}^d$ and  we write $\b{u}=(u_1,\dots,u_d)$ for  $\b{u}\in \mathcal{M}_d$.
In this paper we adopt the convention that bold symbols denote points in $\mathbb{R}^d$.


Given  an $N$-function $\Phi$ we define the \emph{modular function} 
$\rho_{\Phi}:\mathcal{M}_d\to \mathbb{R}^+\cup\{+\infty\}$ by
\[\rho_{\Phi}(\b{u}):= \int_0^T \Phi(|\b{u}|)\ dt.\]
Here $|\cdot|$ is the euclidean norm of $\mathbb{R}^d$.
The \emph{Orlicz class} $C_d^{\Phi}=C_d^{\Phi}([0,T])$  is given  by
\begin{equation}\label{claseOrlicz}
  C^{\Phi}_d:=\left\{\b{u}\in \mathcal{M}_d | \rho_{\Phi}(\b{u})< \infty \right\}.
\end{equation}
The \emph{Orlicz space} $\lphi_d=L^{\Phi}_d([0,T])$ is the linear hull of $\claseor_d$;
equivalently,
\begin{equation}\label{espacioOrlicz}
\lphi_d:=\left\{ \b{u}\in \mathcal{M}_d | \exists \lambda>0: \rho_{\Phi}(\lambda \b{u}) < \infty   \right\}.
\end{equation}
  The Orlicz space $\lphi_d$ equipped with the \emph{Orlicz norm}
\[
\|  \b{u}  \orlnor:=\sup \left\{  \int_0^T \b{u}\b{\cdot} \b{v}\ dt \big| \rho_{\Psi}(\b{v})\leq 1\right\},
\]
is a Banach space. By $\b{u}\b{\cdot} \b{v}$ we denote the usual dot product in $\mathbb{R}^{d}$ between $\b{u}$ and $\b{v}$.  
The following alternative expression for the norm, known as \emph{Amemiya norm},     will  be useful (see \cite[Thm. 10.5]{KR} and \cite{hudzik2000amemiya}). For every $\b{u}\in\lphi$,

\begin{equation}\label{amemiya}
\|\b{u}\orlnor=\inf\limits_{k>0}\frac{1}{k}\left\{1+\rho_{\Phi}(k\b{u})\right\}.
\end{equation}



The subspace $\ephi_d=\ephi_d([0,T])$ is defined as the closure in $\lphi_d$ of the subspace $L^{\infty}_d$ of all $\mathbb{R}^d$-valued essentially bounded functions. It is shown that  $\ephi_d$ is the only one maximal subspace contained in the Orlicz class $\claseor_d$, i.e. 
$\b{u}\in\ephi_d$ if and only if $\rho_{\Phi}(\lambda \b{u})<\infty$ for any $\lambda>0$.  

A generalized version of \emph{H\"older's inequality} holds in Orlicz spaces. Namely, if $\b{u}\in\lphi_d$ and $\b{v}\in\lpsi_d$ then $\b{u}\ccdot\b{v}\in L_1^1$ and
\begin{equation}\label{holder}
\int_0^T\b{v}\ccdot\b{u}\ dt\leq \|\b{u}\orlnor\|\b{v}\|_{L^{\Psi}}.
\end{equation}




If $X$ and $Y$ are  Banach spaces, with $Y\subset X^*$ we denote by $\langle\cdot,\cdot\rangle:Y\times X\to\mathbb{R}$ the bilinear pairing  map given by $\langle x^*,x\rangle=x^*(x)$. H\"older's inequality shows that $\lpsi_d\subset \left[\lphi_d\right]^*$, where the pairing  
$\langle \b{v}, \b{u}\rangle$
is defined by 
\begin{equation}\label{pairing}
  \langle \b{v},\b{u}\rangle=\int_0^T\b{v}\ccdot\b{u}\ dt
\end{equation}
with  $\b{u}\in\lphi_d$ and $\b{v}\in\lpsi_d$.
 Unless $\Phi \in \Delta_2$, the relation $\lpsi_d= \left[\lphi_d\right]^*$ will not hold. In general, it is true  that  $\left[\ephi_d\right]^*=\lpsi_d$.


Like in \cite{KR}, we will consider the subset $\Pi(\ephi_d,r)$ of $\lphi_d$ given by
\[\Pi(\ephi_d,r):=\{\b{u}\in\lphi_d| d(\b{u},\ephi_d)<r\}.\]
This set is related to the Orlicz class $\claseor_d$ by means of inclusions, namely,
\begin{equation}\label{inclusiones}\Pi(\ephi_d, r )\subset r \claseor_d\subset\overline{\Pi(\ephi_d,r)}
\end{equation}
for any positive $r$.
If $\Phi \in \Delta_2$,  then the sets $\lphi_d$, $\ephi_d$, $\Pi(\ephi_d,r)$ and $\claseor_d$ are equal.

%Frequently, we will use the following elementary fact 
%\begin{equation}\label{inclusion2}
%\b{u}\in\Pi(\ephi_d,\lambda)\implies \frac{\b{u}}{\lambda}\in\Pi(\ephi_d,1)\subset\claseor_d.
%\end{equation}

We define the \emph{Sobolev-Orlicz space} $\wphi_d$ (see \cite{adams_sobolev}) by
\[\wphi_d:=\{\b{u}| \b{u} \hbox{ is absolutely continuous and } \b{u},\b{\dot{u}}\in \lphi_d\}.\]
$\wphi_d$ is a Banach space when equipped with the norm
\[
\|  \b{u}  \|_{\wphi}= \|  \b{u}  \|_{\lphi} + \|\b{\dot{u}}\orlnor.
\]



For a  function $\b{u}\in L^1_d([0,T])$, we write $\b{u}=\overline{\b{u}}+\widetilde{\b{u}}$ where $\overline{\b{u}} =\frac1T\int_0^T \b{u}(t)\ dt$ and $\widetilde{\b{u}}=\b{u}-\overline{\b{u}}$.

As usual, if $(X,\|\cdot\|_X)$ is a Banach space and $(Y,\|\cdot \|_Y)$ is a subespace of $X$,  we write $Y\hookrightarrow X$ and we say that $Y$ is \emph{embedded} in $X$  when the restricted identity map $i_Y:Y\to X$ is bounded. That is, there exists $C>0$ such that  for any $y\in Y$ we have $\|y\|_X\leq C\|y\|_Y$.  With this notation, H\"older's inequality states that  $\lpsi_d\hookrightarrow  \left[\lphi_d\right]^*$. It is easy to see that for every $N$-function $\Phi$ we have that $L^{\infty}_d\hookrightarrow\lphi_d \hookrightarrow L^1_d$.


 As usual, a function   $w:\mathbb{R}^+\to \mathbb{R}^+$ is called  a \emph{modulus of continuity} if $w$ is a continuous increasing function which satisfies $w(0)=0$. For example, it can be proved easily that $w(s)=s\Phi^{-1}(1/s)$ is a modulus of  continuity for every $N$-function $\Phi$.  We say that $\b{u}:[0,T]\to\rr^d$  has modulus of continuity $w$  when there exists a constant $C>0$ such that 
\begin{equation}\label{w-holder}|\b{u}(t)-\b{u}(s)|\leq Cw(|t-s|).
\end{equation}
% The inequality \eqref{in-sob-cont}
% establishes that $w(s):=s\Phi^{-1}(1/s)$ is a modulus of continuity for all functions $\b{u}\in\wphi$. 
% Since $\Phi$ is an $N$-function then $w(0)=0$.
% and, from the concavity of $\Phi^{-1}$, we have that $w$ is increasing.
% Inequalies \eqref{in-sob-cont} and \eqref{wirtinger}  proves that the embedding  $\wphi_d\hookrightarrow C^w([0,T],\rr^d)$ holds, where 
We denote by $C^w([0,T],\rr^d)$  the space of  $w$-H\"older continuous functions. This is the space of all functions satisfying \eqref{w-holder} for some $C>0$. It is a Banach space with norm
\[\|\b{u}\|_{  C^w([0,T],\rr^d) }  :=\|\b{u}\|_{L^{\infty}}+\sup\limits_{t\neq s}\frac{|\b{u}(t)-\b{u}(s)|}{w(|t-s|)}.\]





 An important aspect of the theory of Sobolev spaces is related to embedding theorems. There is an extensive literature on this question in the  Orlicz-Sobolev space setting, see for example
 \cite{cianchi2000fully,cianchi1999some,claverooptimal,edmunds2000optimal,kerman2006optimal}.
The following simple lemma is essentially known and we will use it systematically. For the sake of completeness, we include a brief proof of it.



\begin{lem}\label{inclusion orlicz} Let  $w(s):= s\Phi^{-1}(1/s)$. Then the following statements holds
\begin{enumerate}
\item\label{inclusion orlicz_item1} $\wphi\hookrightarrow C^w([0,T],\rr^d) $ and for every $\b{u}\in\wphi$
\begin{align}
 &\left|\b{u}(t)-\b{u}(s) \right| \leq  \|\b{\dot{u}}\orlnor w(| t-s|),&\label{in-sob-cont}
\\
& \|\b{u}\|_{L^{\infty}} \leq\Phi^{-1}\left(\frac{1}{T}\right)\max\{1,T\}\|\b{u}\sobnor&\text{(Sobolev's inequality),}\label{sobolev}
\end{align}
\item For every $\b{u}\in\wphi$ we have $\widetilde{\b{u}}\in L^{\infty}_d$ and 
\begin{align}
& \|\widetilde{\b{u}}\|_{L^{\infty}} \leq T\Phi^{-1}\left(\frac{1}{T}\right)\|\b{\dot u}\orlnor&\text{  (Wirtinger's inequality),}\label{wirtinger}
\end{align}




\end{enumerate}
\end{lem}

\begin{proof}
For $0 \leq
s\leq t \leq T $, we get
\begin{equation}\label{equicont}
\begin{split}
\left|\b{u}(t)-\b{u}(s) \right| &\leq \int_{s}^t \left| \b{\dot{u}}(\tau)\right|\ \ d\tau\\
&\leq \| \chi_{[s,t]}\|_{\lpsi}\|\b{\dot{u}}\|_{\lphi}\\
&= \|\b{\dot{u}}\|_{\lphi} ( t-s)\Phi^{-1}\left(\frac{1}{t-s}\right),
\end{split}
\end{equation}
using H\"older's inequality and \cite[Eq. (9.11)]{KR}.
This proves the inequality \eqref{in-sob-cont}.

Since $u_i$ is continuous, from Mean Value Theorem for integrals, 
there exists  $s_i\in [0,T]$ such that $u_i(s_i)=\overline{u}_i$.
Using this $s_i$ value in \eqref{in-sob-cont} with $u_i$ instead of $\b{u}$ and taking into account that $s\Phi^{-1}(1/s)$ is increasing, 
we obtain  Wirtinger's inequality for each $u_i$. The inequality \eqref{wirtinger} 
follows easily from the corresponding result for each component of $\b{u}$.

On the other hand, again by H\"older's inequality and \cite[Eq. (9.11)]{KR}, we have
\begin{equation}\label{desigualdad2}\begin{split}
|\overline{\b{u}}|= \frac{1}{T}\int\limits_{0}^{T}|\b{u}(s)|ds\leq \Phi^{-1}\left(\frac{1}{T}\right)\|\b{u}\orlnor.
\end{split}
\end{equation}
From \eqref{wirtinger}, \eqref{desigualdad2} and the fact that $\b{u}=\overline{\b{u}}+\widetilde{\b{u}}$,  we obtain \eqref{sobolev}. This completes the proof of item \ref{inclusion orlicz_item1}.
\end{proof}

\begin{comentario}
As a consequence of previous Lemma  there exist a constant $C$, only dependent on $T$, such that for every $\b{u}\in\wphi_d$ we have that

\begin{equation}\label{cota_prome}\|\b{u}\sobnor\leq C\left(|\b{\overline{u}}|+\|\b{\dot{u}}\orlnor\right).\end{equation}

\end{comentario}

The Arzela-Ascoli Theorem implies that  $C^w([0,T],\rr^d)\hookrightarrow C([0,T],\rr^d)$ is a compact  embedding (see \cite[Ch. 5]{driver} for the case $w(s)=|s|^{\alpha}$, $0< \alpha\leq 1$ and if  $w$ is arbitrary, the proof follows with some obvious modifications). Therefore we have the following result.

 \begin{cor}\label{unif_conv} Every bounded sequence $\{\b{u}_n\}$ in  $\wphi_d$  has an uniformly convergent subsequence. 
\end{cor}



 Given a continuous function $a\in C(\mathbb{R}^+,\mathbb{R}^+)$, we define the composition operator $\b{a}:\mathcal{M}_d\to \mathcal{M}_d$ by $\b{a}(\b{u})(t)= a(|\b{u}(t)|)$.
We will often use the following elementary consequence of the previous lemma. 
\begin{cor}\label{a_bound} If $a\in C(\mathbb{R}^+,\mathbb{R}^+)$ then $\b{a}:\wphi_d\to L^{\infty}_1([0,T])$ is bounded. 
More concretely,  there exists a non decreasing function $A:\mathbb{R}^+\to\mathbb{R}^+$ such that
 $\|\b{a}(\b{u})\|_{L^{\infty}([0,T])}\leq A(\|\b{u}\|_{\wphi})$.
\end{cor}

\begin{proof}  Let $\alpha\in C(\mathbb{R}^+,\mathbb{R}^+)$ be a  non-decreasing majorant of $a$, for example 
$\alpha(s):=\sup_{0\leq t\leq s}a(t)$.  If $\b{u}\in \wphi_d$ then, by Lemma \ref{inclusion orlicz}, 
\[a(|\b{u}(t)|)\leq \alpha(\|\b{u}\|_{L^{\infty}})\leq 
\alpha\left(\Phi^{-1}\left(\frac{1}{T}\right)\max\{1,T\} \|\b{u}\|_{\wphi}\right)=: 
A(\|\b{u}\|_{\wphi}).\]
\end{proof}


The following lemma is an inmediate consequence of principles  related to  operators of Nemitskii type, see \cite[\textsection17]{KR}.

\begin{lem}\label{phi_comp}   
The  composition operator  $\boldsymbol{\varphi}$  acts from $\Pi(\ephi_d,1)$ into $C_1^{\Psi}$.
\end{lem}
\begin{proof}
  As a consequence of \cite[Lemma 9.1]{KR} we have that  $\boldsymbol{\varphi}\left(B_{\lphi}(0,1)\right)\subset C_1^{\Psi}$, where
$B_{X}(\b{u}_0,r)$ is the open ball with center $\b{u}_0$ and radius $r>0$ in the space $X$. Therefore, applying \cite[Lemma 17.1]{KR}, we deduce that $\boldsymbol{\varphi}$ acts from $\Pi(\ephi_d,1)$ into $C_1^{\Psi}$.
\end{proof}

We also need the following technical lemma.
\begin{lem}\label{segundo lema}
Let $\lambda>0$ and let $\{\b{u}_n\}_{n\in \mathbb{N}}$ be a sequence of  functions in $\Pi(\ephi_d,\lambda)$ converging to  $\b{u}\in \Pi(\ephi_d,\lambda)$  in the $\lphi$-norm. Then, there exist a subsequence
$\b{u}_{n_k}$ and a real valued function $h\in\Pi\left(\ephi_1\left([0,T]\right),\lambda\right)$ such that $\b{u}_{n_k}\rightarrow \b{u} \quad\text{a.e.}$ and $|\b{u}_{n_k}|\leq h\quad\text{a.e.}$
\end{lem}



\begin{proof}
Let $r:=d(\b{u},\ephi_d)$, $r<\lambda$. As $\b{u}_n$ converges to $\b{u}$, there exists a subsequence $(n_k)$ such that
\[\|\b{u}_{n_k}-\b{u}\orlnor<\frac{\lambda-r}{2}\quad \text{ and }\quad \|\b{u}_{n_k}-\b{u}_{n_{k+1}}\orlnor<2^{-(k+1)}(\lambda-r).\]
Let $h:[0,T]\rightarrow\mathbb{R}$ defined by
\begin{equation}\label{serie} h(x)=|\b{u}_{n_1}(x)|+\sum_{k=2}^{\infty}|\b{u}_{n_k}(x)-\b{u}_{n_{k-1}}(x)|.
\end{equation}
As a consequence  of \cite[Lemma 10.1]{KR} (see \cite[Thm. 5.5]{Orliczvectorial2005} for vector valued functions)  we have that $d(\b{v},\ephi_d)=d(|\b{v}|,\ephi_1)$ for any $\b{v}\in\lphi_d$. 
Now
\[d(|\b{u}_{n_1}|,\ephi_1)= d(\b{u}_{n_1},\ephi_d)\leq d(\b{u}_{n_1},\b{u})+d(\b{u},\ephi_d)<\frac{\lambda+r}{2}.\]
Then
\[d(h,\ephi_1)\leq d(h,|\b{u}_{n_1}|)+d(|\b{u}_{n_1}|,\ephi_1)< \lambda.\]
Therefore, $h\in\Pi(\ephi_1,\lambda)$ and  $|h|<\infty$ a.e. 
We conclude that the series  $\b{u}_{n_1}(x)+\sum_{k=2}^{\infty}(\b{u}_{n_k}(x)-\b{u}_{n_{k-1}}(x))$
is absolutely convergent a.e. and this fact implies that $\b{u}_{n_k}\rightarrow \b{u} \quad\text{a.e.}$ 
The inequality $|\b{u}_{n_k}|\leq h$ follows straightforwardly from the definition of $h$.
\end{proof}

A common obstacle in Orlicz spaces, that distinguishes them from $L^p$ spaces, is that a  sequence $\b{u}_n\in\lphi_d$ which is  uniformly bounded by $ h\in\lphi_1$ and a.e. convergent to $\b{u}$ is not necessarily norm convergent.
Fortunately, the subspace $\ephi_d$ has this property. 

\begin{lem}\label{lema_conv_may}
Suppose that $\b{u}_n \in\lphi_d$ is a sequence such that $\b{u}_n\to \b{u}$ a.e. and assume that there exist $h\in\ephi_1$ with $|\b{u}_n|\leq h$ a.e. 
then $\|\b{u}_n-\b{u}\orlnor\to 0$.
\end{lem}
\begin{proof}\cite[p. 84]{rao1991theory} and \cite[Thm. 10.3]{KR}.
\end{proof}


  We recall some useful concepts.

	\begin{defi} 
	Given a function $I:U\to\mathbb{R}$ where $U$ is an open set of a Banach space $X$,
we say that $I$ has a G\^ateaux derivative at $\b{u} \in U$ if there exists $\b{u}^*\in X^*$ such that for every $\b{v} \in X$
\[
\lim_{s \rightarrow 0}\frac{I(\b{u}+s\b{v})-I(\b{u}) }{s}=\langle \b{u}^* , \b{v}\rangle.
\]
See \cite{ambrosetti} for details. 
\end{defi}

%We recall the following definition. 
\begin{defi} Let $X$ be a Banach space and let $D\subset X$. A non linear operator $T:D\to X^*$ is called \emph{demicontinuous} if it is continuous when $X$ is equipped with the strong topology and $X^*$ with the weak${}^*$ topology 
(see \cite{kato1964demicontinuity}).
\end{defi} 

\section{Differentiability of action integrals in Orlicz spaces}\label{sec:dif}





We take a moment for  discussing the relevance of the function $f$ in the inequalities \eqref{cotaL},  \eqref{cotaDxL} and \eqref{cotaDyL}. These conditions are a direct  generalization of the conditions \cite[Eq (a), p. 10]{mawhin2010critical}.  In particular, we are interested in seeing when for every  $f\in \ephi_1$ there exist
$b\in L^1_1$ and a constant $C>0$ such that 
\begin{equation}\label{cotadb}
\Phi(s+f(t))\leq C\Phi(s)+b(t)\;\;\mbox{for every}\;s>0.
\end{equation} 
If \eqref{cotadb} is true, then we can suppose $f=0$  in the equations \eqref{cotaL} and \eqref{cotaDxL}. The same considerations should be done with $\varphi\left(s+f(t)\right)$.
 As a direct consequence of convexity, we can  bound the term $\Phi(s+f(t))$ by the expression  $\frac12\Phi(2s)+b(t)$ where $b(t):=\tfrac12\Phi(2f(t))\in L^1_1$ and $f\in \ephi_1$. Therefore, we can always assume $f = 0$ in \eqref{cotaL} and \eqref{cotaDxL} at the price of making  the value of $\lambda$ smaller. In the special case that $\Phi\in\Delta_2$, the inequality \eqref{delta2defi} implies \eqref{cotadb}. 

 If $\Phi\notin\Delta_2$, then \eqref{cotadb}  may not be true.  For example, if we consider the $N$-function $\Phi(s)=e^s-s-1$ which does not satisfy the $\Delta_2$-condition and $f(t)=\ln|\ln(t)|$, for $t\in [0,e^{-1}]$. We note that $f(t)\geq 0$ on $[0,e^{-1}]$ and that for $\lambda>1$

\[\int_0^{e^{-1}}\Phi(\lambda f(t))dt\leq 
\int_0^{e^{-1}}e^{\lambda f(t)}dt\leq \int_0^{e^{-1}}|\ln(t)|^{\lambda}dt<\infty.\]
Therefore $f\in\ephi_1$. Now we assume that there exist $b\in L^1_1$ and $C>0$ satisfying \eqref{cotadb}. From the inequality 
 $1/2e^s\leq \Phi(s)+1$ we obtain
\[\frac12 e^se^{f(t)}\leq \Phi(s+f(t))+1\leq C\Phi(s)+b(t)+1.\]
Dividing by $e^s$ and taking $s\to\infty$ we get $\tfrac12|\ln(t)|\leq\tfrac12 e^f(t)\leq C$, which is a contradiction. Thus there are not $b\in L^1_1$ and $C>0$ satisfying \eqref{cotadb}.

 Before addressing the main results in this section, we recall a definition.  


\begin{defi} We say that a function $\mathcal{L}:[0,T]\times \mathbb{R}^d \times \mathbb{R}^d \rightarrow \mathbb{R}$ is a Carath\'eodory function if for fixed $(\b{x},\b{y})$
the map $t \mapsto \mathcal{L}(t, \b{x},\b{y})$ is measurable  and for fixed $t$ the map  $(\b{x},\b{y}) \mapsto \mathcal{L}(t, \b{x}, \b{y})$ is continuously differentiable for almost everywhere $t\in [0,T]$.

\end{defi}



\begin{thm}\label{teorema_acotacion}
Let $\mathcal{L}$ be a Carath\'eodory function satisfying \eqref{cotaL}, \eqref{cotaDxL} and \eqref{cotaDyL}. 
Then the following statements hold:
\begin{enumerate}
\item \label{T1item1} \label{A1} The action integral given by \eqref{integral_accion}
is finitely defined on $\domi:=W^{1}\lphi_d\cap\{\b{u}|\b{\dot{u}}\in\Pi(\ephi_d,\lambda)\}$.

\item\label{T1item3} The function  $I$ is G\^ateaux differentiable on $\domi$ and  its derivative $I'$ is demicontinuous from $\domi$  into $\left[\wphi_d \right]^*$. Moreover, $I'$ is given by the following expression
\begin{equation}\label{DerAccion}
\langle  I'(\b{u}),\b{v}\rangle= \int_0^T \left\{D_{\b{x}}\mathcal{L}\big(t,\b{u},\b{\dot{u}}\big)\ccdot \b{v}+ D_{\b{y}}\mathcal{L}\big(t,\b{u},\b{\dot{u}}\big)\ccdot\b{\dot{v}}\right\} \ dt.
\end{equation}

\item\label{T1item4}  If  $\Psi \in \Delta_2$ then 
  $I'$ is continuous from $\domi$ into $\left[\wphi_d\right]^*$ when both spaces are equipped with the strong topology.


\end{enumerate}
\end{thm}
\begin{proof} Let $\b{u}\in \domi$.
 Since  $\lambda\Pi(\ephi_d,r)=\Pi(\ephi_d,\lambda r)$, we have   $\b{\dot{u}}/\lambda\in\Pi(\ephi_d,1)$. 
Thus, as $f\in\ephi_1$ and attending to \eqref{inclusiones}, we get 

\begin{equation}\label{inclusion3}
|\b{\dot{u}}|/\lambda+f\in\Pi(\ephi_1,1)\subset \claseor_1.
\end{equation}
By Corollary \ref{a_bound} and \eqref{cotaL}, we get 
 \[|\mathcal{L}(\cdot,\b{u},\b{\dot{u}})| \leq A(\|\b{u}\sobnor ) \left(b+ \Phi\left (\frac{|\b{\dot{u}}|}{\lambda}+f\right)  \right)\in
 L^1_1.\]
This fact proves item \ref{T1item1}.

 We split up the proof of item \ref{T1item3} into four steps.

\noindent\emph{Step 1. The non linear operator  $\b{u} \mapsto D_{\b{x}}\mathcal{L}(t,\b{u},\b{\dot{u}})$ is continuous from $\domi$ into $L^{1}_d([0,T])$ with the strong topology on both sets.} 


If $\b{u}\in \domi$, from \eqref{cotaDxL} and \eqref{inclusion3}, we obtain 
\begin{equation}\label{DxL1}
|D_{\b{x}}\mathcal{L}(\cdot,\b{u},\b{\dot{u}})|\leq A(\|u\sobnor) \left(b+\Phi\left(\frac{|\b{\dot{u}}|}{\lambda}+f\right)\right) \in L^1_1.
\end{equation}


Let   $\{\b{u}_n\}_{n\in \mathbb{N}}$ be a sequence of  functions in $\domi$  and let $\b{u}\in \domi$  such that $\b{u}_n\rightarrow \b{u}$ in $\wphi_d$.
From  $\b{u}_n\rightarrow \b{u}$ in $\lphi_d$, there exist a subsequence $\b{u}_{n_k}$ such that $\b{u}_{n_k}\rightarrow \b{u} \quad\text{a.e.}$ and as $\b{\dot{u}}_n\rightarrow \b{\dot{u}}\in\domi$, by 
  Lemma \ref{segundo lema}, there exist a subsequence of  $\b{u}_{n_k}$ (again denoted $\b{u}_{n_k}$) and a function  $h\in \Pi(\ephi_1,\lambda))$
such that  $\b{\dot{u}}_{n_k}\rightarrow \b{\dot{u}} \quad\text{a.e.}$ and $|\b{\dot{u}}_{n_k}|\leq h\quad\text{a.e}$.  Since $\b{u}_{n_k}$, $k=1,2,\ldots,$ is a strong convergent sequence in $\wphi_d$, it is a bounded sequence in $\wphi_d$. According to Lemma \ref{inclusion orlicz} and Corollary \ref{a_bound}, there exists $M>0$ such that $\|\b{a}(\b{u}_{n_k})\|_{L^{\infty}} \leq M$, $k=1,2,\ldots$.  From the previous facts and \eqref{DxL1}, we get
\begin{equation*}\label{DxL1-bis}
|D_{\b{x}}\mathcal{L}(\cdot,\b{u}_{n_k},\b{\dot{u}}_{n_k})|\leq M\left(b+\Phi\left(\frac{|h|}{\lambda}+f\right)\right) \in L^1_1.
\end{equation*}
On the other hand, by the Carath\'eodory condition, we have
\[D_{\b{x}}\mathcal{L}(t,\b{u}_{n_k}(t),\b{\dot{u}}_{n_k}(t))\to D_{\b{x}}\mathcal{L}(t,\b{u}(t),\b{\dot{u}}(t))\quad\hbox{ for a.e. } t\in[0,T].\]
Applying the Dominated Convergence Theorem we conclude the proof of step 1.

\noindent\emph{Step 2. The non linear operator   $\b{u}
 \mapsto  D_{y}\mathcal{L}(t,\b{u},\b{\dot{u}})$ is continuous from $\domi$ with the strong topology  into $\left[\lphi_d\right]^*$  with the weak$^*$ topology.}

 Let $\b{u}\in \domi$.  From  \eqref{inclusion3} andLemma \ref{phi_comp}  it follows that 
\begin{equation}\label{AcotOperphi}
\varphi\left(\frac{|\b{\dot{u}}|}{\lambda}+f\right)\in C^{\Psi}_1
\end{equation}
and Corollary \ref{a_bound} implies $\b{a}(\b{u})\in L^{\infty}_1$. 
Therefore, in virtue of  \eqref{cotaDyL} we get
\begin{equation}\label{DyLpsi}
   \left|D_{\b{y}}\mathcal{L}(\cdot,\b{u},\b{\dot{u}})\right|\leq  A(\|\b{u}\|_{\wphi} )  \left(c+\varphi\left( \frac{|\b{\dot{u}}|}{\lambda}+f\right  ) \right)\in\lpsi_1.
\end{equation}
 We note that \eqref{DxL1},  \eqref{DyLpsi} and the imbeddings $\wphi_d \hookrightarrow L_d^{\infty}$ and  $\lpsi_d\hookrightarrow  \left[\lphi_d\right]^*$ imply that the second member of
\eqref{DerAccion} defines an element in $\left[\wphi_d\right]^*$.

Let $\b{u}_n,\b{u}\in \domi$ such that $\b{u}_n\to \b{u}$ in the norm of $\wphi_d$. 
We must prove that  $D_{\b{y}}\mathcal{L}(\cdot,\b{u}_n,\dot{\b{u}}_n)\overset{w^*}{\rightharpoonup} D_{\b{y}}\mathcal{L}(\cdot,\b{u},\b{\dot{u}})$. On the contrary, there exist $\b{v}\in\lphi_d$, $\epsilon>0$ and a subsequence of $\{\b{u}_n\}$ (denoted  $\{\b{u}_n\}$ for simplicity)  such that
\begin{equation}\label{cota_eps}
 \left| \langle D_{\b{y}}\mathcal{L}(\cdot,\b{u}_n,\b{\dot{u}}_n),\b{v} \rangle - \langle  D_{\b{y}}\mathcal{L}(\cdot,\b{u},\b{\dot{u}}),\b{v} \rangle\right|\geq \epsilon.
\end{equation}
We have $\b{u}_n\rightarrow \b{u}$ in $\lphi_d$ and
$\b{\dot{u}}_n\rightarrow \b{\dot{u}}$ in $\lphi_d$. By Lemma \ref{segundo lema}, there exist a subsequence $\b{u}_{n_k}$ and a function $h\in \Pi(\ephi_1,\lambda)$ such that $\b{u}_{n_k}\rightarrow \b{u} \quad\text{a.e.}$, $\b{\dot{u}}_{n_k}\rightarrow \b{\dot{u}} \quad\text{a.e.}$ and $|\b{\dot{u}}_{n_k}|\leq h\quad\text{a.e.}$ 
As in the previous step, since $\b{u}_n$ is a convergent sequence, the Corollary \ref{a_bound} implies that $a(|\b{u}_n(t)|)$ is uniformly bounded by a certain constant $M>0$. 
Therefore,  with $\b{u}_{n_k}$ instead of $\b{u}$, inequality  \eqref{DyLpsi} becomes 
\begin{equation}\label{Dy-suc}
  \left | D_{\b{y}}\mathcal{L}(\cdot,\b{u}_{n_k},\b{\dot{u}}_{n_k})  \right| 
	\leq M\left(c+\varphi\left(\frac{h}{\lambda}+f\right)\right)\in \lpsi_1.
\end{equation}
Consequently, as $v \in \lphi_d$ and employing H\"older's inequality, we obtain that
\[\sup_k|D_{\b{y}}\mathcal{L}(\cdot,\b{u}_{n_k},\b{\dot{u}}_{n_k})\ccdot v| \in L^1_1.\]
  Finally, from the Lebesgue Dominated Convergence Theorem, we deduce
\begin{equation}\label{conv_debil}\int_0^T  D_{\b{y}}\mathcal{L}(t,\b{u}_{n_k},\b{\dot{u}}_{n_k})\ccdot\b{ v} \ dt \to \int_0^T D_{\b{y}}\mathcal{L}(t,\b{u},\b{\dot{u}})\ccdot\b{ v}\ dt \end{equation}
which contradicts the inequality \eqref{cota_eps}. This completes the proof of step 2.

\emph{Step 3.} We will prove \eqref{DerAccion}. The proof follows similar lines that \cite[Thm. 1.4]{mawhin2010critical}. For $\b{u}\in \domi$ and $\b{0}\neq\b{v}\in\wphi_d$, we define the function
\[H(s,t):=\mathcal{L}(t,\b{u}(t)+s\b{v}(t),\b{\dot{u}}(t)+s\b{\dot{v}}(t)).\]

From \cite[Lemma 10.1]{KR} (or \cite[Thm. 5.5]{Orliczvectorial2005} ) we obtain that if $|\b{u}|\leq |\b{v}|$ then    $d(\b{u},\ephi_d)\leq d(\b{v},\ephi_d)$. 
Therefore, for  $|s|\leq s_0:=\left(\lambda-d(\b{\dot{u}},\ephi_d)\right)/\|\b{v}\sobnor$ we have
\[
d \left(\b{\dot{u}}+s\b{\dot{v}}, \ephi_d \right)
\leq
d \left(|\b{\dot{u}}|+s|\b{\dot{v}}|, \ephi_1 \right)
\leq d \left(|\b{\dot{u}}|,\ephi_1 \right)+ s \|\b{\dot{v}}\orlnor < \lambda.
\]
As a consequence $\b{\dot{u}}+s\b{\dot{v}} \in \Pi(\ephi_d,\lambda)$ and  $|\b{\dot{u}}|+s|\b{\dot{v}}| \in \Pi(\ephi_1,\lambda)$. These facts imply, in virtue of Theorem \ref{teorema_acotacion} item \ref{T1item1}, that $I(\b{u}+s\b{v})$ is well defined and finite for $|s|\leq s_0$. 
Using  Corollary \ref{a_bound} we see that
\[ \|a(|\b{u}+s\b{v}|)\|_{L^{\infty}}\leq  A(\|\b{u}+s\b{v}\sobnor)\leq
 A(\|\b{u}\sobnor+s_0\|\b{v}\sobnor)=:M
\]
Now, applying Chain Rule, \eqref{DxL1}, \eqref{DyLpsi} the monotonicity of $\varphi$ and $\Phi$, 
the fact that $\b{v}\in L^{\infty}_d$ and $\b{\dot{v}}\in\lphi_d$ and H\"older's inequality, we have
\begin{equation}\label{ctg}
\begin{split}
|D_s H(s,t)|&=\left| D_{\b{x}}\mathcal{L}(t,\b{u}+s\b{v},\b{\dot{u}}+s\b{\dot{v}})\ccdot \b{v} +  D_{\b{y}}\mathcal{L}(t,\b{u}+s\b{v},\b{\dot{u}}+s\b{\dot{v}})\ccdot\b{\dot{v}}\right| \\
 & \leq M \left[\left( b(t)+ \Phi\left(\frac{|\b{\dot{u}}|+s_0|\b{\dot{v}}|}{\lambda}+f(t)\right)\right)|\b{v}|\right.\\
&\left. \quad+ \left(c(t)+ \varphi\left (\frac{|\b{\dot{u}}|+s_0|\b{\dot{v}}|}{\lambda}+f(t)\right)\right)|\b{\dot{v}}| \right]\in L^1_1.
\end{split}
\end{equation}
Consequently, $I$ has a directional derivative and
\[
\langle I'(\b{u}),\b{v} \rangle=\frac{d}{ds}I(\b{u}+s\b{v})\big|_{s=0}=\int_0^T  
\left\{D_{\b{x}}\mathcal{L}(t,\b{u},\b{\dot{u}})\ccdot \b{v}+ D_{\b{y}}\mathcal{L}(t,\b{u},\b{\dot{u}})\ccdot\b{\dot{v}}\right\} \ dt.
\]
Moreover, from \eqref{DxL1}, \eqref{DyLpsi}, Lemma \ref{inclusion orlicz} and previous formula, we have
\[
|\langle I'(\b{u}),\b{v} \rangle| \leq \|D_{\b{x}}\mathcal{L}\|_{L^1} \| \b{v}\linf + 
\|D_{\b{y}}\mathcal{L}\|_{\lpsi} \|\b{\dot{v}}\orlnor \leq C \|\b{v}\sobnor
\]
with a appropriate constant $C$.
This completes the proof of the G\^ateaux differentiability of $I$. 

\emph{Step 4. The operator $I':\domi  \to \left[\wphi_d
\right]^* $ is demicontinuous.}
This is a consequence  of the continuity of the mappings $\b{u} \mapsto D_{\b{x}}\mathcal{L}(t,\b{u},\b{\dot{u}})$ and $\b{u} \mapsto
D_{\b{y}}\mathcal{L}(t,\b{u},\b{\dot{u}})$. Indeed, if $\b{u}_n,\b{u}\in \domi$ with $\b{u}_n\to \b{u}$ in the norm of $\wphi_d$ and $\b{v} \in
\wphi_d$, then
\[
\begin{split}
\left\langle  I'(\b{u}_{n}),\b{v} \right\rangle &= \int_0^T \left\{  D_{\b{x}}\mathcal{L}\left(t,\b{u}_n,\b{\dot{u}}_n\right)\ccdot
\b{v} +
 D_{\b{y}}\mathcal{L}\left(t,\b{u}_n,\b{\dot{u}}_n\right)\ccdot\b{\dot{v}}\right\} \ dt\\
&\rightarrow \int_0^T \left\{ D_{\b{x}}\mathcal{L}\left(t,\b{u},\b{\dot{u}}\right)\ccdot \b{v}+ 
D_{\b{y}}\mathcal{L}\left(t,\b{u},\b{\dot{u}}\right)\ccdot\b{\dot{v}}\right\} \ dt\\
&=\left\langle  I'(\b{u}),\b{v} \right\rangle.
\end{split}
\]


In order to prove item  \ref{T1item4}, it is necessary to see that the maps $\b{u}\mapsto D_{\b{x}}\mathcal{L}(t,\b{u},\b{\dot{u}})$  and $\b{u}\mapsto D_{\b{y}}\mathcal{L}(t,\b{u},\b{\dot{u}})$  be norm continuous
from $\domi $ into $L^1_d$ and
 $\lpsi_d$ respectively.  The continuity of the first map has already been proved in step 1. We consider $\b{u}_n$ and $\b{u}$ in $\domi$ with $\|\b{u}_n- \b{u}\sobnor\to 0$. Therefore   there exist a subsequence $\b{u}_{n_k}\in \domi$ and a function $h\in\Pi(\ephi_1,\lambda)$  such that   \eqref{Dy-suc} holds true. As  $\Psi\in\Delta_2$ 
the right hand side of  \eqref{Dy-suc} belongs to $\epsi_1$. Now, invoking  Lemma \ref{lema_conv_may}, we  prove that
  from any sequence $\b{u}_n$ which converges to $\b{u}$ in $\wphi_d$ we can
extract a subsequence such that   $D_{\b{y}}\mathcal{L}(t,\b{u}_{n_k},\b{\dot{u}}_{n_k})\to D_{\b{y}}\mathcal{L}(t,\b{u},\b{\dot{u}})$ in the strong topology. The desired result is obtained by a standard argument.

The continuity of $I'$  follows  from the continuity 
of $D_{\b{x}}\mathcal{L}$ and $D_{\b{y}}\mathcal{L}$ by using the formula \eqref{DerAccion}.
\end{proof}



\section{Critical points and Euler-Lagrange equations}\label{sec:equa-min}


In this section we derive the Euler-Lagrange equations associated to critical points of action integrals on the subspace of $T$-periodic functions.  We denote by $\wphi_T$ the subspace of $\wphi_d$ containing all  $T$-periodic functions. As usual, when $Y$ is a subspace of
the Banach space $X$, we denote by $Y^{\perp}$ the \emph{annihilator subspace} of $X^*$, i.e. the subspace
that consists of all  bounded linear functions which are identically zero on $Y$.

We recall that  a function $f: \mathbb{R}^d \to \mathbb{R}$ is called \emph{strictly convex} if 

$f\left(\tfrac{\b{x}+\b{y}}{2}\right)< \tfrac{1}{2} \left(f\left(
\b{x}\right)+f\left( \b{y}\right)\right)$ for  $\b{x}\neq\b{y}$.  It is  well known that if $f$ is a strictly convex and differentiable function, then
$D_{\b{x}}f:\mathbb{R}^d\to\mathbb{R}^d$ is a one-to-one map  (see, for instance \cite[Thm. 12.17]{rockafellar2009variational}).


\begin{thm}\label{critpoint} Let $\b{u}\in\domi$ be  a $T$-periodic function. The following statements are equivalent:
\begin{enumerate}
 \item $I'(\b{u})\in\left( \wphi_T\right)^{\perp}$.
 \item  $D_{\b{y}}\mathcal{L}(t,\b{u}(t),\b{\dot{u}}(t))$ is an absolutely continuous function and $\b{u}$ solves the following boundary value problem
 \begin{equation}\label{ecualagran2}
    \left\{%
\begin{array}{ll}
   \frac{d}{dt} D_{y}\mathcal{L}(t,\b{u}(t),\b{\dot{u}}(t))= D_{\b{x}}\mathcal{L}(t,\b{u}(t),\b{\dot{u}}(t)) \quad \hbox{a.e.}\ t \in (0,T)\\
    \b{u}(0)-\b{u}(T)=D_{\b{y}}\mathcal{L}(0,\b{u}(0),\b{\dot{u}}(0))-D_{\b{y}}\mathcal{L}(T,\b{u}(T),\b{\dot{u}}(T))=0.
\end{array}%
\right.
\end{equation}
\end{enumerate}
Moreover if $D_{\b{y}}\mathcal{L}(t,x,y)$ is $T$-periodic with respect to the variable $t$ and strictly convex with respect to $\b{y}$, then
$D_{\b{y}}\mathcal{L}(0,\b{u}(0),\b{\b{\dot{\b{u}}}}(0))-D_{\b{y}}\mathcal{L}(T,\b{u}(T),\b{\dot{u}}(T))=0$ is equivalent to $\b{\dot{u}}(0)=\b{\dot{u}}(T)$.
\end{thm}

\begin{proof} The condition  $I'(\b{u})\in\left( \wphi_T\right)^{\perp}$ and \eqref{DerAccion} imply 
\[\int_0^T  D_{\b{y}} \mathcal{L}(t,\b{u}(t),\b{\dot{u}}(t))\ccdot \b{\dot{v}}(t)\ dt
=-\int_0^T  D_{\b{x}}\mathcal{L}(t,\b{u}(t),\b{\dot{u}}(t)) \ccdot\b{ v}(t)\ dt, \]
for every $\b{v}\in \wphi_T$. Using \cite[pp. 6-7]{mawhin2010critical} we obtain that  $D_{\b{y}}\mathcal{L}(t,\b{u}(t),\b{\dot{u}}(t))$ is absolutely continuous and 
$T$-periodic, therefore it is differentiable a.e. on $[0,T]$ and the first equality of \eqref{ecualagran2} holds true.
This completes the proof of  1 implies 2. The proof of 2 implies  1  follows easily 
from \eqref{DerAccion}  and \eqref{ecualagran2}.

The last part of the theorem is a consequence of 
$D_{\b{y}}\mathcal{L}(T,\b{u}(T),\b{\dot{u}}(T))=D_{\b{y}}\mathcal{L}(0,\b{u}(0),\b{\dot{u}}(0))=D_{\b{y}}\mathcal{L}(T,u(T),\b{\dot{u}}(0))$ and the injectivity of $D_{\b{y}}\mathcal{L}(T,u(T),\cdot)$.
\end{proof}


\section{Coercivity discussion}\label{sec:coer}

We recall a usual definition in the context of calculus of variations. 

\begin{defi} Let $X$ be a Banach space and let $D$ be an unbounded subset of $X$. Suppose $J:D\subset X\to\rr$. We say that $J$ is \emph{coercive} if $J(u)\to +\infty$ when  $\|\b{u}\|_X\to +\infty$. 
\end{defi}

It is well known that coercivity is a useful ingredient in order to establish existence of minima. Therefore, we are interested in finding conditions which ensure the coercivity of the action integral $I$ acting on $\domi$. For this purpose, we need to introduce the following  extra condition on lagrangian function $\mathcal{L}$  
\begin{equation}\label{cota_inf}
\mathcal{L}(t,\b{x},\b{y})\geq \alpha_0\Phi\left(\frac{|\b{y}|}{\Lambda}\right)+ F(t,\b{x}),
\end{equation}
where $\alpha_0,\Lambda>0$ and  $F:\rr\times\rr^d\to\rr$ is a Carath\'eodory function, i.e. $F(t,\b{x})$ is  measurable with respect to $t$ for every fixed  $\b{x}\in\rr^d$ and it is continuous at $\b{x}$ for a.e. $t\in [0,T]$. We note that, in virtue of \eqref{cota_inf} and \eqref{cotaL}, we have $F(t,\b{x})\leq a(|\b{x}|)b_0(t)$  with $b_0(t):=b(t)+\Phi(f(t))\in L^1_1([0,T])$. In order to ensure that integral $\int_0^TF(t,\b{u})\ dt$ is finite for $\b{u}\in\wphi$,  we need to assume 
\begin{equation}\label{condA1} |F(t,\b{x})|\leq a(|\b{x}|)b_0(t),\quad\text{for \,a.e. }t\in [0,T] \quad\text{and for every } \b{x}\in\rr^d.
\end{equation}
As we shall see in Theorem \ref{coercitividad1}, when $\mathcal{L}$ satisfies \eqref{cotaL}, \eqref{cotaDxL}, \eqref{cotaDyL}, \eqref{cota_inf} and \eqref{condA1},  the coercivity of the action integral $I$ is related to the coercivity of the functional
\begin{equation}\label{func_phi}
  J_{C,\nu}(\b{u}):= \rho_{\Phi}\left(\frac{\b{u}}{\Lambda}\right)-C\|\b{u}\orlnor^{\nu},
\end{equation}
for $C,\nu>0$. If $\Phi(x)=|x|^p/p$ then $J_{C,\nu}$ is clearly coercive for $\nu<p$. For more general $\Phi$ the situation is more interesting   as it will be shown in the following lemma.

\begin{lem}\label{lem_coer} Let $\Phi$ and $\Psi$ be complementary $N$-functions. Then:
\begin{enumerate}
  \item If $C\Lambda<1$ then $J_{C,1}$ is coercive. 
  
  \item If $\Psi \in \Delta_2$ globally, then there exists a constant $\alpha_{\Phi}>1$ such that, for any $0<\mu<\alpha_{\Phi}$,
\begin{equation}\label{coer_modular} \lim\limits_{\|\b{u}\orlnor \to \infty} \frac{\rho_{\Phi}\left(\frac{\b{u}}{\Lambda}\right)}{\|\b{u}\orlnor^{\mu}}=+\infty.
\end{equation}
In particular, the functional $J_{C,\mu}$ is coercive for every $C>0$ and  $0<\mu<a_{\Phi}$. The constant $\alpha_{\Phi}$ is one of the so-called \emph{ Matuszewska-Orlicz indices} (see \cite[Ch. 11]{M}).
\item If $J_{C,1}$ is coercive with $C\Lambda>1$, then $\Psi \in \Delta_2$.  
\end{enumerate}
\end{lem}

\begin{proof} By \eqref{amemiya} we have
\[(1-C\Lambda)\|\b{u}\orlnor+C\Lambda\|\b{u}\orlnor=\|\b{u}\orlnor\leq \Lambda +\Lambda \rho_{\Phi}\left(\frac{\b{u}}{\Lambda}\right),\]
then
\[\frac{(1-C\Lambda)}{\Lambda}\|\b{u}\orlnor-1\leq \rho_{\Phi}\left(\frac{\b{u}}{\Lambda}\right)- C\|\b{u}\orlnor=J_{C,1}(\b{u}).\]
This shows that $J_{C,1}$ is coercive and therefore item 1 is proved.  

In virtue of \cite[Eq. (2.8)]{AGMS}, the $\Delta_2$-condition on $\Psi$, \cite[Thm. 11.7]{M} and \cite[Cor. 11.6]{M}, we obtain constants $K>0$ and $\alpha_{\Phi}>1$ such that 
\begin{equation}\label{delta2-consecuencia}
\Phi(r s)\geq Kr^{\nu}\Phi(s)
\end{equation}
for any $0<\nu<\alpha_{\Phi}$,  $s\geq 0$ and $r>1$.

Let $1<\mu<\nu<\alpha_{\Phi}$ and let $r>\Lambda$ be a constant that will be specified later.  
Then, from \eqref{delta2-consecuencia} and \eqref{amemiya}, we get
\[
\begin{split}
\frac{\int_0^T \Phi\left(\frac{|\b{u}|}{\Lambda}\right)\ dt}{\|\b{u}\orlnor^{\mu}}
&\geq
K \left(\frac{r}{\Lambda}\right)^{\nu}\frac{\int_0^T \Phi(r^{-1}|\b{u}|)\ dt}{\|\b{u}\orlnor^{\mu}}\\
&\geq
K \left(\frac{r}{\Lambda}\right)^{\nu}\frac{r^{-1}\|\b{u}\orlnor-1}{\|\b{u}\orlnor^{\mu}}.\\
\end{split}
\]
We choose $r=\|\b{u}\orlnor/2$. Since $\|\b{u}\orlnor\to+\infty$   we can assume $\|\b{u}\orlnor>2\Lambda$.  Thus $r>\Lambda$ and 

\[
\frac{\int_0^T \Phi\left(\frac{|\b{u}|}{\Lambda}\right) dt}{\|\b{u}\orlnor^{\mu}}\geq
\frac{K}{2^{\nu}\Lambda^{\nu}} \|\b{u}\orlnor^{\nu-\mu}\to +\infty\quad\text{as }\|\b{u}\orlnor\to+\infty,
\]
because $\nu>\mu$.

In order to prove the last item, we assume that $\Psi \notin \Delta_2$. 
By \cite[Thm. 4.1]{KR},  there exists a sequence of real  numbers  $r_n$ such that
$r_n \to \infty$ and 
\begin{equation}\label{eq: un-_tiende_inf}
\lim\limits_{n \to \infty} \frac{r_n \psi(r_n)}{\Psi(r_n)}=+\infty.
\end{equation}
Now, we choose $\b{u}_n$ such that
$|\b{u}_n|=\Lambda\psi(r_n)\chi_{[0,\frac{1}{\Psi(r_n)}]}$. Then, 
by \cite[Eq. (9.11)]{KR}, we get 
\[
\|\b{u}_n\orlnor =\Lambda\frac{\psi(r_n)}{\Psi(r_n)}\Psi^{-1}(\Psi(r_n))=
\Lambda\frac{r_n\psi(r_n)}{\Psi(r_n)}\to \infty,\quad\text{as}\quad n \to \infty.
\]
Next, using Young's equality (see \cite[Eq. (2.7)]{KR}), we have
\[
\begin{split}
J_{C,1}(\b{u}_n)&=\int_0^T \Phi\left(\frac{|\b{u}_n|}{\Lambda}\right)\,dt-C\|\b{u}_n\orlnor\\
&=
\frac{1}{\Psi(r_n)}\left[\Phi(\psi(r_n))  -C\Lambda r_n\psi(r_n)\right]\\
&=
\frac{1}{\Psi(r_n)} \left[ r_n\psi(r_n)-\Psi(r_n)- C\Lambda r_n\psi(r_n) \right]\\
&=\frac{(1- C\Lambda) r_n\psi(r_n)}{\Psi(r_n)}-1.
\end{split}
\]
From \eqref{eq: un-_tiende_inf} and the condition $C\Lambda>1$, we obtain  $J_{C,1}(\b{u}_n)\to-\infty$, which contradicts the coercivity  of $J_{C,1}$.
\end{proof}

We need the following fact which is an easy consequence of Sobolev and Wirtinger inequalities.

\begin{lem} 
\end{lem}















Next, we present two theorems that establish coercivity of action integrals under different assumptions. 



\begin{thm}\label{coercitividad1}
Let  $\mathcal{L}$ be a Lagrangian function satisfying \eqref{cotaL}, \eqref{cotaDxL}, \eqref{cotaDyL}, \eqref{cota_inf} and \eqref{condA1}. We assume the following conditions:
\begin{enumerate}
\item There exist a non negative function  $b_1 \in L^1_1$ and a constant $\mu>0$  such that for any $\b{x_1},\b{x_2}\in\rr^d$ and a.e. $t\in [0,T]$
\begin{equation}\label{holder_cont}
  \left| F(t,\b{x_2})- F(t,\b{x_1}) \right|\leq b_1(t)(1+|\b{x_2}-\b{x_1}|^{\mu}).
\end{equation}
We suppose that $\mu< \alpha_{\Phi}$,  with $\alpha_{\Phi}$ as in Lemma \ref{lem_coer}, in the case that $\Psi\in\Delta_2$; and we suppose $\mu=1$  if $\Psi$ is an  arbitrary $N$-function. 
\item
\begin{equation}\label{propiedad1coercividad}
\int_{0}^{T}F(t,\b{x})\ dt \rightarrow \infty \quad \hbox{as} \quad |\b{x}|\rightarrow \infty.
\end{equation}
\item\label{hipot_coer}  $\Psi\in\Delta_2$ or, alternatively, 
$\alpha_0^{-1}T\Phi^{-1}\left(1/T\right)\|b_1\|_{L^1}\Lambda<1$.
\end{enumerate}
Then  the action integral $I$ is coercive.
\end{thm}




\begin{proof} In the following estimates, we will use \eqref{cota_inf}, the decomposition $\b{u}=\b{\overline{u}}+\b{\tilde{u}}$, H\"older's inequality and Wirtinger's inequality. 
\begin{equation}\label{cota_inf_I}
\begin{split}
I(\b{u})&\geq\alpha_0\rho_{\Phi}\left( \frac{\b{\dot{u}}}{\Lambda}\right)+\int_0^TF(t,\b{u})\ dt\\ 
&=\alpha_0\rho_{\Phi}\left( \frac{\b{\dot{u}}}{\Lambda}\right)+ \int_0^T \left[F(t,\b{u})-F(t,\b{\overline{u}})\right]\ dt +  \int_0^TF(t,\b{\overline{u}})\ dt\\
&\geq\alpha_0\rho_{\Phi}\left( \frac{\b{\dot{u}}}{\Lambda}\right)- \int_0^Tb_1(t)(1+|\b{\tilde{u}}(t)|^{\mu})\ dt +  \int_0^TF(t,\b{\overline{u}})\ dt\\
&\geq \alpha_0\rho_{\Phi}\left( \frac{\b{\dot{u}}}{\Lambda}\right)- \|b_1\|_{L^1}(1+\|\b{\tilde{u}}\|_{L^{\infty}}^{\mu}) +  \int_0^TF(t,\b{\overline{u}})\ dt\\
&\geq\alpha_0\rho_{\Phi}\left( \frac{\b{\dot{u}}}{\Lambda}\right)- \|b_1\|_{L^1}\left(1+\left[T\Phi^{-1}\left(\frac{1}{T}\right)\right]^{\mu}\|\b{\dot u}\orlnor^{\mu}\right) \\
&\quad+  \int_0^TF(t,\b{\overline{u}})\ dt\\
&=\alpha_0J_{C,\mu}(\b{\dot{u}})- \|b_1\|_{L^1}+ \int_0^TF(t,\b{\overline{u}})\ dt,
\end{split}
\end{equation}
where $C=\alpha_0^{-1}\left[T\Phi^{-1}\left(1/T\right)\right]^{\mu}\|b_1\|_{L^1}$.
Suppose that $\b{u}_n$ is a sequence in $\domi$ with  
$\|\b{u}_n\sobnor\to\infty$. We have to prove that $I(\b{u}_n)\to\infty$. On the contrary, suppose  that for a subsequence, which we
still denote $\b{u}_n$, $I(\b{u}_n)$ is bounded from above. Then, in virtue of \eqref{cota_prome} and by passing to a subsequence, we can assume that $\b{\dot{u}}_n$ is unbounded in $\lphi_d$ or $\b{\overline{u}}_n$ is unbounded in $\mathbb{R}^d$.
On the other hand,  \eqref{condA1} and \eqref{propiedad1coercividad}
imply that the integral $\int_0^TF(t,\b{\overline{u}}_n)\ dt$ is lower bounded. 
These observations, the lower bound for $I$ given in \eqref{cota_inf_I}, 
assumption \ref{hipot_coer} in Theorem \ref{coercitividad1} and Lemma \ref{lem_coer} imply that $I(\b{u}_n)$ is not bounded from above. This contradiction imply the desired result.
\end{proof}


Following \cite{mawhin2010critical} we say that $F$ satisfies the condition (A) if  $F(t,\b{x})$ is a Carath\'eodory function, $F$ verifies \eqref{condA1} and $F$ is continuously differentiable with respect to $\b{x}$. Moreover, the next inequality holds 

\begin{equation}\label{condA2} |D_{\b{x}}F(t,\b{x})|\leq a(|\b{x}|)b_0(t),\quad\text{for a.e. }t\in [0,T] \text{ and for every }\b{x}\in\rr^d.
\end{equation}
The following result was proved in \cite[p. 18]{mawhin2010critical}. 
\begin{lem}\label{lema_pto_cri} Suppose that $F$ satisfies condition (A) and \eqref{propiedad1coercividad}, $F(t,\cdot)$ is  differentiable and convex  a.e. $t\in [0,T]$. Then there exists $\b{x}_0\in\rr^d$ such that
\begin{equation}\label{der_cero}
 \int_0^T D_{\b{x}} F(t,\b{x}_0)\ dt=0.
\end{equation}
\end{lem}


\begin{thm}\label{segcoerthm}
Let $\mathcal{L}$ be as in Theorem \ref{coercitividad1} and let $F$ be as in Lemma \ref{lema_pto_cri}. In addition, assume that $\Psi\in\Delta_2$ or, alternatively  $\alpha_0^{-1}T\Phi^{-1}\left(1/T\right)a(|\b{x}_0|)\|b_0\|_{L^1} \Lambda<1$, with $a$ and $b_0$ as in \eqref{condA1} and $\b{x}_0\in\rr^d$  any point satisfying  \eqref{der_cero}. Then $I$ is coercive. 

\end{thm}


\begin{proof}  Using \eqref{cota_inf}, \cite[Eq. (18), p.17]{mawhin2010critical},  the decomposition $\b{u}=\b{\overline{u}}+\b{\tilde{u}}$,  \eqref{der_cero}, \eqref{holder} and Wirtinger's inequality, we get
%
\begin{equation}\label{cota_con _upunto}
\begin{split}
I(\b{u})&\geq\alpha_0\rho_{\Phi}\left( \frac{\b{\dot{u}}}{\Lambda}\right)+\int_0^T F(t,\b{x}_0)\ dt 
+ \int_0^T D_{\b{x}} F (t,\b{x}_0) \ccdot (\b{u}-\b{x}_0)\ dt\\
&=\alpha_0\rho_{\Phi}\left( \frac{\b{\dot{u}}}{\Lambda}\right) +\int_0^T F(t,\b{x}_0)\ dt 
+ \int_0^TD_{\b{x}}F (t,\b{x}_0) \ccdot \b{\widetilde{u}} \ dt\\
&\quad + \int_0^T D_{\b{x}} F (t,x_0) \ccdot (\b{\overline{u}}  -\b{x}_0) \ dt\\
&=\alpha_0\rho_{\Phi}\left( \frac{\b{\dot{u}}}{\Lambda}\right)+\int_0^T F(t,\b{x}_0)\ dt + 
\int_0^T D_{\b{x}} F (t,\b{x}_0) \ccdot \b{\widetilde{u}}\ dt\\
&\geq\alpha_0 \rho_{\Phi}\left( \frac{\b{\dot{u}}}{\Lambda}\right)-a(|\b{x}_0|)\|b_0\|_{L^1} 
-a(|\b{x}_0|)\|b_0\|_{L^1}T\Phi^{-1}\left(\frac{1}{T}\right) \|\b{\dot{u}}  \|_{\lphi}\\
&= \alpha_0 J_{C,1}(\b{\dot{u}})-a(|\b{x}_0|)\|b_0\|_{L^1} 
\end{split}
\end{equation}
with $C:=\alpha_0^{-1}a(|\b{x}_0|)\|b_0\|_{L^1}T\Phi^{-1}(1/T)$. 

Let $\alpha$ be as in Corollary \ref{a_bound}, i.e. $\alpha]$ is  a non decreasing majorant of $a$. Using that  $F(t, \b{\overline{u}} /2) \leq (1/2)F(t,\b{u}) + (1/2) F(t, -\b{\widetilde{u}})$ and taking into account that $\Phi$ is a non negative function, inequality \eqref{condA1}, H\"older's inequality,  Corollary \ref{a_bound} and Wirtinger's inequality, we obtain
\begin{equation}\label{cota_con _ubarra}
\begin{split}
I(\b{u}) &\geq\alpha_0\rho_{\Phi}\left( \frac{\b{\dot{u}}}{\Lambda}\right)  +2 \int_0^T F(t,\b{\overline{u}} /2)\ dt - \int_0^T F(t, -\b{\widetilde{u}})\ dt\\
&\geq 2 \int_0^T F(t,\b{\overline{u}} /2)\ dt -\|b_0\|_{L^1} \|\b{a}(\b{\tilde{u}})\|_{L^{\infty}}\\
&\geq 2 \int_0^T F(t,\b{\overline{u}} /2)\ dt -\|b_0\|_{L^1} \alpha(\|\b{\tilde{u}}\|_{L^{\infty}})\\
&\geq 2 \int_0^T F(t,\b{\overline{u}} /2)\  dt - C_1 \alpha(C_2\|\b{\dot{u}}\orlnor)
\end{split}
\end{equation}
for certain constants $C_1,C_2>0$.


Now reasoning in a similar way as  in the end of  the proof of Theorem \ref{coercitividad1} we show that  $I(\b{u}_n)\to\infty$. 
\end{proof}



\section{Main result}\label{sec:main}



We need to find conditions for the lower semicontinuity of  $I$. To this end we perform a little adaptation of  a result of \cite{ekeland1999convex}. 


\begin{lem}\label{semicontinf}
We suppose that $\mathcal{L}(t,\b{x},\b{y})$, $F(t,\b{x})$ are Carath\'eodory functions satisfying
\begin{equation}\label{cota_inf_2}
\mathcal{L}(t,\b{x},\b{y})\geq \Phi\left(|\b{y}|\right)+ F(t,\b{x}),
\end{equation}
where $\Phi$ is an $N$-function. 
We also assume that the function $F$ satisfies inequality \eqref{condA1} and $\mathcal{L}(t,\b{x},\cdot)$ is convex in $\rr^d$ for each $(t,\b{x})\in [0,T]\times\rr^d$.  Let $\{\b{u}_n\}\subset\wphi$ be a sequence such that $\b{u}_n$ converges  uniformly  to a function $\b{u}\in\wphi$ and $\b{\dot{u}}_n$ converges in the weak topology of $L^1_d$ to $\b{\dot{u}}$.   Then
\begin{equation}\label{liminf0}I(\b{u})\leq \liminf_{n\to\infty}I(\b{u}_n).
\end{equation}

\end{lem}

\begin{proof} First we note that \eqref{cota_inf_2} and \eqref{condA1} imply that $I$ is defined on $\wphi$ taking values in the interval $(-\infty,+\infty]$. 

Let $\{\b{u}_n\}$ be a sequence  satisfying the assumptions of the theorem.   We define the Carath\'eodory function $\mathcal{\hat{L}}=\mathcal{L}-F$ and we denote by $\hat{I}$ its  associated action integral. Using  \cite[Thm. 2.1, p. 243]{ekeland1999convex}, we have that
\begin{equation}\label{liminf1}
\int_0^T\mathcal{\hat{L}}(t,\b{u},\b{\dot{u}})\ dt\leq \liminf_{n\to\infty}\int_0^T\mathcal{\hat{L}}(t,\b{u}_n,\b{\dot{u}}_n)\ dt.
\end{equation}

From the uniform convergence of $\b{u}_n$ and the Carath\'eodory conditions for $F$ we obtain that $F(t,\b{u}_n(t))\to F(t,\b{u}(t))$ a.e. $t\in[0,T]$.  Since $\b{u}_n$ are uniformly bounded, the inequality  \eqref{condA1} imply that there exists $g\in L_1^1([0,T])$ such that $|F(t,\b{u}_n(t))|\leq g(t)$. From the Dominated Convergence Theorem we have that 
\begin{equation}\label{liminf2}
\lim_{n\to\infty}\int_0^TF(t,\b{u}_n(t))\ dt=\int_0^TF(t,\b{u}(t))\ dt.
\end{equation}
Taking account of \eqref{liminf1} and  \eqref{liminf2} we obtain \eqref{liminf0}.

\end{proof}


\begin{thm} We suppose $\Phi$ and $\Psi$ complementary $N$-functions. We assume that the Carath\'eodory function $\mathcal{L}(t,\b{x},\b{y})$ is strictly convex in $\b{y}$, $D_{\b{y}}\mathcal{L}$ is $T$-periodic with respect to $T$  and that it satisfies \eqref{cotaL}, \eqref{cotaDxL}, \eqref{cotaDyL}, \eqref{cota_inf}, \eqref{condA1} and \eqref{propiedad1coercividad}. In addition, we suppose that some of the following statements hold (we recall the definitions and properties of $\alpha_0$, $b_1$, $\b{x}_0$ and $b_0$ from \eqref{cota_inf}, \eqref{holder_cont}, \eqref{der_cero} and \eqref{condA2} respectively)
\begin{enumerate}
  \item\label{item1prin} $\Psi\in\Delta_2$ and  \eqref{holder_cont}.
\item \eqref{holder_cont} and  $\alpha_0^{-1}T\Phi^{-1}\left(1/T\right)\|b_1\|_{L^1}\Lambda<1$.

\item\label{item3prin} $\Psi\in\Delta_2$,  $F$ satisfies condition (A) and  $F(t,\cdot)$ is  convex  a.e. $t\in [0,T]$.

\item\label{item4prin} As item \ref{item3prin} but with $\alpha_0^{-1}T\Phi^{-1}\left(1/T\right)a(|\b{x}_0|)\|b_0\|_{L^1} \Lambda<1$ instead of $\Psi\in\Delta_2$.

\end{enumerate}
Then problem \eqref{ProbPrin} has a solution.
\end{thm}

\begin{proof} First note that  \eqref{cota_inf}, \eqref{condA1} and \eqref{propiedad1coercividad} imply that $I$ is bounded from below on $\wphi_T$. Let $\{\b{u}_n\}\subset \wphi_T$ be a  minimizing sequence for the problem  $\min\{I(\b{u})|\b{u}\in\wphi_T\}$.
Since  $I(\b{u}_n)$, $n=1,2,\ldots$  is bounded, Theorem \ref{coercitividad1} (or Theorem \ref{segcoerthm} according to which of the items \ref{item1prin}-\ref{item4prin} hold true) implies that $\{\b{u}_n\}$ is norm bounded in $\wphi_d$. Hence, in virtue of Corollary \ref{unif_conv} we can assume that $\b{u}_n$ converges uniformly to a $T$-periodic continuous function $\b{u}$.  The space  $\lphi_d$ is a predual space, concretely $\lphi_d=\left[\epsi_d\right]^*$. Therefore, from \cite[Cor. 5, p. 148]{rao1991theory} and since $\b{\dot{u}}_n$ is bounded in $\lphi_d$,  there exists a subsequence (again denoted $\b{\dot{u}}_n$) such that $\b{\dot{u}}_n$ converges to a function $\b{v}\in\lphi_d$ in the weak* topology of $\lphi_d$. From this fact and the uniform convergence of $\b{u}_n$ to $\b{u}$ we obtain for every $T$-periodic function $\b{\xi}\in C^{\infty}([0,T],\rr^d)\subset\epsi_d$
\[
\int_0^T\b{\dot{\xi}}\ccdot\b{u}\ dt=\lim_{n\to\infty}\int_0^T\b{\dot{\xi}}\ccdot\b{u}_n \ dt=-\lim_{n\to\infty}\int_0^T\b{\xi}\ccdot\b{\dot{u}}_n\ dt=-\int_0^T\b{\xi}\ccdot\b{v}\ dt.
\]
Thus $\b{v}=\b{\dot{u}}$ a.e. $t\in [0,T]$ (see \cite[p. 6]{mawhin2010critical}) and $\b{u}\in\wphi_T$.

Finally from the relations $\left[L^1_d\right]^*=L^{\infty}_d\subset  \epsi_d$ and $\lphi_d\subset L^1_d$ we obtain that $\b{\dot{u}}_n$ converges to $\b{\dot{u}}$ in the weak topology of $L^1_d$. Consequently Theorem \ref{semicontinf}, applying with the $N$-function $\alpha_0\Phi\left(|\ccdot|/\Lambda\right)$, implies 
\[I(\b{u})\leq  \liminf_{n\to\infty}I(\b{u}_n)=\min\limits_{\b{u}\in\wphi_T}I(\b{u}).\]
Hence $\b{u}$ is a minimun, and therefore a critical point, of $I$. Invoking Theorem \ref{critpoint} we conclude the proof.\end{proof}

\printbibliography

\end{document}
