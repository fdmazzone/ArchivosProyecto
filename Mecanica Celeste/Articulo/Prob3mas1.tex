\documentclass[twoside]{article}


%\usepackage{hyperref}
\usepackage{amssymb,amsthm}
\usepackage{amsmath}
\usepackage{color}
\usepackage{ esint }
\usepackage{mathabx}
\usepackage{MnSymbol}
\usepackage{fancyhdr}
%\usepackage{times}

%\usepackage[latin1]{inputenc}

\usepackage{comment}
\usepackage{url}
\usepackage{xcolor}
\usepackage{adjustbox}
\usepackage{hyperref}

\newtheorem{thm}{Theorem}[section]
\newtheorem{cor}[thm]{Corollary}
\newtheorem{lem}[thm]{Lemma}

\newtheorem{defi}[thm]{Definition}
\newtheorem{prop}[thm]{Proposition}
\theoremstyle{remark}
\newtheorem{comentario}{Remark}


\makeatletter
\newcommand{\labitem}[2]{%
\def\@itemlabel{\textbf{#1}}
\item
\def\@currentlabel{#1}\label{#2}}
\makeatother




\title{Periodic solutions for a Sitnikov restricted $n+1$-body problem with primaries in rigid motion}
\author{Gast\'on Beltritti \thanks{SECyT-UNRC and CONICET}\\
Dpto. de Matem\'atica, Facultad de Ciencias Exactas Físico-Químicas y Naturales\\
Universidad Nacional de R\'{i}o Cuarto\
(5800) R\'{\i}o Cuarto, C\'ordoba, Argentina,\\
\url{gbeltritti@exa.unrc.edu.ar}\\[3mm]
Fernando D. Mazzone \thanks{SECyT-UNRC, FCEyN-UNLPam}\\
Dpto. de Matem\'atica, Facultad de Ciencias Exactas, F\'{\i}sico-Qu\'{\i}micas y Naturales\\
Universidad Nacional de R\'{i}o Cuarto\\
(5800) R\'{\i}o Cuarto, C\'ordoba, Argentina,\\
\url{fmazzone@exa.unrc.edu.ar}\\
Martina G. Oviedo \thanks{SECyT-UNRC, CIN}\\
Dpto. de Matem\'atica, Facultad de Ciencias Exactas, F\'{\i}sico-Qu\'{\i}micas y Naturales\\
Universidad Nacional de R\'{i}o Cuarto\\
(5800) R\'{\i}o Cuarto, C\'ordoba, Argentina,\\
\url{martinagoviedo@gmail.com}
}

\date{}


\newcommand{\rr}{\mathbb{R}}
\newcommand{\nn}{\mathbb{N}}


\newcounter{example}

\setcounter{example}{1}


\newenvironment{example}{\noindent\textit{Example} \arabic{example}.}{\addtocounter{example}{1}}




\begin{document}


\maketitle
%
% \begingroup%Locallizing the change to `thefootnote'.
%     \renewcommand{\thefootnote}{}%Removing the footnote symbol.
%     %
%     \footnotetext{%
%     %   2010 Mathematics Subject Classification
%     %   http://www.ams.org/msc/
%     \textbf{2010  AMS Subject Classification.} Primary: .
%     Secondary: .
%     }%
%         \footnotetext{%
%     \textbf{Keywords and phrases.}  .
%     }%
%     \endgroup
%
%
%
%

\begin{abstract}


\end{abstract}




\pagestyle{fancy} \headheight 35pt \fancyhead{} \fancyfoot{}

\fancyfoot[C]{\thepage} \fancyhead[CE]{\nouppercase{G. Beltritti, F. Mazzone, M. Oviedo}} \fancyhead[CO]{\nouppercase{\section}}

\fancyhead[CO]{\nouppercase{\leftmark}}


%\tableofcontents




\section{Introduction}
In this paper we discuss existence of periodic solutions for the following restricted nonplanar Newtonian $n+1$-body problem $P$ (see figure \ref{fig:conf_esp}):
\begin{itemize}
 \item[$P_1$] We have $n$ primary bodies of masses $m_1,\ldots,m_n$ and an additional masless body.
 \item[$P_2$] The primary bodies are in a central  configuration rigid motion (see \cite[Section 2.9]{JaumeLlibre276}). This motion is carried out in a plane $\Pi$.
 \item[$P_3$] The massless particle is moving periodically on the perpendicular line to $\Pi$ passing through the center of masses.
\end{itemize}


\begin{figure}[h]
 \begin{center}


\setlength{\unitlength}{4cm}
\begin{picture}(2.5, 1)(-.5, -.5)
  \setlength{\unitlength}{2cm}
    \put(.5,0){
}
  \setlength{\unitlength}{5cm}
    \put(-.1,0){
    \qbezier(0, 0)(0,.25)(.5, .25)
  \qbezier(1.01, 0)(1.01,.25)(.5, .25)
    \qbezier(0, 0)(0,-.25)(.5, -.25)
  \qbezier(1.01, 0)(1.01,-.25)(.5, -.25)
}
\put(-.2,0){\line(1,0){1.2}}
\put(-0.1,0){\circle*{.04}} \put(-0.19,-0.05){$m_1$}
\put(0.65,-0.24){\circle*{.04}} \put(0.61,-0.2){$m_2$}
\put(0.6,.235){\circle*{.04}}\put(0.55,.18){$m_3$}
\put(0.5,0){\circle*{.03}}\put(0.44,-0.05){$c$}
\put(0.5,-0.08){\line(0,1){.6}}
\put(0.5,0.4){\circle*{.04}}\put(0.53,0.4){$m_4\approx 0$}
\put(-.3,-0.3){\line(1,0){1.5}}
\put(-.3,-0.3){\line(1,5){.12}}
\put(1.2,-0.3){\line(-1,5){.12}}
\put(-0.18,0.3){\line(1,0){1.26}}
\put(1.1,-0.25){$\Pi$}
\end{picture}\caption{Four-body problem with three primaries}\label{fig:conf_esp}
 \end{center}

\end{figure}

Problems like the one presented have been extensively discussed in the literature. In \cite{sitnikov1960existence} K. Sitnikov considered the problem of two body in a Keplerian motion and a massless particle moving in the perpendicular line to the orbital plane passing
through the center of masses. Sitnikov obtained deep results about existence of solutions, some of them periodic (see \cite[III(5)]{moser2016stable}). Since then many  other authors have studied Sitnikov problem, for instance  Liu, Zhou, and Sun \cite{liu1991numerical},  Hagel and Trenkler \cite{hagel1993computer}, Dvorak \cite{dvorak1993numerical}, Dankowicz and Holmes \cite{dankowicz1995existence}, Llibre, Meyer and Soler \cite{llibre1999bridges}, Chesley \cite{chesley1999global}, Jim{\'e}nez-Lara a and Escalona-Buend{\'\i}a \cite{jimenez2001symmetries},
 Llibre and Ortega \cite{llibre2008families}, P{\'e}rez, Jim{\'e}nez and Lacomba \cite{perez2009periodic}.

Problems like the Sitnikov problem for four bodies  were addressed more recently.
In \cite{soulis2008periodic} Soulis, Papadakis and Bountis studied existence, linear stability and bifurcations for a problem similar to $P$, where in place to have a Eulerian collinear configuration they had a Lagrangian equilateral triangle configuration for the primaries bodies, which are supposed to have the same mass $m_1=m_2=m_3$. Later, In \cite{baltagiannis2011families} Baltagiannis nad Papadakis considered more general masses and in \cite{pandey2013periodic} Pandey and Ahmad extend the analysis started in \cite{soulis2008periodic} to the case when the primaries are oblate (not mass points).
In \cite{zhao2015nonplanar}, Zhao and Zhang proved existence of periodic solutions for a problem similar to the one dealt with in \cite{soulis2008periodic}.  They used a variational approach. In the present paper we extend the analysis in \cite{zhao2015nonplanar} to the case of a collinear central configuration for the primaries.
In \cite{li2013characterization} Li, Zhang and Zhao studied a special type of
restricted circular $N+1$-body problem  with equal masses for the primaries. 

Given that in our problem the primaries are no longer equidistant and their relative 
position is determined by a polynomial equation of fifth degree, the calculations involved 
here are tedious to reproduce completely and difficult for that the reader to check them 
by hand. For this reason we have prepared a jupyter-notebook (see \cite{CalAux}) with some 
of these calculations. With a little knowledge of Python-Sympy (see \cite{sympy}) the reader 
can check and reproduce them easily.


\section{Preliminaries and Main Results}

We start considering $n$ bodies, $n>2$, of masses $m_1,\ldots,m_n$ moving in a Euclidean 3-dimensional space according to Newton's laws of motion. We assume that $q_1(t),\ldots,q_n(t)$ are the coordinates (column vectors) of the bodies in some inertial Cartesian coordinate system. We denote by $r_{ij}=|q_i-q_j|$  the  Euclidean distance between $q_i$ and $q_j$. We can suppose without any loos generality, we can assume the center of mass   $c:=\sum_jm_jq_j/M$ ($M:=\sum_j m_j$) is fixed at the origin ($c=0$).

We assume that these bodies are in a \emph{rigid motion}. We recall that a \emph{rigid  motion}, is a solution of motions equations with $r_{ij}$ constant.  It is known (see -buscar referencias ) that a rigid motion is performed in a plane $\Pi$. We assume that $\Pi$ is the plane determined by the first two coordinates axes. Then a rigid motion has the form
\[q_j(t)=Q(\nu t)q^0_j,\]
where
\[
 Q(\nu t)=\begin{pmatrix}
           \cos(\nu t) & -\sin(\nu t) & 0\\
           \sin(\nu t) & \cos(\nu t) & 0\\
           0            &     0     &  1\\
          \end{pmatrix}
\]
and $q^0_j\in\Pi$, $j=1,\ldots,n$ are vectors in a planar \emph{central configuration} (CC) in $\rr^3$, i.e. there exists $\lambda\in\rr$ such that
\[ \nabla_jU(q^0_1,\ldots,q^0_n)+\lambda m_jq^0_j=0,\quad j=1,\ldots,n.\]
where the \emph{potential function} $U$ is defined by:
\begin{equation}\label{eq:potencial}
 U(x)=\sum_{i<j}\frac{m_im_j}{r_{ij}},
\end{equation}
and $\nabla_j$ denotes the $3$-dimensional partial gradient with respect to $q_j$.
According to \cite[Eq. (2.16)]{JaumeLlibre276} we have $\nu^2=\lambda$. Then the primaries bodies perform a periodic motion with period $T:=2\pi/\nu$ .

We suppose that we have a massless particle with coordinates $q(t)=(x(t),y(t),z(t))\in\rr^3$. This particle does not disturb the rigid motion of  primaries.  We want to find conditions under which this particle perform a $T$-periodic motion on the third axis of coordinates.

The particle $q$ satisfies the Newtonian equations of motion
\begin{equation}\label{eq:newton}
 \ddot{q}=\sum_{i=1}^n\frac{m_i(q_i-q)}{|q_i-q|^3},
\end{equation}




\begin{thm}\label{thm:prim} There exists a non-stationary  solution of \eqref{eq:newton} with $x(t)=y(t)=0$ if and only if $q^0_1,\ldots,q^ 0_n$ satisfy that for any $r>0$, such that the set
\[F_r:=\{i:|q_i^ 0|=r\}\]
is non empty, that
\begin{equation}\label{eq:suma0}\sum_{i\in F_r}m_iq_i^ 0=0.\end{equation}
i.e. every maximal equidistant from origin set of bodies has center of mass equal to $0$.
\end{thm}

If condition \eqref{eq:suma0} holds then equation \eqref{eq:newton} is reduced to
\begin{equation}\label{eq:eq_new_red}
 \ddot{z}=-\sum_{i=1 }^n\frac{m_iz}{(s_i^2+z^2)^{\frac32}},
\end{equation}
with  $s_i=|q_i^0|$.



We note that in order to get collisionless solutions of problem $P$ we need that no primary body is located in the center of mass. We say that a
CC is \emph{admissible} if it is non-collisional and satisfies \eqref{eq:suma0}. In the following theorem we characterize all admisible configurations with 3 or 4 bodies.

\begin{thm}\label{thm:caracterizacion}
The only 3-body admissible CC is the equilateral triangle with three equal masses. In the case of 4-body, an admissible CC  has two pairs of equal masses and satisfies some of the following properties: it is collinear and symmetric around the center of mass or it is a rhombus with the equal masses in opposite vertices, being the minor masses near from origin. In the particular case that the four masses are equal the CC is a square.
\end{thm}

\begin{thm}\label{thm:prin_ine} We assume that $q_1,\ldots,q_n$ is a admisible CC. A sufficient condition for that the problem $(P)$ has non trivial sotutions is that
\begin{equation}\label{eq:ine_prin}
 \sum_{i<j}\frac{m_im_j}{r_{ij}}<\left(\sum_{i=1}^n\frac{m_i}{s_i^3}\right)\left(\sum_{i=1}^nm_is_i^2\right).
\end{equation}



\end{thm}






\section{Proofs}



\begin{lem}\label{lem:1} For $c>0$ we define the function $y_c(t):=(c+t)^{-3/2}$. If $0<t_1<t_2<\ldots<t_k$ then the functions $y_j(t):=y_{t_j}(t)$  are linearly independent on  each open interval   $I\subset \mathbb{R}^+$.
\end{lem}
\begin{proof} It is sufficient to prove that Wronskian

 \[W:=W(y_1,\ldots,y_k)(t)=\det\begin{pmatrix}
			      y_1 & \cdots & y_k\\
			      \frac{dy_1}{dt}&  \cdots & \frac{dy_k}{dt}\\
			      \vdots & \ddots & \vdots \\
			      \frac{d^{k-1}y_1}{dt^{k-1}}&  \cdots & \frac{d^{k-1}y_k}{dt^{k-1}}\\
                           \end{pmatrix}
\]
is not null on $I$.

Using induction is easy to show that
\begin{equation}\label{eq:der_ind}\frac{d^iy_c}{dt^i}=\beta_{i}y_{c}^{\frac{2i+3}{3}},\quad\hbox{for some }\beta_{i}\neq 0, \hbox{ and for all }i=1,\ldots.
\end{equation}
Fix any $t\in I$. Then, according to \eqref{eq:der_ind} and writing $\lambda_j:=(t+t_j)^{-1}$, we have

\[
\begin{split}
  W(t)&=\det
    \begin{pmatrix}
      \lambda_1^{3/2} & \lambda_2^{3/2} &\cdots & \lambda_k^{3/2} \\
      \beta_1\lambda_1^{5/2} &\beta_1 \lambda_2^{5/2} &\cdots &\beta_1 \lambda_k^{5/2}\\
      \vdots & \vdots &\ddots & \vdots\\
      \beta_{k-1}\lambda_1^{k+1/2} & \beta_{k-1}\lambda_2^{k+1/2} &\cdots & \beta_{k-1}\lambda_k^{k+1/2}
    \end{pmatrix}
  \\
  &= \beta_1\beta_2\cdots\beta_{k-1} \lambda_1^{3/2}\lambda_2^{3/2}\cdots \lambda_k^{3/2}
     \det \begin{pmatrix}
      1& 1 &\cdots & 1 \\
      \lambda_1 & \lambda_2 &\cdots & \lambda_k\\
      \vdots & \vdots &\ddots & \vdots\\
      \lambda_1^{k-1} & \lambda_2^{k-1} &\cdots & \lambda_k^{k-1}
    \end{pmatrix}
  \\
  &= \beta_1\beta_2\cdots\beta_{k-1} \lambda_1^{3/2}\lambda_2^{3/2}\cdots \lambda_k^{3/2}
  \prod_{1\leq i<j\leq n}(\lambda_j-\lambda_i)
,
\end{split}
\]
where the last equality follows of the well known Vandermonde determinant identity. Therefore $W\neq 0$ if and only if $\lambda_i\neq\lambda_j$, $i\neq j$,
which in turn is equivalent to $t_i\neq t_j$, $i\neq j$.
\end{proof}


\begin{proof} [Proof Theorem \ref{thm:prim}] We use a rotating coordinate system where the primaries are fixed. Concretely we  put
\[\xi=Q(-\nu t)q.\]
In this system the motion equations are
\begin{equation}\label{eq:mov_rot}\ddot{\xi}+2\nu B\dot{\xi}+\nu^2 C\xi=\sum_{i=1}^n\frac{m_i(q_i^0-\xi)}{|q_i^0-\xi|^3},\end{equation}
where
\[B:=\begin{pmatrix}
       J & 0_{2\times 1}\\
       0_{1\times 2} & 0\\
     \end{pmatrix},\quad J:=\begin{pmatrix}
       0 & -1\\
       1 & 0\\
     \end{pmatrix}\quad\hbox{and}\quad C=\begin{pmatrix}
       -I_{2} & 0_{2\times 1}\\
       0_{1\times 2} & 0\\
     \end{pmatrix},
\]
where $0_{n\times m}$ and $I_{n}$ denote the null $n\times m$ matrix  and the identity $n\times n$ matrix respectively. Assuming that the masless particle is moving on the $z$-axis then $\xi=q=(0,0,z)$ and the Coriolis and centrifugal forces,   $2\nu B\dot{\xi}$ and $\nu^2 C\xi$ respectively, are null. Therefore, taking account in the first two equation in \eqref{eq:mov_rot} and identifying the vectors $q_i^0$, $i=1,\ldots,n$ with vectors in $\rr^2$, we have
\[
\sum_{i=1}^n\frac{m_iq_i^0}{|q_i^0-\xi|^3}=0.
\]

Let $D=\{|q_i^0|: i=1,\ldots,n\}$.  Suppose that $D=\{r_1,\ldots,r_k\}$, with $r_i\neq r_j$ for $i\neq j$, and  $\{1,\ldots,n\}=F_1\cup \cdots\cup F_k$, where if $i\in F_j$ then $|q_i^0|=r_j$. Then
\[\sum_{j=1}^k\left\{\frac{1}{(r_j^{2}+z^2)^{3/2}}\sum_{i\in F_j}m_iq_i^0\right\}=0.\]

Since we are considering a non-stationary solution, we have that $z(t)$ is not constant. Therefore there exists an interval $I\subset\rr^+$ where
\[\sum_{j=1}^k\left\{\frac{1}{(r_j^{2}+s)^{3/2}}\sum_{i\in F_j}m_iq_i^0\right\}=0,\quad s\in I.\]
Then, according to Lemma \ref{lem:1}, we obtain \eqref{eq:suma0}.

If  \eqref{eq:suma0} is satisfied then the force field $F$ acting on the masless
particle carries the $z$ axis in itself. Therefore, from the existence and uniqueness theorem and other elementary properties of system of ODEs we obtain a solution  of \eqref{eq:newton} with $x(t)=y(t)=0$.  \end{proof}




\begin{proof}[Proof Theorem \ref{thm:caracterizacion}]
For the case of 3-bodies, we note that the Theorem \ref{thm:prim} and the fact that the center of masses is an excluded position imply that if $F_r$ is not empty then $\# F_r\geq 2$. Hence an admissible 3-body CC consists of three equidistant bodies from the origin. Therefore, it must to be the Lagrangian equilateral triangle configuration. Now, equation \eqref{eq:suma0} implies that every bodies has the same mass.

In the case of 4-bodies, we have again that $\# F_r\geq 2$.  We consider two cases, the first one  $|q_1|\neq|q_2|$.  Therefore we can suppose that $|q_1|=|q_3|$ and $|q_2|=|q_4|$. Now \eqref{eq:suma0} implies that
 $m_1=m_3$ and $m_2=m_4$.  We divide the plane in two cones by means of  the line $L$ joining $q_1$  and $q_3=-q_1$ together with its perpendicular bisector $M$.  From the Perpendicular Bisector Theorem (see \cite{moeckel1990central}), we have that if  $q_2$  is in a open cone, then  $q_4$ is in the other one. But on the other hand \eqref{eq:suma0} implies $q_2=-q_4$, which is a contradiction. Then $q_2,q_4\in M$ or $q_2,q_4\in L$ (since $q_2=-q_4$ the case $q_2\in L$ and $q_4\in M$ is impossible). In the first case the CC is a rhombus with the larger masses closer to the origin (see  \cite{perez2007convex}). The second case we have a collinear CC which is also symmetric by  \eqref{eq:suma0}.
 It remains to discuss the case of $|q_1|=|q_2|=|q_3|=|q_4|$. In this situation, in \cite{hampton2005co} was proved that the configuration is the equal mass square.

\end{proof} 



\begin{proof}[Proof Theorem \ref{thm:prin_ine}] The following reasoning follows ideas developed in \cite{zhao2015nonplanar} and \cite{li2013characterization}.  We consider the action integral
\[I(z)=\int_0^T\frac12|z'|^2+\sum_{i=1}^n\frac{m_i}{\sqrt{s_i^2+z^2}}dt,\]
Then $T$-periodic solutions of \eqref{eq:eq_new_red} are critical points of $I$ in the space $H^1_T([0,T],\rr)$ of the functions  absolutely continuous, $T$-periodic with $u'\in L^2([0,T],\rr)$ (see \cite[Cor. 1.1]{Mawhin2010}). We prove the existence of critical points by means of the direct method of calculus of variations, i.e. we will prove that $I$ has minimum.  The functional $I$ is not coercive in $H^1_T([0,T],\rr)$,  this deficiency is drawn with symmetry techniques (see \cite{David-2004}). There is an action of the group $\mathbb{Z}_2$ on $H^1_T([0,T],\rr)$ defined by $(\bar{0}\cdot u)(t)=u(t)$ and $(\bar{1}\cdot u)(t)=-u(t+\frac{T}{2})$. The functional $I$ is $\mathbb{Z}_2$-invariant. We define the space of all $\mathbb{Z}_2$-symmetric funcions (assosiated to the italian symetry) \[\Lambda([0,T],\mathbb{R}):=\left\{ u\in H^1_T([0,T],\rr) | u(t)=-u\left(t+\frac{T}{2}\right)\right\}.\]
Then a critical point of $I$ in $\Lambda([0,T],\mathbb{R})$ is a critical point of $I$ in $H^1_T([0,T],\rr)$ (see \cite{David-2004} and \cite{RichardPalais274}). Moreover, the funciontal $I$ restricted to $\Lambda$  is coercive. This follows from an obvious adaptation of proposition 4.1 of \cite{David-2004}.



\end{proof}



\section*{Acknowledgments}



\bibliographystyle{plainurl}
 %\bibliographystyle{apalike-url}
 \bibliography{mecanica_celeste}


\end{document}


