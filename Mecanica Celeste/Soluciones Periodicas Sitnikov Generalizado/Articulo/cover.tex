\documentclass{article}
\usepackage{url}
\begin{document}

\begin{flushright}
G. Beltritti
F. Mazzone,
M. Oviedo


Dpto de Matem\'atica

Facultad de Ciencias Exactas, F\'isico-Qu\'imicas y Naturales

Universidad Nacional de R\'io Cuarto

Ruta 36, Km 601

X5804BYA-R\'io Cuarto

Argentina
\end{flushright}

\today
\vspace{1cm}

Dear Editors,

Celestial Mechanics and Dynamical Astronomy

\vspace{.5cm}

We  submit a revised version of our manuscript previously entitled "A generalized Sitnikov problem". Currently, at the suggestion of the commentators, the title of the article is  "The Sitnikov problem for several primary bodies configuration".

As the Editors suggest, all changes are marked in the manuscript with boldface. The only exception are those changes which are repetitive. Specifically, the change from "balanced" to "admissible", the replacement of "for every $t$" by "for every $t \in \mathcal{O}$" and the elimination of the word "collisionless".

\textbf{Change list:}
\begin{enumerate}
 \item New title.
 \item We introduce a square symbol to indicate the end of proofs.
 \item The following sentences were added in the introduction on page 3:

"The Perpendicular Bisector Theorem of Moeckel (see [21]) was an important help to solve
this question".

"We use two alternative techniques to solve
the problems raised in that section: i) elementary arguments using energy
conservation ([2, Ch. 2]) and ii) variational techniques inspired in [15, 34, 35]".

"For example, the results in Section 5 concerning admissible configurations".
\item 

\end{enumerate}


Next we respond to the questions raised by the  reviewers and list the changes introduced.

Answers  reviewer # 1:

1) We introduce a square symbol to indicate the end of proofs.

2) We expand more broadly on the question and introduce a reference justifying the statement commented by the reviewer. In order to give more clarity to what was indicated by the reviewer, we also introduced a set $ \ mathcal {O} $  for indicate the domain of solutions when they are not defined in all time  and we replaced in several places the phrase "for every $ t $" by "for every $ t \ in \ mathcal {O} $"

3) In the introduction we add brief comments on the techniques used.

4) We add information about what we consider are the main novelties contributed by the article and the research lines that are opened from it.


Answers  reviewer # 2:

1) We changed the title to a more specific one.

2) We add a reference and a brief explanation that justifies our affirmation.

3) We changed ``balanced'' by ``admissible''

4) We omit the invocation to the concept of "collisionless". Instead, we incorporate the condition $q_i \ neq 0$ in definition 2. This does not change the results obtained in the article at all. It was only necessary to eliminate the word "collisionless" every time it appeared in the text.

5) We think that no change should be made. We agree with the reviewer, any masses in an equilateral triangle form a central configuration, but the condition of admissibility is satisfied only if the masses are equal.

6) We rewrite the definition 4.

7) We use the computer as a calculator to check the validity of an inequality in a finite number of cases.This is rigorous since the differences obtained between the sides of the inequality were   higher order than the typical rounding errors in the operations carried out. Note that for the rest of the cases, which are infinite, the inequality is proven by a mathematical reasoning. We think that it is not necessary to modify the article, however if in the opinion of the editors this should be done with pleasure we will do it.

8) We rewrite the theorem 7.

9) We clarify this point.

10) See the response to the Editor-in-Chief.

Answers  Editor-in-Chief:

Unfortunately, we are not familiar enough with any native English-speaking mathematician we may ask you to review our article. We have consulted another person for whom English is a second language. We hope that most of the problems are solved.

Sincerely,
\vspace{.5cm}

Fernando Mazzone

\end{document}
