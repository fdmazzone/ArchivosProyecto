\documentclass{article}
\usepackage{url}
\begin{document}

\begin{flushright}
G. Beltritti
F. Mazzone,
M. Oviedo


Dpto de Matem\'atica

Facultad de Ciencias Exactas, F\'isico-Qu\'imicas y Naturales

Universidad Nacional de R\'io Cuarto

Ruta 36, Km 601

X5804BYA-R\'io Cuarto

Argentina
\end{flushright}

\today
\vspace{1cm}

Dear Editors,

Celestial Mechanics and Dynamical Astronomy

\vspace{.5cm}

We  submit a revised version of our manuscript previously entitled "A generalized Sitnikov problem". Currently, following the suggestion of the reviewers, the title of the article is  "The Sitnikov problem for several primary bodies configurations".


The editors suggested that we send a detailed list with the corrections and that we mark them in boldface in the manuscript. We think that enumerating the changes in a list can be impractical, given that whole paragraphs were added and that the use of bold letters can be confusing with parts of the text that are naturally written in boldface. We adopted the criterion of marking all phrases that were modified in red. However, if the publishers request it, we gladly make the list of changes.


The only change that is not indicated in red in the manuscript is the following: we deleted from page 6 the phrase: "Hereafter,  we say that   $q_1,\ldots,q_n$ is a collisionless configuration  when $q_i\neq 0$ for $i=1,\ldots,n$. We note that despite of this fact the system can have collisions; for example,  when the primaries have a homothetic collapsing motion."



Next we respond to the questions raised by the  Editor-in-Chief and the  reviewers and list the changes introduced as consequence of them.

Answers  Editor-in-Chief:

Unfortunately, we are not familiar enough with any native English-speaking mathematician  to ask for help in this regard.   We have consulted with another two person. One mathematician and the other one is english teacher. We hope that most of the problems are solved.

Answers  reviewer # 1:

1) We introduce a square symbol to indicate the end of proofs.

2) We explain more broadly the question and introduce a reference justifying the statement commented by the reviewer. In order to give more clarity to what was indicated by the reviewer, we also introduced a set $ \mathcal {O} $  for indicate the domain of solutions when they are not defined in all time  and we replaced in several places the phrase "for every $ t $" by "for every $ t \in \mathcal {O} $"

3) In the introduction we add comments on the techniques used.

4) We add information about what we consider are the main novelties contributed by the article.


Answers  reviewer # 2:

1) We change the title to a more specific one.

2) We add a reference and a brief explanation that justifies our affirmation.

3) We change ``balanced'' by ``admissible''

4) We omit the invocation to the concept  "collisionless". Instead, we incorporate the condition $q_i \neq 0$ in definition 2. This presentation is equivalent to the previous one. It was only necessary to eliminate the word "collisionless" every time it appeared in the text.

5) We think that no change should be made. We agree with the reviewer, any masses in an equilateral triangle form a central configuration, but the condition of admissibility is satisfied only if the masses are equal.

6) We rewrite definition 4.

7) We use the computer as a calculator to check the validity of an inequality in a finite number of cases.This is rigorous since the differences obtained between the sides of the inequality were  of an higher order than the typical rounding errors in the operations carried out. Note that for the rest of the cases, which are infinite, the inequality is proven by a mathematical reasoning. We think that it is not necessary to modify the article, however if in the opinion of the editors this should be done with pleasure we will do it.

8) We rewrite the theorem 7.

9) We clarify this point.

10) See the response to the Editor-in-Chief.



Sincerely,
\vspace{.5cm}

Gast\'on Beltritti

\end{document}
