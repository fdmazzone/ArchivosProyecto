\documentclass[twoside]{article}


%%%%%%%%% Packages %%%%%%%%%%%%%%%%%%%%%%%%%%%%
\usepackage{amssymb,amsthm}
\usepackage{amsmath}
\usepackage{mathabx}
\usepackage{fancyhdr}
\usepackage{url}
\usepackage{hyperref}
\usepackage{enumitem}
\usepackage{cleveref}
\usepackage{color}



%%%%%%%%%%%%%%%% Theorems like environments
\newtheorem{thm}{Theorem}[section]
\newtheorem{cor}[thm]{Corollary}
\newtheorem{lem}[thm]{Lemma}
\newtheorem{defi}[thm]{Definition}
\newtheorem{prop}[thm]{Proposition}
\theoremstyle{remark}
\newtheorem{comentario}{Remark}




\newcounter{example}[section]
\setcounter{example}{0}
\makeatletter
\renewcommand{\p@example}{\thesection.} % "prefix" for cross-referencing
\makeatother
%\newenvironment{example}{\noindent\textit{Example \arabic{example}}.}{\addtocounter{example}{1}}
\newenvironment{example}{\refstepcounter{example}\noindent\textit{Example \arabic{section}.\arabic{example}}.}{ }


%%%%%%%%%%%%% New item with label%%%%%%%%%%%%%%%%%%%%%%%%%
\makeatletter
\newcommand{\labitem}[2]{%
\def\@itemlabel{#1}
\item
\def\@currentlabel{#1}\label{#2}}
\makeatother
\makeatletter
\def\namedlabel#1#2{\begingroup
    #2%
    \def\@currentlabel{#2}%
    \phantomsection\label{#1}\endgroup
}
\makeatother



%%%%%%%%%%%%%%%% Title 

\title{Clarke dual method for Hamiltonian systems with non standard grow}
\author{(In alphabetical order)\\[3mm]
Sonia Acinas \thanks{SECyT-UNRC,  FCEyN-UNLPam and UNSL}\\
Dpto. de Matem\'atica, Facultad de Ciencias Exactas y Naturales\\
Universidad Nacional de La Pampa\\
(L6300CLB) Santa Rosa, La Pampa, Argentina\\
\url{sonia.acinas@gmail.com}\\[3mm]
Jakub Maksymiuk \\[3mm]
 Fernando D. Mazzone \thanks{SECyT-UNRC, FCEyN-UNLPam and CONICET}\\
Dpto. de Matem\'atica, Facultad de Ciencias Exactas, F\'{\i}sico-Qu\'{\i}micas y Naturales\\
Universidad Nacional de R\'{i}o Cuarto\\
(5800) R\'{\i}o Cuarto, C\'ordoba, Argentina,\\
\url{fmazzone@exa.unrc.edu.ar} \\
}

\date{}

%%%%%%%%%%%%%%%%New commands

\newcommand{\orlnor}{\|_{L^{\Phi}}}
\newcommand{\linf}{\|_{L^{\infty}}}
\newcommand{\lphi}{L^{\Phi}}
\newcommand{\lpsi}{L^{\Phi^{\star}}}
\newcommand{\ephi}{E^{\Phi}}
\newcommand{\claseor}{C^{\Phi}}
\newcommand{\wphi}{W^{1}\lphi}
\newcommand{\wphit}{W^{1}\lphi_T}
\newcommand{\sobnor}{\|_{W^{1}\lphi}}
\newcommand{\domi}{\mathcal{E}^{\Phi}}
\renewcommand{\b}[1]{\boldsymbol{#1}}
\newcommand{\rr}{\mathbb{R}}
\newcommand{\nn}{\mathbb{N}}
%\newcommand{\cdot}{\b{\cdot}}
\renewcommand{\leq}{\leqslant} 
\renewcommand{\geq}{\geqslant} 
\newcommand{\epsi}{E^{\Phi^{\star}}}
\newcommand{\Phie}{\Phi^{\star}}
\newcommand{\lip}{\mathop{\rm Lip}}




\DeclareSymbolFont{symbolsC}{U}{txsyc}{m}{n}
\DeclareMathSymbol{\strictif}{\mathrel}{symbolsC}{74}



\begin{document}


\maketitle
%
\begingroup%Locallizing the change to `thefootnote'.
    \renewcommand{\thefootnote}{}%Removing the footnote symbol.
    %
    \footnotetext{%
    %   2010 Mathematics Subject Classification
    %   http://www.ams.org/msc/
    \textbf{2010  AMS Subject Classification.} Primary: 34C25.
    Secondary: 34B15.
    }%
        \footnotetext{%
    \textbf{Keywords and phrases.} 
Periodic Solutions,  Orlicz Spaces,   Euler-Lagrange,   Critical Points.
    }%
    \endgroup
%
%
%
%

\begin{abstract}
In this paper we consider the problem of finding periodic solutions of certain Hamiltonian systems .....blablabla

\end{abstract}






\pagestyle{fancy} \headheight 35pt \fancyhead{} \fancyfoot{}

\fancyfoot[C]{\thepage} \fancyhead[CE]{\nouppercase{F.D. Mazzone   and S. Acinas}} \fancyhead[CO]{\nouppercase{\section}}

\fancyhead[CO]{\nouppercase{\leftmark}}


%\tableofcontents

\section{Main problem}
Let $H:[0,T]\times \rr^d\times \rr^d \to \rr$. 
We are looking for periodic solutions of the Hamiltonian system
\begin{equation}\label{eq:ham-sis-qp}
\left\{\begin{array}{ll}
\dot{q}(t)&=D_pH(t,q(t),p(t))\\
\dot{p}(t)&=-D_qH(t,q(t),p(t))\\
p(0)&=p(T)\hbox{ , } q(0)=q(T)
\end{array}
\right.
\end{equation}
for $t \in [0,T]$. \textcolor{red}{I think that, like in \cite{Mawhin2010}, is better to present the Hamiltonian problem as the main problem}


An alternative writing of \eqref{eq:ham-sis-qp} using the combined variable $u=(q,p)$ and the canonical symplectic matrix 

\[J=
\begin{pmatrix}
0&I_{d\times d}
\\
-I_{d\times d}&0
\end{pmatrix}
\]
is the following 
\begin{equation}
\dot{u}=J \nabla H(t,u(t))
\end{equation}
or equivalently
\begin{equation}
J\dot{u}=- \nabla H(t,u(t))
\end{equation}
where $\nabla H$ is the gradient of $H$ with respect to the combined variable.

\section{Preliminaries}
We will use some basic concepts of convex analysis that we list below.

Let $\Gamma_0(\rr^d)=\{F:\rr^d\to (-\infty,+\infty]
\\
\mbox {convex, lower semicontinous functions with non-empty effective domain.}\}$

The Fenchel conjugate of $F$ is given by
\[
F^\star(p)=\sup\limits_{q \in \rr^d} \left\langle p,q\right\rangle-F(q)
\]
The Fenchel conjugate satisfies the following properties:
\begin{enumerate}
\item $F^\star \in \Gamma_0(\rr^d)$
\item If $F\leq G$, then  $G^\star \leq F^\star$.
\item If $G(q)=\alpha F(\beta q)+\sigma$ with $\alpha,\beta,\sigma>0$ then
$G^\star(p)=\alpha F^\star(\frac{p}{\beta \alpha})-\sigma$
\end{enumerate}

Let $\Phi:\mathbb{R}^d\to [0,+\infty)$ be  a differentiable, convex function such that $\Phi(0)=0$, $\Phi(q)>0$ if $q\neq 0$, $\Phi(-q)=\Phi(q)$,
 and
\begin{equation}\label{eq:N-sub-inf}
\lim_{|q|\to\infty}\frac{\Phi(q)}{|q|}=+\infty,
\end{equation}
where $|\cdot|$ denotes the euclidean norm on $\rr^d$. From now on, we say that $\Phi$ is an $G$-function if $\Phi$ satisfies the previous properties.

We write $\Phie$ for the Fenchel conjugate of $\Phi$.

We do not assume  that $\Phi$ and $\Phi'$ satisfy the $\Delta_2$-condition.


We denote by  $\partial F(q)$ the subdifferential of $F$ in the sense of convex analysis (see \cite{clarke1990optimization,clarke2013functional})

The next result is a generalization of \cite[Prop. 2.2, p.34]{mawhin2010critical}

\begin{prop}\label{prop: cota-conj-phi}
Let $F \in  \Gamma_0(\rr^d)$. Suppose that there exist an anisotropic function $\Phi$ and
non negative  constants $\beta,\gamma$ such that
\begin{equation}\label{eq:cotas-F-aniso}
-\beta \leq F(q) \leq \Phi(q)+\gamma, \mbox{ for all } q \in \rr^d. 
\end{equation}
Now, if $p \in \partial F(q)$ then 
\begin{equation}\label{eq:trans-grad-aniso}
\Phie(p)\leq \Phi(2q)+2(\beta+\gamma).
\end{equation}
\end{prop}

\begin{proof}
If $p \in \partial F(q)$, from \cite[Thm. 2.2, p.33]{mawhin2010critical}, 
\begin{equation}\label{eq:trans-grad}
F^\star(p)=\langle p,q \rangle-F(q)
\end{equation}
Conjugating \eqref{eq:cotas-F-aniso}, we have
\begin{equation}\label{eq:trans-conj}
F^\star(p)\geq \Phie(p)-\gamma.
\end{equation}
From Young's inequality, we get
\begin{equation}\label{eq:trans-young}
\langle p,q \rangle =\frac{1}{2} \langle p,2q \rangle \leq \frac{1}{2 }\Phie(p)+\frac{1}{2}\Phi(2q)
\end{equation}
By \cref{eq:cotas-F-aniso,eq:trans-grad,eq:trans-conj,eq:trans-young}, we get
\[
\Phie(p)\leq \frac{1}{2}\Phie(p)+\frac{1}{2}\Phi(2q)+\beta+\gamma
\]
which implies \eqref{eq:trans-grad-aniso}
\end{proof}

\begin{comentario}\label{com:mejora_desi} Inequality \eqref{eq:trans-grad-aniso} is a few better than the corresponding in \cite[Prop. 2.2]{mawhin2010critical} because the the case of power function  we obtain $(\beta+\gamma)^{1/p}$, meanwhile in \cite{mawhin2010critical} appears $(\beta+\gamma)^{1/(p-1)}$ .

\end{comentario}


\section{Optimal bounds for a symplectic bilinear form}

We  consider the  Euclidean space $\rr^{2n}$ equipped with the standard  symplectic structure given by bilinear canonical symplectic 2-form

\[\Omega(u,v):=\langle Ju,v\rangle .\]

\textcolor{red}{As Jakub observed we can not consider any $G$-function on the symplectic manifold $\rr^{2n}$. I think that the following can be the appropriate form of the $G$-function defined on the symplectic manifold $\rr^{2n}$}

\begin{defi}
% Let $\Phi$ a $G$-function defined in the symplectic manifold $\rr^{2n}$. 
%We say that  $\Phi$ is a \emph{symplectic $G$-function} if
 %\begin{equation}\label{eq:J-phi1}
 %\Phi(Ju)= \Phie(u).
%\end{equation}
Let $\Psi:\rr^{2n}\to \rr$ be an anisotropic function G-function.
We say that $\Psi$ is a \emph{symplectic $G$-function} if $\Psi^*(Ju)\prec \Psi(u)$, i.e.
there exists $C,k>0$ such that
\[
\Psi^*(Ju)\leq \Psi(ku)+C
\]
\end{defi}

Given $u=(q,p)\in \rr^{2n}$ and $\Phi:\rr^n \to \rr$, 
\textcolor[rgb]{1,0,0}{Jakub} define $\hat{\Phi}:\rr^{2n}\to \rr$ by
\[
\hat{\Phi}(q,p)=\Phi(q)+\Phi^*(p)
\]
with $\Phi^* \prec \Phi$.

\begin{prop}
$\hat{\Phi}$ is symplectic.
\end{prop}

\begin{proof}
We have ${\hat\Phi}^*(q,p)=\Phi^*(q)+\Phi^*(p)$, then 
${\hat \Phi}^*(Ju)=\Phi^*(p)+\Phi^*(-q)\prec \Phi(q)+\Phi(p)=\hat{\Phi}(u)$.
\end{proof}

\textcolor[rgb]{1,0,0}{Fernando} suggests  $\overline{\Phi}(q,p)=\Phi(q)+\Phi^*(p)$.

\begin{prop}
$\overline \Phi$ is symplectic.
\end{prop}

\begin{proof}
We have $\Phi^*(q,p)=\Phi(p)+\Phi^*(q)$, then 
$\overline{\Phi}^*(Ju)=\Phi(-q)+\Phi^*(p)=\overline{\Phi}(u)$.
\end{proof}




\begin{thm}
$J$ induces an embedding of $L^{\Psi^*}([0,T],\rr^{2n})$
into $L^{\Psi}([0,T],\rr^{2n})$ when $\Psi$ is symplectic.
\end{thm}


\begin{proof}
As $\Psi$ is symplectic, there exist $k,c$ such that
$
\Psi^*(Ju)<\Psi(ku)+c
$
then 
\[
\int \Psi^*\left(\frac{Ju}{k \lambda}\right) \leq cT+\int \Psi\left(\frac{u}{\lambda}\right)<\infty
\]

If $\|u\|_{L^{\Psi}}=1$, then $\int \Psi(u)\leq 1$ and
\[
\int \Psi^*\left(\frac{Ju}{k}\right) \leq cT+1 
\]
As $\Psi^*$ is convex, we have 
\[
\int \Psi^*\left(\frac{Ju}{(cT+1) \lambda}\right) 
\leq \frac{1}{cT+1} \int \Psi^*\left(\frac{Ju}{k}\right)\leq 1
\]
then 
\[
\|Ju \|_{L^{\Psi^*}}\leq (cT+1)k:=c_0
\]
Finally, for any $u$, 
\[
\|Ju\|_{L^{\Psi^*}}\leq c_0\|u\|_{L^{\Psi}}.
\]
\end{proof}



\begin{cor}
If $
\Omega(u,v)=\int J v\cdot u
$
and $\Psi$ is symplectic, then $\Omega$ is well defined in $L^{\psi} \times L^{\Psi}$.
\end{cor}







\begin{thm}
Let $\Phi$ be a symplectic G-function. 
There exist  $C_{\Phi},C_1$ and $\Lambda>0$ such that for every $u \in W_T^1L^{\Phi}([0,T],\rr^{2n})$
we have
\begin{equation}\label{eq:cotaJu}
\Omega(\dot{u},u):=\int_0^T J\dot{u}\cdot u\,dt\geq -C_{\Phi}\int_0^T \Phi\left(\frac{\dot{u}}{\Lambda}\right)\,dt-C_1
\end{equation}
\end{thm}


\begin{proof}
Let  $u\in W^1L^{\Phi}([0,T],\rr^{2n})$. As usual we write $u=\tilde{u}+\overline{u}$ where
 \[\overline{u}=\frac{1}{T}\int_0^Tu(t)dt.\]
 From \cite[Lem. 2.4]{MA2017} we have that
\begin{equation}\label{eq:PoinW-ineq}
\int_0^T\Phi(\tilde{u})dt\leq\int_0^T\Phi(T\dot{u})dt.
 \end{equation}

By Young's inequality, the fact that a $\Phi$ is a simplectic G-function and \eqref{eq:PoinW-ineq}, we obtain
%the inequality
 %\[\int_0^T\Phi(\tilde{u})dt\leq\int_0^T\Phi(T\dot{u})dt.\]
%given by \cite[Lem. 2.4]{MA2017}, we get
\[
\begin{split}
\int_0^T J\dot{u} \cdot u \,dt=\frac{k}{T}\int_0^T J\frac{T\dot{u}}{k}\cdot \tilde{u} \,dt\geq 
\\
-\frac{k}{T}\left\{\int_0^T \Phi^* \left(J\frac{T\dot{u}}{k}\right) \,dt+\int_0^T \Phi(\tilde{u}) \,dt\right\}\geq
\\
-\frac{k}{T}\left\{ 2\int_0^T \Phi(T\dot{u})\,dt+C\right\}
\end{split}
\]
\end{proof}




\section{Differentiability of Hamiltonian dual action}

\begin{thm}\label{thm:DiffDualAct}
Suppose that $\Phi:\rr^{2n}\to [0,+\infty)$ is a differentiable $G$-function, not necessarily symplectic. Additionally
\begin{enumerate}
\item \label{it:h1-prop-H}
$H:[0,T]\times\rr^{2n}\to \rr$ is measurable in $t$, continuously differentiable with respect to $u$.
\item \label{it:h2-cotaH-conjphi}there exist $\beta, \gamma \in L^1([0,T],\rr)$, $\Lambda>\lambda>0$ such that
\begin{equation}\label{eq:cota-H-phi-conj}
\Phi^{\star}\left(\frac{u}{\Lambda}\right)-\beta(t)\leq H(t,u) \leq \Phi^{\star}\left(\frac{u}{\lambda}\right)+\gamma(t)
\end{equation}
\end{enumerate}
Then there exists $\Lambda_0$ such that the dual action
\begin{equation}\label{eq:DualAct}
 \chi(v)=\int_0^T \frac{1}{2} \langle J\dot{v},v\rangle+H^{\star}(t,\dot{v})  dt
\end{equation}



is continuously differentiable in $\wphit([0,T],\rr^{2n}) \cap \{u|d(\dot{u},L^{\infty})<\Lambda_0\}$.

If $v$ is a critical point of $\chi$ with $d(\dot{v},L^{\infty})<\Lambda_0$, the function defined by 
$u(t)=\nabla  H^{\star}(t,\dot{v})$
solves 
\[
\left\{\begin{array} {lll}
\dot{u}&=&J\nabla H (t,u)
\\
u(t)&=&u(T)
\end{array}
\right.
\]
\end{thm}


\begin{proof}
Conjugating \ref{it:h2-cotaH-conjphi} we obtain
\begin{equation}\label{eq:cotaH*-phi}
\Phi(\lambda u)-\gamma(t)\leq H^{\star}(t,v)\leq \Phi(\Lambda v)+\beta(t)
\end{equation}
Since $H^{\star}$ is smooth, we have $\partial_vH^{\star}(t,v)=\{\nabla_v H^{\star}(t,v)\}$.
Applying Proposition \ref{prop: cota-conj-phi} with $F=H^{\star}$, $\Phi(\Lambda v)$ instead of $\Phi(u)$
and $u=\nabla H^{\star}(t,v) \in \partial_v H(t,v)$, inequality \eqref{eq:cota-H-phi-conj} becomes
\begin{equation}\label{eq:main-phiconj-phi}
\Phi^{\star}\left(\frac{\nabla H^{\star}(t,v)}{\Lambda}\right)\leq \Phi(2\Lambda v)+2(\beta+\gamma).
\end{equation}
which will be the main inequality in the proof.

We are planning to obtain  the structure condition \eqref{eq:condicion-estructura} of \cite{MA2017} which guarantees
differentiability. 

We consider the Lagrangian 
\begin{equation}
\mathcal{L}(t,v,\xi)=\frac{1}{2}\langle J\xi,v\rangle + H^{\star}(t,\xi)
\end{equation}
and we have to prove that there exist $\Lambda_0>\lambda_0>0$
such that 
\begin{equation}\label{eq:cota-L-gradL-conj}
|\mathcal{L}|+|\nabla_v\mathcal{L}|+\Phi^{\star}\left(\frac{\nabla_{\xi}\mathcal{L}}{\lambda_0}\right)\leq
a(v)\left(b(t)+\Phi\left(\frac{\xi}{\Lambda_0}\right)\right)
\end{equation}
We start with $|\mathcal{L}|$. From \eqref{eq:cotaH*-phi}, 
\[
|\mathcal{L}|\leq \frac{1}{2} |\langle J\xi,v\rangle|+H^{\star}(t,\xi)\leq \frac{1}{2}|\xi||v|+\Phi(\Lambda \xi)+\beta(t). 
\]
Since $\frac{\Phi(x)}{|x|}\to \infty$ as $|x|\to \infty$, there exists $C>0$ such that $|x|\leq \Phi(x)+C$ for all $x \in \rr^d$.
Then, 
\[
|\mathcal{L}|\leq \frac{1}{2} \frac{|v|}{\Lambda}\left(\Phi( \Lambda \xi)+C\right)+\Phi(\Lambda \xi)+\beta(t)
\leq \max\left\{\Lambda, \frac{|v|}{2\Lambda}\right\}
\left[\Phi(\Lambda \xi)+C+\beta(t)\right]
\] 
which is an estimate like  the second member of \eqref{eq:cota-L-gradL-conj}.

Now, we treat $|\nabla_v \mathcal{L}|$ and we get
\begin{equation}\label{eq:cota-grad-v}
|\nabla_v \mathcal{L}|=\frac{1}{2}|J\xi|\leq |\xi| \leq \frac{1}{2 \Lambda}(\Phi(\Lambda \xi)+C).
\end{equation}
which is also an estimate of the desired type.

Finally, we deal with $\Phi(\nabla_{\xi}\mathcal{L}{\lambda_0})$. 
As $\Phi^{\star}$ is a convex, even function, we have
\[\Phi^{\star}\left(\frac{\nabla_{\xi}\mathcal{L}}{\lambda_0}\right)=
\Phi^{\star}\left(\frac{-\frac{1}{2}Jv}{\lambda_0}+\frac{\nabla H^{\star}(t,\xi)}{\lambda_0}\right)
\leq
\frac{1}{2} \Phi^{\star}\left(\frac{Jv}{\lambda_0}\right)+
\frac{1}{2}\Phi^{\star}\left(\frac{2 \nabla H^{\star}(t,\xi)}{\lambda_0}\right).
\]
We choose $\frac{2}{\lambda_0}=\frac{1}{\Lambda}$ with $\Lambda$ as in \eqref{eq:main-phiconj-phi} and 
we finally have
\begin{equation}\label{eq:cota-conj-grad}
\begin{split}
\Phi^{\star}\left(\frac{\nabla_{\xi}\mathcal{L}}{\lambda_0}\right)\leq 
\Phi^{\star}\left(\frac{Jv}{2\Lambda}\right)+\Phi(2\Lambda\xi)+2(\beta+\gamma)=\\
\max\left\{\Phi^{\star}\left(\frac{Jv}{2\Lambda}\right),1\right\}
\left[\Phi(2\Lambda\xi)+2(\beta+\gamma)\right]
\end{split}
\end{equation}
which is a bound like the second member of \eqref{eq:cota-L-gradL-conj}.

Therefore, from \eqref{eq:cota-L-gradL-conj}, \eqref{eq:cota-grad-v}, \eqref{eq:cota-conj-grad} and choosing the worst functions $a$ and $b$, 
we obtain condition \eqref{eq:condicion-estructura}.

Next, \cite[Thm. 4.5]{MA2017} implies differentiability of $\chi$ in a set like 
$\wphit([0,T],\rr^d)\cap\{u|d(\dot{u},L^{\infty})<\lambda_0\}$.

If $v \in \wphit([0,T],\rr^d)$ is a critical point of $\chi$ with $d(\dot{v},L^{\infty})<\lambda_0$
then, from equations (21) of \cite{MA2017} we obtain
\[
0=\int_0^T \frac{1}{2} \langle J\dot{v},h\rangle-\frac{1}{2} \langle \dot{h},Jv \rangle
+\langle \nabla H^{\star}(t,\dot{v}),\dot{h} \rangle.
\]
The rest of the proof follows as in \cite{mawhin2010critical}.

\end{proof}


\section{Existence periodic solutions Hamiltonian system}

The following theorem extend  to a quite general function $\Phi$ the result in \cite[Thm. 3.1]{mawhin2010critical} formulated for $\Phi_2(u)=|u|^2/2$. Even more, our result improves a little bit  \cite[Thm. 3.1]{mawhin2010critical} in the sense that we obtain existence for $\Phi_2$ when the functions $l$ and $\gamma$,  introduced below, belong to  $L^2$ and $L^1$ respectively instead of $L^4$ and $L^2$ as it is assumed in
\cite[Thm. 3.1]{mawhin2010critical}. This little improvement is due to the observation in Remark \ref{com:mejora_desi}.

\begin{thm}\label{thm:solution-ham} Suppose that $\Phi$ is a symplectic $G$-function and
\begin{enumerate}
 \labitem{H1)}{it:hip1}   There exists $\xi\in E^{\Phi}([0,T],\rr^{2n})$ such that for every $u \in \rr^{2n}$ 
and a.e. $t \in [0,T]$
 \[H(t,u)\geq \langle \xi(t), u\rangle.
  \]
	
 \labitem{H2)}{it:hip2} There exists $\Lambda_0$ \textcolor[rgb]{1,0,0}{(indicar d\'onde vive en funci\'on de $C_{\Phi^{\star}}$???)} 
and $\alpha \in L^1([0,T],\rr) $
such that, for every $(t,u)\in [0,T]\times\rr^{2n}$ and a.e. $t \in [0,T]$, we have

 \[
  H(t,u)\leq \Phi\left( \frac{u}{\Lambda_0}\right)+\alpha(t).
 \]
 \labitem{H3)}{it:hip3} 
 \[
  \int_0^TH(t,u)dt\to+\infty,\quad\hbox{when } |u|\to+\infty.
 \]

\end{enumerate}

Then the problem xxxxx has at least one solution $u$ such that 
\[
v(t)=-J\left[u(t)-\frac{1}{T}\int_0^T u(s)\,ds\right]
\]
minimizes the dual action
\[
\chi(v)= \int_0^T \frac{1}{2} \langle J\dot{v},v\rangle+H^{\star}(t,\dot{v})  dt
\]

\end{thm}

\begin{proof} 
%Let $\delta$ be a positive number such that $\alpha+\delta<C_{\Phi}^{-1}$. Note that from  \eqref{delta2-potencias}  we have that
%\[\frac{K}{\alpha+\delta}\Phi\left( (\alpha+\delta) u\right)-
  %\frac{1}{\alpha}\Phi\left( \alpha u\right)\geq  \frac{K_1}{\alpha}\Phi\left( \alpha u\right), \]
%where $K_1=\left[ (\alpha+\delta)/\alpha\right]^{p_1-1}-1>0$ and $K, p_2$ are the constants in \eqref{delta2-potencias}. The constat $K_1$  depends only on $\alpha,\delta$ and $\Phi$. 

Suppose that $0<r<1$ and $\epsilon<\frac{r}{\Lambda_0}$

We define
\[
  H_{\epsilon}(t,u)=H(t,u)+\Phi(\epsilon u)
\]
 

%Let $\lambda=\min\{1,K_1/2\}$. 
By \ref{it:hip1}, Young's inequality and the convexity of $\Phi$, we have
\begin{equation}\label{eq:CotaSupH}
 \begin{split}
    H_{\epsilon}(t,u)  &\geq \left\langle \xi(t), u \right\rangle +\Phi(\epsilon u)
		\\
				&\geq -\Phi^{\star}\left(\frac{1}{r\epsilon} \xi(t)\right)-\Phi(r\epsilon u)+\Phi(\epsilon u)
		\\
		    &\geq  \Phi((1-r)\epsilon u)-\beta(t)
				  \end{split} 
\end{equation}
where $\beta(t):=\Phi^{\star}( \frac{1}{r\epsilon} \xi(t))\in L^1$. 

On the other hand
\begin{equation}\label{eq:CotaInfH}
  H_{\epsilon}(t,u)  \leq  \Phi\left(\frac{u}{\Lambda_0}\right)+\alpha(t)+\Phi(\epsilon u)\leq 
	(1+r)\Phi\left(\frac{u}{\Lambda_0}\right)+\alpha(t)
\end{equation}


From  \eqref{eq:CotaSupH}, \eqref{eq:CotaInfH} and  properties of Fenchel conjugate, we get
\begin{equation}\label{eq:CotaH*}
(1+r)\Phi^{\star}\left(\frac{\Lambda_0 u}{1+r}\right)-\alpha(t)\leq H^{\star}_{\epsilon}(t,u)\leq \Phi^{\star}\left(\frac{u}{(1-r)\epsilon}\right)+\beta(t).
\end{equation}

The perturbed Hamiltonian $H_{\epsilon}$ verifies the assumptions of Theorem \ref{thm:DiffDualAct}, 
then the dual action $\chi_{\epsilon}$
is continuously differentiable in $\wphit([0,T],\rr^{2n}) \cap \{u|d(\dot{u},L^{\infty})<\lambda_0\}$.

\textcolor[rgb]{1,0,0}{VER QUIEN EN $\lambda_0???$
}

%Moreover, since $\Phi\in\Delta_2\cap \nabla_2$ we have that the dual action
%
%\begin{equation}\label{eq:DualActDelta}
 %\chi_{\epsilon}(v)=\int_0^T \frac{1}{2} \langle J\dot{v},v\rangle+H_{\epsilon}^{\star}(t,\dot{v})  dt
%\end{equation}
%is continuously differentiable in $ W^1\lpsi_T([0,T],\rr^{2d})$. 


Now, we deal with the coercivity  of $\chi_{\epsilon}$ given by 
\begin{equation}\label{eq:DualActDelta}
\chi_{\epsilon}(v)=\int_0^T \frac{1}{2} \langle J\dot{v},v\rangle+H_{\epsilon}^{\star}(t,\dot{v})  dt
\end{equation}

From \eqref{eq:CotaH*} and \eqref{eq:cotaJu}, we have
\[
  \chi_{\epsilon}(v)\geq 
	-\frac{C_{\Phi^{\star}}}{2} \int_0^T \Phi^{\star} (T\dot{v})\,dt+(1+r)\int_0^T \Phi^{\star}
	\left(\frac{\Lambda_0 \dot{v}}{1+r}\right)\,dt-\int_0^T \alpha(t)\,dt-\frac{C_1}{2}T
\]

\textcolor[rgb]{1,0,0}{En el papel, no aparece el 2 dividiendo. No ser\'ia necesario, porque acotar\'iamos por -1, pero no recuerdo  
si esa era la idea o fue un olvido. En la cuenta final, no  genera problemas dejar el 2.
}

Let $\Lambda_0 \geq \max \{(1+r)T, C_{\Phi^{\star}}T\}$, then 
$\Lambda_0>\max\{T,C_{\Phi^{\star}}T\}$, $\frac{\Lambda_0}{T}>1$ and there exists $r>0$ such that $\frac{\Lambda_0}{T}=(1+r)$. 

Thus, 
\[
\begin{split}
  \chi_{\epsilon}(v)\geq 
	-\frac{C_{\Phi^{\star}}}{2} \int_0^T \Phi^{\star} (T\dot{v})\,dt+\frac{\Lambda_0}{T}\int_0^T \Phi^{\star}
	\left(T\dot{v}\right)\,dt-\int_0^T \alpha(t)\,dt-\frac{C_1}{2}T
	\\
	=	\left(-\frac{C_{\Phi^{\star}}}{2}+\frac{\Lambda_0}{T}\right) \int_0^T \Phi^{\star} (T\dot{v})\,dt
		-\int_0^T \alpha(t)\,dt-\frac{C_1}{2}T
		\\
		> \left(-\frac{C_{\Phi^{\star}}}{2}+C_{\Phi^{\star}}\right)\int_0^T \Phi^{\star} (T\dot{v})\,dt
		-\int_0^T \alpha(t)\,dt-\frac{C_1}{2}T
		\\
		=\frac{C_{\Phi^{\star}}}{2}\int_0^T \Phi^{\star} (T\dot{v})\,dt
		-\int_0^T \alpha(t)\,dt-\frac{C_1}{2}T
				\end{split}
\]
\textcolor[rgb]{1,0,0}{No creo que sea necesario tanto detalle en la cuenta anterior, 
pero como no la hab\'iamos escrito en el papel, la hice para ver c\'omo sal\'ia.
}
\end{proof}

\textcolor[rgb]{1,0,0}{Observaciones de \'ultimo momento!
}
\begin{itemize}
\item
Habr\'ia que definir si usamos $n$ o $d$ porque un poco de $\rr^d$ y otro poco de $\rr^n$.
\item Un problema similar tenemos con $\Phi$ y $\Psi$.
\item Ahora recuerdo que dijiste algo sobre NO escribir UN resultado como el Teorema 3.1, sino
varios resultados individuales (difrenciabilidad, coercividad, minimización, etc)
\end{itemize}


%\section{Optimal constant in Poincar\'e-Wirtinger inequality }


\section{Cota \'optima y muchas otras cosas...}

\textcolor[rgb]{1,0,0}{Quiz\'as aqu\'i haya cosas que deben  colocarse antes, pero me generaban confusi\'on en el medio y por eso las acumul\'e ac\'a.
}
\subsection{Caso $L^p$ para cota \'optima }
Let $u \in H^{1,p}_T ([0,T],\rr^d)$. Then
\begin{equation}
\int_0^T |u-\overline{u}|^{p'}\,dt\leq C_p T^p \int_0^T |u'|^p\,dt
\end{equation}
where the optimal constant satisfies
\begin{equation}\label{eq:def-cp}
\begin{split}
C_p^{-1}:=
\inf
\left\{
T^p \frac{\int_0^T |u'|^p\,dt}{\int_0^T |u-\overline{u}|^p\,dt}| u\in H^{1,p}_T
\right\}=
\\
\inf
\left\{
T^p \frac{\int_0^T |u'|^p\,dt}{\int_0^T |u|^p\,dt}| u\in H^{1,p}_T,\;\int_0^T u\,dt=0
\right\}
\end{split}
\end{equation}
 
\begin{lem}
$C_p$ given by \eqref{eq:def-cp} is independent of $T$.
\end{lem}

\begin{proof}
Let $T\neq T'$. If $u$  is a function such that 
\begin{equation}
C_p^{-1}(T)+\epsilon> T^p \frac{\int_0^T |u'|^p\,dt}{\int_0^T |u|^p\,dt}
\end{equation}  
Performing the change of variable $s=\frac{T'}{T}t$ and calling $r=\frac{T}{T´}$, we have
\begin{equation}
T^p \frac{\int_0^T |u'(t)|^p\,dt}{\int_0^T |u(t)|^p\,dt}=
T^p \frac{\int_0^T |u'(rs)|^p\,ds}{\int_0^T |u(rs)|^p\,dt}=
(T')^p \frac{\int_0^T |v'(s)|^p\,ds}{\int_0^T |v(s)|^p\,dt}\geq C_p^{-1}(T')
\end{equation}
where $v(s)=u(rs)$. Therefore, $C_p^{-1}(T')\leq C_p^{-1}(T)$ and consequently $C_p(T')=C_p(T)$.
\end{proof}

\begin{lem}
\begin{equation}
C_p^{-1}=\inf \left\{T^p \int_0^T |u'|^p\,dt | u\in H^{1,p}_T,\;\int_0^T u\,dt=0, 
\int_0^T |u|^p\,dt=1\right\}
\end{equation}
\end{lem}
\begin{proof}
The existence of a minimum follows as usual by means of a minimizing sequence. 

More details....?

We employ the method of Lagrange multipliers to solve an optimization problem with constraints. 
We will look for critical points of 
\begin{equation}
I=\int_0^T |u'|^p\,dt-\lambda \int_0^T |u|^p\,dt+\mu \cdot \int_0^T u\,dt,\;\; u\in H^{1,p}_T 
\end{equation}
The G\^ateaux derivative of the functional is given by
\begin{equation}
\begin{split}
\left\langle I'(u),v\right\rangle=
\int_0^T p |u'|^{p-2}u'\cdot v'\,dt-p\lambda \int_0^T |u|^{p-2}u\cdot v\,dt+ \mu \int_0^T v\,dt=
\\
\int_0^T \left\{\frac{d}{dt}(p |u'|^{p-2}u')-p\lambda |u|^{p-2}u+\mu\right\}\cdot v\,dt+p |u'|^{p-2}u'\cdot v|_0^T=0
\end{split}
\end{equation}
Since $v$ is an arbitrary function, we choose $v$ such that $v(0)=v(T)=0$ and we obtain 
\begin{equation}\label{eq:int-nulo}
\frac{d}{dt}(p |u'|^{p-2}u')-p\lambda |u|^{p-2}u+\mu=0 \;\;a.e.
\end{equation}
This fact implies that 
$p |u'|^{p-2}u'\cdot v|_0^T=0 \; \forall v \in H^{1,p}_T$, that is
 \begin{equation}
[p |u'(T)|^{p-2}u'(T)-p|u'(0)|^{p-2}u'(0)]\cdot v(0)=0.
\end{equation}
Then $u'(T)=u'(0)$. 
Now, integrating \eqref{eq:int-nulo}, we get
\[
p\lambda \int_0^T |u|^{p-2}u\,dt+\mu T=0.
\]
If $p=2$ then $\mu=0$ and 
\[
\left\{
\begin{array}{lll}
u''+ \lambda u&=&0\\
u(0)&=&u(T)\\
\int_0^T u\,dt&=&0
\end{array}
\right.
\]
The normalized solution is
$u(t)=\cos(\sqrt\lambda t)u_0+\sin(\sqrt\lambda t)u_1$
with $u_0,u_1\in \rr^d$.

As $u(0)=u(T)$ and $u'(0)=u'(T)$, 
the function $u(t)$ has minimal  period $\frac{2\pi}{\sqrt \lambda}$ and it solves the second order ODE 
$u''+ \lambda u=0$

Then $u(0)=u(T)$, $u'(0)=u'(T)$ imply that the function $u$ has period T.

\'Esto ....

As $u \neq 0$, we have $k \frac{2\pi}{\sqrt \lambda}=T$ with $k=1,3,\dots$.
Then $\lambda=k^2 \frac{4\pi^2}{T^2}$.

Now, if $u_k(t)=\cos(\frac{2k\pi}{T}t)u_0+\sin(\frac{2k\pi}{T}t)u_1$, then 
\[
\begin{split}
1=&\int_0^T |u_k|^2\,dt
\\
=&\int_0^T \left[\cos\left(\frac{2k\pi}{T}t\right)\right]^2\,dt |u_0|^2+
\int_0^T \left[\sin\left(\frac{2k\pi}{T}t\right)\right]^2\,dt |u_1|^2
\\
+&\int_0^T \cos\left(\frac{2k\pi}{T}t\right)\sin\left(\frac{2k\pi}{T}t\right)\,dt\; u_0\cdot u_1
\\
=&\frac{T}{2} (|u_0|^2+|u_1|^2)
\end{split}
\]
and 
\[
\begin{split}
&T^2 \int_0^T |u'_k|^2\,dt
\\
&=T^2 \left(\frac{2k\pi}{t}\right)^2
\left\{
\int_0^T \left[\sin\left(\frac{2k\pi t}{T}\right)\right]^2|u_0|^2+
\int_0^T \left[\cos\left(\frac{2k\pi t}{T}\right)\right]^2|u_1|^2+
0
\right\}
\\
&=
\left(\frac{2k\pi}{t}\right)^2\frac{T}{2}(|u_0|^2+|u_1|^2)
\\
&=
4k^2\pi^2
\end{split}
\]
The minimum occurs when $k=1$ and we get $C_2^{-1}=4\pi^2$.
Then, 
$\int_0^T |u|^2\,dt\leq \frac{T^2}{4\pi^2} \int_0^T |u'|^2\,dt$

Or...

From $u''+\lambda u=0$, we have $u'' u+\lambda u^2=0$ and integrating over $[0,T]$ we obtain
$0=\int_0^Tu''u+\lambda \int_0^t u^2=-\int_0^T (u')^2+\lambda \int_0^T u^2 +u'u|_0^T=
-\int_0^T (u')^2+\lambda \int_0^T u^2 +u'(T)u(T)-u'(0)u(0)=
-\int_0^T (u')^2+\lambda \int_0^T u^2 $, then 
$\frac{4 \pi^2 k}{T^2}=\lambda=\frac{\int_0^T (u')^2}{\int_0^T u^2}=\frac{1}{C_2}$ 
The minimum value is attained at $k=1$ and therefore $C_2=\frac{T^2}{4\pi^2}$.
\end{proof}



\subsection{$L^{\Phi}$ case where $\Phi:\rr^d \to \rr$ is an anisotropic function}


Now, we are looking for the optimal constant $C(\Phi, T)$ such that
\begin{equation}
\int_0^T \Phi(u-\overline{u_{\Phi}})\,dt\leq
C(\Phi, T) \int_0^T\Phi(u')\,dt\;\; u \in \wphit
\end{equation}

Then, 
\begin{equation}
\int_0^T \Phi(u-\overline{u}_{\Phi})\,dt\leq
C(\Phi, T) \int_0^T\Phi(u-a)\,dt\;\; \forall a  \in \rr^d 
\end{equation}
where $a=\overline{u}_{\Phi}$ is the unique vector of $\rr^d$ such that
\begin{equation}
\int_0^T \nabla \Phi(u-a)\,dt=0.
\end{equation}  
Thus, 
$C^{-1}=\inf\left\{
\frac{\int_0^T \Phi(u')\,dt}{\int_0^T \Phi(u-\overline{u}_{\Phi})}| u \in \wphit
\right\}$
Let $v=u-\overline{u}_{\Phi}$, $v'=u'$ and $\overline{v}_{\Phi}=0$ then
\begin{equation}
\lambda:=C^{-1}=\inf \left\{
\frac{\int_0^T \Phi(u')\,dt}{\int_0^T \Phi(u)\,dt}|u \in \wphit, \int_0^T \nabla \Phi(u)\,dt=0
\right\}
\end{equation}
Let 
\begin{equation}
L(u,u')=\frac{\int_0^T \Phi(u')\,dt}{\int_0^T \Phi(u)\,dt}-\mu \cdot\int_0^T \nabla \Phi(u)\,dt
\end{equation}
with $\mu \in \rr^d$. 
By G\^ateaux derivative we have
\begin{equation}
0=\frac{\int_0^T \Phi(u)\,dt \int_0^T \nabla\Phi(u')\cdot v'\,dt-
\int_0^T \Phi(u')\,dt \int_0^T \nabla \Phi(u)\cdot v\,dt}{(\int_0^T \Phi(u)\,dt)^2}-\mu \int_0^T D^2\Phi(u)\cdot v \,dt
\end{equation}
then 
\begin{equation}
\begin{split}
0=\int_0^T \nabla\Phi(u')\cdot v'\,dt-\lambda \int_0^T \nabla \Phi(u)\cdot v\,dt-
\int_0^T \Phi(u)\,dt \mu \cdot\int_0^T D^2\Phi(u)\cdot v \,dt=
\\
\int_0^T \left\{
-\frac{d}{dt}\nabla\Phi(u')-\lambda \nabla\Phi(u)
\right\} \cdot v\,dt+\nabla\Phi(u')\cdot v|_0^T-\int_0^T \Phi(u)\,dt\, \mu \cdot\int_0^T D^2\Phi(u)\cdot v \,dt=
\\
\int_0^T \left\{
-\frac{d}{dt}\nabla\Phi(u')-\lambda \nabla\Phi(u)
 -\int_0^T \Phi(u)\,dt \,\mu \cdot D^2\Phi(u)\right\}\cdot v \,dt+\nabla\Phi(u')\cdot v|_0^T
\end{split}
\end{equation}
$\forall v \in \wphit$. 

Now, we consider any $v \in W_0^1L^{\Phi}$ and we get
\begin{equation}\label{eq:ed-phi-aniso}
-\frac{d}{dt}\nabla\Phi(u')-\lambda \nabla\Phi(u)
 -\int_0^T \Phi(u)\,dt\, \mu \cdot D^2\Phi(u)=0
\end{equation}
%+\nabla\Phi(u')=0$. 
Then, 
\[
 \nabla\Phi(u') v|_0^T=0\;\;\forall v \in \wphit,
\]
that is
\[
 \{\nabla\Phi(u'(T))-\nabla\Phi(u'(0))\}\cdot v(0)=0
\]
for any $v \in \wphit$. 
Thus,  $\nabla\Phi(u'(T))=\nabla\Phi(u'(0))$

As $\Phi$ is strictly convex, then $\nabla \Phi$ is injective and consequently $u'(T)=u'(0)$.


Post-multiplying \eqref{eq:ed-phi-aniso} by $\mu^t$ and integrating over $[0,T]$, we get
\begin{equation}
0=\int_0^T -\frac{d}{dt} \nabla \Phi(u')\,dt \cdot \mu^t
-\lambda \int_0^T \nabla \Phi(u)\,dt \cdot \mu^t-\int_0^T\Phi(u)\,dt \int_0^T \mu\cdot D^2\Phi(u)\,dt\cdot \mu^t
\end{equation}
with $\nabla\Phi(u'(T))=\nabla\Phi(u'(0))$.

We know that $\mu\cdot D^2\Phi(u)\cdot\mu^t=0$ iff $\mu=0$. And, as $\int_0^T \Phi(u)\,dt\neq 0$, then 
$\int_0^T\Phi(u)\,dt \int_0^T \mu\cdot D^2\Phi(u)\cdot\mu^t\,dt\cdot =0$ implies that $\mu=0$.

Therefore, 
\begin{equation}
\left\{
\begin{array}{l}
\frac{d}{dt} \nabla \Phi(u')+\lambda \nabla\Phi(u)=0
\\
u(0)=u(T),\;u'(0)=u'(T),\;\int_0^T \nabla \Phi(u)\,dt=0
\end{array}
\right.
\end{equation}



If $\Phi(u)=\frac{|u|^p}{p}$ and $u \in \rr$. We know that 
$T=\frac{4\pi (p-1)^{\frac{1}{p}}}{p \sin(\frac{\pi}{p})\lambda^{\frac{1}{p}}}k$ for $k=+1,+2,\dots$.




Cuestiones a resolver: 
\begin{enumerate}
\item Qu\'e dejar? Caso p o caso $\Phi$?
\item En  el caso $\Phi$? Analizamos la existencia en detalle?
\item Consideramos el caso $\Phi(x_1,\dots,x_d)=\Phi(x_1)+\dots+\Phi(x_d)$?
\end{enumerate}






\begin{example} Let $\Phi:\rr^d\to [0,+\infty)$ be  a $G$-function. Then the $G$-function
\[\Phi(u)=\Phi(q,p):= \Phi(q)+\Phie(p).\]
is a symplectic $G$-function.
\end{example}

\textcolor{red}{PROBLEM 0: It is the previous the general form of any symplectic $G$-function? It is possible to find other example of these functions?}

We note that if $\Phi$ is symplectic then

\begin{equation}\label{eq:J-phi2}
 \nabla\Phi(Ju)= J\Phie(u).
\end{equation}
Here we are agreeing that $\nabla\Phi$ is a column vector.

As a consequence of \eqref{eq:J-phi1}, the matrix $J$ induce a isometry  between the spaces $L^{\Phi}([0,T],\rr^{2d})$ and $L^{\Phie}([0,T],\rr^{2d})$.  Therefore we candefine  a bilinear form $\overline{\Omega}$  on $L^{\Phi}([0,T],\rr^{2d})$ of the following way

\[\overline{\Omega}(u,v):=\int_0^T\Omega(u, v) dt,\quad u,v\in L^{\Phi}([0,T],\rr^{2d})\]

We consider the following functional

\[\Theta(u):=\overline{\Omega}(u,\dot{u}).\]

We are interested in to find bounds of the quadratic functional $\Theta$ of the following type

\begin{equation}\label{eq:IneQuad}
 \theta(u)\geq -C\int_0^T\Phi\left(\dot{u}\right)dt,
\end{equation}
for $u\in W^1L^{\Phi}([0,T],\rr^{2d})$. It is important to get the best constant $C$ in previous inequality because this constant imposes  restrictions to the Hamiltonian $H$. 

If $\Phi(q)=|q|^2/2$ was proved in \cite[Prop. 3.2]{mawhin2010critical} \eqref{eq:IneQuad} holds width $C=T/\pi$.  Below we prove that this is the optimal constant satisfying \eqref{eq:IneQuad}.   Meanwhile in \cite[Lem. 3.3]{Tian2007192} 
was proved that $C_{\Phi}=2T$ satisfies \eqref{eq:IneQuad} when $\Phi(q)=|q|^{\alpha}/\alpha$, $1<\alpha<\infty$. Since this constant is not equal to $T/\pi$ when $\alpha=2$, it is not optimal.

\begin{prop}
 Let $\Phi$ be any symplectic $G$-function. Then  \eqref{eq:IneQuad} holds for  and $C=2T^{-1}$ for every  $u\in W^1L^{\Phi}([0,T],\rr^{2d})$.
\end{prop}

 \begin{proof} Let  $u\in W^1L^{\Phi}([0,T],\rr^{2d})$. As is usual we write $u=\tilde{u}+\overline{u}$ where
 \[\overline{u}=\frac{1}{T}\int_0^Tu(t)dt.\]
 From \cite[Lem. 2.4]{MA2017} we have that
 \[\int_0^T\Phi(\tilde{u})dt\leq\int_0^T\Phi(T\dot{u})dt.\]
 Then by Young's inequality and using \eqref{eq:J-phi1}
 \[
 \begin{split}
  \int_0^T\Omega\left(\dot{u},u\right)dt &=T\int_0^T\left\langle J\dot{u},T^{-1}\tilde{u}\right\rangle dt\\
  &\geq -T\left\{ \int_0^T\Phie(J\dot{u})dt + \int_0^T\Phi(T^{-1}\tilde{u})dt \right\}\\
  &\geq -2T\left\{ \int_0^T\Phi(\dot{u})dt \right\}
  \end{split}
 \]
 \end{proof}

\textcolor{red}{ Clearly the cosntant $2/T$ is far to be optimal. A possible way of improve $C$ is consider other average $\overline{u}$. The mean value that it was used is the standard condered in the literature. But this value is apropriate for el Hilbert setting $\Phi(q)=|q|^2/2$. In this case, the value of $\overline{u}$ is the nearest (in the $L^2$-norm) constant vector to $u$. For a arbitrary $G$ function, it seem more reasonable consider    the nearest constant vector to $u$ respect to the $\Phi $-integral, i.e.
\[
 \int_0^T\Phi(u-\overline{u})dt\leq \int_0^T\Phi(u-u_0)dt,\quad\hbox{for every } u_0\in\rr^{2n}
\]
Equivalently $\overline{u}$ is characterizate by
\[
 \int_0^T\nabla\Phi(u-\overline{u})dt=0.
\]
There is not a explicit formula as in the Hilbert setting.
\newline
PROBLEM 1. We can get a better constant taking this $\overline{u}$???}

 We call to the best constant in \eqref{eq:IneQuad} $C_{\Phi}$, i.e.
 
\begin{equation}\label{eq:C-optima}
C_{\Phi}=-\inf\left\{\left.\frac{\int_0^T\langle J\dot{u},u\rangle dt}{\int_0^T\Phi(\dot{u})dt}\right|
u\in W^1L^{\Phi}\left([0,T],\rr^{2d}\right)\right\}
 \end{equation}
 
 
\begin{prop} The relation $C_{\Phi}=C_{\Phie}$ holds for every symplectic $\Phi$. 
 \end{prop}
 
 \begin{proof} Since $\Phi$ is symplectic if $u=Jv$
 \[
      \frac{\int_0^T\langle J\dot{u},u\rangle dt}{\int_0^T\Phi(\dot{u})dt}
      =
      \frac{\int_0^T\langle -\dot{v},Jv\rangle dt}{\int_0^T\Phi(J\dot{v})dt}
      =\frac{\int_0^T\langle J\dot{v},v\rangle dt}{\int_0^T\Phie(\dot{v})dt}.
   \]
 Using that $u\mapsto Ju$ is invertible from $W^1L^{\Phie}([0,T],\rr^{2d})$ into $W^1L^{\Phi}([0,T],\rr^{2d})$ the statement follows taking infimum in previous equality.
  
 \end{proof}

 
For the following result we need the theory of indices of $G$-functions, see \cite{fiorenza1997indices,Maligranda} for a complete treatment in the case of $N$-functions defined on $\rr$. The results are easily extended to the anisotropic setting.
We denote by $\alpha_{\Phi}$ and $\beta_{\Phi}$ the so called  \emph{Matuszewska-Orlicz indices} of the function $\Phi$, which are defined next
\begin{equation}\label{eq:MO_indices}
    \alpha_{\Phi}:=\lim\limits_{t\to 0^{+}}\frac{\log \left (\sup\limits_{u>0}\frac{\Phi(t u)}{\Phi(u)} \right ) }{\log(t)},\quad
    \beta_{\Phi}:=\lim\limits_{t\to +\infty}\frac{\log \left  (\sup\limits_{u>0}\frac{\Phi(t u)}{\Phi(u)}\right )}{\log(t)}.
\end{equation}
We have that $1\leq \alpha_{\Phi}\leq \beta_{\Phi}\leq +\infty$. The relation $\beta_{\Phi}<\infty$ holds true if and only if $\Phi$ is a
$\Delta_2$-function. The indices satisfy the following relation
\begin{equation}\label{inv_indices}
    \frac{1}{\alpha_{\Phi}}+\frac{1}{\beta_{\Phie}}=1.
\end{equation}
Therefore if $\Phie$ is a $\Delta_2$-function (\textcolor{red}{I mean $\Delta_2$ as globally $\Delta_2$}) then $\alpha_{\Phi}>1$. 


We observe that if $\Phi$ is symplectic then $\Phi\in\Delta_2$ implies $\Phie\in\Delta_2$.  It is well known   that if $\Phi$  and $\Phie$ are $\Delta_2$-function, then $\Phi$ is controlled by above and below
 by power functions.  More concretely, for every $\epsilon>0$ there exists a
constant $K=K(\Phi,\epsilon)$ and $p_0,p_1$ with $1<\alpha_{\Phi}-\epsilon<p_1\leq p_2<\beta_{\Phi}+\epsilon<\infty$ such that, for every $t,u\geq 0$,
\begin{equation}\label{delta2-potencias}
    K^{-1}\min\big\{t^{p_2},t^{p_1} \big\}\Phi(u)\leq \Phi(t u)\leq
    K\max\big\{t^{p_2},t^{p_1} \big\}\Phi(u).
\end{equation}





We recall the following result of \cite{ABGMS2015}.

\begin{lem}\label{lem_coer} Let $\Phi$ be a  $G$-function. If $\Phie \in \Delta_2$ globally, then  for any $0<\mu<\alpha_{\Phi}$,
\begin{equation}\label{coer_modular} \lim\limits_{\|\b{u}\orlnor \to \infty} \frac{\int_0^T\Phi\left(\frac{\b{u}}{\Lambda}\right)dt}{\|\b{u}\orlnor^{\mu}}=+\infty.
\end{equation}

\end{lem}





\begin{thm}\label{thm:C-opt-hamil} Suppose that $u\in \wphit([0,T],\rr^{2d})$ attains the minimum in \eqref{eq:C-optima}, then $\lambda=2/C_{\Phi}$ is the first eigenvalue and $u$ the corresponding eigenfunction of the following problem.
\begin{equation}\label{eq:eigen_prob}
\left\{
\begin{array}{l}
 \frac{d}{dt}\nabla\Phie(\dot{u})+\lambda\nabla\Phie(\lambda u)=0\\
 u(0)=u(T), \, \int_0^T\nabla\Phie (\lambda u)dt=0
\end{array}
\right.\tag{Eig}
\end{equation}
\begin{proof} 
 
\end{proof}






 
\end{thm}


















%{ \bf Assumption:}
 %
%For any $\Lambda>0$ there exists $C_{\Phi,\Lambda}$ such that
%\[
%\Omega(\dot{u},u)\geq -C_{\Psi,\Lambda} \int \Psi\left(\frac{u}{\Lambda}\right)
%\]
%
%The assumption is true when $\Psi \in \Delta_2$.


\section*{Acknowledgments}
The authors are partially supported by  UNRC and UNLPam grants. 

% \bibliographystyle{apalike}
 \bibliographystyle{plain}
 
\bibliography{biblio}


\end{document}


