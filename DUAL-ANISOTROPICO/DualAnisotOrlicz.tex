\documentclass[twoside]{article}


%%%%%%%%% Packages %%%%%%%%%%%%%%%%%%%%%%%%%%%%
\usepackage{amssymb,amsthm}
\usepackage{amsmath}
\usepackage{mathabx}
\usepackage{fancyhdr}
\usepackage{url}
\usepackage{hyperref}
\usepackage{enumitem}
\usepackage{cleveref}
\usepackage{color}



%%%%%%%%%%%%%%%% Theorems like environments
\newtheorem{thm}{Theorem}[section]
\newtheorem{cor}[thm]{Corollary}
\newtheorem{lem}[thm]{Lemma}
\newtheorem{defi}[thm]{Definition}
\newtheorem{prop}[thm]{Proposition}
\theoremstyle{remark}
\newtheorem{comentario}{Remark}



\newcounter{example}[section]
\setcounter{example}{0}
\makeatletter
\renewcommand{\p@example}{\thesection.} % "prefix" for cross-referencing
\makeatother
%\newenvironment{example}{\noindent\textit{Example \arabic{example}}.}{\addtocounter{example}{1}}
\newenvironment{example}{\refstepcounter{example}\noindent\textit{Example \arabic{section}.\arabic{example}}.}{ }


%%%%%%%%%%%%% New item with label%%%%%%%%%%%%%%%%%%%%%%%%%
\makeatletter
\newcommand{\labitem}[2]{%
\def\@itemlabel{#1}
\item
\def\@currentlabel{#1}\label{#2}}
\makeatother
\makeatletter
\def\namedlabel#1#2{\begingroup
    #2%
    \def\@currentlabel{#2}%
    \phantomsection\label{#1}\endgroup
}
\makeatother



%%%%%%%%%%%%%%%% Title 

\title{Clarke dual method for Hamiltonian systems with non standard grow}
\author{(In alphabetical order)\\[3mm]
Sonia Acinas \thanks{SECyT-UNRC,  FCEyN-UNLPam and UNSL}\\
Dpto. de Matem\'atica, Facultad de Ciencias Exactas y Naturales\\
Universidad Nacional de La Pampa\\
(L6300CLB) Santa Rosa, La Pampa, Argentina\\
\url{sonia.acinas@gmail.com}\\[3mm]
Jakub Maksymiuk \\[3mm]
 Fernando D. Mazzone \thanks{SECyT-UNRC, FCEyN-UNLPam and CONICET}\\
Dpto. de Matem\'atica, Facultad de Ciencias Exactas, F\'{\i}sico-Qu\'{\i}micas y Naturales\\
Universidad Nacional de R\'{i}o Cuarto\\
(5800) R\'{\i}o Cuarto, C\'ordoba, Argentina,\\
\url{fmazzone@exa.unrc.edu.ar} \\
}

\date{}

%%%%%%%%%%%%%%%%New commands

\newcommand{\orlnor}{\|_{L^{\Phi}}}
\newcommand{\linf}{\|_{L^{\infty}}}
\newcommand{\lphi}{L^{\Phi}}
\newcommand{\lpsi}{L^{\Phi^{\star}}}
\newcommand{\ephi}{E^{\Phi}}
\newcommand{\claseor}{C^{\Phi}}
\newcommand{\wphi}{W^{1}\lphi}
\newcommand{\wphit}{W^{1}\lphi_T}
\newcommand{\sobnor}{\|_{W^{1}\lphi}}
\newcommand{\domi}{\mathcal{E}^{\Phi}}
\renewcommand{\b}[1]{\boldsymbol{#1}}
\newcommand{\rr}{\mathbb{R}}
\newcommand{\nn}{\mathbb{N}}
%\newcommand{\cdot}{\b{\cdot}}
\renewcommand{\leq}{\leqslant} 
\renewcommand{\geq}{\geqslant} 
\newcommand{\epsi}{E^{\Phi^{\star}}}
\newcommand{\Phie}{\Psi}%{\Phi^{\star}}
\newcommand{\lip}{\mathop{\rm Lip}}
\newcommand{\phih}{\overline{\Phi}}
\newcommand{\phihe}{\overline{\Psi}}



\DeclareSymbolFont{symbolsC}{U}{txsyc}{m}{n}
\DeclareMathSymbol{\strictif}{\mathrel}{symbolsC}{74}



\begin{document}


\maketitle
%
\begingroup%Locallizing the change to `thefootnote'.
    \renewcommand{\thefootnote}{}%Removing the footnote symbol.
    %
    \footnotetext{%
    %   2010 Mathematics Subject Classification
    %   http://www.ams.org/msc/
    \textbf{2010  AMS Subject Classification.} Primary: 34C25.
    Secondary: 34B15.
    }%
        \footnotetext{%
    \textbf{Keywords and phrases.} 
Periodic Solutions,  Orlicz Spaces,   Euler-Lagrange,   Critical Points.
    }%
    \endgroup
%
%
%
%

\begin{abstract}
In this paper we consider the problem of finding periodic solutions of certain Hamiltonian systems .....blablabla

\end{abstract}






\pagestyle{fancy} \headheight 35pt \fancyhead{} \fancyfoot{}

\fancyfoot[C]{\thepage} \fancyhead[CE]{\nouppercase{F.D. Mazzone   and S. Acinas}} \fancyhead[CO]{\nouppercase{\section}}

\fancyhead[CO]{\nouppercase{\leftmark}}


%\tableofcontents

\section{Main problem}
Let $H:[0,T]\times \rr^d\times \rr^d \to \rr$. 
We are looking for periodic solutions of the Hamiltonian system
\begin{equation}\label{eq:ham-sis-qp}
\left\{\begin{array}{ll}
\dot{q}(t)&=D_pH(t,q(t),p(t))\\
\dot{p}(t)&=-D_qH(t,q(t),p(t))\\
p(0)&=p(T)\hbox{ , } q(0)=q(T)
\end{array}
\right.
\end{equation}
for $t \in [0,T]$. \textcolor{red}{I think that, like in \cite{Mawhin2010}, is better to present the Hamiltonian problem as the main problem}


An alternative writing of \eqref{eq:ham-sis-qp} using the combined variable $u=(q,p)$ and the canonical symplectic matrix 

\[J=
\begin{pmatrix}
0&I_{d\times d}
\\
-I_{d\times d}&0
\end{pmatrix}
\]
is the following 
\begin{equation}
\dot{u}=J \nabla H(t,u(t))
\end{equation}
or equivalently
\begin{equation}
J\dot{u}=- \nabla H(t,u(t))
\end{equation}
where $\nabla H$ is the gradient of $H$ with respect to the combined variable.

\section{Preliminaries}
We will use some basic concepts of convex analysis that we list below.

Let $\Gamma_0(\rr^d)=\{F:\rr^d\to (-\infty,+\infty]
\\
\mbox {convex, lower semicontinous functions with non-empty effective domain.}\}$

The Fenchel conjugate of $F$ is given by
\[
F^*(p)=\sup\limits_{q \in \rr^d} \left\langle p,q\right\rangle-F(q)
\]
The Fenchel conjugate satisfies the following properties:
\begin{enumerate}
\item $F^* \in \Gamma_0(\rr^d)$
\item If $F\leq G$, then  $G^* \leq F^*$.
\item If $G(q)=\alpha F(\beta q)+\sigma$ with $\alpha,\beta,\sigma>0$ then
$G^*(p)=\alpha F^*(\frac{p}{\beta \alpha})-\sigma$
\end{enumerate}

Let $\Phi:\mathbb{R}^d\to [0,+\infty)$ be  a differentiable, convex function such that $\Phi(0)=0$, $\Phi(q)>0$ if $q\neq 0$, $\Phi(-q)=\Phi(q)$,
 and
\begin{equation}\label{eq:N-sub-inf}
\lim_{|q|\to\infty}\frac{\Phi(q)}{|q|}=+\infty,
\end{equation}
where $|\cdot|$ denotes the euclidean norm on $\rr^d$. From now on, we say that $\Phi$ is an $G$-function if $\Phi$ satisfies the previous properties.

We write $\Psi$ for the Fenchel conjugate of $\Phi$.

We do not assume  that $\Phi$ and $\Phi'$ satisfy the $\Delta_2$-condition.


We denote by  $\partial F(q)$ the subdifferential of $F$ in the sense of convex analysis (see \cite{clarke1990optimization,clarke2013functional})

The next result is a generalization of \cite[Prop. 2.2, p.34]{mawhin2010critical}

\begin{prop}\label{prop: cota-conj-phi}
Let $F \in  \Gamma_0(\rr^d)$. Suppose that there exist an anisotropic function $\Phi$ and
non negative  constants $\beta,\gamma$ such that
\begin{equation}\label{eq:cotas-F-aniso}
-\beta \leq F(q) \leq \Phi(q)+\gamma, \mbox{ for all } q \in \rr^d. 
\end{equation}
Now, if $p \in \partial F(q)$ then 
\begin{equation}\label{eq:trans-grad-aniso}
\Psi(p)\leq \Phi(2q)+2(\beta+\gamma).
\end{equation}
\end{prop}

\begin{proof}
If $p \in \partial F(q)$, from \cite[Thm. 2.2, p.33]{mawhin2010critical}, 
\begin{equation}\label{eq:trans-grad}
F^*(p)=\langle p,q \rangle-F(q)
\end{equation}
Conjugating \eqref{eq:cotas-F-aniso}, we have
\begin{equation}\label{eq:trans-conj}
F^*(p)\geq \Psi(p)-\gamma.
\end{equation}
From Young's inequality, we get
\begin{equation}\label{eq:trans-young}
\langle p,q \rangle =\frac{1}{2} \langle p,2q \rangle \leq \frac{1}{2 }\Psi(p)+\frac{1}{2}\Phi(2q)
\end{equation}
By \cref{eq:cotas-F-aniso,eq:trans-grad,eq:trans-conj,eq:trans-young}, we get
\[
\Psi(p)\leq \frac{1}{2}\Psi(p)+\frac{1}{2}\Phi(2q)+\beta+\gamma
\]
which implies \eqref{eq:trans-grad-aniso}
\end{proof}


\section{Optimal bounds for a symplectic bilinear form}

We  consider the  Euclidean space $\rr^{2d}$ equipped with the standard  symplectic structure given by bilinear canonical symplectic 2-form

\[\Omega(u,v):=\langle Ju,v\rangle .\]


Let $\Phi$ a $G$-function. We consider the \emph{symplectic $G$-function} $\phih$ defined symplectic manifold $\rr^{2d}$
\[\phih(u)=\phih(q,p):= \Phi(q)+\Phie(p).\]

\textcolor{red}{I think the $\phih$ is the appropriate form of the $G$-function defined on the symplectic manifold $\rr^{2n}$}

The $G$-function $\phih$ has the following important property

\begin{equation}\label{eq:J-phi1}
 \phih(Ju)= \phihe(u).
\end{equation}

and 

\begin{equation}\label{eq:J-phi2}
 \nabla\phih(Ju)= J\phihe(u).
\end{equation}
Here we are agreeing that $\nabla\Phi$ is a column vector.

As a consequence of \eqref{eq:J-phi1}, the matrix $J$ induce a isometry  between the spaces $L^{\phih}([0,T],\rr^{2d})$ and $L^{\phihe}([0,T],\rr^{2d})$.  Therefore we can extend $\Omega$ to a bilinear form $\overline{\Omega}$  on $L^{\phih}([0,T],\rr^{2d})$ of the following way

\[\overline{\Omega}(u,v):=\int_0^T\Omega(u, v) dt,\quad u,v\in L^{\phih}([0,T],\rr^{2d})\]

We consider the following functional

\[\Theta(u):=\overline{\Omega}(u,\dot{u}).\]

We are interested in to find bounds of the quadratic functional $\Theta$ of the following type

\begin{equation}\label{eq:IneQuad}
 \theta(u)\geq -C\int_0^T\phih\left(\dot{u}\right)dt,
\end{equation}
for $u\in W^1L^{\phih}([0,T],\rr^{2d})$. It is important to get the best constant $C$ in previous inequality because this constant imposes  restrictions to the Hamiltonian $H$.  We call to the best constant in \eqref{eq:IneQuad} $C_{\Phi}$

If $\Phi(q)=|q|^2/2$ was proved in \cite[Prop. 3.2]{mawhin2010critical} that $C_{\Phi,1}=T/\pi$.  Below we prove that this is the optimal one.  In \cite[Lem. 3.3]{Tian2007192} was proved that \eqref{eq:IneQuad} holds for $\Phi(q)=|q|^{\alpha}/\alpha$, $1<\alpha<\infty$ and $C_{\Phi}=2T$. Since this constant is not equal to $T/\pi$ when $\alpha=2$ it is not optimal.

\begin{prop}
 Let $\Phi:\rr^d\to [0,+\infty)$ be any $G$-function and $\phih$. Then  \eqref{eq:IneQuad} holds for  and $C=2T^{-1}$ for every  $u\in W^1L^{\phih}([0,T],\rr^{2d})$.
\end{prop}

 \begin{proof} Let  $u\in W^1L^{\phih}([0,T],\rr^{2d})$. As is usual we write $u=\tilde{u}+\overline{u}$ where
 \[\overline{u}=\frac{1}{T}\int_0^Tu(t)dt.\]
 From \cite[Lem. 2.4]{MA2017} we have that
 \[\int_0^T\phih(\tilde{u})dt\leq\int_0^T\phih(T\dot{u})dt.\]
 Then by Young's inequality and using \eqref{eq:J-phi1}
 \[
 \begin{split}
  \int_0^T\Omega\left(\dot{u},u\right)dt &=T\int_0^T\left\langle J\dot{u},T^{-1}\tilde{u}\right\rangle dt\\
  &\geq -T\left\{ \int_0^T\overline{\Psi}(J\dot{u})dt + \int_0^T\phih(T^{-1}\tilde{u})dt \right\}\\
  &\geq -2T\left\{ \int_0^T\phih(\dot{u})dt \right\}
  \end{split}
 \]
 \end{proof}

\textcolor{red}{ Clearly the cosntant $2/T$ is far to be optimal. A possible way of improve $C$ is consider other average $\overline{u}$. The mean value that it was used is the standard condered in the literature. But this value is apropriate for el Hilbert setting $\Phi(q)=|q|^2/2$. In this case, the value of $\overline{u}$ is the nearest (in the $L^2$-norm) constant vector to $u$. For a arbitrary $G$ function, it seem more reasonable consider    the nearest constant vector to $u$ respect to the $\phih $-integral, i.e.
\[
 \int_0^T\phih(u-\overline{u})dt\leq \int_0^T\phih(u-u_0)dt,\quad\hbox{for every } u_0\in\rr^{2n}
\]
Equivalently $\overline{u}$ is characterizate by
\[
 \int_0^T\nabla\phih(u-\overline{u})dt=0.
\]
There is not a explicit formula as in the Hilbert setting.
\newline
PROBLEM 1. We can get a better constant taking this $\overline{u}$???}














Now, we produce a generalization of \cite[Thm. 2.3, pp.]{mawhin2010critical}.

\begin{thm}
Suppose that 
\begin{enumerate}
\item \label{it:h1-prop-H}
$H:[0,T]\times\rr^{2d}\to \rr$ is measurable in $t$, continuously differentiable with respect to $u$.
\item \label{it:h2-cotaH-conjphi}there exist $\beta, \gamma \in L^1([0,T],\rr)$, $\Lambda>\lambda>0$ such that
\begin{equation}\label{eq:cota-H-phi-conj}
\Phi^*\left(\frac{u}{\Lambda}\right)-\beta(t)\leq H(t,u) \leq \Phi^*\left(\frac{u}{\lambda}\right)+\gamma(t)
\end{equation}
\end{enumerate}
Then there exists $\Lambda_0$ such that the dual action
\[
\chi(v)=\int_0^T \frac{1}{2} \langle J\dot{v},v\rangle+H^*(t,\dot{v})\,dt
\]
is continuously differentiable in $\wphit([0,T],\rr^d) \cap \{u|d(\dot{u},L^{\infty})<\Lambda_0\}$.

If $v$ is a critical point of $\chi$ with $d(\dot{v},L^{\infty})<\Lambda_0$, the function defined by 
$u(t)=\nabla  H^*(t,\dot{v})$
solves 
\[
\left\{\begin{array} {lll}
\dot{u}&=&J\nabla H (t,u)
\\
u(t)&=&u(T)
\end{array}
\right.
\]
\end{thm}


\begin{proof}
Conjugating \ref{it:h2-cotaH-conjphi} we obtain
\begin{equation}\label{eq:cotaH*-phi}
\Phi(\lambda u)-\gamma(t)\leq H^*(t,v)\leq \Phi(\Lambda v)+\beta(t)
\end{equation}
Since $H^*$ is smooth, we have $\partial_vH^*(t,v)=\{\nabla_v H^*(t,v)\}$.
Applying Proposition \ref{prop: cota-conj-phi} with $F=H^*$, $\Phi(\Lambda v)$ instead of $\Phi(u)$
and $u=\nabla H^*(t,v) \in \partial_v H(t,v)$, inequality \eqref{eq:cota-H-phi-conj} becomes
\begin{equation}\label{eq:main-phiconj-phi}
\Phi^*\left(\frac{\nabla H^*(t,v)}{\Lambda}\right)\leq \Phi(2\Lambda v)+2(\beta+\gamma).
\end{equation}
which will be the main inequality in the proof.

We are planning to obtain  the structure condition \eqref{eq:condicion-estructura} of \cite{MA2017} which guarantees
differentiability. 

We consider the Lagrangian 
\begin{equation}
\mathcal{L}(t,v,\xi)=\frac{1}{2}\langle J\xi,v\rangle + H^*(t,\xi)
\end{equation}
and we have to prove that there exist $\Lambda_0>\lambda_0<0$
such that 
\begin{equation}\label{eq:cota-L-gradL-conj}
|\mathcal{L}|+|\nabla_v\mathcal{L}|+\Phi^*\left(\frac{\nabla_{\xi}\mathcal{L}}{\lambda_0}\right)\leq
a(v)\left(b(t)+\Phi\left(\frac{\xi}{\Lambda_0}\right)\right)
\end{equation}
We start with $|\mathcal{L}|$. From \eqref{eq:cotaH*-phi}, 
\[
|\mathcal{L}|\leq \frac{1}{2} |\langle J\xi,v\rangle|+H^*(t,\xi)\leq \frac{1}{2}|\xi||v|+\Phi(\Lambda \xi)+\beta(t). 
\]
Since $\frac{\Phi(x)}{|x|}\to \infty$ as $|x|\to \infty$, there exists $C>0$ such that $|x|\leq \Phi(|x|)+C$ for all $x \in \rr^d$. 
Then, 
\[
|\mathcal{L}|\leq \frac{1}{2} \frac{|v|}{\Lambda}\left(\Phi( \Lambda \xi)+C\right)+\Phi(\Lambda \xi)+\beta(t)
\leq \max\left\{\Lambda, \frac{|v|}{2\Lambda}\right\}
\left[\Phi(\Lambda \xi)+C+\beta(t)\right]
\] 
which is an estimate like  the second member of \eqref{eq:cota-L-gradL-conj}.

Now, we treat $|\nabla_v \mathcal{L}|$ and we get
\begin{equation}\label{eq:cota-grad-v}
|\nabla_v \mathcal{L}|=\frac{1}{2}|J\xi|\leq |\xi| \leq \frac{1}{2 \Lambda}(\Phi(\Lambda \xi)+C).
\end{equation}
which is also an estimate of the desired type.

Finally, we deal with $\Phi(\nabla_{\xi}\mathcal{L}{\lambda_0})$. 
As $\Phi^*$ is a convex, pair??? function, we have
\[\Phi^*\left(\frac{\nabla_{\xi}\mathcal{L}}{\lambda_0}\right)=
\Phi^*\left(\frac{-\frac{1}{2}Jv}{\lambda_0}+\frac{\nabla H^*(t,\xi)}{\lambda_0}\right)
\leq
\frac{1}{2} \Phi^*\left(\frac{Jv}{\lambda_0}\right)+
\frac{1}{2}\Phi^*\left(\frac{2 \nabla H^*(t,\xi)}{\lambda_0}\right).
\]
We choose $\frac{2}{\lambda_0}=\frac{1}{\Lambda}$ with $\Lambda$ as in \eqref{eq:main-phiconj-phi} and 
we finally have
\begin{equation}\label{eq:cota-conj-grad}
\begin{split}
\Phi^*\left(\frac{\nabla_{\xi}\mathcal{L}}{\lambda_0}\right)\leq 
\Phi^*\left(\frac{Jv}{2\Lambda}\right)+\Phi(2\Lambda\xi)+2(\beta+\gamma)=\\
\max\left\{\Phi^*\left(\frac{Jv}{2\Lambda}\right),1\right\}
\left[\Phi(2\Lambda\xi)+2(\beta+\gamma)\right]
\end{split}
\end{equation}
which is a bound like the second member of \eqref{eq:cota-L-gradL-conj}.

Therefore, from \eqref{eq:cota-L-gradL-conj}, \eqref{eq:cota-grad-v}, \eqref{eq:cota-conj-grad} and choosing the worst functions $a$ and $b$, 
we obtain condition \eqref{eq:condicion-estructura}.

Next, \cite[Thm. 4.5]{MA2017} implies differentiability of $\chi$ in a set like 
$\wphit([0,T],\rr^d)\cap\{u|d(\dot{u},L^{\infty})<\lambda_0\}$.

If $v \in \wphit([0,T],\rr^d)$ is a critical point of $\chi$ with $d(\dot{v},L^{\infty})<\lambda_0$
then, from equations (21) of \cite{MA2017} we obtain
\[
0=\int_0^T \frac{1}{2} \langle J\dot{v},h\rangle-\frac{1}{2} \langle \dot{h},Jv \rangle
+\langle \nabla H^*(t,\dot{v}),\dot{h} \rangle.
\]
The rest of the proof follows as in \cite{mawhin2010critical}.

\end{proof}


\section*{Acknowledgments}
The authors are partially supported by  UNRC and UNLPam grants. The second author is  partially supported by a  UNSL grant. 




% \bibliographystyle{apalike}
 \bibliographystyle{plain}
 
\bibliography{biblio}


\end{document}


