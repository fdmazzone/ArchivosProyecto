\documentclass[final,hyperref={pdfpagelabels=false}]{beamer}
\usepackage{grffile}
\mode<presentation>{\usetheme{I6pd2}}
\usepackage[english]{babel}
\usepackage[latin1]{inputenc}
\usepackage{amsmath,amsthm, amssymb, latexsym}
%\usepackage{times}\usefonttheme{professionalfonts}  % obsolete
%\usefonttheme[onlymath]{serif}
%\boldmath
\usepackage[orientation=portrait,width=900mm,height=1500mm,debug]{beamerposter}
% change list indention level
% \setdefaultleftmargin{3em}{}{}{}{}{}


%\usepackage{snapshot} % will write a .dep file with all dependencies, allows for easy bundling

\usepackage{array,booktabs,tabularx}
\newcolumntype{Z}{>{\centering\arraybackslash}X} % centered tabularx columns
\newcommand{\pphantom}{\textcolor{ta3aluminium}} % phantom introduces a vertical space in p formatted table columns??!!
\newtheorem{thm}{Theorem}
\newtheorem{cor}[thm]{Corollary}
\newtheorem{lem}[thm]{Lemma}
\newtheorem{rem}[thm]{Remark}
\newtheorem{defn}[thm]{Definition}
\newtheorem{prop}[thm]{Proposition}
\newtheorem{exmp}[thm]{Example}
\listfiles

%%%%%%%%%%%%%%%%%%%%%%%%%%%%%%%%%%%%%%%%%%%%%%%%%%%%%%%%%%%%%%%%%%%%%%%%%%%%%%%%%%%%%%
\graphicspath{{figures/}}
 
\title{\huge Some existence results on periodic solutions of 
Euler-Lagrange equations in an Orlicz-Sobolev space setting\\
\vspace{1cm}
{ X Americas Conference on  Differential Equations and Nonlinear Analysis}
 }
\author{S. ~Acinas${}^{(3)}$, L. ~Buri${}^{(1)}$, G. ~Giubergia${}^{(1)}$, F. ~Mazzone${}^{(1,2)}$, E. ~Schwindt${}^{(4)}$}
\institute[UNRC]{ 
 ${}^{(1)}$ Depto de Matem�tica. Universidad Nacional de R\'{\i}o Cuarto.

${}^{(2)}$ CONICET

 ${}^{(3)}$ Instituto de Matem�tica Aplicada San Luis (CONICET-UNSL) y Depto de Matem�tica, Universidad Nacional de La Pampa.

 ${}^{(4)}$ Universit� d'Orl�ans,  Laboratoire MAPMO, Francia.
}
\date[Feb. 18th, 2015]{X Americas Conference on  Differential Equations and Nonlinear Analysis}

%%%%%%%%%%%%%%%%%%%%%%%%%%%%%%%%%%%%%%%%%%%%%%%%%%%%%%%%%%%%%%%%%%%%%%%%%%%%%%%%%%%%%%
\newlength{\columnheight}
\setlength{\columnheight}{105cm}


%%%%%%%%%%%%%%%%%%%%%%%%%%%%%%%%%%%%%%%%%%%%%%%%%%%%%%%%%%%%%%%%%%%%%%%%%%%%%%%%%%%%%%
\begin{document}

\newcommand{\orlnor}{\|_{L^{\Phi}}}
\newcommand{\lurnor}{\|^{*}_{L^{\Phi}}}
\newcommand{\linf}{\|_{L^{\infty}}}
\newcommand{\lphi}{L^{\Phi}}
\newcommand{\lpsi}{L^{\Psi}}
\newcommand{\ephi}{E^{\Phi}}
\newcommand{\claseor}{C^{\Phi}}
\newcommand{\wphi}{W^{1}\lphi}
\newcommand{\sobnor}{\|_{W^{1}\lphi}}
\newcommand{\domi}{\mathcal{E}^{\Phi}_d(\lambda)}
\renewcommand{\b}[1]{\boldsymbol{#1}}
\newcommand{\rr}{\mathbb{R}}
\newcommand{\nn}{\mathbb{N}}
\newcommand{\ccdot}{\b{\cdot}}
\renewcommand{\leq}{\leqslant} 
\newcommand{\epsi}{E^{\Psi}}



\newcommand{\pro}{\operatornamewithlimits{\int\hspace{-4.6mm}{\diagup}}}
\newcommand{\es}{\operatornamewithlimits{\hbox{ess sup }}}
\newcommand{\dive}{\hbox{div }}



\begin{frame}
  \begin{columns}
    % ---------------------------------------------------------%
    % Set up a column 
    \begin{column}{.49\textwidth}
      \begin{beamercolorbox}[center,wd=\textwidth]{postercolumn}
        \begin{minipage}[T]{.95\textwidth}  % tweaks the width, makes a new \textwidth
          \parbox[t][\columnheight]{\textwidth}{ % must be some better way to set the the height, width and textwidth simultaneously
            % Since all columns are the same length, it is all nice and tidy.  You have to get the height empirically
            % ---------------------------------------------------------%
            % fill each column with content            
            \begin{block}{Introduction}
             
              This work  is concerned with the existence of periodic solutions of the problem
              \begin{equation}\label{ProbPrin}
                \left\{%
                  \begin{array}{ll}
                    \frac{d}{dt} D_{y}\mathcal{L}(t,\b{u}(t),\b{\dot{u}}(t))= D_{\b{x}}\mathcal{L}(t,\b{u}(t),\b{\dot{u}}(t)) \quad \hbox{a.e.}\ t \in (0,T)\\
                    \b{u}(0)-\b{u}(T)=\b{\dot{u}}(0)-\b{\dot{u}}(T)=0
                  \end{array}%
                \right.
              \end{equation}
              where $T>0$, $\b{u}:[0,T]\to\rr^d$ is absolutely continuous and the \emph{Lagrangian} $\mathcal{L}:[0,T]\times\rr^d\times\rr^d\to\rr$ is a Carath\'eodory function satisfying the conditions
              \begin{equation}\label{cotaL}
                |\mathcal{L}(t,\b{x},\b{y})| &\leq a(|\b{x}|)\left(b(t)+ \Phi\left(\frac{|\b{y}|}{\lambda}+f(t) \right)\right)
              \end{equation}
     \begin{equation}\label{cotaDxL}         
|D_{\b{x}}\mathcal{L}(t,\b{x},\b{y})| &\leq a(|\b{x}|)\left(b(t)+ \Phi\left(\frac{|\b{y}|}{\lambda}+f(t) \right)\right),
\end{equation}
  \begin{equation}\label{cotaDyL} 
|D_{\b{y}}\mathcal{L}(t,\b{x},\b{y})| &\leq a(|\b{x}|)\left(c(t)+ \varphi\left(\frac{|\b{y}|}{\lambda}+f(t)\right)  \right).
\end{equation}

              In these inequalities we assume that  $\lambda>0$ and (see below for definitions)

              \begin{itemize}

              \item  $a\in C(\mathbb{R}^+,\mathbb{R}^+)$.

              \item  $\Phi$ is an  $N$-function.

              \item $b\in L^1([0,T])$  and $c\in \lpsi_1$, where $\Psi$ is the complementary $N$-function of $\Phi$.

              \item $f\in\ephi_1$. 

              \end{itemize}
          
            \end{block}


  \begin{block}{Definitions}
\begin{itemize}
\item  $\Phi$ is an $N$-function if  
                \[
                \Phi(t)=\int_{0}^t \varphi(\tau)\ d\tau,\quad\hbox{for } t\geq 0,
                \]
                where $\varphi:\mathbb{R}^+\rightarrow \mathbb{R}^+$ is a right continuous, non decreasing function  satisfying   $\varphi(0)=0$, $\varphi(t)>0$ for $t>0$ and
                $\lim_{t\rightarrow \infty}\varphi(t)=+\infty$. We denote by $\lphi_d$ the Orlicz space associated to the $N$-function $\Phi$ of functions defined on $[0,T]$ taking values in $\rr^d$.

\item Given a function $\varphi$ as above, we  consider the so-called right inverse function $\psi$ of $\varphi$ which is 
defined by $\psi(s)=\sup_{\varphi(t)\leq s}t$.
The function $\psi$ satisfies the same properties as the function $\varphi$, therefore we have an $N$-function $\Psi$ such that $\Psi'=\psi$ .
 The function $\Psi$ is called the \emph{complementary function} of $\Phi$.

\item For $\b{u}:[0,T]\to\rr^d$ measurable we define the \emph{modular function}
 \[\rho_{\Phi}(\b{u}):= \int_0^T \Phi(|\b{u}|)\ dt.\]

\item The \emph{Orlicz class} $C_d^{\Phi}=C_d^{\Phi}([0,T])$  is given  by
\begin{equation}\label{claseOrlicz}
  C^{\Phi}_d:=\left\{\b{u} | \rho_{\Phi}(\b{u})< \infty \right\}.
\end{equation}
\item The \emph{Orlicz space} $\lphi_d=L^{\Phi}_d([0,T])$ is the linear hull of $\claseor_d$;
equivalently,
\begin{equation}\label{espacioOrlicz}
\lphi_d:=\left\{ \b{u}| \exists \lambda>0: \rho_{\Phi}(\lambda \b{u}) < \infty   \right\}.
\end{equation}
  \item The Orlicz space $\lphi_d$ equipped with the \emph{Orlicz norm}
\[
\|  \b{u}  \orlnor:=\sup \left\{  \int_0^T \b{u}\b{\cdot} \b{v}\ dt \big| \rho_{\Psi}(\b{v})\leq 1\right\},
\]
is a Banach space. 

\item  The subspace $\ephi_d=\ephi_d([0,T])$ is defined as the closure in $\lphi_d$ of the subspace $L^{\infty}_d$ of all $\mathbb{R}^d$-valued essentially bounded functions.

\item We define the \emph{Sobolev-Orlicz space} $\wphi_d$ by
\[\wphi_d:=\{\b{u}| \b{u} \hbox{ is absolutely continuous and } \b{u},\b{\dot{u}}\in \lphi_d\}.\]
$\wphi_d$ is a Banach space when it is equipped with the norm
\[
\|  \b{u}  \|_{\wphi}= \|  \b{u}  \|_{\lphi} + \|\b{\dot{u}}\orlnor.
\]

\end{itemize}


  \end{block}
  










  \begin{block}{Methodology}
              Our approach will be through the direct method of the calculus of variations, i.e. we will find solutions of \eqref{ProbPrin} 
              exhibiting extreme points of the  \emph{action integral}
              \begin{equation}\label{integral_accion}
                I(\b{u})=\int_{0}^T \mathcal{L}(t,\b{u}(t),\b{\dot{u}}(t))\ dt.
              \end{equation}
            \end{block}
            

   %         \vfill
            \begin{block}{Differentiability of action integrals in Orlicz spaces}

As a first step we need to show that the action integral is well defined and  differentiable on a subset of a Banach space. 
For this purpose we introduce the sets 
\[\Pi(\ephi_d,r):=\{\b{u}\in\lphi_d| d(\b{u},\ephi_d)<r\}\]
and 
\[\domi:=W^{1}\lphi_d\cap\{\b{u}|\b{\dot{u}}\in\Pi(\ephi_d,\lambda)\}.\]
\begin{minipage}[T]{.9\textwidth}
\begin{thm}\label{teorema_acotacion}
Let $\mathcal{L}$ be a Carath\'eodory function satisfying \eqref{cotaL}, \eqref{cotaDxL} and \eqref{cotaDyL}. 
Then the following statements hold:
\begin{itemize}
\item \label{T1item1} \label{A1} The action integral given by \eqref{integral_accion}
is finitely defined on $\domi$.

\item\label{T1item3} The function  $I$ is G\^ateaux differentiable on $\domi$ and  its derivative $I'$ is continuous from $\domi$  into $\left[\wphi_d \right]^*$. Here we consider   $\domi$ equipped with the strong topology and  $\left[\wphi_d \right]^*$ with the weak${}^*$ topology. Moreover, $I'$ is given by the following expression
\begin{equation}\label{DerAccion}
\langle  I'(\b{u}),\b{v}\rangle= \int_0^T \left\{D_{\b{x}}\mathcal{L}\big(t,\b{u},\b{\dot{u}}\big)\ccdot \b{v}+ D_{\b{y}}\mathcal{L}\big(t,\b{u},\b{\dot{u}}\big)\ccdot\b{\dot{v}}\right\} \ dt.
\end{equation}

\item\label{T1item4}  If  $\Psi \in \Delta_2$ then 
  $I'$ is continuous from $\domi$ into $\left[\wphi_d\right]^*$ when both spaces are equipped with the strong topology.



\end{itemize}
\end{thm}
\end{minipage}



                         \end{block}
            \vfill
           
          






                   }
        \end{minipage}
      \end{beamercolorbox}
    \end{column}
    % ---------------------------------------------------------%
    % end the column

    % ---------------------------------------------------------%
    % Set up a column 
    \begin{column}{.49\textwidth}
      \begin{beamercolorbox}[center,wd=\textwidth]{postercolumn}
        \begin{minipage}[T]{.95\textwidth} % tweaks the width, makes a new \textwidth
          \parbox[t][\columnheight]{\textwidth}{ % must be some better way to set the the height, width and textwidth simultaneously
            % Since all columns are the same length, it is all nice and tidy.  You have to get the height empirically
            % ---------------------------------------------------------%
            % fill each column with content
            
  \begin{block}{Critical points}


Now we are in condition  to show that  solutions of  \eqref{ProbPrin} are
critical points of $I$ on $\wphi_T$, where $\wphi_T$ denotes the subspace of $\wphi$ consisting of $T$-periodic functions.


\begin{minipage}[T]{.9\textwidth}
\begin{thm}\label{critpoint} Let $\b{u}\in\domi$ be  a $T$-periodic function. The following statements are equivalent:
\begin{itemize}
 \item\label{T1item3b} $I'(\b{u})\in\left( \wphi_T\right)^{\perp}$.
 \item  $D_{\b{y}}\mathcal{L}(t,\b{u}(t),\b{\dot{u}}(t))$ is an absolutely continuous function and $\b{u}$ solves the following boundary value problem
 \begin{equation}\label{ecualagran2}
    \left\{%
\begin{array}{ll}
   \frac{d}{dt} D_{y}\mathcal{L}(t,\b{u}(t),\b{\dot{u}}(t))= D_{\b{x}}\mathcal{L}(t,\b{u}(t),\b{\dot{u}}(t)) \quad \hbox{a.e.}\ t \in (0,T)\\
    \b{u}(0)-\b{u}(T)=D_{\b{y}}\mathcal{L}(0,\b{u}(0),\b{\dot{u}}(0))-D_{\b{y}}\mathcal{L}(T,\b{u}(T),\b{\dot{u}}(T))=0.
\end{array}%
\right.
\end{equation}
\end{itemize}
Moreover if $D_{\b{y}}\mathcal{L}(t,x,y)$ is $T$-periodic with respect to the variable $t$ and strictly convex with respect to $\b{y}$, then
$D_{\b{y}}\mathcal{L}(0,\b{u}(0),\b{\b{\dot{\b{u}}}}(0))-D_{\b{y}}\mathcal{L}(T,\b{u}(T),\b{\dot{u}}(T))=0$ is equivalent to $\b{\dot{u}}(0)=\b{\dot{u}}(T)$.
\end{thm}
\end{minipage}
\end{block}
      %      \vfill
            \begin{block}{Coercivity}

As usual, in order to establish coercivity of our action integral we need to suppose an adequate  bound from below of the lagrangian function. In this respect we assume that   

\begin{equation}\label{cota_inf}
\mathcal{L}(t,\b{x},\b{y})\geq \alpha_0\Phi\left(\frac{|\b{y}|}{\Lambda}\right)+ F(t,\b{x}),
\end{equation}
where $\alpha_0,\Lambda>0$ and  $F:\rr\times\rr^d\to\rr$ is a Carath\'eodory function, i.e. $F(t,\b{x})$ is  measurable with respect to $t$ for every fixed  $\b{x}\in\rr^d$ and $F$ is continuous at $\b{x}$ for a.e. $t\in [0,T]$. We need to assume 
\begin{equation}\label{condA1} |F(t,\b{x})|\leq a(|\b{x}|)b_0(t),\quad\text{for \,a.e. }t\in [0,T] \quad\text{and for every } \b{x}\in\rr^d,
\end{equation}
where $b_0\in L^1_1$. 


We say that $F$ satisfies the condition (A) if  $F(t,\b{x})$ is a Carath\'eo\-dory function, $F$ verifies \eqref{condA1} and $F$ is continuously differentiable with respect to $\b{x}$. Moreover, the next inequality holds 
\begin{equation}\label{condA2} |D_{\b{x}}F(t,\b{x})|\leq a(|\b{x}|)b_0(t),\quad\text{for a.e. }t\in [0,T] \text{ and for every }\b{x}\in\rr^d.
\end{equation}

We can show that the coercivity of the action integral $I$ is obtained if the functional
\begin{equation}\label{func_phi}
  J_{C,\nu}(\b{u}):= \rho_{\Phi}\left(\frac{\b{u}}{\Lambda}\right)-C\|\b{u}\orlnor^{\nu},
\end{equation}
is coercive for $C,\nu>0$.




 If $\Phi(x)=|x|^p/p$ then $J_{C,\nu}$ is clearly coercive for $\nu<p$. For more general $\Phi$ the situation is more interesting.


\begin{minipage}[T]{.9\textwidth}
\begin{lem}\label{lem_coer} Let $\Phi$ and $\Psi$ be complementary $N$-functions. Then:
\begin{itemize}
  \item If $C\Lambda<1$, then $J_{C,1}$ is coercive. 
  
  \item If $\Psi \in \Delta_2$ globally, then there exists a constant $\alpha_{\Phi}>1$ such that, for any $0<\mu<\alpha_{\Phi}$,
\begin{equation}\label{coer_modular} \lim\limits_{\|\b{u}\orlnor \to \infty} \frac{\rho_{\Phi}\left(\frac{\b{u}}{\Lambda}\right)}{\|\b{u}\orlnor^{\mu}}=+\infty.
\end{equation}
In particular, the functional $J_{C,\mu}$ is coercive for every $C>0$ and  $0<\mu<a_{\Phi}$. The constant $\alpha_{\Phi}$ is one of the so-called \emph{ Matuszewska-Orlicz indices}.
\item If $J_{C,1}$ is coercive with $C\Lambda>1$, then $\Psi \in \Delta_2$.  
\end{itemize}
\end{lem}
\end{minipage}

\begin{minipage}[T]{.9\textwidth}
\begin{thm}[Coercivity I]\label{coercitividad1}
Let  $\mathcal{L}$ be a lagrangian function satisfying \eqref{cotaL}, \eqref{cotaDxL}, \eqref{cotaDyL}, \eqref{cota_inf} and \eqref{condA1}. We assume the following conditions:
\begin{itemize}
\item There exist a non negative function  $b_1 \in L^1_1$ and a constant $\mu>0$  such that for any $\b{x_1},\b{x_2}\in\rr^d$ and a.e. $t\in [0,T]$
\begin{equation}\label{holder_cont}
  \left| F(t,\b{x_2})- F(t,\b{x_1}) \right|\leq b_1(t)(1+|\b{x_2}-\b{x_1}|^{\mu}).
\end{equation}
We suppose that $\mu< \alpha_{\Phi}$,  with $\alpha_{\Phi}$ as in previous lemma, in the case that $\Psi\in\Delta_2$; and, we suppose $\mu=1$  if $\Psi$ is an  arbitrary $N$-function. 
\item
\begin{equation}\label{propiedad1coercividad}
\int_{0}^{T}F(t,\b{x})\ dt \rightarrow \infty \quad \hbox{as} \quad |\b{x}|\rightarrow \infty.
\end{equation}
\item\label{hipot_coer}  $\Psi\in\Delta_2$ or, alternatively, 
$\alpha_0^{-1}T\Phi^{-1}\left(1/T\right)\|b_1\|_{L^1}\Lambda<1$.
\end{itemize}
Then  the action integral $I$ is coercive.
\end{thm}
\end{minipage}

\begin{minipage}[T]{.9\textwidth}

We can formulate an alternative coercivity result. We need to mention  the following technical fact.

\begin{lem}[Mawhin-Willem]\label{lema_pto_cri} Suppose that $F$ satisfies condition (A) and \eqref{propiedad1coercividad}, $F(t,\cdot)$ is  differentiable and convex  a.e. $t\in [0,T]$. Then, there exists $\b{x}_0\in\rr^d$ such that
\begin{equation}\label{der_cero}
 \int_0^T D_{\b{x}} F(t,\b{x}_0)\ dt=0.
\end{equation}
\end{lem}
\end{minipage}

\begin{minipage}[T]{.9\textwidth}
\begin{thm}[Coercivity II]\label{segcoerthm}
Let $\mathcal{L}$ be as in Theorem (Coercitivity I) and let $F$ be as in previous lemma. Moreover, assume that $\Psi\in\Delta_2$ or, alternatively  $\alpha_0^{-1}T\Phi^{-1}\left(1/T\right)a(|\b{x}_0|)\|b_0\|_{L^1} \Lambda<1$, with $a$ and $b_0$ as in \eqref{condA1} and $\b{x}_0\in\rr^d$  any point satisfying  \eqref{der_cero}. Then $I$ is coercive. 

\end{thm}
\end{minipage}




            \end{block}
         %   \vfill
            \begin{block}{Main results}

% \begin{minipage}[T]{.9\textwidth}
% \begin{thm} 
Let $\Phi$ and $\Psi$ be complementary $N$-functions. 
Suppose that the Cara\-th\'eodory function $\mathcal{L}(t,\b{x},\b{y})$ is strictly convex at $\b{y}$, $D_{\b{y}}\mathcal{L}$ is $T$-periodic with respect to $T$  and \eqref{cotaL}, \eqref{cotaDxL}, \eqref{cotaDyL}, \eqref{cota_inf}, \eqref{condA1} and \eqref{propiedad1coercividad} are satisfied. In addition, assume that some of the following statements hold (we recall the definitions and properties of $\alpha_0$, $b_1$, $\b{x}_0$ and $b_0$ from \eqref{cota_inf}, \eqref{holder_cont}, \eqref{der_cero} and \eqref{condA2} respectively):
\begin{itemize}
  \item\label{item1prin} $\Psi\in\Delta_2$ and  \eqref{holder_cont}.
\item \eqref{holder_cont} and  $\alpha_0^{-1}T\Phi^{-1}\left(1/T\right)\|b_1\|_{L^1}\Lambda<1$.

\item\label{item3prin} $\Psi\in\Delta_2$,  $F$ satisfies condition (A) and  $F(t,\cdot)$ is  convex  a.e. $t\in [0,T]$.

\item\label{item4prin} As previous  item  but with $\alpha_0^{-1}T\Phi^{-1}\left(1/T\right)a(|\b{x}_0|)\|b_0\|_{L^1} \Lambda<1$ instead of $\Psi\in\Delta_2$.

\end{itemize}
Then, problem \eqref{ProbPrin} has a solution.
% \end{thm}
% \end{minipage}

                      \end{block}
         \vfill

          }
          % ---------------------------------------------------------%
          % end the column
        \end{minipage}
      \end{beamercolorbox}
    \end{column}
    % ---------------------------------------------------------%
    % end the column
  \end{columns}

\end{frame}
\end{document}


%%%%%%%%%%%%%%%%%%%%%%%%%%%%%%%%%%%%%%%%%%%%%%%%%%%%%%%%%%%%%%%%%%%%%%%%%%%%%%%%%%%%%%%%%%%%%%%%%%%%
%%% Local Variables: 
%%% mode: latex
%%% TeX-PDF-mode: t
%%% End:
