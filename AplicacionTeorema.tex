\documentclass[twoside]{elsarticle}

\usepackage[spanish]{babel}
\usepackage[utf8x]{inputenc}
\usepackage{amssymb,amsthm}
\usepackage{amsmath}
\usepackage{url}



\newtheorem{thm}{Theorem}[section]
\newtheorem{cor}[thm]{Corollary}
\newtheorem{lem}[thm]{Lemma}
\newtheorem{rem}[thm]{Remark}
\newtheorem{defi}[thm]{Definition}
\newtheorem{prop}[thm]{Proposition}
\theoremstyle{remark}
\newtheorem{comentario}{Remark}




\newcommand{\orlnor}{\|_{L^{\Phi}}}
\newcommand{\lurnor}{\|^{*}_{L^{\Phi}}}
\newcommand{\linf}{\|_{L^{\infty}}}
\newcommand{\lphi}{L^{\Phi}}
\newcommand{\lpsi}{L^{\Psi}}
\newcommand{\ephi}{E^{\Phi}}
\newcommand{\claseor}{C^{\Phi}}
\newcommand{\wphi}{W^{1}\lphi}
\newcommand{\sobnor}{\|_{W^{1}\lphi}}
\newcommand{\domi}{\mathcal{E}^{\Phi}_d(\lambda)}
\renewcommand{\b}[1]{\boldsymbol{#1}}
\newcommand{\rr}{\mathbb{R}}
\newcommand{\nn}{\mathbb{N}}
\newcommand{\ccdot}{\b{\cdot}}
\renewcommand{\leq}{\leqslant} 
\newcommand{\epsi}{E^{\Psi}}



\journal{Nonlinear Analysis: TMA}

\begin{document}



Una de las principales novedades de nuestro trabajo es que se obtiene existencia de soluciones para lagrangianos $\mathcal{L}(t,\b{x},\b{y})$ donde la nolinearidad no es una potencia, ni tan siquiera, en virtud de que no hemos supuesto que las N-funciones  $\Phi$ con las que trabajamos sean de tipo $\Delta_2$,  tienen un crecimiento acotado por potencias. Por ejemplo nuestro teorema principal se aplica a lagrangianos del tipo 
 
 \[\mathcal{L}(t,\b{x},\b{y})=e^{|\b{y}|}+F(t,\b{x}),\] 
cuando $F$ satisface la condición (A) y es convexa o satisface (38).  En este caso podemos considerar la $N$ función $\Phi(x)=e^x-x-1$ que satisface las condiciones (2),(3) y (4) y su complementaria $\Psi=$???? es $\Delta_2$.. 

Otra novedad que queremos destacar es la condición (38).  Esta condición es una relajación de la hipótesis de que exista $g(t)\in L^1$ con 
\begin{equation}\label{eq:hip_ma_whi}
  |\nabla F(t,\b{x})|\leq g(t),
\end{equation}
considerada por Mawhin y Willem (ver \cite[Th.1.5]{mawhin2010critical}).  Notar que el Teorema de Valor Medio para derivadas  y   \eqref{eq:hip_ma_whi} implican  nuestra condición (38). En una serie de papers (ver por ejemplo  \cite{wu1999periodic,zhao2004periodic,tang2010periodic}) se consideraron  relajaciones de la condición \eqref{eq:hip_ma_whi} para lagrangianos del tipo potencia.
 \begin{equation}\label{eq:la_pot}\mathcal{L}(t,\b{x},\b{y})=|\b{y}|^p+F(t,\b{x}),\quad p>1
 \end{equation}
 En los papers mencionados  \eqref{eq:hip_ma_whi} fue  sustituída por la condición más débil
 \begin{equation}\label{eq:sublinear}|\nabla F(t,\b{x})|\leq b(t)|x|^{\mu-1}+b_2(t),\quad b_1,b_2\in L^1, \mu\geq 1.
 \end{equation}
 Además o bien  $\mu$ es subcrítico, esto es  $\mu<p$ o, asumiendo condiciones extras  sobre $b_1$, $\mu$ puede ser es crítico, i.e. $\mu=p$.  Nuestra condición (38) es diferente que \eqref{eq:sublinear}, por lo cual nuestros resultados son aún nuevos para Lagrangianos del tipo \eqref{eq:la_pot}. Por ejemplo sea $f(x)$ una función definida en $\rr$, continuamente diferenciable con $f(x)=|x|$, para $|x|>1$ y definamos $F(t,x)=g(t)(f(x)+\cos(e^x))$, donde $g$ es cualquier función en $L^1$. Entonces $F$ satisface (38), puesto que como $f$ es Lipschitz tenemos
\[|F(t,x_1)-F(t,x_2)|\leq |g(t)|(K|x_1-x_2|+2).\]
Además como $f(x)=|x|$ para $|x|>1$ y como $\cos(e^x)$ es acotada, la función $F$ satisface la condición de coercitividad (39). De allí que nuestro Teorema principal se aplica a esta función $F$.  Pero $\frac{d}{dx}F=g(t)(f'(x)+\cos(e^x)e^x)$ no está acotada por ninguna potencia de $x$ a menos que $g$ sea trivial, i.e. $F$ no satisface \eqref{eq:sublinear}.

Alternativamente, consideramos la hipótesis de convexidad de $F$. Queremos destacar que una función convexa no necesariamente satisface (38). Esto se justifica por la siguiente observación: supongamos $F$ independiente de $t$ ($F(t,x)=F(x)$) y que satisface (38), entonces $F$ es sublinear, i.e.
\[|F(x)|\leq a|x|+b,\quad a>0,b>0\]
Justifiquemos esta aseveración. La desigualdad (38) implica que si $|x-y|\leq 1$, entonces $|F(x)-F(y)|\leq c$, con $c>0$. Luego si $x\in\rr^n$ buscamos $n\in\nn$ tal que $n-1\leq |x|<n$. Entonces

\[
\begin{split}
|F(x)|&=\left|F(x)-F\left(\frac{n-1}{n}x\right)+ F\left(\frac{n-1}{n}x\right)-  \cdots
+F\left(\frac{1}{n}x\right)-F(0)\right|\\
&\leq nc\\
&\leq c(|x|+1).
\end{split}
\]

Ahora vemos que la funcion $F(t,x)=|x|^2$ satisface la hipotesis de convexidad y la condición (A) se aplica a la función $F$ , sin embargo como no  essublineal  no satisface (38). La misma función muestra que nuestra condición (38) no implica \eqref{eq:sublinear}.  Tampoco (38) implica convexidad, esto es claro como muestra la función  $F=g(t)(f(x)+\cos(e^x))$ considerada antes. 


\section*{References}

 \bibliographystyle{elsarticle-num} 
 \bibliography{Biblio-ABGMS.bib}

\end{document}
