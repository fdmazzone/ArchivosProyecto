\documentclass[a4paper,10pt,draft]{article}

\usepackage{amsmath,amsthm,amsfonts,amssymb,color,enumerate}
\usepackage[matrix,arrow]{xy}
\usepackage{graphicx}
\usepackage [spanish]{babel}
\usepackage[latin1]{inputenc} % Caracteres con acentos.
\usepackage[spanish]{babel} % Titulos en espa�ol.
\usepackage{color}



\newtheorem{lem}{Lema}
\newtheorem{teo}{Teorema}

%...............................................................................

\begin{document}

Hip�tesis Generales:

\begin{enumerate}
 \item $X$ espacio de Banach separable
 \item $\phi: X \rightarrow \mathbb{R}$ $N_{\infty}$ funci�n
 \item $(T; \tau ; \mu )$ espacio de medida finito $\mu(T)< \infty$
 \item $F: \tau $x$ X \rightarrow \mathbb{R}$, funci�n medible para cada $x \in X$ (medible respecto a la variable tiempo)
 \item $f_t = F(t,.)$ es localmente Lipschitz.
 \item $ \forall x$, $ \forall \xi \in \partial{f_t}(x)$, $  \exists $$ \lambda, \Lambda > 0$ tal que:

      \[\phi^* \Big(\frac{\xi}{\lambda}\Big) \leq C \Big(\phi \Big(\frac{x}{\Lambda}\Big)+ b(t)\Big)\]
			
			con $b(t) \in L^1(T;[0,\infty))$
			
\end{enumerate}

Observaci�n: Veamos que la Hip�tesis B del libro de Clarke (P�g 83), est� incluida en esta nueva hip�tesis, cuando tomamos $\phi(x)=||x||_{X}^p$.
Nuestra hip�tesis $6$, tomando la funci�n $\phi$ como dijimos anteriormente, $b(t)=\frac{1}{\Lambda^p}$, entonces $\phi^*(\xi)=||\xi||_{X^*}^q$. As� nuestra hip�tesis $6$ se escribe

\[ \frac{||\xi||_{X^*}^q}{\lambda^q} \leq C \Big(\frac{||x||_{X}^p}{\Lambda^p} +\frac{1}{\Lambda^p} \Big)\]
elevando todo a la $\frac{1}{q}$
 
\[ \frac{||\xi||_{X^*}}{\lambda} \leq C_1 \Big(\frac{||x||_{X}^p}{\Lambda^p} +\frac{1}{\Lambda^p}\Big)^\frac{1}{q}\leq C_2 \Big(\frac{||x||_{X}^{\frac{p}{q}}}{\Lambda^\frac{p}{q}} +\frac{1}{\Lambda^\frac{p}{q}} \Big) \leq \frac{C_2}{\Lambda^{(p-1)}} (||x||_X^{(p-1)}+1)\] 

As�, obetenemos que 

\[||\xi||_{X*} \leq C_3 (||x||_X^{(p-1)}+1)\]

Luego se cumple la Hip�tesis B.

Ahora, reciprocamente, si se cumple la Hip�tesis B entonces tenemos

\[||\xi||_{X*} \leq C (||x||_X^{(p-1)}+1)\]

elevando la desigualdad anterior a la $q$

\[||\xi||_{X*}^q \leq C (||x||_X^{(p-1)}+1)^q \leq C_1 (||x||_X^{(p-1)q}+1)= C_1 (||x||_X^{p}+1)\]

As� tomando $ \Lambda=1$, $\lambda=1$ y $b(t)=1$ se cumple nuestra hip�tesis $6$.


\begin{lem}
Bajo las hip�tesis anteriores, la funci�n \[ F(x)= \int_T{f_t(x(t)) d\mu(t)}\] es globalmente Lipschitz sobre conjuntos acotados de $X$.
\end{lem}

\textbf{Dem:}
Como $f_t$ es localmente Lipschitz, entonces por el Teorema del Valor Medio (teo $10.17$ p�g $201$, libro de Clarke), $\exists z \in (u;x)$ :
\[ f_t(u(t))-f_t(x(t)) = < \xi_t ; u(t)-x(t)>\] donde $\xi_t \in \partial f_t(z)$ 

\textcolor{red}{como $f_t$ es medible (respecto a la variable temporal), entonces $< \xi_t ; u(t)-x(t)>$ tambi�n es una funci�n medible respecto de $t$.}

Entonces 

\[\begin{array}{lcl}
\Big| \int_T{f_t(u(t))-f_t(x(t)) d\mu(t)}\Big| &=& \Big|\int_T{<\xi_t ; u(t)-x(t)> d\mu(t)}\Big| \\
							&=& \Big|\int_T{<\frac{\xi_t}{\lambda} \lambda ; \frac{u(t)-x(t)}{||u(t)-x(t)||_{L^\phi}}||u(t)-x(t)||_{L^\phi}> d\mu(t)}\Big| \\
							&\leq& \Big| \lambda ||u(t)-x(t)||_{L^\phi} \Big(\int_T{\phi^*\Big( \frac{\xi_t}{\lambda}\Big) d\mu(t)} +\int_T{\phi \Big( \frac{u(t)-x(t)}{||u(t)-x(t)||_{L^\phi}}}\Big) d\mu(t) \Big)  \Big| \\
							& \leq & |\lambda| ||u(t)-x(t)||_{L^\phi} \Big( \int_T{C \Big (\phi\Big( \frac{z}{\Lambda} \Big) + b(t)\Big) d\mu(t)}  + \int_T{\phi\Big( \frac{u(t)-x(t)}{||u(t)-x(t)||_{L^\phi}}\Big) d\mu(t)}\Big)
\end{array}
\]

donde hemos utilizado la hip�tesis y la desigualdad de Young.

Como sabemos que $z= \alpha u(t) + (1- \alpha) x(t)$ entonces usando la convexidad de $\phi$ obtenemos: \[\phi \Big( \frac{z}{\Lambda}\Big) =\phi \Big( \frac{\alpha u(t) + (1- \alpha) x(t)}{\Lambda}\Big) \leq \alpha \phi \Big( \frac{u(t)}{\Lambda} \Big) + (1- \alpha)\phi \Big( \frac{x(t)}{\Lambda} \Big) \]

Como $\phi (\frac{u(t)}{\Lambda})$ y $ \phi (\frac{x(t)}{\Lambda}) \in \textcolor{red}{L^1}$ \textcolor{red}{(Fernando y Sonia)} y $\int_T{\phi\Big( \frac{u(t)-x(t)}{||u(t)-x(t)||_{L^\phi}}\Big) d\mu(t)} \leq 1$

Entonces llegamos a:
\[ |F(u(t))-F(x(t))| \leq |\lambda| ||u(t)-x(t)||_{L^\phi} \Big( C \alpha \int_T{ \phi \Big( \frac{u(t)}{\Lambda} \Big) d\mu(t)} + C (1- \alpha) \int_T{\phi \Big( \frac{x(t)}{\Lambda} \Big) d\mu(t)} + C\int_T{b(t) d\mu(t)} +  1  \Big)   \]

Es decir: \[ |F(u(t))-F(x(t))| \leq K ||u(t)-x(t)||_{L^\phi} \]

Luego es Lipschitz.

\begin{teo}
Bajo las hip�tesis $1-6 $ tenemos que:
\[ \partial f(x) \subset \int_T{\partial f_t(x(t)) d\mu(t)}\]
\end{teo}

\textbf{Dem:}
 Veamos que $\int_T{f_t^0(x(t); v(t))d\mu(t)} \geq f^0(x(t), v(t)) \geq <\xi,(t)> \forall v \in \textcolor{red}{X}, \xi \in \partial f(x(t))$.

Por el Lema anterior $f(x(t)) = \int_T{f_t(x(t)) d\mu(t)}$ es Lipschitz, 
\[ | f(y(t))-f(x(t))| \leq K || y(t)-x(t)||_{L^\phi}\].

Sea $\xi \in \partial f(x)$ y sea $v(t) \in \textcolor{red}{X}$ entonces
\[ f^0(x(t), v(t))= \limsup\limits_{y\rightarrow x, \lambda \downarrow 0} \int_T \frac{ f_t (y(t)+ \lambda v(t))- f_t(y(t))}{\lambda}\]

El l�mite superior se alcanza en una subsucesi�n

\[  \limsup\limits_{y_n\rightarrow x, \lambda_n \downarrow 0} \int_T \frac{ f_t (y_n(t)+ \lambda_n v(t))- f_t(y_n(t))}{\lambda_n}  \]

En el l�mite anterior podemos usar Fatou ya que:


\[\begin{split}
\Big| \frac{ f_t (y_n(t)+ \lambda_n v(t))- f_t(y_n(t))}{\lambda_n} \Big| &= |<\xi_{n,t};v(t)>| \\
                  % & = & |\lambda| |<\frac{\xi_{n,t}}{\lambda}, v(t)>|
								% & \leq & |\lambda| \Big( \phi^* \Big( \frac{\xi_{n,t}}{\lambda}\Big) + \phi (v(t))\Big)
\end{split}
\]




\end{document}